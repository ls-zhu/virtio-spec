\chapter{Basic Facilities of a Virtio Device}\label{sec:Basic Facilities of a Virtio Device}

A virtio device is discovered and identified by a bus-specific method
(see the bus specific sections: \ref{sec:Virtio Transport Options / Virtio Over PCI Bus}~\nameref{sec:Virtio Transport Options / Virtio Over PCI Bus},
\ref{sec:Virtio Transport Options / Virtio Over MMIO}~\nameref{sec:Virtio Transport Options / Virtio Over MMIO} and \ref{sec:Virtio Transport Options / Virtio Over Channel I/O}~\nameref{sec:Virtio Transport Options / Virtio Over Channel I/O}).  Each
device consists of the following parts:

\begin{itemize}
\item Device status field
\item Feature bits
\item Notifications
\item Device Configuration space
\item One or more virtqueues
\end{itemize}

\section{\field{Device Status} Field}\label{sec:Basic Facilities of a Virtio Device / Device Status Field}
During device initialization by a driver,
the driver follows the sequence of steps specified in
\ref{sec:General Initialization And Device Operation / Device
Initialization}.

The \field{device status} field provides a simple low-level
indication of the completed steps of this sequence.
It's most useful to imagine it hooked up to traffic
lights on the console indicating the status of each device.  The
following bits are defined (listed below in the order in which
they would be typically set):
\begin{description}
\item[ACKNOWLEDGE (1)] Indicates that the guest OS has found the
  device and recognized it as a valid virtio device.

\item[DRIVER (2)] Indicates that the guest OS knows how to drive the
  device.
  \begin{note}
    There could be a significant (or infinite) delay before setting
    this bit.  For example, under Linux, drivers can be loadable modules.
  \end{note}

\item[FAILED (128)] Indicates that something went wrong in the guest,
  and it has given up on the device. This could be an internal
  error, or the driver didn't like the device for some reason, or
  even a fatal error during device operation.

\item[FEATURES_OK (8)] Indicates that the driver has acknowledged all the
  features it understands, and feature negotiation is complete.

\item[DRIVER_OK (4)] Indicates that the driver is set up and ready to
  drive the device.

\item[DEVICE_NEEDS_RESET (64)] Indicates that the device has experienced
  an error from which it can't recover.

\item[SUSPEND (16)] When VIRTIO_F_SUSPEND is negotiated, indicates that the
  device has been suspended by the driver.

\end{description}

The \field{device status} field starts out as 0, and is reinitialized to 0 by
the device during reset.

\drivernormative{\subsection}{Device Status Field}{Basic Facilities of a Virtio Device / Device Status Field}
The driver MUST update \field{device status},
setting bits to indicate the completed steps of the driver
initialization sequence specified in
\ref{sec:General Initialization And Device Operation / Device
Initialization}.
The driver MUST NOT clear a
\field{device status} bit.  If the driver sets the FAILED bit,
the driver MUST later reset the device before attempting to re-initialize.

The driver SHOULD NOT rely on completion of operations of a
device if DEVICE_NEEDS_RESET is set.
\begin{note}
For example, the driver can't assume requests in flight will be
completed if DEVICE_NEEDS_RESET is set, nor can it assume that
they have not been completed.  A good implementation will try to
recover by issuing a reset.
\end{note}

The driver SHOULD NOT set SUSPEND if FEATURES_OK is not set.

The driver SHOULD NOT set SUSPEND if DRIVER_OK is not set.

When setting SUSPEND, the driver MUST re-read \field{device status} to ensure the SUSPEND bit is set.

\devicenormative{\subsection}{Device Status Field}{Basic Facilities of a Virtio Device / Device Status Field}

The device MUST NOT consume buffers or send any used buffer
notifications to the driver before DRIVER_OK.

\label{sec:Basic Facilities of a Virtio Device / Device Status Field / DEVICENEEDSRESET}The device SHOULD set DEVICE_NEEDS_RESET when it enters an error state
that a reset is needed.  If DRIVER_OK is set, after it sets DEVICE_NEEDS_RESET, the device
MUST send a device configuration change notification to the driver.

The device MUST ignore SUSPEND if FEATURES_OK is not set.

The device MUST ignore SUSPEND if VIRTIO_F_SUSPEND is not negotiated.

The device SHOULD allow settings to \field{device status} even when SUSPEND is set.

If VIRTIO_F_SUSPEND is negotiated and SUSPEND is set, the device SHOULD clear SUSPEND
and resumes operation upon DRIVER_OK.

If VIRTIO_F_SUSPEND is negotiated, when the driver sets SUSPEND,
the device SHOULD perform the following actions before presenting SUSPEND bit in the \field{device status}:

\begin{itemize}
\item Stop consuming buffers of any virtqueues and mark all finished descriptors as used.
\item Wait until all buffers that are being processed to finish and mark them as used.
\item Flush all used buffer and send used buffer notifications to the driver.
\item Record Virtqueue State of each enabled virtqueue, see section \ref{sec:Basic Facilities of a Virtio Device / Splited Virtqueues / Virtqueue State}
and \ref{sec:Basic Facilities of a Virtio Device / Packed Virtqueues / Virtqueue State}.
\item Pause its operation except \field{device status} and preserve configurations in its Device Configuration Space, see \ref{sec:Basic Facilities of a Virtio Device / Device Configuration Space}
\end{itemize}

\section{Feature Bits}\label{sec:Basic Facilities of a Virtio Device / Feature Bits}

Each virtio device offers all the features it understands.  During
device initialization, the driver reads this and tells the device the
subset that it accepts.  The only way to renegotiate is to reset
the device.

This allows for forwards and backwards compatibility: if the device is
enhanced with a new feature bit, older drivers will not write that
feature bit back to the device.  Similarly, if a driver is enhanced with a feature
that the device doesn't support, it see the new feature is not offered.

Feature bits are allocated as follows:

\begin{description}
\item[0 to 23, and 50 to 127] Feature bits for the specific device type

\item[24 to 43] Feature bits reserved for extensions to the queue and
  feature negotiation mechanisms

\item[44 to 49, and 128 and above] Feature bits reserved for future extensions.
\end{description}

\begin{note}
For example, feature bit 0 for a network device (i.e.
Device ID 1) indicates that the device supports checksumming of
packets.
\end{note}

In particular, new fields in the device configuration space are
indicated by offering a new feature bit.

To keep the feature negotiation mechanism extensible, it is
important that devices \em{do not} offer any feature bits that
they would not be able to handle if the driver accepted them
(even though drivers are not supposed to accept any unspecified,
reserved, or unsupported features even if offered, according to
the specification.) Likewise, it is important that drivers \em{do
not} accept feature bits they do not know how to handle (even
though devices are not supposed to offer any unspecified,
reserved, or unsupported features in the first place,
according to the specification.) The preferred
way for handling reserved and unexpected features is that the
driver ignores them.

In particular, this is
especially important for features limited to specific transports,
as enabling these for more transports in future versions of the
specification is highly likely to require changing the behaviour
from drivers and devices.  Drivers and devices supporting
multiple transports need to carefully maintain per-transport
lists of allowed features.

\drivernormative{\subsection}{Feature Bits}{Basic Facilities of a Virtio Device / Feature Bits}
The driver MUST NOT accept a feature which the device did not offer,
and MUST NOT accept a feature which requires another feature which was
not accepted.

The driver MUST validate the feature bits offered by the device.
The driver MUST ignore and MUST NOT accept any feature bit that is
\begin{itemize}
\item not described in this specification,
\item marked as reserved,
\item not supported for the specific transport,
\item not defined for the device type.
\end{itemize}

The driver SHOULD go into backwards compatibility mode
if the device does not offer a feature it understands, otherwise MUST
set the FAILED \field{device status} bit and cease initialization.

By contrast, the driver MUST NOT fail solely because a feature
it does not understand has been offered by the device.

\devicenormative{\subsection}{Feature Bits}{Basic Facilities of a Virtio Device / Feature Bits}
The device MUST NOT offer a feature which requires another feature
which was not offered.  The device SHOULD accept any valid subset
of features the driver accepts, otherwise it MUST fail to set the
FEATURES_OK \field{device status} bit when the driver writes it.

The device MUST NOT offer feature bits corresponding to features
it would not support if accepted by the driver (even if the
driver is prohibited from accepting the feature bits by the
specification); for the sake of clarity, this refers to feature
bits not described in this specification, reserved feature bits
and feature bits reserved or not supported for the specific
transport or the specific device type, but this does not preclude
devices written to a future version of this specification from
offering such feature bits should such a specification have a
provision for devices to support the corresponding features.

If a device has successfully negotiated a set of features
at least once (by accepting the FEATURES_OK \field{device
status} bit during device initialization), then it SHOULD
NOT fail re-negotiation of the same set of features after
a device or system reset.  Failure to do so would interfere
with resuming from suspend and error recovery.

\subsection{Legacy Interface: A Note on Feature
Bits}\label{sec:Basic Facilities of a Virtio Device / Feature
Bits / Legacy Interface: A Note on Feature Bits}

Transitional Drivers MUST detect Legacy Devices by detecting that
the feature bit VIRTIO_F_VERSION_1 is not offered.
Transitional devices MUST detect Legacy drivers by detecting that
VIRTIO_F_VERSION_1 has not been acknowledged by the driver.

In this case device is used through the legacy interface.

Legacy interface support is OPTIONAL.
Thus, both transitional and non-transitional devices and
drivers are compliant with this specification.

Requirements pertaining to transitional devices and drivers
is contained in sections named 'Legacy Interface' like this one.

When device is used through the legacy interface, transitional
devices and transitional drivers MUST operate according to the
requirements documented within these legacy interface sections.
Specification text within these sections generally does not apply
to non-transitional devices.

\section{Notifications}\label{sec:Basic Facilities of a Virtio Device
/ Notifications}

The notion of sending a notification (driver to device or device
to driver) plays an important role in this specification. The
modus operandi of the notifications is transport specific.

There are three types of notifications: 
\begin{itemize}
\item configuration change notification
\item available buffer notification
\item used buffer notification. 
\end{itemize}

Configuration change notifications and used buffer notifications are sent
by the device, the recipient is the driver. A configuration change
notification indicates that the device configuration space has changed; a
used buffer notification indicates that a buffer may have been made used
on the virtqueue designated by the notification.

Available buffer notifications are sent by the driver, the recipient is
the device. This type of notification indicates that a buffer may have
been made available on the virtqueue designated by the notification.

The semantics, the transport-specific implementations, and other
important aspects of the different notifications are specified in detail
in the following chapters.

Most transports implement notifications sent by the device to the
driver using interrupts. Therefore, in previous versions of this
specification, these notifications were often called interrupts.
Some names defined in this specification still retain this interrupt
terminology. Occasionally, the term event is used to refer to
a notification or a receipt of a notification.

\section{Device Reset}\label{sec:Basic Facilities of a Virtio Device / Device Reset}

The driver may want to initiate a device reset at various times; notably,
it is required to do so during device initialization and device cleanup.

The mechanism used by the driver to initiate the reset is transport specific.

\devicenormative{\subsection}{Device Reset}{Basic Facilities of a Virtio Device / Device Reset}

A device MUST reinitialize \field{device status} to 0 after receiving a reset.

A device MUST NOT send notifications or interact with the queues after
indicating completion of the reset by reinitializing \field{device status}
to 0, until the driver re-initializes the device.

\drivernormative{\subsection}{Device Reset}{Basic Facilities of a Virtio Device / Device Reset}

The driver SHOULD consider a driver-initiated reset complete when it
reads \field{device status} as 0.

\section{Device Configuration Space}\label{sec:Basic Facilities of a Virtio Device / Device Configuration Space}

Device configuration space is generally used for rarely-changing or
initialization-time parameters.  Where configuration fields are
optional, their existence is indicated by feature bits: Future
versions of this specification will likely extend the device
configuration space by adding extra fields at the tail.

\begin{note}
The device configuration space uses the little-endian format
for multi-byte fields.
\end{note}

Each transport also provides a generation count for the device configuration
space, which will change whenever there is a possibility that two
accesses to the device configuration space can see different versions of that
space.

\drivernormative{\subsection}{Device Configuration Space}{Basic Facilities of a Virtio Device / Device Configuration Space}
Drivers MUST NOT assume reads from
fields greater than 32 bits wide are atomic, nor are reads from
multiple fields: drivers SHOULD read device configuration space fields like so:

\begin{lstlisting}
u32 before, after;
do {
        before = get_config_generation(device);
        // read config entry/entries.
        after = get_config_generation(device);
} while (after != before);
\end{lstlisting}

For optional configuration space fields, the driver MUST check that the
corresponding feature is offered before accessing that part of the configuration
space.
\begin{note}
See section \ref{sec:General Initialization And Device Operation / Device Initialization} for details on feature negotiation.
\end{note}

Drivers MUST
NOT limit structure size and device configuration space size.  Instead,
drivers SHOULD only check that device configuration space is {\em large enough} to
contain the fields necessary for device operation.

\begin{note}
For example, if the specification states that device configuration
space 'includes a single 8-bit field' drivers should understand this to mean that
the device configuration space might also include an arbitrary amount of
tail padding, and accept any device configuration space size equal to or
greater than the specified 8-bit size.
\end{note}

\devicenormative{\subsection}{Device Configuration Space}{Basic Facilities of a Virtio Device / Device Configuration Space}
The device MUST allow reading of any device-specific configuration
field before FEATURES_OK is set by the driver.  This includes fields which are
conditional on feature bits, as long as those feature bits are offered
by the device.

\subsection{Legacy Interface: A Note on Device Configuration Space endian-ness}\label{sec:Basic Facilities of a Virtio Device / Device Configuration Space / Legacy Interface: A Note on Configuration Space endian-ness}

Note that for legacy interfaces, device configuration space is generally the
guest's native endian, rather than PCI's little-endian.
The correct endian-ness is documented for each device.

\subsection{Legacy Interface: Device Configuration Space}\label{sec:Basic Facilities of a Virtio Device / Device Configuration Space / Legacy Interface: Device Configuration Space}

Legacy devices did not have a configuration generation field, thus are
susceptible to race conditions if configuration is updated.  This
affects the block \field{capacity} (see \ref{sec:Device Types /
Block Device / Device configuration layout}) and
network \field{mac} (see \ref{sec:Device Types / Network Device /
Device configuration layout}) fields;
when using the legacy interface, drivers SHOULD
read these fields multiple times until two reads generate a consistent
result.

\section{Virtqueues}\label{sec:Basic Facilities of a Virtio Device / Virtqueues}

The mechanism for bulk data transport on virtio devices is
pretentiously called a virtqueue. Each device can have zero or more
virtqueues\footnote{For example, the simplest network device has one virtqueue for
transmit and one for receive.}.

A virtio device can have maximum of 65536 virtqueues. Each virtqueue is
identified by a virtqueue index. A virtqueue index has a value in the
range of 0 to 65535.

Driver makes requests available to device by adding
an available buffer to the queue, i.e., adding a buffer
describing the request to a virtqueue, and optionally triggering
a driver event, i.e., sending an available buffer notification
to the device.

Device executes the requests and - when complete - adds
a used buffer to the queue, i.e., lets the driver
know by marking the buffer as used. Device can then trigger
a device event, i.e., send a used buffer notification to the driver.

Device reports the number of bytes it has written to memory for
each buffer it uses. This is referred to as ``used length''.

Device is not generally required to use buffers in
the same order in which they have been made available
by the driver.

Some devices always use descriptors in the same order in which
they have been made available. These devices can offer the
VIRTIO_F_IN_ORDER feature. If negotiated, this knowledge
might allow optimizations or simplify driver and/or device code.

Each virtqueue can consist of up to 3 parts:
\begin{itemize}
\item Descriptor Area - used for describing buffers
\item Driver Area - extra data supplied by driver to the device
\item Device Area - extra data supplied by device to driver
\end{itemize}

\begin{note}
Note that previous versions of this spec used different names for
these parts (following \ref{sec:Basic Facilities of a Virtio Device / Split Virtqueues}):
\begin{itemize}
\item Descriptor Table - for the Descriptor Area
\item Available Ring - for the Driver Area
\item Used Ring - for the Device Area
\end{itemize}

\end{note}

Two formats are supported: Split Virtqueues (see \ref{sec:Basic
Facilities of a Virtio Device / Split
Virtqueues}~\nameref{sec:Basic Facilities of a Virtio Device /
Split Virtqueues}) and Packed Virtqueues (see \ref{sec:Basic
Facilities of a Virtio Device / Packed
Virtqueues}~\nameref{sec:Basic Facilities of a Virtio Device /
Packed Virtqueues}).

Every driver and device supports either the Packed or the Split
Virtqueue format, or both.

\subsection{Virtqueue Reset}\label{sec:Basic Facilities of a Virtio Device / Virtqueues / Virtqueue Reset}

When VIRTIO_F_RING_RESET is negotiated, the driver can reset a virtqueue
individually. The way to reset the virtqueue is transport specific.

Virtqueue reset is divided into two parts. The driver first resets a queue and
can afterwards optionally re-enable it.

\subsubsection{Virtqueue Reset}\label{sec:Basic Facilities of a Virtio Device / Virtqueues / Virtqueue Reset / Virtqueue Reset}

\devicenormative{\paragraph}{Virtqueue Reset}{Basic Facilities of a Virtio Device / Virtqueues / Virtqueue Reset / Virtqueue Reset}

After a queue has been reset by the driver, the device MUST NOT execute
any requests from that virtqueue, or notify the driver for it.

The device MUST reset any state of a virtqueue to the default state,
including the available state and the used state.

\drivernormative{\paragraph}{Virtqueue Reset}{Basic Facilities of a Virtio Device / Virtqueues / Virtqueue Reset / Virtqueue Reset}

After the driver tells the device to reset a queue, the driver MUST verify that
the queue has actually been reset.

After the queue has been successfully reset, the driver MAY release any
resource associated with that virtqueue.

\subsubsection{Virtqueue Re-enable}\label{sec:Basic Facilities of a Virtio Device / Virtqueues / Virtqueue Reset / Virtqueue Re-enable}

This process is the same as the initialization process of a single queue during
the initialization of the entire device.

\devicenormative{\paragraph}{Virtqueue Re-enable}{Basic Facilities of a Virtio Device / Virtqueues / Virtqueue Reset / Virtqueue Re-enable}

The device MUST observe any queue configuration that may have been
changed by the driver, like the maximum queue size.

\drivernormative{\paragraph}{Virtqueue Re-enable}{Basic Facilities of a Virtio Device / Virtqueues / Virtqueue Reset / Virtqueue Re-enable}

When re-enabling a queue, the driver MUST configure the queue resources
as during initial virtqueue discovery, but optionally with different
parameters.

\section{Split Virtqueues}\label{sec:Basic Facilities of a Virtio Device / Split Virtqueues}
The split virtqueue format was the only format supported
by the version 1.0 (and earlier) of this standard.

The split virtqueue format separates the virtqueue into several
parts, where each part is write-able by either the driver or the
device, but not both. Multiple parts and/or locations within
a part need to be updated when making a buffer
available and when marking it as used.

Each queue has a 16-bit queue size
parameter, which sets the number of entries and implies the total size
of the queue.

Each virtqueue consists of three parts:

\begin{itemize}
\item Descriptor Table - occupies the Descriptor Area
\item Available Ring - occupies the Driver Area
\item Used Ring - occupies the Device Area
\end{itemize}

where each part is physically-contiguous in guest memory,
and has different alignment requirements.

The memory alignment and size requirements, in bytes, of each part of the
virtqueue are summarized in the following table:

\begin{tabular}{|l|l|l|}
\hline
Virtqueue Part    & Alignment & Size \\
\hline \hline
Descriptor Table  & 16        & $16 * $(Queue Size) \\
\hline
Available Ring    & 2         & $6 + 2 * $(Queue Size) \\
 \hline
Used Ring         & 4         & $6 + 8 * $(Queue Size) \\
 \hline
\end{tabular}

The Alignment column gives the minimum alignment for each part
of the virtqueue.

The Size column gives the total number of bytes for each
part of the virtqueue.

Queue Size corresponds to the maximum number of buffers in the
virtqueue\footnote{For example, if Queue Size is 4 then at most 4 buffers
can be queued at any given time.}.  Queue Size value is always a
power of 2.  The maximum Queue Size value is 32768.  This value
is specified in a bus-specific way.

When the driver wants to send a buffer to the device, it fills in
a slot in the descriptor table (or chains several together), and
writes the descriptor index into the available ring.  It then
notifies the device. When the device has finished a buffer, it
writes the descriptor index into the used ring, and sends a
used buffer notification.

\drivernormative{\subsection}{Virtqueues}{Basic Facilities of a Virtio Device / Virtqueues}
The driver MUST ensure that the physical address of the first byte
of each virtqueue part is a multiple of the specified alignment value
in the above table.

\subsection{Legacy Interfaces: A Note on Virtqueue Layout}\label{sec:Basic Facilities of a Virtio Device / Virtqueues / Legacy Interfaces: A Note on Virtqueue Layout}

For Legacy Interfaces, several additional
restrictions are placed on the virtqueue layout:

Each virtqueue occupies two or more physically-contiguous pages
(usually defined as 4096 bytes, but depending on the transport;
henceforth referred to as Queue Align)
and consists of three parts:

\begin{tabular}{|l|l|l|}
\hline
Descriptor Table & Available Ring (\ldots padding\ldots) & Used Ring \\
\hline
\end{tabular}

The bus-specific Queue Size field controls the total number of bytes
for the virtqueue.
When using the legacy interface, the transitional
driver MUST retrieve the Queue Size field from the device
and MUST allocate the total number of bytes for the virtqueue
according to the following formula (Queue Align given in qalign and
Queue Size given in qsz):

\begin{lstlisting}
#define ALIGN(x) (((x) + qalign) & ~qalign)
static inline unsigned virtq_size(unsigned int qsz)
{
     return ALIGN(sizeof(struct virtq_desc)*qsz + sizeof(u16)*(3 + qsz))
          + ALIGN(sizeof(u16)*3 + sizeof(struct virtq_used_elem)*qsz);
}
\end{lstlisting}

This wastes some space with padding.
When using the legacy interface, both transitional
devices and drivers MUST use the following virtqueue layout
structure to locate elements of the virtqueue:

\begin{lstlisting}
struct virtq {
        // The actual descriptors (16 bytes each)
        struct virtq_desc desc[ Queue Size ];

        // A ring of available descriptor heads with free-running index.
        struct virtq_avail avail;

        // Padding to the next Queue Align boundary.
        u8 pad[ Padding ];

        // A ring of used descriptor heads with free-running index.
        struct virtq_used used;
};
\end{lstlisting}

\subsection{Legacy Interfaces: A Note on Virtqueue Endianness}\label{sec:Basic Facilities of a Virtio Device / Virtqueues / Legacy Interfaces: A Note on Virtqueue Endianness}

Note that when using the legacy interface, transitional
devices and drivers MUST use the native
endian of the guest as the endian of fields and in the virtqueue.
This is opposed to little-endian for non-legacy interface as
specified by this standard.
It is assumed that the host is already aware of the guest endian.

\subsection{Message Framing}\label{sec:Basic Facilities of a Virtio Device / Virtqueues / Message Framing}
The framing of messages with descriptors is
independent of the contents of the buffers. For example, a network
transmit buffer consists of a 12 byte header followed by the network
packet. This could be most simply placed in the descriptor table as a
12 byte output descriptor followed by a 1514 byte output descriptor,
but it could also consist of a single 1526 byte output descriptor in
the case where the header and packet are adjacent, or even three or
more descriptors (possibly with loss of efficiency in that case).

Note that, some device implementations have large-but-reasonable
restrictions on total descriptor size (such as based on IOV_MAX in the
host OS). This has not been a problem in practice: little sympathy
will be given to drivers which create unreasonably-sized descriptors
such as by dividing a network packet into 1500 single-byte
descriptors!

\devicenormative{\subsubsection}{Message Framing}{Basic Facilities of a Virtio Device / Message Framing}
The device MUST NOT make assumptions about the particular arrangement
of descriptors.  The device MAY have a reasonable limit of descriptors
it will allow in a chain.

\drivernormative{\subsubsection}{Message Framing}{Basic Facilities of a Virtio Device / Message Framing}
The driver MUST place any device-writable descriptor elements after
any device-readable descriptor elements.

The driver SHOULD NOT use an excessive number of descriptors to
describe a buffer.

\subsubsection{Legacy Interface: Message Framing}\label{sec:Basic Facilities of a Virtio Device / Virtqueues / Message Framing / Legacy Interface: Message Framing}

Regrettably, initial driver implementations used simple layouts, and
devices came to rely on it, despite this specification wording.  In
addition, the specification for virtio_blk SCSI commands required
intuiting field lengths from frame boundaries (see
 \ref{sec:Device Types / Block Device / Device Operation / Legacy Interface: Device Operation}~\nameref{sec:Device Types / Block Device / Device Operation / Legacy Interface: Device Operation})

Thus when using the legacy interface, the VIRTIO_F_ANY_LAYOUT
feature indicates to both the device and the driver that no
assumptions were made about framing.  Requirements for
transitional drivers when this is not negotiated are included in
each device section.

\subsection{The Virtqueue Descriptor Table}\label{sec:Basic Facilities of a Virtio Device / Virtqueues / The Virtqueue Descriptor Table}

The descriptor table refers to the buffers the driver is using for
the device. \field{addr} is a physical address, and the buffers
can be chained via \field{next}. Each descriptor describes a
buffer which is read-only for the device (``device-readable'') or write-only for the device (``device-writable''), but a chain of
descriptors can contain both device-readable and device-writable buffers.

The actual contents of the memory offered to the device depends on the
device type.  Most common is to begin the data with a header
(containing little-endian fields) for the device to read, and postfix
it with a status tailer for the device to write.

\begin{lstlisting}
struct virtq_desc {
        /* Address (guest-physical). */
        le64 addr;
        /* Length. */
        le32 len;

/* This marks a buffer as continuing via the next field. */
#define VIRTQ_DESC_F_NEXT   1
/* This marks a buffer as device write-only (otherwise device read-only). */
#define VIRTQ_DESC_F_WRITE     2
/* This means the buffer contains a list of buffer descriptors. */
#define VIRTQ_DESC_F_INDIRECT   4
        /* The flags as indicated above. */
        le16 flags;
        /* Next field if flags & NEXT */
        le16 next;
};
\end{lstlisting}

The number of descriptors in the table is defined by the queue size
for this virtqueue: this is the maximum possible descriptor chain length.

If VIRTIO_F_IN_ORDER has been negotiated, driver uses
descriptors in ring order: starting from offset 0 in the table,
and wrapping around at the end of the table.

\begin{note}
The legacy \hyperref[intro:Virtio PCI Draft]{[Virtio PCI Draft]}
referred to this structure as vring_desc, and the constants as
VRING_DESC_F_NEXT, etc, but the layout and values were identical.
\end{note}

\devicenormative{\subsubsection}{The Virtqueue Descriptor Table}{Basic Facilities of a Virtio Device / Virtqueues / The Virtqueue Descriptor Table}
A device MUST NOT write to a device-readable buffer, and a device SHOULD NOT
read a device-writable buffer (it MAY do so for debugging or diagnostic
purposes). A device MUST NOT write to any descriptor table entry.

\drivernormative{\subsubsection}{The Virtqueue Descriptor Table}{Basic Facilities of a Virtio Device / Virtqueues / The Virtqueue Descriptor Table}
Drivers MUST NOT add a descriptor chain longer than $2^{32}$ bytes in total;
this implies that loops in the descriptor chain are forbidden!

If VIRTIO_F_IN_ORDER has been negotiated, and when making a
descriptor with VRING_DESC_F_NEXT set in \field{flags} at offset
$x$ in the table available to the device, driver MUST set
\field{next} to $0$ for the last descriptor in the table
(where $x = queue\_size - 1$) and to $x + 1$ for the rest of the descriptors.

\subsubsection{Indirect Descriptors}\label{sec:Basic Facilities of a Virtio Device / Virtqueues / The Virtqueue Descriptor Table / Indirect Descriptors}

Some devices benefit by concurrently dispatching a large number
of large requests. The VIRTIO_F_INDIRECT_DESC feature allows this (see \ref{sec:virtio-queue.h}~\nameref{sec:virtio-queue.h}). To increase
ring capacity the driver can store a table of indirect
descriptors anywhere in memory, and insert a descriptor in main
virtqueue (with \field{flags}\&VIRTQ_DESC_F_INDIRECT on) that refers to memory buffer
containing this indirect descriptor table; \field{addr} and \field{len}
refer to the indirect table address and length in bytes,
respectively.

The indirect table layout structure looks like this
(\field{len} is the length of the descriptor that refers to this table,
which is a variable, so this code won't compile):

\begin{lstlisting}
struct indirect_descriptor_table {
        /* The actual descriptors (16 bytes each) */
        struct virtq_desc desc[len / 16];
};
\end{lstlisting}

The first indirect descriptor is located at start of the indirect
descriptor table (index 0), additional indirect descriptors are
chained by \field{next}. An indirect descriptor without a valid \field{next}
(with \field{flags}\&VIRTQ_DESC_F_NEXT off) signals the end of the descriptor.
A single indirect descriptor
table can include both device-readable and device-writable descriptors.

If VIRTIO_F_IN_ORDER has been negotiated, indirect descriptors
use sequential indices, in-order: index 0 followed by index 1
followed by index 2, etc.

\drivernormative{\paragraph}{Indirect Descriptors}{Basic Facilities of a Virtio Device / Virtqueues / The Virtqueue Descriptor Table / Indirect Descriptors}
The driver MUST NOT set the VIRTQ_DESC_F_INDIRECT flag unless the
VIRTIO_F_INDIRECT_DESC feature was negotiated.   The driver MUST NOT
set the VIRTQ_DESC_F_INDIRECT flag within an indirect descriptor (ie. only
one table per descriptor).

A driver MUST NOT create a descriptor chain longer than the Queue Size of
the device.

A driver MUST NOT set both VIRTQ_DESC_F_INDIRECT and VIRTQ_DESC_F_NEXT
in \field{flags}.

If VIRTIO_F_IN_ORDER has been negotiated, indirect descriptors
MUST appear sequentially, with \field{next} taking the value
of 1 for the 1st descriptor, 2 for the 2nd one, etc.

\devicenormative{\paragraph}{Indirect Descriptors}{Basic Facilities of a Virtio Device / Virtqueues / The Virtqueue Descriptor Table / Indirect Descriptors}
The device MUST ignore the write-only flag (\field{flags}\&VIRTQ_DESC_F_WRITE) in the descriptor that refers to an indirect table.

The device MUST handle the case of zero or more normal chained
descriptors followed by a single descriptor with \field{flags}\&VIRTQ_DESC_F_INDIRECT.

\begin{note}
While unusual (most implementations either create a chain solely using
non-indirect descriptors, or use a single indirect element), such a
layout is valid.
\end{note}

\subsection{The Virtqueue Available Ring}\label{sec:Basic Facilities of a Virtio Device / Virtqueues / The Virtqueue Available Ring}

The available ring has the following layout structure:

\begin{lstlisting}
struct virtq_avail {
#define VIRTQ_AVAIL_F_NO_INTERRUPT      1
        le16 flags;
        le16 idx;
        le16 ring[ /* Queue Size */ ];
        le16 used_event; /* Only if VIRTIO_F_EVENT_IDX */
};
\end{lstlisting}

The driver uses the available ring to offer buffers to the
device: each ring entry refers to the head of a descriptor chain.  It is only
written by the driver and read by the device.

\field{idx} field indicates where the driver would put the next descriptor
entry in the ring (modulo the queue size). This starts at 0, and increases.

\begin{note}
The legacy \hyperref[intro:Virtio PCI Draft]{[Virtio PCI Draft]}
referred to this structure as vring_avail, and the constant as
VRING_AVAIL_F_NO_INTERRUPT, but the layout and value were identical.
\end{note}

\drivernormative{\subsubsection}{The Virtqueue Available Ring}{Basic Facilities of a Virtio Device / Virtqueues / The Virtqueue Available Ring}
A driver MUST NOT decrement the available \field{idx} on a virtqueue (ie.
there is no way to ``unexpose'' buffers).

\subsection{Used Buffer Notification Suppression}\label{sec:Basic
Facilities of a Virtio Device / Virtqueues / Used Buffer Notification Suppression}

If the VIRTIO_F_EVENT_IDX feature bit is not negotiated,
the \field{flags} field in the available ring offers a crude mechanism for the driver to inform
the device that it doesn't want notifications when buffers are used.  Otherwise
\field{used_event} is a more performant alternative where the driver
specifies how far the device can progress before a notification is
required.

Neither of these notification suppression methods are reliable, as they
are not synchronized with the device, but they serve as
useful optimizations.

\drivernormative{\subsubsection}{Used Buffer Notification Suppression}{Basic Facilities of a Virtio Device / Virtqueues / Used Buffer Notification Suppression}
If the VIRTIO_F_EVENT_IDX feature bit is not negotiated:
\begin{itemize}
\item The driver MUST set \field{flags} to 0 or 1.
\item The driver MAY set \field{flags} to 1 to advise
the device that notifications are not needed.
\end{itemize}

Otherwise, if the VIRTIO_F_EVENT_IDX feature bit is negotiated:
\begin{itemize}
\item The driver MUST set \field{flags} to 0.
\item The driver MAY use \field{used_event} to advise the device that
notifications are unnecessary until the device writes an entry with an index specified by \field{used_event} into the used ring (equivalently, until \field{idx} in the
used ring will reach the value \field{used_event} + 1).
\end{itemize}

The driver MUST handle spurious notifications from the device.

\devicenormative{\subsubsection}{Used Buffer Notification Suppression}{Basic Facilities of a Virtio Device / Virtqueues / Used Buffer Notification Suppression}

If the VIRTIO_F_EVENT_IDX feature bit is not negotiated:
\begin{itemize}
\item The device MUST ignore the \field{used_event} value.
\item After the device writes a descriptor index into the used ring:
  \begin{itemize}
  \item If \field{flags} is 1, the device SHOULD NOT send a notification.
  \item If \field{flags} is 0, the device MUST send a notification.
  \end{itemize}
\end{itemize}

Otherwise, if the VIRTIO_F_EVENT_IDX feature bit is negotiated:
\begin{itemize}
\item The device MUST ignore the lower bit of \field{flags}.
\item After the device writes a descriptor index into the used ring:
  \begin{itemize}
  \item If the \field{idx} field in the used ring (which determined
    where that descriptor index was placed) was equal to
    \field{used_event}, the device MUST send a notification.
  \item Otherwise the device SHOULD NOT send a notification.
  \end{itemize}
\end{itemize}

\begin{note}
For example, if \field{used_event} is 0, then a device using

  VIRTIO_F_EVENT_IDX would send a used buffer notification
  to the driver after the first buffer is
  used (and again after the 65536th buffer, etc).
\end{note}

\subsection{The Virtqueue Used Ring}\label{sec:Basic Facilities of a Virtio Device / Virtqueues / The Virtqueue Used Ring}

The used ring has the following layout structure:

\begin{lstlisting}
struct virtq_used {
#define VIRTQ_USED_F_NO_NOTIFY  1
        le16 flags;
        le16 idx;
        struct virtq_used_elem ring[ /* Queue Size */];
        le16 avail_event; /* Only if VIRTIO_F_EVENT_IDX */
};

/* le32 is used here for ids for padding reasons. */
struct virtq_used_elem {
        /* Index of start of used descriptor chain. */
        le32 id;
        /*
         * The number of bytes written into the device writable portion of
         * the buffer described by the descriptor chain.
         */
        le32 len;
};
\end{lstlisting}

The used ring is where the device returns buffers once it is done with
them: it is only written to by the device, and read by the driver.

Each entry in the ring is a pair: \field{id} indicates the head entry of the
descriptor chain describing the buffer (this matches an entry
placed in the available ring by the guest earlier), and \field{len} the total
of bytes written into the buffer.

\begin{note}
\field{len} is particularly useful
for drivers using untrusted buffers: if a driver does not know exactly
how much has been written by the device, the driver would have to zero
the buffer in advance to ensure no data leakage occurs.

For example, a network driver may hand a received buffer directly to
an unprivileged userspace application.  If the network device has not
overwritten the bytes which were in that buffer, this could leak the
contents of freed memory from other processes to the application.
\end{note}

\field{idx} field indicates where the device would put the next descriptor
entry in the ring (modulo the queue size). This starts at 0, and increases.

\begin{note}
The legacy \hyperref[intro:Virtio PCI Draft]{[Virtio PCI Draft]}
referred to these structures as vring_used and vring_used_elem, and
the constant as VRING_USED_F_NO_NOTIFY, but the layout and value were
identical.
\end{note}

\subsubsection{Legacy Interface: The Virtqueue Used
Ring}\label{sec:Basic Facilities of a Virtio Device / Virtqueues
/ The Virtqueue Used Ring/ Legacy Interface: The Virtqueue Used
Ring}

Historically, many drivers ignored the \field{len} value, as a
result, many devices set \field{len} incorrectly.  Thus, when
using the legacy interface, it is generally a good idea to ignore
the \field{len} value in used ring entries if possible.  Specific
known issues are listed per device type.

\devicenormative{\subsubsection}{The Virtqueue Used Ring}{Basic Facilities of a Virtio Device / Virtqueues / The Virtqueue Used Ring}

The device MUST set \field{len} prior to updating the used \field{idx}.

The device MUST write at least \field{len} bytes to descriptor,
beginning at the first device-writable buffer,
prior to updating the used \field{idx}.

The device MAY write more than \field{len} bytes to descriptor.

\begin{note}
There are potential error cases where a device might not know what
parts of the buffers have been written.  This is why \field{len} is
permitted to be an underestimate: that's preferable to the driver believing
that uninitialized memory has been overwritten when it has not.
\end{note}

\drivernormative{\subsubsection}{The Virtqueue Used Ring}{Basic Facilities of a Virtio Device / Virtqueues / The Virtqueue Used Ring}

The driver MUST NOT make assumptions about data in device-writable buffers
beyond the first \field{len} bytes, and SHOULD ignore this data.

\subsection{In-order use of descriptors}
\label{sec:Basic Facilities of a Virtio Device / Virtqueues / In-order use of descriptors}

Some devices always use descriptors in the same order in which
they have been made available. These devices can offer the
VIRTIO_F_IN_ORDER feature. If negotiated, this knowledge allows
devices to notify the use of a batch of buffers to the driver by
only writing out a single used ring entry with the \field{id}
corresponding to the head entry of the
descriptor chain describing the last buffer in the batch.

The device then skips forward in the ring according to the size of
the batch. Accordingly, it increments the used \field{idx} by the
size of the batch.

The driver needs to look up the used \field{id} and
calculate the batch size to be able to advance to where the next
used ring entry will be written by the device.

This will result in the used ring entry at an offset matching the
first available ring entry in the batch, the used ring entry for
the next batch at an offset matching the first available ring
entry in the next batch, etc.

The skipped buffers (for which no used ring entry was written)
are assumed to have been used (read or written) by the
device completely.

\subsection{Available Buffer Notification Suppression}\label{sec:Basic Facilities of a Virtio Device / Virtqueues / Virtqueue Notification Suppression}

The device can suppress available buffer notifications in a manner
analogous to the way drivers can suppress used buffer notifications as
detailed in section \ref{sec:Basic Facilities of a Virtio Device /
Virtqueues / Used Buffer Notification Suppression}.
The device manipulates \field{flags} or \field{avail_event} in the used ring the
same way the driver manipulates \field{flags} or \field{used_event} in the available ring.

\drivernormative{\subsubsection}{Available Buffer Notification Suppression}{Basic Facilities of a Virtio Device / Virtqueues / Available Buffer Notification Suppression}

The driver MUST initialize \field{flags} in the used ring to 0 when
allocating the used ring.

If the VIRTIO_F_EVENT_IDX feature bit is not negotiated:
\begin{itemize}
\item The driver MUST ignore the \field{avail_event} value.
\item After the driver writes a descriptor index into the available ring:
  \begin{itemize}
        \item If \field{flags} is 1, the driver SHOULD NOT send a notification.
        \item If \field{flags} is 0, the driver MUST send a notification.
  \end{itemize}
\end{itemize}

Otherwise, if the VIRTIO_F_EVENT_IDX feature bit is negotiated:
\begin{itemize}
\item The driver MUST ignore the lower bit of \field{flags}.
\item After the driver writes a descriptor index into the available ring:
  \begin{itemize}
        \item If the \field{idx} field in the available ring (which determined
          where that descriptor index was placed) was equal to
          \field{avail_event}, the driver MUST send a notification.
        \item Otherwise the driver SHOULD NOT send a notification.
  \end{itemize}
\end{itemize}

\devicenormative{\subsubsection}{Available Buffer Notification Suppression}{Basic Facilities of a Virtio Device / Virtqueues / Available Buffer Notification Suppression}
If the VIRTIO_F_EVENT_IDX feature bit is not negotiated:
\begin{itemize}
\item The device MUST set \field{flags} to 0 or 1.
\item The device MAY set \field{flags} to 1 to advise
the driver that notifications are not needed.
\end{itemize}

Otherwise, if the VIRTIO_F_EVENT_IDX feature bit is negotiated:
\begin{itemize}
\item The device MUST set \field{flags} to 0.
\item The device MAY use \field{avail_event} to advise the driver that notifications are unnecessary until the driver writes entry with an index specified by \field{avail_event} into the available ring (equivalently, until \field{idx} in the
available ring will reach the value \field{avail_event} + 1).
\end{itemize}

The device MUST handle spurious notifications from the driver.

\subsection{Helpers for Operating Virtqueues}\label{sec:Basic Facilities of a Virtio Device / Virtqueues / Helpers for Operating Virtqueues}

The Linux Kernel Source code contains the definitions above and
helper routines in a more usable form, in
include/uapi/linux/virtio_ring.h. This was explicitly licensed by IBM
and Red Hat under the (3-clause) BSD license so that it can be
freely used by all other projects, and is reproduced (with slight
variation) in \ref{sec:virtio-queue.h}~\nameref{sec:virtio-queue.h}.

\subsection{Virtqueue Operation}\label{sec:Basic Facilities of a Virtio Device / Virtqueues / Virtqueue Operation}

There are two parts to virtqueue operation: supplying new
available buffers to the device, and processing used buffers from
the device.

\begin{note} As an
example, the simplest virtio network device has two virtqueues: the
transmit virtqueue and the receive virtqueue. The driver adds
outgoing (device-readable) packets to the transmit virtqueue, and then
frees them after they are used. Similarly, incoming (device-writable)
buffers are added to the receive virtqueue, and processed after
they are used.
\end{note}

What follows is the requirements of each of these two parts
when using the split virtqueue format in more detail.

\subsection{Supplying Buffers to The Device}\label{sec:Basic Facilities of a Virtio Device / Virtqueues / Supplying Buffers to The Device}

The driver offers buffers to one of the device's virtqueues as follows:

\begin{enumerate}
\item\label{itm:Basic Facilities of a Virtio Device / Virtqueues / Supplying Buffers to The Device / Place Buffers} The driver places the buffer into free descriptor(s) in the
   descriptor table, chaining as necessary (see \ref{sec:Basic Facilities of a Virtio Device / Virtqueues / The Virtqueue Descriptor Table}~\nameref{sec:Basic Facilities of a Virtio Device / Virtqueues / The Virtqueue Descriptor Table}).

\item\label{itm:Basic Facilities of a Virtio Device / Virtqueues / Supplying Buffers to The Device / Place Index} The driver places the index of the head of the descriptor chain
   into the next ring entry of the available ring.

\item Steps \ref{itm:Basic Facilities of a Virtio Device / Virtqueues / Supplying Buffers to The Device / Place Buffers} and \ref{itm:Basic Facilities of a Virtio Device / Virtqueues / Supplying Buffers to The Device / Place Index} MAY be performed repeatedly if batching
  is possible.

\item The driver performs a suitable memory barrier to ensure the device sees
  the updated descriptor table and available ring before the next
  step.

\item The available \field{idx} is increased by the number of
  descriptor chain heads added to the available ring.

\item The driver performs a suitable memory barrier to ensure that it updates
  the \field{idx} field before checking for notification suppression.

\item The driver sends an available buffer notification to the device if
    such notifications are not suppressed.
\end{enumerate}

Note that the above code does not take precautions against the
available ring buffer wrapping around: this is not possible since
the ring buffer is the same size as the descriptor table, so step
(1) will prevent such a condition.

In addition, the maximum queue size is 32768 (the highest power
of 2 which fits in 16 bits), so the 16-bit \field{idx} value can always
distinguish between a full and empty buffer.

What follows is the requirements of each stage in more detail.

\subsubsection{Placing Buffers Into The Descriptor Table}\label{sec:Basic Facilities of a Virtio Device / Virtqueues / Supplying Buffers to The Device / Placing Buffers Into The Descriptor Table}

A buffer consists of zero or more device-readable physically-contiguous
elements followed by zero or more physically-contiguous
device-writable elements (each has at least one element). This
algorithm maps it into the descriptor table to form a descriptor
chain:

for each buffer element, b:

\begin{enumerate}
\item Get the next free descriptor table entry, d
\item Set \field{d.addr} to the physical address of the start of b
\item Set \field{d.len} to the length of b.
\item If b is device-writable, set \field{d.flags} to VIRTQ_DESC_F_WRITE,
    otherwise 0.
\item If there is a buffer element after this:
    \begin{enumerate}
    \item Set \field{d.next} to the index of the next free descriptor
      element.
    \item Set the VIRTQ_DESC_F_NEXT bit in \field{d.flags}.
    \end{enumerate}
\end{enumerate}

In practice, \field{d.next} is usually used to chain free
descriptors, and a separate count kept to check there are enough
free descriptors before beginning the mappings.

\subsubsection{Updating The Available Ring}\label{sec:Basic Facilities of a Virtio Device / Virtqueues / Supplying Buffers to The Device / Updating The Available Ring}

The descriptor chain head is the first d in the algorithm
above, ie. the index of the descriptor table entry referring to the first
part of the buffer.  A naive driver implementation MAY do the following (with the
appropriate conversion to-and-from little-endian assumed):

\begin{lstlisting}
avail->ring[avail->idx % qsz] = head;
\end{lstlisting}

However, in general the driver MAY add many descriptor chains before it updates
\field{idx} (at which point they become visible to the
device), so it is common to keep a counter of how many the driver has added:

\begin{lstlisting}
avail->ring[(avail->idx + added++) % qsz] = head;
\end{lstlisting}

\subsubsection{Updating \field{idx}}\label{sec:Basic Facilities of a Virtio Device / Virtqueues / Supplying Buffers to The Device / Updating idx}

\field{idx} always increments, and wraps naturally at
65536:

\begin{lstlisting}
avail->idx += added;
\end{lstlisting}

Once available \field{idx} is updated by the driver, this exposes the
descriptor and its contents.  The device MAY
access the descriptor chains the driver created and the
memory they refer to immediately.

\drivernormative{\paragraph}{Updating idx}{Basic Facilities of a Virtio Device / Virtqueues / Supplying Buffers to The Device / Updating idx}
The driver MUST perform a suitable memory barrier before the \field{idx} update, to ensure the
device sees the most up-to-date copy.

\subsubsection{Notifying The Device}\label{sec:Basic Facilities of a Virtio Device / Virtqueues / Supplying Buffers to The Device / Notifying The Device}

The actual method of device notification is bus-specific, but generally
it can be expensive.  So the device MAY suppress such notifications if it
doesn't need them, as detailed in section \ref{sec:Basic Facilities of a Virtio Device / Virtqueues / Virtqueue Notification Suppression}.

The driver has to be careful to expose the new \field{idx}
value before checking if notifications are suppressed.

\drivernormative{\paragraph}{Notifying The Device}{Basic Facilities of a Virtio Device / Virtqueues / Supplying Buffers to The Device / Notifying The Device}
The driver MUST perform a suitable memory barrier before reading \field{flags} or
\field{avail_event}, to avoid missing a notification.

\subsection{Receiving Used Buffers From The Device}\label{sec:Basic Facilities of a Virtio Device / Virtqueues / Receiving Used Buffers From The Device}

Once the device has used buffers referred to by a descriptor (read from or written to them, or
parts of both, depending on the nature of the virtqueue and the
device), it sends a used buffer notification to the driver as detailed
in section \ref{sec:Basic Facilities of a Virtio Device / Virtqueues /
Used Buffer Notification Suppression}.

\begin{note}

For optimal performance, a driver MAY disable used buffer notifications
while processing the used ring, but beware the problem of missing
notifications between emptying the ring and reenabling notifications.  This
is usually handled by re-checking for more used buffers after
notifications are re-enabled:

\begin{lstlisting}
virtq_disable_used_buffer_notifications(vq);

for (;;) {
        if (vq->last_seen_used != le16_to_cpu(virtq->used.idx)) {
                virtq_enable_used_buffer_notifications(vq);
                mb();

                if (vq->last_seen_used != le16_to_cpu(virtq->used.idx))
                        break;

                virtq_disable_used_buffer_notifications(vq);
        }

        struct virtq_used_elem *e = virtq.used->ring[vq->last_seen_used%vsz];
        process_buffer(e);
        vq->last_seen_used++;
}
\end{lstlisting}
\end{note}

\subsection{Virtqueue State}\label{sec:Basic Facilities of a Virtio Device / Splited Virtqueues / Virtqueue State}

When VIRTIO_F_QUEUE_STATE has been negotiated, the driver can set and
get the device internal virtqueue state through the following
fields. The implementation of the interfaces is transport specific.

\subsubsection{\field{Available State} Field}

The available state field is two bytes of virtqueue state that is used by
the device to read the next available buffer. It is presented in the following format:

\begin{lstlisting}
le16 last_avail_idx;
\end{lstlisting}

The \field{last_avail_idx} field is the free-running available ring
index where the device will read the next available head of a
descriptor chain.

See also \ref{sec:Basic Facilities of a Virtio Device / Virtqueues / The Virtqueue Available Ring}.

\drivernormative{\subsubsection}{Virtqueue State}{Basic Facilities of a Virtio Device / Splited Virtqueues/ Virtqueue State}

The driver SHOULD NOT access \field{Used State} of any splited virtqueues, it SHOULD use the
used index in the used ring.

\devicenormative{\subsubsection}{Virtqueue State}{Basic Facilities of a Virtio Device / Splited Virtqueues/ Virtqueue State}

The device SHOULD only accept setting Virtqueue State of any splited virtqueues
when DRIVER_OK is not set in \field{device status} or SUSPEND is set in \field{device status}.
Otherwise the device MUST ignore any writes to Virtqueue State of any splited virtqueues.

When SUSPEND is set, the device MUST record the Available State of every enabled splited virtqueue
in \field{Available State} field,
and correspondingly restore the Available State of every enabled splited virtqueue
from \field{Available State} field when DRIVER_OK is set.

The device SHOULD reset \field{Available State} field upon a device reset.


\section{Packed Virtqueues}\label{sec:Basic Facilities of a Virtio Device / Packed Virtqueues}

Packed virtqueues is an alternative compact virtqueue layout using
read-write memory, that is memory that is both read and written
by both the device and the driver.

Use of packed virtqueues is negotiated by the VIRTIO_F_RING_PACKED
feature bit.

Packed virtqueues support up to $2^{15}$ entries each.

With current transports, virtqueues are located in guest memory
allocated by the driver.
Each packed virtqueue consists of three parts:

\begin{itemize}
\item Descriptor Ring - occupies the Descriptor Area
\item Driver Event Suppression - occupies the Driver Area
\item Device Event Suppression - occupies the Device Area
\end{itemize}

Where the Descriptor Ring in turn consists of descriptors,
and where each descriptor can contain the following parts:

\begin{itemize}
\item Buffer ID
\item Element Address
\item Element Length
\item Flags
\end{itemize}

A buffer consists of zero or more device-readable physically-contiguous
elements followed by zero or more physically-contiguous
device-writable elements (each buffer has at least one element).

When the driver wants to send such a buffer to the device, it
writes at least one available descriptor describing elements of
the buffer into the Descriptor Ring.  The descriptor(s) are
associated with a buffer by means of a Buffer ID stored within
the descriptor.

The driver then notifies the device. When the device has finished
processing the buffer, it writes a used device descriptor
including the Buffer ID into the Descriptor Ring (overwriting a
driver descriptor previously made available), and sends a
used event notification.

The Descriptor Ring is used in a circular manner: the driver writes
descriptors into the ring in order. After reaching the end of the ring,
the next descriptor is placed at the head of the ring.  Once the ring is
full of driver descriptors, the driver stops sending new requests and
waits for the device to start processing descriptors and to write out
some used descriptors before making new driver descriptors
available.

Similarly, the device reads descriptors from the ring in order and
detects that a driver descriptor has been made available.  As
processing of descriptors is completed, used descriptors are
written by the device back into the ring.

Note: after reading driver descriptors and starting their
processing in order, the device might complete their processing out
of order.  Used device descriptors are written in the order
in which their processing is complete.

The Device Event Suppression data structure is write-only by the
device. It includes information for reducing the number of
device events, i.e., sending fewer available buffer notifications
to the device.

The Driver Event Suppression data structure is read-only by the
device. It includes information for reducing the number of
driver events, i.e., sending fewer used buffer notifications 
to the driver.

\subsection{Driver and Device Ring Wrap Counters}
\label{sec:Packed Virtqueues / Driver and Device Ring Wrap Counters}
Each of the driver and the device are expected to maintain,
internally, a single-bit ring wrap counter initialized to 1.

The counter maintained by the driver is called the Driver
Ring Wrap Counter. The driver changes the value of this counter
each time it makes available the
last descriptor in the ring (after making the last descriptor
available).

The counter maintained by the device is called the Device Ring Wrap
Counter.  The device changes the value of this counter
each time it uses the last descriptor in
the ring (after marking the last descriptor used).

It is easy to see that the Driver Ring Wrap Counter in the driver matches
the Device Ring Wrap Counter in the device when both are processing the same
descriptor, or when all available descriptors have been used.

To mark a descriptor as available and used, both the driver and
the device use the following two flags:
\begin{lstlisting}
#define VIRTQ_DESC_F_AVAIL     (1 << 7)
#define VIRTQ_DESC_F_USED      (1 << 15)
\end{lstlisting}

To mark a descriptor as available, the driver sets the
VIRTQ_DESC_F_AVAIL bit in Flags to match the internal Driver
Ring Wrap Counter.  It also sets the VIRTQ_DESC_F_USED bit to match the
\emph{inverse} value (i.e. to not match the internal Driver Ring
Wrap Counter).

To mark a descriptor as used, the device sets the
VIRTQ_DESC_F_USED bit in Flags to match the internal Device
Ring Wrap Counter.  It also sets the VIRTQ_DESC_F_AVAIL bit to match the
\emph{same} value.

Thus VIRTQ_DESC_F_AVAIL and VIRTQ_DESC_F_USED bits are different
for an available descriptor and equal for a used descriptor.

Note that this observation is mostly useful for sanity-checking
as these are necessary but not sufficient conditions - for
example, all descriptors are zero-initialized. To detect used and
available descriptors it is possible for drivers and devices to
keep track of the last observed value of
VIRTQ_DESC_F_USED/VIRTQ_DESC_F_AVAIL.  Other techniques to detect
VIRTQ_DESC_F_AVAIL/VIRTQ_DESC_F_USED bit changes might also be
possible.

\subsection{Polling of available and used descriptors}
\label{sec:Packed Virtqueues / Polling of available and used descriptors}

Writes of device and driver descriptors can generally be
reordered, but each side (driver and device) are only required to
poll (or test) a single location in memory: the next device descriptor after
the one they processed previously, in circular order.

Sometimes the device needs to only write out a single used descriptor
after processing a batch of multiple available descriptors.  As
described in more detail below, this can happen when using
descriptor chaining or with in-order
use of descriptors.  In this case, the device writes out a used
descriptor with the buffer id of the last descriptor in the group.
After processing the used descriptor, both device and driver then
skip forward in the ring the number of the remaining descriptors
in the group until processing (reading for the driver and writing
for the device) the next used descriptor.

\subsection{Write Flag}
\label{sec:Packed Virtqueues / Write Flag}

In an available descriptor, the VIRTQ_DESC_F_WRITE bit within Flags
is used to mark a descriptor as corresponding to a write-only or
read-only element of a buffer.

\begin{lstlisting}
/* This marks a descriptor as device write-only (otherwise device read-only). */
#define VIRTQ_DESC_F_WRITE     2
\end{lstlisting}

In a used descriptor, this bit is used to specify whether any
data has been written by the device into any parts of the buffer.


\subsection{Element Address and Length}
\label{sec:Packed Virtqueues / Element Address and Length}

In an available descriptor, Element Address corresponds to the
physical address of the buffer element. The length of the element assumed
to be physically contiguous is stored in Element Length.

In a used descriptor, Element Address is unused. Element Length
specifies the length of the buffer that has been initialized
(written to) by the device.

Element Length is reserved for used descriptors without the
VIRTQ_DESC_F_WRITE flag, and is ignored by drivers.

\subsection{Scatter-Gather Support}
\label{sec:Packed Virtqueues / Scatter-Gather Support}

Some drivers need an ability to supply a list of multiple buffer
elements (also known as a scatter/gather list) with a request.
Two features support this: descriptor chaining and indirect descriptors.

If neither feature is in use by the driver, each buffer is
physically-contiguous, either read-only or write-only and is
described completely by a single descriptor.

While unusual (most implementations either create all lists
solely using non-indirect descriptors, or always use a single
indirect element), if both features have been negotiated, mixing
indirect and non-indirect descriptors in a ring is valid, as long as each
list only contains descriptors of a given type.

Scatter/gather lists only apply to available descriptors. A
single used descriptor corresponds to the whole list.

The device limits the number of descriptors in a list through a
transport-specific and/or device-specific value. If not limited,
the maximum number of descriptors in a list is the virt queue
size.

\subsection{Next Flag: Descriptor Chaining}
\label{sec:Packed Virtqueues / Next Flag: Descriptor Chaining}

The packed ring format allows the driver to supply
a scatter/gather list to the device
by using multiple descriptors, and setting the VIRTQ_DESC_F_NEXT bit in
Flags for all but the last available descriptor.

\begin{lstlisting}
/* This marks a buffer as continuing. */
#define VIRTQ_DESC_F_NEXT   1
\end{lstlisting}

Buffer ID is included in the last descriptor in the list.

The driver always makes the first descriptor in the list
available after the rest of the list has been written out into
the ring. This guarantees that the device will never observe a
partial scatter/gather list in the ring.

Note: all flags, including VIRTQ_DESC_F_AVAIL, VIRTQ_DESC_F_USED,
VIRTQ_DESC_F_WRITE must be set/cleared correctly in all
descriptors in the list, not just the first one.

The device only writes out a single used descriptor for the whole
list. It then skips forward according to the number of
descriptors in the list. The driver needs to keep track of the size
of the list corresponding to each buffer ID, to be able to skip
to where the next used descriptor is written by the device.

For example, if descriptors are used in the same order in which
they are made available, this will result in the used descriptor
overwriting the first available descriptor in the list, the used
descriptor for the next list overwriting the first available
descriptor in the next list, etc.

VIRTQ_DESC_F_NEXT is reserved in used descriptors, and
should be ignored by drivers.

\subsection{Indirect Flag: Scatter-Gather Support}
\label{sec:Packed Virtqueues / Indirect Flag: Scatter-Gather Support}

Some devices benefit by concurrently dispatching a large number
of large requests. The VIRTIO_F_INDIRECT_DESC feature allows this. To increase
ring capacity the driver can store a (read-only by the device) table of indirect
descriptors anywhere in memory, and insert a descriptor in the main
virtqueue (with \field{Flags} bit VIRTQ_DESC_F_INDIRECT on) that refers to
a buffer element
containing this indirect descriptor table; \field{addr} and \field{len}
refer to the indirect table address and length in bytes,
respectively.
\begin{lstlisting}
/* This means the element contains a table of descriptors. */
#define VIRTQ_DESC_F_INDIRECT   4
\end{lstlisting}

The indirect table layout structure looks like this
(\field{len} is the Buffer Length of the descriptor that refers to this table,
which is a variable):

\begin{lstlisting}
struct pvirtq_indirect_descriptor_table {
        /* The actual descriptor structures (struct pvirtq_desc each) */
        struct pvirtq_desc desc[len / sizeof(struct pvirtq_desc)];
};
\end{lstlisting}

The first descriptor is located at the start of the indirect
descriptor table, additional indirect descriptors come
immediately afterwards. The VIRTQ_DESC_F_WRITE \field{flags} bit is the
only valid flag for descriptors in the indirect table. Others
are reserved and are ignored by the device.
Buffer ID is also reserved and is ignored by the device.

In descriptors with VIRTQ_DESC_F_INDIRECT set VIRTQ_DESC_F_WRITE
is reserved and is ignored by the device.

\subsection{In-order use of descriptors}
\label{sec:Packed Virtqueues / In-order use of descriptors}

Some devices always use descriptors in the same order in which
they have been made available. These devices can offer the
VIRTIO_F_IN_ORDER feature. If negotiated, this knowledge allows
devices to notify the use of a batch of buffers to the driver by
only writing out a single used descriptor with the Buffer ID
corresponding to the last descriptor in the batch.

The device then skips forward in the ring according to the size of
the batch. The driver needs to look up the used Buffer ID and
calculate the batch size to be able to advance to where the next
used descriptor will be written by the device.

This will result in the used descriptor overwriting the first
available descriptor in the batch, the used descriptor for the
next batch overwriting the first available descriptor in the next
batch, etc.

The skipped buffers (for which no used descriptor was written)
are assumed to have been used (read or written) by the
device completely.

\subsection{Multi-buffer requests}
\label{sec:Packed Virtqueues / Multi-buffer requests}
Some devices combine multiple buffers as part of processing of a
single request.  These devices always mark the descriptor
corresponding to the first buffer in the request used after the
rest of the descriptors (corresponding to rest of the buffers) in
the request - which follow the first descriptor in ring order -
has been marked used and written out into the ring.  This
guarantees that the driver will never observe a partial request
in the ring.

\subsection{Driver and Device Event Suppression}
\label{sec:Packed Virtqueues / Driver and Device Event Suppression}
In many systems used and available buffer notifications involve
significant overhead. To mitigate this overhead,
each virtqueue includes two identical structures used for
controlling notifications between the device and the driver.

The Driver Event Suppression structure is read-only by the
device and controls the used buffer notifications sent by the device
to the driver.

The Device Event Suppression structure is read-only by
the driver and controls the available buffer notifications sent by the
driver to the device.

Each of these Event Suppression structures includes the following fields:

\begin{description}
\item [Descriptor Ring Change Event Flags] Takes values:
\begin{lstlisting}
/* Enable events */
#define RING_EVENT_FLAGS_ENABLE 0x0
/* Disable events */
#define RING_EVENT_FLAGS_DISABLE 0x1
/*
 * Enable events for a specific descriptor
 * (as specified by Descriptor Ring Change Event Offset/Wrap Counter).
 * Only valid if VIRTIO_F_EVENT_IDX has been negotiated.
 */
#define RING_EVENT_FLAGS_DESC 0x2
/* The value 0x3 is reserved */
\end{lstlisting}
\item [Descriptor Ring Change Event Offset] If Event Flags set to descriptor
specific event: offset within the ring (in units of descriptor
size). Event will only trigger when this descriptor is
made available/used respectively.
\item [Descriptor Ring Change Event Wrap Counter] If Event Flags set to descriptor
specific event: offset within the ring (in units of descriptor
size). Event will only trigger when Ring Wrap Counter
matches this value and a descriptor is
made available/used respectively.
\end{description}

After writing out some descriptors, both the device and the driver
are expected to consult the relevant structure to find out
whether a used respectively an available buffer notification should be sent.

\subsubsection{Structure Size and Alignment}
\label{sec:Packed Virtqueues / Structure Size and Alignment}

Each part of the virtqueue is physically-contiguous in guest memory,
and has different alignment requirements.

The memory alignment and size requirements, in bytes, of each part of the
virtqueue are summarized in the following table:

\begin{tabular}{|l|l|l|}
\hline
Virtqueue Part    & Alignment & Size \\
\hline \hline
Descriptor Ring  & 16        & $16 * $(Queue Size) \\
\hline
Device Event Suppression    & 4         & 4 \\
 \hline
Driver Event Suppression         & 4         & 4 \\
 \hline
\end{tabular}

The Alignment column gives the minimum alignment for each part
of the virtqueue.

The Size column gives the total number of bytes for each
part of the virtqueue.

Queue Size corresponds to the maximum number of descriptors in the
virtqueue\footnote{For example, if Queue Size is 4 then at most 4 buffers
can be queued at any given time.}.  The Queue Size value does not
have to be a power of 2.

\drivernormative{\subsection}{Virtqueues}{Basic Facilities of a Virtio Device / Packed Virtqueues}
The driver MUST ensure that the physical address of the first byte
of each virtqueue part is a multiple of the specified alignment value
in the above table.

\devicenormative{\subsection}{Virtqueues}{Basic Facilities of a Virtio Device / Packed Virtqueues}
The device MUST start processing driver descriptors in the order
in which they appear in the ring.
The device MUST start writing device descriptors into the ring in
the order in which they complete.
The device MAY reorder descriptor writes once they are started.

\subsection{The Virtqueue Descriptor Format}\label{sec:Basic
Facilities of a Virtio Device / Packed Virtqueues / The Virtqueue
Descriptor Format}

The available descriptor refers to the buffers the driver is sending
to the device. \field{addr} is a physical address, and the
descriptor is identified with a buffer using the \field{id} field.

\begin{lstlisting}
struct pvirtq_desc {
        /* Buffer Address. */
        le64 addr;
        /* Buffer Length. */
        le32 len;
        /* Buffer ID. */
        le16 id;
        /* The flags depending on descriptor type. */
        le16 flags;
};
\end{lstlisting}

The descriptor ring is zero-initialized.

\subsection{Event Suppression Structure Format}\label{sec:Basic
Facilities of a Virtio Device / Packed Virtqueues / Event Suppression Structure
Format}

The following structure is used to reduce the number of
notifications sent between driver and device.

\begin{lstlisting}
struct pvirtq_event_suppress {
        le16 {
             desc_event_off : 15; /* Descriptor Ring Change Event Offset */
             desc_event_wrap : 1; /* Descriptor Ring Change Event Wrap Counter */
        } desc; /* If desc_event_flags set to RING_EVENT_FLAGS_DESC */
        le16 {
             desc_event_flags : 2, /* Descriptor Ring Change Event Flags */
             reserved : 14; /* Reserved, set to 0 */
        } flags;
};
\end{lstlisting}

\devicenormative{\subsection}{The Virtqueue Descriptor Table}{Basic Facilities of a Virtio Device / Packed Virtqueues / The Virtqueue Descriptor Table}
A device MUST NOT write to a device-readable buffer, and a device SHOULD NOT
read a device-writable buffer.
A device MUST NOT use a descriptor unless it observes
the VIRTQ_DESC_F_AVAIL bit in its \field{flags} being changed
(e.g. as compared to the initial zero value).
A device MUST NOT change a descriptor after changing it's
the VIRTQ_DESC_F_USED bit in its \field{flags}.

\drivernormative{\subsection}{The Virtqueue Descriptor Table}{Basic Facilities of a Virtio Device / PAcked Virtqueues / The Virtqueue Descriptor Table}
A driver MUST NOT change a descriptor unless it observes
the VIRTQ_DESC_F_USED bit in its \field{flags} being changed.
A driver MUST NOT change a descriptor after changing
the VIRTQ_DESC_F_AVAIL bit in its \field{flags}.
When notifying the device, driver MUST set
\field{next_off} and
\field{next_wrap} to match the next descriptor
not yet made available to the device.
A driver MAY send multiple available buffer notifications without making
any new descriptors available to the device.

\drivernormative{\subsection}{Scatter-Gather Support}{Basic Facilities of a Virtio Device / Packed Virtqueues / Scatter-Gather Support}
A driver MUST NOT create a descriptor list longer than allowed
by the device.

A driver MUST NOT create a descriptor list longer than the Queue
Size.

This implies that loops in the descriptor list are forbidden!

The driver MUST place any device-writable descriptor elements after
any device-readable descriptor elements.

A driver MUST NOT depend on the device to use more descriptors
to be able to write out all descriptors in a list. A driver
MUST make sure there's enough space in the ring
for the whole list before making the first descriptor in the list
available to the device.

A driver MUST NOT make the first descriptor in the list available
before all subsequent descriptors comprising the list are made
available.

\devicenormative{\subsection}{Scatter-Gather Support}{Basic Facilities of a Virtio Device / Packed Virtqueues / Scatter-Gather Support}
The device MUST use descriptors in a list chained by the
VIRTQ_DESC_F_NEXT flag in the same order that they
were made available by the driver.

The device MAY limit the number of buffers it will allow in a
list.

\drivernormative{\subsection}{Indirect Descriptors}{Basic Facilities of a Virtio Device / Packed Virtqueues / The Virtqueue Descriptor Table / Indirect Descriptors}
The driver MUST NOT set the VIRTQ_DESC_F_INDIRECT flag unless the
VIRTIO_F_INDIRECT_DESC feature was negotiated.   The driver MUST NOT
set any flags except DESC_F_WRITE within an indirect descriptor.

A driver MUST NOT create a descriptor chain longer than allowed
by the device.

A driver MUST NOT write direct descriptors with
VIRTQ_DESC_F_INDIRECT set in a scatter-gather list linked by
VIRTQ_DESC_F_NEXT.
\field{flags}.

\subsection{Virtqueue Operation}\label{sec:Basic Facilities of a Virtio Device / Packed Virtqueues / Virtqueue Operation}

There are two parts to virtqueue operation: supplying new
available buffers to the device, and processing used buffers from
the device.

What follows is the requirements of each of these two parts
when using the packed virtqueue format in more detail.

\subsection{Supplying Buffers to The Device}\label{sec:Basic Facilities of a Virtio Device / Packed Virtqueues / Supplying Buffers to The Device}

The driver offers buffers to one of the device's virtqueues as follows:

\begin{enumerate}
\item The driver places the buffer into free descriptor(s) in the Descriptor Ring.

\item The driver performs a suitable memory barrier to ensure that it updates
  the descriptor(s) before checking for notification suppression.

\item If notifications are not suppressed, the driver notifies the device
    of the new available buffers.
\end{enumerate}

What follows are the requirements of each stage in more detail.

\subsubsection{Placing Available Buffers Into The Descriptor Ring}\label{sec:Basic Facilities of a Virtio Device / Virtqueues / Supplying Buffers to The Device / Placing Available Buffers Into The Descriptor Ring}

For each buffer element, b:

\begin{enumerate}
\item Get the next descriptor table entry, d
\item Get the next free buffer id value
\item Set \field{d.addr} to the physical address of the start of b
\item Set \field{d.len} to the length of b.
\item Set \field{d.id} to the buffer id
\item Calculate the flags as follows:
\begin{enumerate}
\item If b is device-writable, set the VIRTQ_DESC_F_WRITE bit to 1, otherwise 0
\item Set the VIRTQ_DESC_F_AVAIL bit to the current value of the Driver Ring Wrap Counter
\item Set the VIRTQ_DESC_F_USED bit to inverse value
\end{enumerate}
\item Perform a memory barrier to ensure that the descriptor has
      been initialized
\item Set \field{d.flags} to the calculated flags value
\item If d is the last descriptor in the ring, toggle the
      Driver Ring Wrap Counter
\item Otherwise, increment d to point at the next descriptor
\end{enumerate}

This makes a single descriptor buffer available. However, in
general the driver MAY make use of a batch of descriptors as part
of a single request. In that case, it defers updating
the descriptor flags for the first descriptor
(and the previous memory barrier) until after the rest of
the descriptors have been initialized.

Once the descriptor \field{flags} field is updated by the driver, this exposes
the descriptor and its contents.  The device MAY
access the descriptor and any following descriptors the driver created and the
memory they refer to immediately.

\drivernormative{\paragraph}{Updating flags}{Basic Facilities of a Virtio Device / Packed Virtqueues / Supplying Buffers to The Device / Updating flags}
The driver MUST perform a suitable memory barrier before the
\field{flags} update, to ensure the
device sees the most up-to-date copy.

\subsubsection{Sending Available Buffer Notifications}\label{sec:Basic Facilities
of a Virtio Device / Packed Virtqueues / Supplying Buffers to The Device
/ Sending Available Buffer Notifications}

The actual method of device notification is bus-specific, but generally
it can be expensive.  So the device MAY suppress such notifications if it
doesn't need them, using the Event Suppression structure comprising the
Device Area as detailed in section \ref{sec:Basic
Facilities of a Virtio Device / Packed Virtqueues / Event
Suppression Structure Format}.

The driver has to be careful to expose the new \field{flags}
value before checking if notifications are suppressed.

\subsubsection{Implementation Example}\label{sec:Basic Facilities of a Virtio Device / Packed Virtqueues / Supplying Buffers to The Device / Implementation Example}

Below is a driver code example. It does not attempt to reduce
the number of available buffer notifications, neither does it support
the VIRTIO_F_EVENT_IDX feature.

\begin{lstlisting}
/* Note: vq->avail_wrap_count is initialized to 1 */
/* Note: vq->sgs is an array same size as the ring */

id = alloc_id(vq);

first = vq->next_avail;
sgs = 0;
for (each buffer element b) {
        sgs++;

        vq->ids[vq->next_avail] = -1;
        vq->desc[vq->next_avail].address = get_addr(b);
        vq->desc[vq->next_avail].len = get_len(b);

        avail = vq->avail_wrap_count ? VIRTQ_DESC_F_AVAIL : 0;
        used = !vq->avail_wrap_count ? VIRTQ_DESC_F_USED : 0;
        f = get_flags(b) | avail | used;
        if (b is not the last buffer element) {
                f |= VIRTQ_DESC_F_NEXT;
        }

        /* Don't mark the 1st descriptor available until all of them are ready. */
        if (vq->next_avail == first) {
                flags = f;
        } else {
                vq->desc[vq->next_avail].flags = f;
        }

        last = vq->next_avail;

        vq->next_avail++;

        if (vq->next_avail >= vq->size) {
                vq->next_avail = 0;
                vq->avail_wrap_count ^= 1;
        }
}
vq->sgs[id] = sgs;
/* ID included in the last descriptor in the list */
vq->desc[last].id = id;
write_memory_barrier();
vq->desc[first].flags = flags;

memory_barrier();

if (vq->device_event.flags != RING_EVENT_FLAGS_DISABLE) {
        notify_device(vq);
}

\end{lstlisting}


\drivernormative{\paragraph}{Sending Available Buffer Notifications} {Basic Facilities of a Virtio Device / Packed Virtqueues / Supplying Buffers to The Device / Sending Available Buffer Notifications}
The driver MUST perform a suitable memory barrier before reading
the Event Suppression structure occupying the Device Area. Failing
to do so could result in mandatory available buffer notifications
not being sent.

\subsection{Receiving Used Buffers From The Device}\label{sec:Basic Facilities of a Virtio Device / Packed Virtqueues / Receiving Used Buffers From The Device}

Once the device has used buffers referred to by a descriptor (read from or written to them, or
parts of both, depending on the nature of the virtqueue and the
device), it sends a used buffer notification to the driver
as detailed in section \ref{sec:Basic
Facilities of a Virtio Device / Packed Virtqueues / Event
Suppression Structure Format}.

\begin{note}

For optimal performance, a driver MAY disable used buffer notifications
while processing the used buffers, but beware the problem of missing
notifications between emptying the ring and reenabling used buffer
notifications.  This is usually handled by re-checking for more used
buffers after notifications are re-enabled:

\end{note}

\begin{lstlisting}
/* Note: vq->used_wrap_count is initialized to 1 */

vq->driver_event.flags = RING_EVENT_FLAGS_DISABLE;

for (;;) {
        struct pvirtq_desc *d = vq->desc[vq->next_used];

        /*
         * Check that
         * 1. Descriptor has been made available. This check is necessary
         *    if the driver is making new descriptors available in parallel
         *    with this processing of used descriptors (e.g. from another thread).
         *    Note: there are many other ways to check this, e.g.
         *    track the number of outstanding available descriptors or buffers
         *    and check that it's not 0.
         * 2. Descriptor has been used by the device.
         */
        flags = d->flags;
        bool avail = flags & VIRTQ_DESC_F_AVAIL;
        bool used = flags & VIRTQ_DESC_F_USED;
        if (avail != vq->used_wrap_count || used != vq->used_wrap_count) {
                vq->driver_event.flags = RING_EVENT_FLAGS_ENABLE;
                memory_barrier();

                /*
                 * Re-test in case the driver made more descriptors available in
                 * parallel with the used descriptor processing (e.g. from another
                 * thread) and/or the device used more descriptors before the driver
                 * enabled events.
                 */
                flags = d->flags;
                bool avail = flags & VIRTQ_DESC_F_AVAIL;
                bool used = flags & VIRTQ_DESC_F_USED;
                if (avail != vq->used_wrap_count || used != vq->used_wrap_count) {
                        break;
                }

                vq->driver_event.flags = RING_EVENT_FLAGS_DISABLE;
        }

        read_memory_barrier();

        /* skip descriptors until the next buffer */
        id = d->id;
        assert(id < vq->size);
        sgs = vq->sgs[id];
        vq->next_used += sgs;
        if (vq->next_used >= vq->size) {
                vq->next_used -= vq->size;
                vq->used_wrap_count ^= 1;
        }

        free_id(vq, id);

        process_buffer(d);
}
\end{lstlisting}


\section{Driver Notifications} \label{sec:Basic Facilities of a Virtio Device / Driver notifications}
The driver is sometimes required to send an available buffer
notification to the device.

When VIRTIO_F_NOTIFICATION_DATA has not been negotiated,
this notification contains either a virtqueue index if
VIRTIO_F_NOTIF_CONFIG_DATA is not negotiated or device supplied virtqueue
notification config data if VIRTIO_F_NOTIF_CONFIG_DATA is negotiated.

The notification method and supplying any such virtqueue notification config data
is transport specific.

However, some devices benefit from the ability to find out the
amount of available data in the queue without accessing the virtqueue in memory:
for efficiency or as a debugging aid.

To help with these optimizations, when VIRTIO_F_NOTIFICATION_DATA
has been negotiated, driver notifications to the device include
the following information:

\begin{description}
\item [vq_index or vq_notif_config_data] Either virtqueue index or device
      supplied queue notification config data corresponding to a virtqueue.
\item [next_off] Offset
      within the ring where the next available ring entry
      will be written.
      When VIRTIO_F_RING_PACKED has not been negotiated this refers to the
      15 least significant bits of the available index.
      When VIRTIO_F_RING_PACKED has been negotiated this refers to the offset
      (in units of descriptor entries)
      within the descriptor ring where the next available
      descriptor will be written.
\item [next_wrap] Wrap Counter.
      With VIRTIO_F_RING_PACKED this is the wrap counter
      referring to the next available descriptor.
      Without VIRTIO_F_RING_PACKED this is the most significant bit
      (bit 15) of the available index.
\end{description}

Note that the driver can send multiple notifications even without
making any more buffers available. When VIRTIO_F_NOTIFICATION_DATA
has been negotiated, these notifications would then have
identical \field{next_off} and \field{next_wrap} values.

\input{shared-mem.tex}

\section{Exporting Objects}\label{sec:Basic Facilities of a Virtio Device / Exporting Objects}

When an object created by one virtio device needs to be
shared with a seperate virtio device, the first device can
export the object by generating a UUID which can then
be passed to the second device to identify the object.

What constitutes an object, how to export objects, and
how to import objects are defined by the individual device
types. It is RECOMMENDED that devices generate version 4
UUIDs as specified by \hyperref[intro:rfc4122]{[RFC4122]}.

\input{admin.tex}

\chapter{General Initialization And Device Operation}\label{sec:General Initialization And Device Operation}

We start with an overview of device initialization, then expand on the
details of the device and how each step is preformed.  This section
is best read along with the bus-specific section which describes
how to communicate with the specific device.

\section{Device Initialization}\label{sec:General Initialization And Device Operation / Device Initialization}

\drivernormative{\subsection}{Device Initialization}{General Initialization And Device Operation / Device Initialization}
The driver MUST follow this sequence to initialize a device:

\begin{enumerate}
\item Reset the device.

\item Set the ACKNOWLEDGE status bit: the guest OS has noticed the device.

\item Set the DRIVER status bit: the guest OS knows how to drive the device.

\item\label{itm:General Initialization And Device Operation /
Device Initialization / Read feature bits} Read device feature bits, and write the subset of feature bits
   understood by the OS and driver to the device.  During this step the
   driver MAY read (but MUST NOT write) the device-specific configuration fields to check that it can support the device before accepting it.

\item\label{itm:General Initialization And Device Operation / Device Initialization / Set FEATURES-OK} Set the FEATURES_OK status bit.  The driver MUST NOT accept
   new feature bits after this step.

\item\label{itm:General Initialization And Device Operation / Device Initialization / Re-read FEATURES-OK} Re-read \field{device status} to ensure the FEATURES_OK bit is still
   set: otherwise, the device does not support our subset of features
   and the device is unusable.

\item\label{itm:General Initialization And Device Operation / Device Initialization / Device-specific Setup} Perform device-specific setup, including discovery of virtqueues for the
   device, optional per-bus setup, reading and possibly writing the
   device's virtio configuration space, and population of virtqueues.

\item\label{itm:General Initialization And Device Operation / Device Initialization / Set DRIVER-OK} Set the DRIVER_OK status bit.  At this point the device is
   ``live''.
\end{enumerate}

If any of these steps go irrecoverably wrong, the driver SHOULD
set the FAILED status bit to indicate that it has given up on the
device (it can reset the device later to restart if desired).  The
driver MUST NOT continue initialization in that case.

The driver MUST NOT send any buffer available notifications to
the device before setting DRIVER_OK.

\subsection{Legacy Interface: Device Initialization}\label{sec:General Initialization And Device Operation / Device Initialization / Legacy Interface: Device Initialization}
Legacy devices did not support the FEATURES_OK status bit, and thus did
not have a graceful way for the device to indicate unsupported feature
combinations.  They also did not provide a clear mechanism to end
feature negotiation, which meant that devices finalized features on
first-use, and no features could be introduced which radically changed
the initial operation of the device.

Legacy driver implementations often used the device before setting the
DRIVER_OK bit, and sometimes even before writing the feature bits
to the device.

The result was the steps \ref{itm:General Initialization And
Device Operation / Device Initialization / Set FEATURES-OK} and
\ref{itm:General Initialization And Device Operation / Device
Initialization / Re-read FEATURES-OK} were omitted, and steps
\ref{itm:General Initialization And Device Operation /
Device Initialization / Read feature bits},
\ref{itm:General Initialization And Device Operation / Device Initialization / Device-specific Setup} and \ref{itm:General Initialization And Device Operation / Device Initialization / Set DRIVER-OK}
were conflated.

Therefore, when using the legacy interface:
\begin{itemize}
\item
The transitional driver MUST execute the initialization
sequence as described in \ref{sec:General Initialization And Device
Operation / Device Initialization}
but omitting the steps \ref{itm:General Initialization And Device
Operation / Device Initialization / Set FEATURES-OK} and
\ref{itm:General Initialization And Device Operation / Device
Initialization / Re-read FEATURES-OK}.

\item
The transitional device MUST support the driver
writing device configuration fields
before the step \ref{itm:General Initialization And Device Operation /
Device Initialization / Read feature bits}.
\item
The transitional device MUST support the driver
using the device before the step \ref{itm:General Initialization
And Device Operation / Device Initialization / Set DRIVER-OK}.
\end{itemize}

\section{Device Operation}\label{sec:General Initialization And Device Operation / Device Operation}

When operating the device, each field in the device configuration
space can be changed by either the driver or the device.

Whenever such a configuration change is triggered by the device,
driver is notified. This makes it possible for drivers to
cache device configuration, avoiding expensive configuration
reads unless notified.


\subsection{Notification of Device Configuration Changes}\label{sec:General Initialization And Device Operation / Device Operation / Notification of Device Configuration Changes}

For devices where the device-specific configuration information can be
changed, a configuration change notification is sent when a
device-specific configuration change occurs.

In addition, this notification is triggered by the device setting
DEVICE_NEEDS_RESET (see \ref{sec:Basic Facilities of a Virtio Device / Device Status Field / DEVICENEEDSRESET}).

\section{Device Cleanup}\label{sec:General Initialization And Device Operation / Device Cleanup}

Once the driver has set the DRIVER_OK status bit, all the configured
virtqueue of the device are considered live.  None of the virtqueues
of a device are live once the device has been reset.

\drivernormative{\subsection}{Device Cleanup}{General Initialization And Device Operation / Device Cleanup}

A driver MUST NOT alter virtqueue entries for exposed buffers,
i.e., buffers which have been
made available to the device (and not been used by the device)
of a live virtqueue.

Thus a driver MUST ensure a virtqueue isn't live (by device reset) before removing exposed buffers.

\chapter{Virtio Transport Options}\label{sec:Virtio Transport Options}

Virtio can use various different buses, thus the standard is split
into virtio general and bus-specific sections.

\section{Virtio Over PCI Bus}\label{sec:Virtio Transport Options / Virtio Over PCI Bus}

Virtio devices are commonly implemented as PCI devices.

A Virtio device can be implemented as any kind of PCI device:
a Conventional PCI device or a PCI Express
device.  To assure designs meet the latest level
requirements, see
the PCI-SIG home page at \url{http://www.pcisig.com} for any
approved changes.

\devicenormative{\subsection}{Virtio Over PCI Bus}{Virtio Transport Options / Virtio Over PCI Bus}
A Virtio device using Virtio Over PCI Bus MUST expose to
guest an interface that meets the specification requirements of
the appropriate PCI specification: \hyperref[intro:PCI]{[PCI]}
and \hyperref[intro:PCIe]{[PCIe]}
respectively.

\subsection{PCI Device Discovery}\label{sec:Virtio Transport Options / Virtio Over PCI Bus / PCI Device Discovery}

Any PCI device with PCI Vendor ID 0x1AF4, and PCI Device ID 0x1000 through
0x107F inclusive is a virtio device. The actual value within this range
indicates which virtio device is supported by the device.
The PCI Device ID is calculated by adding 0x1040 to the Virtio Device ID,
as indicated in section \ref{sec:Device Types}.
Additionally, devices MAY utilize a Transitional PCI Device ID range,
0x1000 to 0x103F depending on the device type.

\devicenormative{\subsubsection}{PCI Device Discovery}{Virtio Transport Options / Virtio Over PCI Bus / PCI Device Discovery}

Devices MUST have the PCI Vendor ID 0x1AF4.
Devices MUST either have the PCI Device ID calculated by adding 0x1040
to the Virtio Device ID, as indicated in section \ref{sec:Device
Types} or have the Transitional PCI Device ID depending on the device type,
as follows:

\begin{tabular}{|l|c|}
\hline
Transitional PCI Device ID  &  Virtio Device    \\
\hline \hline
0x1000      &   network device     \\
\hline
0x1001     &   block device     \\
\hline
0x1002     & memory ballooning (traditional)  \\
\hline
0x1003     &      console       \\
\hline
0x1004     &     SCSI host      \\
\hline
0x1005     &  entropy source    \\
\hline
0x1009     &   9P transport     \\
\hline
\end{tabular}

For example, the network device with the Virtio Device ID 1
has the PCI Device ID 0x1041 or the Transitional PCI Device ID 0x1000.

The PCI Subsystem Vendor ID and the PCI Subsystem Device ID MAY reflect
the PCI Vendor and Device ID of the environment (for informational purposes by the driver).

Non-transitional devices SHOULD have a PCI Device ID in the range
0x1040 to 0x107f.
Non-transitional devices SHOULD have a PCI Revision ID of 1 or higher.
Non-transitional devices SHOULD have a PCI Subsystem Device ID of 0x40 or higher.

This is to reduce the chance of a legacy driver attempting
to drive the device.

\drivernormative{\subsubsection}{PCI Device Discovery}{Virtio Transport Options / Virtio Over PCI Bus / PCI Device Discovery}
Drivers MUST match devices with the PCI Vendor ID 0x1AF4 and
the PCI Device ID in the range 0x1040 to 0x107f,
calculated by adding 0x1040 to the Virtio Device ID,
as indicated in section \ref{sec:Device Types}.
Drivers for device types listed in section \ref{sec:Virtio
Transport Options / Virtio Over PCI Bus / PCI Device Discovery}
MUST match devices with the PCI Vendor ID 0x1AF4 and
the Transitional PCI Device ID indicated in section
 \ref{sec:Virtio
Transport Options / Virtio Over PCI Bus / PCI Device Discovery}.

Drivers MUST match any PCI Revision ID value.
Drivers MAY match any PCI Subsystem Vendor ID and any
PCI Subsystem Device ID value.

\subsubsection{Legacy Interfaces: A Note on PCI Device Discovery}\label{sec:Virtio Transport Options / Virtio Over PCI Bus / PCI Device Discovery / Legacy Interfaces: A Note on PCI Device Discovery}
Transitional devices MUST have a PCI Revision ID of 0.
Transitional devices MUST have the PCI Subsystem Device ID
matching the Virtio Device ID, as indicated in section \ref{sec:Device Types}.
Transitional devices MUST have the Transitional PCI Device ID in
the range 0x1000 to 0x103f.

This is to match legacy drivers.

\subsection{PCI Device Layout}\label{sec:Virtio Transport Options / Virtio Over PCI Bus / PCI Device Layout}

The device is configured via I/O and/or memory regions (though see
\ref{sec:Virtio Transport Options / Virtio Over PCI Bus / PCI Device Layout / PCI configuration access capability}
for access via the PCI configuration space), as specified by Virtio
Structure PCI Capabilities.

Fields of different sizes are present in the device
configuration regions.
All 64-bit, 32-bit and 16-bit fields are little-endian.
64-bit fields are to be treated as two 32-bit fields,
with low 32 bit part followed by the high 32 bit part.

\drivernormative{\subsubsection}{PCI Device Layout}{Virtio Transport Options / Virtio Over PCI Bus / PCI Device Layout}

For device configuration access, the driver MUST use 8-bit wide
accesses for 8-bit wide fields, 16-bit wide and aligned accesses
for 16-bit wide fields and 32-bit wide and aligned accesses for
32-bit and 64-bit wide fields. For 64-bit fields, the driver MAY
access each of the high and low 32-bit parts of the field
independently.

\devicenormative{\subsubsection}{PCI Device Layout}{Virtio Transport Options / Virtio Over PCI Bus / PCI Device Layout}

For 64-bit device configuration fields, the device MUST allow driver
independent access to high and low 32-bit parts of the field.

\subsection{Virtio Structure PCI Capabilities}\label{sec:Virtio Transport Options / Virtio Over PCI Bus / Virtio Structure PCI Capabilities}

The virtio device configuration layout includes several structures:
\begin{itemize}
\item Common configuration
\item Notifications
\item ISR Status
\item Device-specific configuration (optional)
\item PCI configuration access
\end{itemize}

Each structure can be mapped by a Base Address register (BAR) belonging to
the function, or accessed via the special VIRTIO_PCI_CAP_PCI_CFG field in the PCI configuration space.

The location of each structure is specified using a vendor-specific PCI capability located
on the capability list in PCI configuration space of the device.
This virtio structure capability uses little-endian format; all fields are
read-only for the driver unless stated otherwise:

\begin{lstlisting}
struct virtio_pci_cap {
        u8 cap_vndr;    /* Generic PCI field: PCI_CAP_ID_VNDR */
        u8 cap_next;    /* Generic PCI field: next ptr. */
        u8 cap_len;     /* Generic PCI field: capability length */
        u8 cfg_type;    /* Identifies the structure. */
        u8 bar;         /* Where to find it. */
        u8 id;          /* Multiple capabilities of the same type */
        u8 padding[2];  /* Pad to full dword. */
        le32 offset;    /* Offset within bar. */
        le32 length;    /* Length of the structure, in bytes. */
};
\end{lstlisting}

This structure can be followed by extra data, depending on
\field{cfg_type}, as documented below.

The fields are interpreted as follows:

\begin{description}
\item[\field{cap_vndr}]
        0x09; Identifies a vendor-specific capability.

\item[\field{cap_next}]
        Link to next capability in the capability list in the PCI configuration space.

\item[\field{cap_len}]
        Length of this capability structure, including the whole of
        struct virtio_pci_cap, and extra data if any.
        This length MAY include padding, or fields unused by the driver.

\item[\field{cfg_type}]
        identifies the structure, according to the following table:

\begin{lstlisting}
/* Common configuration */
#define VIRTIO_PCI_CAP_COMMON_CFG        1
/* Notifications */
#define VIRTIO_PCI_CAP_NOTIFY_CFG        2
/* ISR Status */
#define VIRTIO_PCI_CAP_ISR_CFG           3
/* Device specific configuration */
#define VIRTIO_PCI_CAP_DEVICE_CFG        4
/* PCI configuration access */
#define VIRTIO_PCI_CAP_PCI_CFG           5
/* Shared memory region */
#define VIRTIO_PCI_CAP_SHARED_MEMORY_CFG 8
/* Vendor-specific data */
#define VIRTIO_PCI_CAP_VENDOR_CFG        9
\end{lstlisting}

        Any other value is reserved for future use.

        Each structure is detailed individually below.

        The device MAY offer more than one structure of any type - this makes it
        possible for the device to expose multiple interfaces to drivers.  The order of
        the capabilities in the capability list specifies the order of preference
        suggested by the device.  A device may specify that this ordering mechanism be
        overridden by the use of the \field{id} field.
        \begin{note}
          For example, on some hypervisors, notifications using IO accesses are
        faster than memory accesses. In this case, the device would expose two
        capabilities with \field{cfg_type} set to VIRTIO_PCI_CAP_NOTIFY_CFG:
        the first one addressing an I/O BAR, the second one addressing a memory BAR.
        In this example, the driver would use the I/O BAR if I/O resources are available, and fall back on
        memory BAR when I/O resources are unavailable.
        \end{note}

\item[\field{bar}]
        values 0x0 to 0x5 specify a Base Address register (BAR) belonging to
        the function located beginning at 10h in PCI Configuration Space
        and used to map the structure into Memory or I/O Space.
        The BAR is permitted to be either 32-bit or 64-bit, it can map Memory Space
        or I/O Space.

        Any other value is reserved for future use.

\item[\field{id}]
        Used by some device types to uniquely identify multiple capabilities
        of a certain type. If the device type does not specify the meaning of
        this field, its contents are undefined.


\item[\field{offset}]
        indicates where the structure begins relative to the base address associated
        with the BAR.  The alignment requirements of \field{offset} are indicated
        in each structure-specific section below.

\item[\field{length}]
        indicates the length of the structure.

        \field{length} MAY include padding, or fields unused by the driver, or
        future extensions.

        \begin{note}
        For example, a future device might present a large structure size of several
        MBytes.
        As current devices never utilize structures larger than 4KBytes in size,
        driver MAY limit the mapped structure size to e.g.
        4KBytes (thus ignoring parts of structure after the first
        4KBytes) to allow forward compatibility with such devices without loss of
        functionality and without wasting resources.
        \end{note}
\end{description}

A variant of this type, struct virtio_pci_cap64, is defined for
those capabilities that require offsets or lengths larger than
4GiB:

\begin{lstlisting}
struct virtio_pci_cap64 {
        struct virtio_pci_cap cap;
        le32 offset_hi;
        le32 length_hi;
};
\end{lstlisting}

Given that the \field{cap.length} and \field{cap.offset} fields
are only 32 bit, the additional \field{offset_hi} and \field{length_hi}
fields provide the most significant 32 bits of a total 64 bit offset and
length within the BAR specified by \field{cap.bar}.

\drivernormative{\subsubsection}{Virtio Structure PCI Capabilities}{Virtio Transport Options / Virtio Over PCI Bus / Virtio Structure PCI Capabilities}

The driver MUST ignore any vendor-specific capability structure which has
a reserved \field{cfg_type} value.

The driver SHOULD use the first instance of each virtio structure type they can
support.

The driver MUST accept a \field{cap_len} value which is larger than specified here.

The driver MUST ignore any vendor-specific capability structure which has
a reserved \field{bar} value.

        The drivers SHOULD only map part of configuration structure
        large enough for device operation.  The drivers MUST handle
        an unexpectedly large \field{length}, but MAY check that \field{length}
        is large enough for device operation.

The driver MUST NOT write into any field of the capability structure,
with the exception of those with \field{cap_type} VIRTIO_PCI_CAP_PCI_CFG as
detailed in \ref{drivernormative:Virtio Transport Options / Virtio Over PCI Bus / PCI Device Layout / PCI configuration access capability}.

\devicenormative{\subsubsection}{Virtio Structure PCI Capabilities}{Virtio Transport Options / Virtio Over PCI Bus / Virtio Structure PCI Capabilities}

The device MUST include any extra data (from the beginning of the \field{cap_vndr} field
through end of the extra data fields if any) in \field{cap_len}.
The device MAY append extra data
or padding to any structure beyond that.

If the device presents multiple structures of the same type, it SHOULD order
them from optimal (first) to least-optimal (last).

\subsubsection{Common configuration structure layout}\label{sec:Virtio Transport Options / Virtio Over PCI Bus / PCI Device Layout / Common configuration structure layout}

The common configuration structure is found at the \field{bar} and \field{offset} within the VIRTIO_PCI_CAP_COMMON_CFG capability; its layout is below.

\begin{lstlisting}
struct virtio_pci_common_cfg {
        /* About the whole device. */
        le32 device_feature_select;     /* read-write */
        le32 device_feature;            /* read-only for driver */
        le32 driver_feature_select;     /* read-write */
        le32 driver_feature;            /* read-write */
        le16 config_msix_vector;        /* read-write */
        le16 num_queues;                /* read-only for driver */
        u8 device_status;               /* read-write */
        u8 config_generation;           /* read-only for driver */

        /* About a specific virtqueue. */
        le16 queue_select;              /* read-write */
        le16 queue_size;                /* read-write */
        le16 queue_msix_vector;         /* read-write */
        le16 queue_enable;              /* read-write */
        le16 queue_notify_off;          /* read-only for driver */
        le64 queue_desc;                /* read-write */
        le64 queue_driver;              /* read-write */
        le64 queue_device;              /* read-write */
        le16 queue_notif_config_data;   /* read-only for driver */
        le16 queue_reset;               /* read-write */

        /* About the administration virtqueue. */
        le16 admin_queue_index;         /* read-only for driver */
        le16 admin_queue_num;         /* read-only for driver */
};
\end{lstlisting}

\begin{description}
\item[\field{device_feature_select}]
        The driver uses this to select which feature bits \field{device_feature} shows.
        Value 0x0 selects Feature Bits 0 to 31, 0x1 selects Feature Bits 32 to 63, etc.

\item[\field{device_feature}]
        The device uses this to report which feature bits it is
        offering to the driver: the driver writes to
        \field{device_feature_select} to select which feature bits are presented.

\item[\field{driver_feature_select}]
        The driver uses this to select which feature bits \field{driver_feature} shows.
        Value 0x0 selects Feature Bits 0 to 31, 0x1 selects Feature Bits 32 to 63, etc.

\item[\field{driver_feature}]
        The driver writes this to accept feature bits offered by the device.
        Driver Feature Bits selected by \field{driver_feature_select}.

\item[\field{config_msix_vector}]
        Set by the driver to the MSI-X vector for configuration change notifications.

\item[\field{num_queues}]
        The device specifies the maximum number of virtqueues supported here.
        This excludes administration virtqueues if any are supported.

\item[\field{device_status}]
        The driver writes the device status here (see \ref{sec:Basic Facilities of a Virtio Device / Device Status Field}). Writing 0 into this
        field resets the device.

\item[\field{config_generation}]
        Configuration atomicity value.  The device changes this every time the
        configuration noticeably changes.

\item[\field{queue_select}]
        Queue Select. The driver selects which virtqueue the following
        fields refer to.

\item[\field{queue_size}]
        Queue Size.  On reset, specifies the maximum queue size supported by
        the device. This can be modified by the driver to reduce memory requirements.
        A 0 means the queue is unavailable.

\item[\field{queue_msix_vector}]
        Set by the driver to the MSI-X vector for virtqueue notifications.

\item[\field{queue_enable}]
        The driver uses this to selectively prevent the device from executing requests from this virtqueue.
        1 - enabled; 0 - disabled.

\item[\field{queue_notify_off}]
        The driver reads this to calculate the offset from start of Notification structure at
        which this virtqueue is located.
        \begin{note} this is \em{not} an offset in bytes.
        See \ref{sec:Virtio Transport Options / Virtio Over PCI Bus / PCI Device Layout / Notification capability} below.
        \end{note}

\item[\field{queue_desc}]
        The driver writes the physical address of Descriptor Area here.  See section \ref{sec:Basic Facilities of a Virtio Device / Virtqueues}.

\item[\field{queue_driver}]
        The driver writes the physical address of Driver Area here.  See section \ref{sec:Basic Facilities of a Virtio Device / Virtqueues}.

\item[\field{queue_device}]
        The driver writes the physical address of Device Area here.  See section \ref{sec:Basic Facilities of a Virtio Device / Virtqueues}.

\item[\field{queue_notif_config_data}]
        This field exists only if VIRTIO_F_NOTIF_CONFIG_DATA has been negotiated.
        The driver will use this value when driver sends available buffer
        notification to the device.
        See section \ref{sec:Virtio Transport Options / Virtio Over PCI Bus / PCI-specific Initialization And Device Operation / Available Buffer Notifications}.
        \begin{note}
        This field provides the device with flexibility to determine how virtqueues
        will be referred to in available buffer notifications.
        In a trivial case the device can set \field{queue_notif_config_data} to
        the virtqueue index. Some devices may benefit from providing another value,
        for example an internal virtqueue identifier, or an internal offset
        related to the virtqueue index.
        \end{note}
        \begin{note}
        This field was previously known as queue_notify_data.
        \end{note}

\item[\field{queue_reset}]
        The driver uses this to selectively reset the queue.
        This field exists only if VIRTIO_F_RING_RESET has been
        negotiated. (see \ref{sec:Basic Facilities of a Virtio Device / Virtqueues / Virtqueue Reset}).

\item[\field{admin_queue_index}]
        The device uses this to report the index of the first administration virtqueue.
        This field is valid only if VIRTIO_F_ADMIN_VQ has been negotiated.
\item[\field{admin_queue_num}]
	The device uses this to report the number of the
	supported administration virtqueues.
	Virtqueues with index
	between \field{admin_queue_index} and (\field{admin_queue_index} +
	\field{admin_queue_num} - 1) inclusive serve as administration
	virtqueues.
	The value 0 indicates no supported administration virtqueues.
	This field is valid only if VIRTIO_F_ADMIN_VQ has been
	negotiated.
\end{description}

\devicenormative{\paragraph}{Common configuration structure layout}{Virtio Transport Options / Virtio Over PCI Bus / PCI Device Layout / Common configuration structure layout}
\field{offset} MUST be 4-byte aligned.

The device MUST present at least one common configuration capability.

The device MUST present the feature bits it is offering in \field{device_feature}, starting at bit \field{device_feature_select} $*$ 32 for any \field{device_feature_select} written by the driver.
\begin{note}
  This means that it will present 0 for any \field{device_feature_select} other than 0 or 1, since no feature defined here exceeds 63.
\end{note}

The device MUST present any valid feature bits the driver has written in \field{driver_feature}, starting at bit \field{driver_feature_select} $*$ 32 for any \field{driver_feature_select} written by the driver.  Valid feature bits are those which are subset of the corresponding \field{device_feature} bits.  The device MAY present invalid bits written by the driver.

\begin{note}
  This means that a device can ignore writes for feature bits it never
  offers, and simply present 0 on reads.  Or it can just mirror what the driver wrote
  (but it will still have to check them when the driver sets FEATURES_OK).
\end{note}

\begin{note}
  A driver shouldn't write invalid bits anyway, as per \ref{drivernormative:General Initialization And Device Operation / Device Initialization}, but this attempts to handle it.
\end{note}

The device MUST present a changed \field{config_generation} after the
driver has read a device-specific configuration value which has
changed since any part of the device-specific configuration was last
read.
\begin{note}
As \field{config_generation} is an 8-bit value, simply incrementing it
on every configuration change could violate this requirement due to wrap.
Better would be to set an internal flag when it has changed,
and if that flag is set when the driver reads from the device-specific
configuration, increment \field{config_generation} and clear the flag.
\end{note}

The device MUST reset when 0 is written to \field{device_status}, and
present a 0 in \field{device_status} once that is done.

The device MUST present a 0 in \field{queue_enable} on reset.

If VIRTIO_F_RING_RESET has been negotiated, the device MUST present a 0 in
\field{queue_reset} on reset.

If VIRTIO_F_RING_RESET has been negotiated, the device MUST present a 0 in
\field{queue_reset} after the virtqueue is enabled with \field{queue_enable}.

The device MUST reset the queue when 1 is written to \field{queue_reset}. The
device MUST continue to present 1 in \field{queue_reset} as long as the queue reset
is ongoing. The device MUST present 0 in both \field{queue_reset} and \field{queue_enable}
when queue reset has completed.
(see \ref{sec:Basic Facilities of a Virtio Device / Virtqueues / Virtqueue Reset}).

The device MUST present a 0 in \field{queue_size} if the virtqueue
corresponding to the current \field{queue_select} is unavailable.

If VIRTIO_F_RING_PACKED has not been negotiated, the device MUST
present either a value of 0 or a power of 2 in
\field{queue_size}.

If VIRTIO_F_ADMIN_VQ has been negotiated, the value
\field{admin_queue_index} MUST be equal to, or bigger than
\field{num_queues}; also, \field{admin_queue_num} MUST be
smaller than, or equal to 0x10000 - \field{admin_queue_index},
to ensure that indices of valid admin queues fit into
a 16 bit range beyond all other virtqueues.

\drivernormative{\paragraph}{Common configuration structure layout}{Virtio Transport Options / Virtio Over PCI Bus / PCI Device Layout / Common configuration structure layout}

The driver MUST NOT write to \field{device_feature}, \field{num_queues},
\field{config_generation}, \field{queue_notify_off} or
\field{queue_notif_config_data}.

If VIRTIO_F_RING_PACKED has been negotiated,
the driver MUST NOT write the value 0 to \field{queue_size}.
If VIRTIO_F_RING_PACKED has not been negotiated,
the driver MUST NOT write a value which is not a power of 2 to \field{queue_size}.

The driver MUST configure the other virtqueue fields before enabling the virtqueue
with \field{queue_enable}.

After writing 0 to \field{device_status}, the driver MUST wait for a read of
\field{device_status} to return 0 before reinitializing the device.

The driver MUST NOT write a 0 to \field{queue_enable}.

If VIRTIO_F_RING_RESET has been negotiated, after the driver writes 1 to
\field{queue_reset} to reset the queue, the driver MUST NOT consider queue
reset to be complete until it reads back 0 in \field{queue_reset}. The driver
MAY re-enable the queue by writing 1 to \field{queue_enable} after ensuring
that other virtqueue fields have been set up correctly. The driver MAY set
driver-writeable queue configuration values to different values than those that
were used before the queue reset.
(see \ref{sec:Basic Facilities of a Virtio Device / Virtqueues / Virtqueue Reset}).

If VIRTIO_F_ADMIN_VQ has been negotiated, and if the driver
configures any administration virtqueues, the driver MUST
configure the administration virtqueues using the index
in the range \field{admin_queue_index} to
\field{admin_queue_index} + \field{admin_queue_num} - 1 inclusive.
The driver MAY configure fewer administration virtqueues than
supported by the device.

\subsubsection{Notification structure layout}\label{sec:Virtio Transport Options / Virtio Over PCI Bus / PCI Device Layout / Notification capability}

The notification location is found using the VIRTIO_PCI_CAP_NOTIFY_CFG
capability.  This capability is immediately followed by an additional
field, like so:

\begin{lstlisting}
struct virtio_pci_notify_cap {
        struct virtio_pci_cap cap;
        le32 notify_off_multiplier; /* Multiplier for queue_notify_off. */
};
\end{lstlisting}

\field{notify_off_multiplier} is combined with the \field{queue_notify_off} to
derive the Queue Notify address within a BAR for a virtqueue:

\begin{lstlisting}
        cap.offset + queue_notify_off * notify_off_multiplier
\end{lstlisting}

The \field{cap.offset} and \field{notify_off_multiplier} are taken from the
notification capability structure above, and the \field{queue_notify_off} is
taken from the common configuration structure.

\begin{note}
For example, if \field{notifier_off_multiplier} is 0, the device uses
the same Queue Notify address for all queues.
\end{note}

\devicenormative{\paragraph}{Notification capability}{Virtio Transport Options / Virtio Over PCI Bus / PCI Device Layout / Notification capability}
The device MUST present at least one notification capability.

For devices not offering VIRTIO_F_NOTIFICATION_DATA:

The \field{cap.offset} MUST be 2-byte aligned.

The device MUST either present \field{notify_off_multiplier} as an even power of 2,
or present \field{notify_off_multiplier} as 0.

The value \field{cap.length} presented by the device MUST be at least 2
and MUST be large enough to support queue notification offsets
for all supported queues in all possible configurations.

For all queues, the value \field{cap.length} presented by the device MUST satisfy:
\begin{lstlisting}
cap.length >= queue_notify_off * notify_off_multiplier + 2
\end{lstlisting}

For devices offering VIRTIO_F_NOTIFICATION_DATA:

The device MUST either present \field{notify_off_multiplier} as a
number that is a power of 2 that is also a multiple 4,
or present \field{notify_off_multiplier} as 0.

The \field{cap.offset} MUST be 4-byte aligned.

The value \field{cap.length} presented by the device MUST be at least 4
and MUST be large enough to support queue notification offsets
for all supported queues in all possible configurations.

For all queues, the value \field{cap.length} presented by the device MUST satisfy:
\begin{lstlisting}
cap.length >= queue_notify_off * notify_off_multiplier + 4
\end{lstlisting}

\subsubsection{ISR status capability}\label{sec:Virtio Transport Options / Virtio Over PCI Bus / PCI Device Layout / ISR status capability}

The VIRTIO_PCI_CAP_ISR_CFG capability
refers to at least a single byte, which contains the 8-bit ISR status field
to be used for INT\#x interrupt handling.

The \field{offset} for the \field{ISR status} has no alignment requirements.

The ISR bits allow the driver to distinguish between device-specific configuration
change interrupts and normal virtqueue interrupts:

\begin{tabular}{ |l||l|l|l| }
\hline
Bits       & 0                               & 1               &  2 to 31 \\
\hline
Purpose    & Queue Interrupt  & Device Configuration Interrupt & Reserved \\
\hline
\end{tabular}

To avoid an extra access, simply reading this register resets it to 0 and
causes the device to de-assert the interrupt.

In this way, driver read of ISR status causes the device to de-assert
an interrupt.

See sections \ref{sec:Virtio Transport Options / Virtio Over PCI Bus / PCI-specific Initialization And Device Operation / Used Buffer Notifications} and \ref{sec:Virtio Transport Options / Virtio Over PCI Bus / PCI-specific Initialization And Device Operation / Notification of Device Configuration Changes} for how this is used.

\devicenormative{\paragraph}{ISR status capability}{Virtio Transport Options / Virtio Over PCI Bus / PCI Device Layout / ISR status capability}

The device MUST present at least one VIRTIO_PCI_CAP_ISR_CFG capability.

The device MUST set the Device Configuration Interrupt bit
in \field{ISR status} before sending a device configuration
change notification to the driver.

If MSI-X capability is disabled, the device MUST set the Queue
Interrupt bit in \field{ISR status} before sending a virtqueue
notification to the driver.

If MSI-X capability is disabled, the device MUST set the Interrupt Status
bit in the PCI Status register in the PCI Configuration Header of
the device to the logical OR of all bits in \field{ISR status} of
the device.  The device then asserts/deasserts INT\#x interrupts unless masked
according to standard PCI rules \hyperref[intro:PCI]{[PCI]}.

The device MUST reset \field{ISR status} to 0 on driver read.

\drivernormative{\paragraph}{ISR status capability}{Virtio Transport Options / Virtio Over PCI Bus / PCI Device Layout / ISR status capability}

If MSI-X capability is enabled, the driver SHOULD NOT access
\field{ISR status} upon detecting a Queue Interrupt.

\subsubsection{Device-specific configuration}\label{sec:Virtio Transport Options / Virtio Over PCI Bus / PCI Device Layout / Device-specific configuration}

The device MUST present at least one VIRTIO_PCI_CAP_DEVICE_CFG capability for
any device type which has a device-specific configuration.

\devicenormative{\paragraph}{Device-specific configuration}{Virtio Transport Options / Virtio Over PCI Bus / PCI Device Layout / Device-specific configuration}

The \field{offset} for the device-specific configuration MUST be 4-byte aligned.

\subsubsection{Shared memory capability}\label{sec:Virtio Transport Options / Virtio Over PCI Bus / PCI Device Layout / Shared memory capability}

Shared memory regions \ref{sec:Basic Facilities of a Virtio
Device / Shared Memory Regions} are enumerated on the PCI transport
as a sequence of VIRTIO_PCI_CAP_SHARED_MEMORY_CFG capabilities, one per region.

The capability is defined by a struct virtio_pci_cap64 and
utilises the \field{cap.id} to allow multiple shared memory
regions per device.
The identifier in \field{cap.id} does not denote a certain order of
preference; it is only used to uniquely identify a region.

\devicenormative{\paragraph}{Shared memory capability}{Virtio Transport Options / Virtio Over PCI Bus / PCI Device Layout / Shared memory capability}

The region defined by the combination of the \field{cap.offset},
\field{offset_hi}, and \field{cap.length}, \field{length_hi}
fields MUST be contained within the BAR specified by
\field{cap.bar}.

The \field{cap.id} MUST be unique for any one device instance.

\subsubsection{Vendor data capability}\label{sec:Virtio
Transport Options / Virtio Over PCI Bus / PCI Device Layout /
Vendor data capability}

The optional Vendor data capability allows the device to present
vendor-specific data to the driver, without
conflicts, for debugging and/or reporting purposes,
and without conflicting with standard functionality.

This capability augments but does not replace the standard
subsystem ID and subsystem vendor ID fields
(offsets 0x2C and 0x2E in the PCI configuration space header)
as specified by \hyperref[intro:PCI]{[PCI]}.

Vendor data capability is enumerated on the PCI transport
as a VIRTIO_PCI_CAP_VENDOR_CFG capability.

The capability has the following structure:
\begin{lstlisting}
struct virtio_pci_vndr_data {
        u8 cap_vndr;    /* Generic PCI field: PCI_CAP_ID_VNDR */
        u8 cap_next;    /* Generic PCI field: next ptr. */
        u8 cap_len;     /* Generic PCI field: capability length */
        u8 cfg_type;    /* Identifies the structure. */
        u16 vendor_id;  /* Identifies the vendor-specific format. */
	/* For Vendor Definition */
	/* Pads structure to a multiple of 4 bytes */
	/* Reads must not have side effects */
};
\end{lstlisting}

Where \field{vendor_id} identifies the PCI-SIG assigned Vendor ID
as specified by \hyperref[intro:PCI]{[PCI]}.

Note that the capability size is required to be a multiple of 4.

To make it safe for a generic driver to access the capability,
reads from this capability MUST NOT have any side effects.

\devicenormative{\paragraph}{Vendor data capability}{Virtio
Transport Options / Virtio Over PCI Bus / PCI Device Layout /
Vendor data capability}

Devices CAN present \field{vendor_id} that does not match
either the PCI Vendor ID or the PCI Subsystem Vendor ID.

Devices CAN present multiple Vendor data capabilities with
either different or identical \field{vendor_id} values.

The value \field{vendor_id} MUST NOT equal 0x1AF4.

The size of the Vendor data capability MUST be a multiple of 4 bytes.

Reads of the Vendor data capability by the driver MUST NOT have any
side effects.

\drivernormative{\paragraph}{Vendor data capability}{Virtio
Transport Options / Virtio Over PCI Bus / PCI Device Layout /
Vendor data capability}

The driver SHOULD NOT use the Vendor data capability except
for debugging and reporting purposes.

The driver MUST qualify the \field{vendor_id} before
interpreting or writing into the Vendor data capability.

\subsubsection{PCI configuration access capability}\label{sec:Virtio Transport Options / Virtio Over PCI Bus / PCI Device Layout / PCI configuration access capability}

The VIRTIO_PCI_CAP_PCI_CFG capability
creates an alternative (and likely suboptimal) access method to the
common configuration, notification, ISR and device-specific configuration regions.

The capability is immediately followed by an additional field like so:

\begin{lstlisting}
struct virtio_pci_cfg_cap {
        struct virtio_pci_cap cap;
        u8 pci_cfg_data[4]; /* Data for BAR access. */
};
\end{lstlisting}

The fields \field{cap.bar}, \field{cap.length}, \field{cap.offset} and
\field{pci_cfg_data} are read-write (RW) for the driver.

To access a device region, the driver writes into the capability
structure (ie. within the PCI configuration space) as follows:

\begin{itemize}
\item The driver sets the BAR to access by writing to \field{cap.bar}.

\item The driver sets the size of the access by writing 1, 2 or 4 to
  \field{cap.length}.

\item The driver sets the offset within the BAR by writing to
  \field{cap.offset}.
\end{itemize}

At that point, \field{pci_cfg_data} will provide a window of size
\field{cap.length} into the given \field{cap.bar} at offset \field{cap.offset}.

\devicenormative{\paragraph}{PCI configuration access capability}{Virtio Transport Options / Virtio Over PCI Bus / PCI Device Layout / PCI configuration access capability}

The device MUST present at least one VIRTIO_PCI_CAP_PCI_CFG capability.

Upon detecting driver write access
to \field{pci_cfg_data}, the device MUST execute a write access
at offset \field{cap.offset} at BAR selected by \field{cap.bar} using the first \field{cap.length}
bytes from \field{pci_cfg_data}.

Upon detecting driver read access
to \field{pci_cfg_data}, the device MUST
execute a read access of length cap.length at offset \field{cap.offset}
at BAR selected by \field{cap.bar} and store the first \field{cap.length} bytes in
\field{pci_cfg_data}.

\drivernormative{\paragraph}{PCI configuration access capability}{Virtio Transport Options / Virtio Over PCI Bus / PCI Device Layout / PCI configuration access capability}

The driver MUST NOT write a \field{cap.offset} which is not
a multiple of \field{cap.length} (ie. all accesses MUST be aligned).

The driver MUST NOT read or write \field{pci_cfg_data}
unless \field{cap.bar}, \field{cap.length} and \field{cap.offset}
address \field{cap.length} bytes within a BAR range
specified by some other Virtio Structure PCI Capability
of type other than \field{VIRTIO_PCI_CAP_PCI_CFG}.

\subsubsection{Legacy Interfaces: A Note on PCI Device Layout}\label{sec:Virtio Transport Options / Virtio Over PCI Bus / PCI Device Layout / Legacy Interfaces: A Note on PCI Device Layout}

Transitional devices MUST present part of configuration
registers in a legacy configuration structure in BAR0 in the first I/O
region of the PCI device, as documented below.
When using the legacy interface, transitional drivers
MUST use the legacy configuration structure in BAR0 in the first
I/O region of the PCI device, as documented below.

When using the legacy interface the driver MAY access
the device-specific configuration region using any width accesses, and
a transitional device MUST present driver with the same results as
when accessed using the ``natural'' access method (i.e.
32-bit accesses for 32-bit fields, etc).

Note that this is possible because while the virtio common configuration structure is PCI
(i.e. little) endian, when using the legacy interface the device-specific
configuration region is encoded in the native endian of the guest (where such distinction is
applicable).

When used through the legacy interface, the virtio common configuration structure looks as follows:

\begin{tabularx}{\textwidth}{ |X||X|X|X|X|X|X|X|X| }
\hline
 Bits & 32 & 32 & 32 & 16 & 16 & 16 & 8 & 8 \\
\hline
 Read / Write & R & R+W & R+W & R & R+W & R+W & R+W & R \\
\hline
 Purpose & Device Features bits 0:31 & Driver Features bits 0:31 &
  Queue Address & \field{queue_size} & \field{queue_select} & Queue Notify &
  Device Status & ISR \newline Status \\
\hline
\end{tabularx}

If MSI-X is enabled for the device, two additional fields
immediately follow this header:

\begin{tabular}{ |l||l|l| }
\hline
Bits       & 16             & 16     \\
\hline
Read/Write & R+W            & R+W    \\
\hline
Purpose (MSI-X) & \field{config_msix_vector}  & \field{queue_msix_vector} \\
\hline
\end{tabular}

Note: When MSI-X capability is enabled, device-specific configuration starts at
byte offset 24 in virtio common configuration structure. When MSI-X capability is not
enabled, device-specific configuration starts at byte offset 20 in virtio
header.  ie. once you enable MSI-X on the device, the other fields move.
If you turn it off again, they move back!

Any device-specific configuration space immediately follows
these general headers:

\begin{tabular}{|l||l|l|}
\hline
Bits & Device Specific & \multirow{3}{*}{\ldots} \\
\cline{1-2}
Read / Write & Device Specific & \\
\cline{1-2}
Purpose & Device Specific & \\
\hline
\end{tabular}

When accessing the device-specific configuration space
using the legacy interface, transitional
drivers MUST access the device-specific configuration space
at an offset immediately following the general headers.

When using the legacy interface, transitional
devices MUST present the device-specific configuration space
if any at an offset immediately following the general headers.

Note that only Feature Bits 0 to 31 are accessible through the
Legacy Interface. When used through the Legacy Interface,
Transitional Devices MUST assume that Feature Bits 32 to 63
are not acknowledged by Driver.

As legacy devices had no \field{config_generation} field,
see \ref{sec:Basic Facilities of a Virtio Device / Device
Configuration Space / Legacy Interface: Device Configuration
Space}~\nameref{sec:Basic Facilities of a Virtio Device / Device Configuration Space / Legacy Interface: Device Configuration Space} for workarounds.

\subsubsection{Non-transitional Device With Legacy Driver: A Note
on PCI Device Layout}\label{sec:Virtio Transport Options / Virtio
Over PCI Bus / PCI Device Layout / Non-transitional Device With
Legacy Driver: A Note on PCI Device Layout}

All known legacy drivers check either the PCI Revision or the
Device and Vendor IDs, and thus won't attempt to drive a
non-transitional device.

A buggy legacy driver might mistakenly attempt to drive a
non-transitional device. If support for such drivers is required
(as opposed to fixing the bug), the following would be the
recommended way to detect and handle them.
\begin{note}
Such buggy drivers are not currently known to be used in
production.
\end{note}

\subparagraph{Device Requirements: Non-transitional Device With Legacy Driver}
\label{drivernormative:Virtio Transport Options / Virtio Over PCI
Bus / PCI-specific Initialization And Device Operation /
Device Initialization / Non-transitional Device With Legacy
Driver}
\label{devicenormative:Virtio Transport Options / Virtio Over PCI
Bus / PCI-specific Initialization And Device Operation /
Device Initialization / Non-transitional Device With Legacy
Driver}

Non-transitional devices, on a platform where a legacy driver for
a legacy device with the same ID (including PCI Revision, Device
and Vendor IDs) is known to have previously existed,
SHOULD take the following steps to cause the legacy driver to
fail gracefully when it attempts to drive them:

\begin{enumerate}
\item Present an I/O BAR in BAR0, and
\item Respond to a single-byte zero write to offset 18
   (corresponding to Device Status register in the legacy layout)
   of BAR0 by presenting zeroes on every BAR and ignoring writes.
\end{enumerate}

\subsection{PCI-specific Initialization And Device Operation}\label{sec:Virtio Transport Options / Virtio Over PCI Bus / PCI-specific Initialization And Device Operation}

\subsubsection{Device Initialization}\label{sec:Virtio Transport Options / Virtio Over PCI Bus / PCI-specific Initialization And Device Operation / Device Initialization}

This documents PCI-specific steps executed during Device Initialization.

\paragraph{Virtio Device Configuration Layout Detection}\label{sec:Virtio Transport Options / Virtio Over PCI Bus / PCI-specific Initialization And Device Operation / Device Initialization / Virtio Device Configuration Layout Detection}

As a prerequisite to device initialization, the driver scans the
PCI capability list, detecting virtio configuration layout using Virtio
Structure PCI capabilities as detailed in \ref{sec:Virtio Transport Options / Virtio Over PCI Bus / Virtio Structure PCI Capabilities}

\subparagraph{Legacy Interface: A Note on Device Layout Detection}\label{sec:Virtio Transport Options / Virtio Over PCI Bus / PCI-specific Initialization And Device Operation / Device Initialization / Virtio Device Configuration Layout Detection / Legacy Interface: A Note on Device Layout Detection}

Legacy drivers skipped the Device Layout Detection step, assuming legacy
device configuration space in BAR0 in I/O space unconditionally.

Legacy devices did not have the Virtio PCI Capability in their
capability list.

Therefore:

Transitional devices MUST expose the Legacy Interface in I/O
space in BAR0.

Transitional drivers MUST look for the Virtio PCI
Capabilities on the capability list.
If these are not present, driver MUST assume a legacy device,
and use it through the legacy interface.

Non-transitional drivers MUST look for the Virtio PCI
Capabilities on the capability list.
If these are not present, driver MUST assume a legacy device,
and fail gracefully.

\paragraph{MSI-X Vector Configuration}\label{sec:Virtio Transport Options / Virtio Over PCI Bus / PCI-specific Initialization And Device Operation / Device Initialization / MSI-X Vector Configuration}

When MSI-X capability is present and enabled in the device
(through standard PCI configuration space) \field{config_msix_vector} and \field{queue_msix_vector} are used to map configuration change and queue
interrupts to MSI-X vectors. In this case, the ISR Status is unused.

Writing a valid MSI-X Table entry number, 0 to 0x7FF, to
\field{config_msix_vector}/\field{queue_msix_vector} maps interrupts triggered
by the configuration change/selected queue events respectively to
the corresponding MSI-X vector. To disable interrupts for an
event type, the driver unmaps this event by writing a special NO_VECTOR
value:

\begin{lstlisting}
/* Vector value used to disable MSI for queue */
#define VIRTIO_MSI_NO_VECTOR            0xffff
\end{lstlisting}

Note that mapping an event to vector might require device to
allocate internal device resources, and thus could fail.

\devicenormative{\subparagraph}{MSI-X Vector Configuration}{Virtio Transport Options / Virtio Over PCI Bus / PCI-specific Initialization And Device Operation / Device Initialization / MSI-X Vector Configuration}

A device that has an MSI-X capability SHOULD support at least 2
and at most 0x800 MSI-X vectors.
Device MUST report the number of vectors supported in
\field{Table Size} in the MSI-X Capability as specified in
\hyperref[intro:PCI]{[PCI]}.
The device SHOULD restrict the reported MSI-X Table Size field
to a value that might benefit system performance.
\begin{note}
For example, a device which does not expect to send
interrupts at a high rate might only specify 2 MSI-X vectors.
\end{note}
Device MUST support mapping any event type to any valid
vector 0 to MSI-X \field{Table Size}.
Device MUST support unmapping any event type.

The device MUST return vector mapped to a given event,
(NO_VECTOR if unmapped) on read of \field{config_msix_vector}/\field{queue_msix_vector}.
The device MUST have all queue and configuration change
events are unmapped upon reset.

Devices SHOULD NOT cause mapping an event to vector to fail
unless it is impossible for the device to satisfy the mapping
request.  Devices MUST report mapping
failures by returning the NO_VECTOR value when the relevant
\field{config_msix_vector}/\field{queue_msix_vector} field is read.

\drivernormative{\subparagraph}{MSI-X Vector Configuration}{Virtio Transport Options / Virtio Over PCI Bus / PCI-specific Initialization And Device Operation / Device Initialization / MSI-X Vector Configuration}

Driver MUST support device with any MSI-X Table Size 0 to 0x7FF.
Driver MAY fall back on using INT\#x interrupts for a device
which only supports one MSI-X vector (MSI-X Table Size = 0).

Driver MAY interpret the Table Size as a hint from the device
for the suggested number of MSI-X vectors to use.

Driver MUST NOT attempt to map an event to a vector
outside the MSI-X Table supported by the device,
as reported by \field{Table Size} in the MSI-X Capability.

After mapping an event to vector, the
driver MUST verify success by reading the Vector field value: on
success, the previously written value is returned, and on
failure, NO_VECTOR is returned. If a mapping failure is detected,
the driver MAY retry mapping with fewer vectors, disable MSI-X
or report device failure.

\paragraph{Virtqueue Configuration}\label{sec:Virtio Transport Options / Virtio Over PCI Bus / PCI-specific Initialization And Device Operation / Device Initialization / Virtqueue Configuration}

As a device can have zero or more virtqueues for bulk data
transport\footnote{For example, the simplest network device has two virtqueues.}, the driver
needs to configure them as part of the device-specific
configuration.

The driver typically does this as follows, for each virtqueue a device has:

\begin{enumerate}
\item Write the virtqueue index to \field{queue_select}.

\item Read the virtqueue size from \field{queue_size}. This controls how big the virtqueue is
  (see \ref{sec:Basic Facilities of a Virtio Device / Virtqueues}~\nameref{sec:Basic Facilities of a Virtio Device / Virtqueues}). If this field is 0, the virtqueue does not exist.

\item Optionally, select a smaller virtqueue size and write it to \field{queue_size}.

\item Allocate and zero Descriptor Table, Available and Used rings for the
   virtqueue in contiguous physical memory.

\item Optionally, if MSI-X capability is present and enabled on the
  device, select a vector to use to request interrupts triggered
  by virtqueue events. Write the MSI-X Table entry number
  corresponding to this vector into \field{queue_msix_vector}. Read
  \field{queue_msix_vector}: on success, previously written value is
  returned; on failure, NO_VECTOR value is returned.
\end{enumerate}

\subparagraph{Legacy Interface: A Note on Virtqueue Configuration}\label{sec:Virtio Transport Options / Virtio Over PCI Bus / PCI-specific Initialization And Device Operation / Device Initialization / Virtqueue Configuration / Legacy Interface: A Note on Virtqueue Configuration}
When using the legacy interface, the queue layout follows \ref{sec:Basic Facilities of a Virtio Device / Virtqueues / Legacy Interfaces: A Note on Virtqueue Layout}~\nameref{sec:Basic Facilities of a Virtio Device / Virtqueues / Legacy Interfaces: A Note on Virtqueue Layout} with an alignment of 4096.
Driver writes the physical address, divided
by 4096 to the Queue Address field\footnote{The 4096 is based on the x86 page size, but it's also large
enough to ensure that the separate parts of the virtqueue are on
separate cache lines.
}.  There was no mechanism to negotiate the queue size.

\subsubsection{Available Buffer Notifications}\label{sec:Virtio Transport Options / Virtio Over PCI Bus / PCI-specific Initialization And Device Operation / Available Buffer Notifications}

When VIRTIO_F_NOTIFICATION_DATA has not been negotiated,
the driver sends an available buffer notification to the device by writing
only the 16-bit notification value to the Queue Notify address of the
virtqueue. A notification value depends on the negotiation of
VIRTIO_F_NOTIF_CONFIG_DATA.

If VIRTIO_F_NOTIFICATION_DATA has been negotiated, the driver sends an
available buffer notification to the device by writing the following 32-bit
value to the Queue Notify address:
\lstinputlisting{notifications-data-le.c}

\begin{itemize}
\item When VIRTIO_F_NOTIF_CONFIG_DATA is not negotiated \field{vq_index} is set
to the virtqueue index.

\item When VIRTIO_F_NOTIFICATION_DATA is negotiated,
\field{vq_notif_config_data} is set to \field{queue_notif_config_data}.
\end{itemize}

See \ref{sec:Basic Facilities of a Virtio Device / Driver notifications}~\nameref{sec:Basic Facilities of a Virtio Device / Driver notifications}
for the definition of the components.

See \ref{sec:Virtio Transport Options / Virtio Over PCI Bus / PCI Device Layout / Notification capability}
for how to calculate the Queue Notify address.

\drivernormative{\paragraph}{Available Buffer Notifications}{Virtio Transport Options / Virtio Over PCI Bus / PCI-specific Initialization And Device Operation / Available Buffer Notifications}

If VIRTIO_F_NOTIFICATION_DATA is not negotiated, the driver notification
MUST be a 16-bit notification.

If VIRTIO_F_NOTIFICATION_DATA is negotiated, the driver notification
MUST be a 32-bit notification.

If VIRTIO_F_NOTIF_CONFIG_DATA is not negotiated:
\begin{itemize}
\item If VIRTIO_F_NOTIFICATION_DATA is not negotiated, the driver MUST set the
notification value to the virtqueue index.

\item If VIRTIO_F_NOTIFICATION_DATA is negotiated, the driver MUST set the
\field{vq_index} to the virtqueue index.

\end{itemize}

If VIRTIO_F_NOTIF_CONFIG_DATA is negotiated:
\begin{itemize}
\item If VIRTIO_F_NOTIFICATION_DATA is not negotiated, the driver MUST set
the notification value to \field{queue_notif_config_data}.

\item If VIRTIO_F_NOTIFICATION_DATA is negotiated, the driver MUST set the
\field{vq_notify_config_data} to the \field{queue_notif_config_data} value.
\end{itemize}

\subsubsection{Used Buffer Notifications}\label{sec:Virtio Transport Options / Virtio Over PCI Bus / PCI-specific Initialization And Device Operation / Used Buffer Notifications}

If a used buffer notification is necessary for a virtqueue, the device would typically act as follows:

\begin{itemize}
  \item If MSI-X capability is disabled:
    \begin{enumerate}
    \item Set the lower bit of the ISR Status field for the device.

    \item Send the appropriate PCI interrupt for the device.
    \end{enumerate}

  \item If MSI-X capability is enabled:
    \begin{enumerate}
    \item If \field{queue_msix_vector} is not NO_VECTOR,
      request the appropriate MSI-X interrupt message for the
      device, \field{queue_msix_vector} sets the MSI-X Table entry
      number.
    \end{enumerate}
\end{itemize}

\devicenormative{\paragraph}{Used Buffer Notifications}{Virtio Transport Options / Virtio Over PCI Bus / PCI-specific Initialization And Device Operation / Used Buffer Notifications}

If MSI-X capability is enabled and \field{queue_msix_vector} is
NO_VECTOR for a virtqueue, the device MUST NOT deliver an interrupt
for that virtqueue.

\subsubsection{Notification of Device Configuration Changes}\label{sec:Virtio Transport Options / Virtio Over PCI Bus / PCI-specific Initialization And Device Operation / Notification of Device Configuration Changes}

Some virtio PCI devices can change the device configuration
state, as reflected in the device-specific configuration region of the device. In this case:

\begin{itemize}
  \item If MSI-X capability is disabled:
    \begin{enumerate}
    \item Set the second lower bit of the ISR Status field for the device.

    \item Send the appropriate PCI interrupt for the device.
    \end{enumerate}

  \item If MSI-X capability is enabled:
    \begin{enumerate}
    \item If \field{config_msix_vector} is not NO_VECTOR,
      request the appropriate MSI-X interrupt message for the
      device, \field{config_msix_vector} sets the MSI-X Table entry
      number.
    \end{enumerate}
\end{itemize}

A single interrupt MAY indicate both that one or more virtqueue has
been used and that the configuration space has changed.

\devicenormative{\paragraph}{Notification of Device Configuration Changes}{Virtio Transport Options / Virtio Over PCI Bus / PCI-specific Initialization And Device Operation / Notification of Device Configuration Changes}

If MSI-X capability is enabled and \field{config_msix_vector} is
NO_VECTOR, the device MUST NOT deliver an interrupt
for device configuration space changes.

\drivernormative{\paragraph}{Notification of Device Configuration Changes}{Virtio Transport Options / Virtio Over PCI Bus / PCI-specific Initialization And Device Operation / Notification of Device Configuration Changes}

A driver MUST handle the case where the same interrupt is used to indicate
both device configuration space change and one or more virtqueues being used.

\subsubsection{Driver Handling Interrupts}\label{sec:Virtio Transport Options / Virtio Over PCI Bus / PCI-specific Initialization And Device Operation / Driver Handling Interrupts}
The driver interrupt handler would typically:

\begin{itemize}
  \item If MSI-X capability is disabled:
    \begin{itemize}
      \item Read the ISR Status field, which will reset it to zero.
      \item If the lower bit is set:
        look through all virtqueues for the
        device, to see if any progress has been made by the device
        which requires servicing.
      \item If the second lower bit is set:
        re-examine the configuration space to see what changed.
    \end{itemize}
  \item If MSI-X capability is enabled:
    \begin{itemize}
      \item
        Look through all virtqueues mapped to that MSI-X vector for the
        device, to see if any progress has been made by the device
        which requires servicing.
      \item
        If the MSI-X vector is equal to \field{config_msix_vector},
        re-examine the configuration space to see what changed.
    \end{itemize}
\end{itemize}

\input{transport-mmio.tex}
\input{transport-ccw.tex}

\chapter{Device Types}\label{sec:Device Types}

On top of the queues, config space and feature negotiation facilities
built into virtio, several devices are defined.

The following device IDs are used to identify different types of virtio
devices.  Some device IDs are reserved for devices which are not currently
defined in this standard.

Discovering what devices are available and their type is bus-dependent.

\begin{tabular} { |l|c| }
\hline
Device ID  &  Virtio Device    \\
\hline \hline
0          & reserved (invalid) \\
\hline
1          &   network device     \\
\hline
2          &   block device     \\
\hline
3          &      console       \\
\hline
4          &  entropy source    \\
\hline
5          & memory ballooning (traditional)  \\
\hline
6          &     ioMemory       \\
\hline
7          &       rpmsg        \\
\hline
8          &     SCSI host      \\
\hline
9          &   9P transport     \\
\hline
10         &   mac80211 wlan    \\
\hline
11         &   rproc serial     \\
\hline
12         &   virtio CAIF      \\
\hline
13         &  memory balloon    \\
\hline
16         &   GPU device       \\
\hline
17         &   Timer/Clock device \\
\hline
18         &   Input device \\
\hline
19         &   Socket device \\
\hline
20         &   Crypto device \\
\hline
21         &   Signal Distribution Module \\
\hline
22         &   pstore device \\
\hline
23         &   IOMMU device \\
\hline
24         &   Memory device \\
\hline
25         &   Sound device \\
\hline
26         &   file system device \\
\hline
27         &   PMEM device \\
\hline
28         &   RPMB device \\
\hline
29         &   mac80211 hwsim wireless simulation device \\
\hline
30         &   Video encoder device \\
\hline
31         &   Video decoder device \\
\hline
32         &   SCMI device \\
\hline
33         &   NitroSecureModule \\
\hline
34         &   I2C adapter \\
\hline
35         &   Watchdog \\
\hline
36         &   CAN device \\
\hline
38         &   Parameter Server \\
\hline
39         &   Audio policy device \\
\hline
40         &   Bluetooth device \\
\hline
41         &   GPIO device \\
\hline
42         &   RDMA device \\
\hline
43         &   Camera device \\
\hline
44         &   ISM device \\
\hline
45         &   SPI master \\
\hline
\end{tabular}

Some of the devices above are unspecified by this document,
because they are seen as immature or especially niche.  Be warned
that some are only specified by the sole existing implementation;
they could become part of a future specification, be abandoned
entirely, or live on outside this standard.  We shall speak of
them no further.

\section{Network Device}\label{sec:Device Types / Network Device}

The virtio network device is a virtual network interface controller.
It consists of a virtual Ethernet link which connects the device
to the Ethernet network. The device has transmit and receive
queues. The driver adds empty buffers to the receive virtqueue.
The device receives incoming packets from the link; the device
places these incoming packets in the receive virtqueue buffers.
The driver adds outgoing packets to the transmit virtqueue. The device
removes these packets from the transmit virtqueue and sends them to
the link. The device may have a control virtqueue. The driver
uses the control virtqueue to dynamically manipulate various
features of the initialized device.

\subsection{Device ID}\label{sec:Device Types / Network Device / Device ID}

 1

\subsection{Virtqueues}\label{sec:Device Types / Network Device / Virtqueues}

\begin{description}
\item[0] receiveq1
\item[1] transmitq1
\item[\ldots]
\item[2(N-1)] receiveqN
\item[2(N-1)+1] transmitqN
\item[2N] controlq
\end{description}

 N=1 if neither VIRTIO_NET_F_MQ nor VIRTIO_NET_F_RSS are negotiated, otherwise N is set by
 \field{max_virtqueue_pairs}.

controlq is optional; it only exists if VIRTIO_NET_F_CTRL_VQ is
negotiated.

\subsection{Feature bits}\label{sec:Device Types / Network Device / Feature bits}

\begin{description}
\item[VIRTIO_NET_F_CSUM (0)] Device handles packets with partial checksum offload.

\item[VIRTIO_NET_F_GUEST_CSUM (1)] Driver handles packets with partial checksum.

\item[VIRTIO_NET_F_CTRL_GUEST_OFFLOADS (2)] Control channel offloads
        reconfiguration support.

\item[VIRTIO_NET_F_MTU(3)] Device maximum MTU reporting is supported. If
    offered by the device, device advises driver about the value of
    its maximum MTU. If negotiated, the driver uses \field{mtu} as
    the maximum MTU value.

\item[VIRTIO_NET_F_MAC (5)] Device has given MAC address.

\item[VIRTIO_NET_F_GUEST_TSO4 (7)] Driver can receive TSOv4.

\item[VIRTIO_NET_F_GUEST_TSO6 (8)] Driver can receive TSOv6.

\item[VIRTIO_NET_F_GUEST_ECN (9)] Driver can receive TSO with ECN.

\item[VIRTIO_NET_F_GUEST_UFO (10)] Driver can receive UFO.

\item[VIRTIO_NET_F_HOST_TSO4 (11)] Device can receive TSOv4.

\item[VIRTIO_NET_F_HOST_TSO6 (12)] Device can receive TSOv6.

\item[VIRTIO_NET_F_HOST_ECN (13)] Device can receive TSO with ECN.

\item[VIRTIO_NET_F_HOST_UFO (14)] Device can receive UFO.

\item[VIRTIO_NET_F_MRG_RXBUF (15)] Driver can merge receive buffers.

\item[VIRTIO_NET_F_STATUS (16)] Configuration status field is
    available.

\item[VIRTIO_NET_F_CTRL_VQ (17)] Control channel is available.

\item[VIRTIO_NET_F_CTRL_RX (18)] Control channel RX mode support.

\item[VIRTIO_NET_F_CTRL_VLAN (19)] Control channel VLAN filtering.

\item[VIRTIO_NET_F_CTRL_RX_EXTRA (20)]	Control channel RX extra mode support.

\item[VIRTIO_NET_F_GUEST_ANNOUNCE(21)] Driver can send gratuitous
    packets.

\item[VIRTIO_NET_F_MQ(22)] Device supports multiqueue with automatic
    receive steering.

\item[VIRTIO_NET_F_CTRL_MAC_ADDR(23)] Set MAC address through control
    channel.

\item[VIRTIO_NET_F_HASH_TUNNEL(51)] Device supports inner header hash for encapsulated packets.

\item[VIRTIO_NET_F_VQ_NOTF_COAL(52)] Device supports virtqueue notification coalescing.

\item[VIRTIO_NET_F_NOTF_COAL(53)] Device supports notifications coalescing.

\item[VIRTIO_NET_F_GUEST_USO4 (54)] Driver can receive USOv4 packets.

\item[VIRTIO_NET_F_GUEST_USO6 (55)] Driver can receive USOv6 packets.

\item[VIRTIO_NET_F_HOST_USO (56)] Device can receive USO packets. Unlike UFO
 (fragmenting the packet) the USO splits large UDP packet
 to several segments when each of these smaller packets has UDP header.

\item[VIRTIO_NET_F_HASH_REPORT(57)] Device can report per-packet hash
    value and a type of calculated hash.

\item[VIRTIO_NET_F_GUEST_HDRLEN(59)] Driver can provide the exact \field{hdr_len}
    value. Device benefits from knowing the exact header length.

\item[VIRTIO_NET_F_RSS(60)] Device supports RSS (receive-side scaling)
    with Toeplitz hash calculation and configurable hash
    parameters for receive steering.

\item[VIRTIO_NET_F_RSC_EXT(61)] Device can process duplicated ACKs
    and report number of coalesced segments and duplicated ACKs.

\item[VIRTIO_NET_F_STANDBY(62)] Device may act as a standby for a primary
    device with the same MAC address.

\item[VIRTIO_NET_F_SPEED_DUPLEX(63)] Device reports speed and duplex.
\end{description}

\subsubsection{Feature bit requirements}\label{sec:Device Types / Network Device / Feature bits / Feature bit requirements}

Some networking feature bits require other networking feature bits
(see \ref{drivernormative:Basic Facilities of a Virtio Device / Feature Bits}):

\begin{description}
\item[VIRTIO_NET_F_GUEST_TSO4] Requires VIRTIO_NET_F_GUEST_CSUM.
\item[VIRTIO_NET_F_GUEST_TSO6] Requires VIRTIO_NET_F_GUEST_CSUM.
\item[VIRTIO_NET_F_GUEST_ECN] Requires VIRTIO_NET_F_GUEST_TSO4 or VIRTIO_NET_F_GUEST_TSO6.
\item[VIRTIO_NET_F_GUEST_UFO] Requires VIRTIO_NET_F_GUEST_CSUM.
\item[VIRTIO_NET_F_GUEST_USO4] Requires VIRTIO_NET_F_GUEST_CSUM.
\item[VIRTIO_NET_F_GUEST_USO6] Requires VIRTIO_NET_F_GUEST_CSUM.

\item[VIRTIO_NET_F_HOST_TSO4] Requires VIRTIO_NET_F_CSUM.
\item[VIRTIO_NET_F_HOST_TSO6] Requires VIRTIO_NET_F_CSUM.
\item[VIRTIO_NET_F_HOST_ECN] Requires VIRTIO_NET_F_HOST_TSO4 or VIRTIO_NET_F_HOST_TSO6.
\item[VIRTIO_NET_F_HOST_UFO] Requires VIRTIO_NET_F_CSUM.
\item[VIRTIO_NET_F_HOST_USO] Requires VIRTIO_NET_F_CSUM.

\item[VIRTIO_NET_F_CTRL_RX] Requires VIRTIO_NET_F_CTRL_VQ.
\item[VIRTIO_NET_F_CTRL_VLAN] Requires VIRTIO_NET_F_CTRL_VQ.
\item[VIRTIO_NET_F_GUEST_ANNOUNCE] Requires VIRTIO_NET_F_CTRL_VQ.
\item[VIRTIO_NET_F_MQ] Requires VIRTIO_NET_F_CTRL_VQ.
\item[VIRTIO_NET_F_CTRL_MAC_ADDR] Requires VIRTIO_NET_F_CTRL_VQ.
\item[VIRTIO_NET_F_NOTF_COAL] Requires VIRTIO_NET_F_CTRL_VQ.
\item[VIRTIO_NET_F_RSC_EXT] Requires VIRTIO_NET_F_HOST_TSO4 or VIRTIO_NET_F_HOST_TSO6.
\item[VIRTIO_NET_F_RSS] Requires VIRTIO_NET_F_CTRL_VQ.
\item[VIRTIO_NET_F_VQ_NOTF_COAL] Requires VIRTIO_NET_F_CTRL_VQ.
\item[VIRTIO_NET_F_HASH_TUNNEL] Requires VIRTIO_NET_F_CTRL_VQ along with VIRTIO_NET_F_RSS or VIRTIO_NET_F_HASH_REPORT.
\end{description}

\subsubsection{Legacy Interface: Feature bits}\label{sec:Device Types / Network Device / Feature bits / Legacy Interface: Feature bits}
\begin{description}
\item[VIRTIO_NET_F_GSO (6)] Device handles packets with any GSO type. This was supposed to indicate segmentation offload support, but
upon further investigation it became clear that multiple bits were needed.
\item[VIRTIO_NET_F_GUEST_RSC4 (41)] Device coalesces TCPIP v4 packets. This was implemented by hypervisor patch for certification
purposes and current Windows driver depends on it. It will not function if virtio-net device reports this feature.
\item[VIRTIO_NET_F_GUEST_RSC6 (42)] Device coalesces TCPIP v6 packets. Similar to VIRTIO_NET_F_GUEST_RSC4.
\end{description}

\subsection{Device configuration layout}\label{sec:Device Types / Network Device / Device configuration layout}
\label{sec:Device Types / Block Device / Feature bits / Device configuration layout}

The network device has the following device configuration layout.
All of the device configuration fields are read-only for the driver.

\begin{lstlisting}
struct virtio_net_config {
        u8 mac[6];
        le16 status;
        le16 max_virtqueue_pairs;
        le16 mtu;
        le32 speed;
        u8 duplex;
        u8 rss_max_key_size;
        le16 rss_max_indirection_table_length;
        le32 supported_hash_types;
        le32 supported_tunnel_types;
};
\end{lstlisting}

The \field{mac} address field always exists (although it is only
valid if VIRTIO_NET_F_MAC is set).

The \field{status} only exists if VIRTIO_NET_F_STATUS is set.
Two bits are currently defined for the status field: VIRTIO_NET_S_LINK_UP
and VIRTIO_NET_S_ANNOUNCE.

\begin{lstlisting}
#define VIRTIO_NET_S_LINK_UP     1
#define VIRTIO_NET_S_ANNOUNCE    2
\end{lstlisting}

The following field, \field{max_virtqueue_pairs} only exists if
VIRTIO_NET_F_MQ or VIRTIO_NET_F_RSS is set. This field specifies the maximum number
of each of transmit and receive virtqueues (receiveq1\ldots receiveqN
and transmitq1\ldots transmitqN respectively) that can be configured once at least one of these features
is negotiated.

The following field, \field{mtu} only exists if VIRTIO_NET_F_MTU
is set. This field specifies the maximum MTU for the driver to
use.

The following two fields, \field{speed} and \field{duplex}, only
exist if VIRTIO_NET_F_SPEED_DUPLEX is set.

\field{speed} contains the device speed, in units of 1 MBit per
second, 0 to 0x7fffffff, or 0xffffffff for unknown speed.

\field{duplex} has the values of 0x01 for full duplex, 0x00 for
half duplex and 0xff for unknown duplex state.

Both \field{speed} and \field{duplex} can change, thus the driver
is expected to re-read these values after receiving a
configuration change notification.

The following field, \field{rss_max_key_size} only exists if VIRTIO_NET_F_RSS or VIRTIO_NET_F_HASH_REPORT is set.
It specifies the maximum supported length of RSS key in bytes.

The following field, \field{rss_max_indirection_table_length} only exists if VIRTIO_NET_F_RSS is set.
It specifies the maximum number of 16-bit entries in RSS indirection table.

The next field, \field{supported_hash_types} only exists if the device supports hash calculation,
i.e. if VIRTIO_NET_F_RSS or VIRTIO_NET_F_HASH_REPORT is set.

Field \field{supported_hash_types} contains the bitmask of supported hash types.
See \ref{sec:Device Types / Network Device / Device Operation / Processing of Incoming Packets / Hash calculation for incoming packets / Supported/enabled hash types} for details of supported hash types.

Field \field{supported_tunnel_types} only exists if the device supports inner header hash, i.e. if VIRTIO_NET_F_HASH_TUNNEL is set.

Field \field{supported_tunnel_types} contains the bitmask of encapsulation types supported by the device for inner header hash.
Encapsulation types are defined in \ref{sec:Device Types / Network Device / Device Operation / Processing of Incoming Packets /
Hash calculation for incoming packets / Encapsulation types supported/enabled for inner header hash}.

\devicenormative{\subsubsection}{Device configuration layout}{Device Types / Network Device / Device configuration layout}

The device MUST set \field{max_virtqueue_pairs} to between 1 and 0x8000 inclusive,
if it offers VIRTIO_NET_F_MQ.

The device MUST set \field{mtu} to between 68 and 65535 inclusive,
if it offers VIRTIO_NET_F_MTU.

The device SHOULD set \field{mtu} to at least 1280, if it offers
VIRTIO_NET_F_MTU.

The device MUST NOT modify \field{mtu} once it has been set.

The device MUST NOT pass received packets that exceed \field{mtu} (plus low
level ethernet header length) size with \field{gso_type} NONE or ECN
after VIRTIO_NET_F_MTU has been successfully negotiated.

The device MUST forward transmitted packets of up to \field{mtu} (plus low
level ethernet header length) size with \field{gso_type} NONE or ECN, and do
so without fragmentation, after VIRTIO_NET_F_MTU has been successfully
negotiated.

The device MUST set \field{rss_max_key_size} to at least 40, if it offers
VIRTIO_NET_F_RSS or VIRTIO_NET_F_HASH_REPORT.

The device MUST set \field{rss_max_indirection_table_length} to at least 128, if it offers
VIRTIO_NET_F_RSS.

If the driver negotiates the VIRTIO_NET_F_STANDBY feature, the device MAY act
as a standby device for a primary device with the same MAC address.

If VIRTIO_NET_F_SPEED_DUPLEX has been negotiated, \field{speed}
MUST contain the device speed, in units of 1 MBit per second, 0 to
0x7ffffffff, or 0xfffffffff for unknown.

If VIRTIO_NET_F_SPEED_DUPLEX has been negotiated, \field{duplex}
MUST have the values of 0x00 for full duplex, 0x01 for half
duplex, or 0xff for unknown.

If VIRTIO_NET_F_SPEED_DUPLEX and VIRTIO_NET_F_STATUS have both
been negotiated, the device SHOULD NOT change the \field{speed} and
\field{duplex} fields as long as VIRTIO_NET_S_LINK_UP is set in
the \field{status}.

The device SHOULD NOT offer VIRTIO_NET_F_HASH_REPORT if it
does not offer VIRTIO_NET_F_CTRL_VQ.

The device SHOULD NOT offer VIRTIO_NET_F_CTRL_RX_EXTRA if it
does not offer VIRTIO_NET_F_CTRL_VQ.

\drivernormative{\subsubsection}{Device configuration layout}{Device Types / Network Device / Device configuration layout}

The driver MUST NOT write to any of the device configuration fields.

A driver SHOULD negotiate VIRTIO_NET_F_MAC if the device offers it.
If the driver negotiates the VIRTIO_NET_F_MAC feature, the driver MUST set
the physical address of the NIC to \field{mac}.  Otherwise, it SHOULD
use a locally-administered MAC address (see \hyperref[intro:IEEE 802]{IEEE 802},
``9.2 48-bit universal LAN MAC addresses'').

If the driver does not negotiate the VIRTIO_NET_F_STATUS feature, it SHOULD
assume the link is active, otherwise it SHOULD read the link status from
the bottom bit of \field{status}.

A driver SHOULD negotiate VIRTIO_NET_F_MTU if the device offers it.

If the driver negotiates VIRTIO_NET_F_MTU, it MUST supply enough receive
buffers to receive at least one receive packet of size \field{mtu} (plus low
level ethernet header length) with \field{gso_type} NONE or ECN.

If the driver negotiates VIRTIO_NET_F_MTU, it MUST NOT transmit packets of
size exceeding the value of \field{mtu} (plus low level ethernet header length)
with \field{gso_type} NONE or ECN.

A driver SHOULD negotiate the VIRTIO_NET_F_STANDBY feature if the device offers it.

If VIRTIO_NET_F_SPEED_DUPLEX has been negotiated,
the driver MUST treat any value of \field{speed} above
0x7fffffff as well as any value of \field{duplex} not
matching 0x00 or 0x01 as an unknown value.

If VIRTIO_NET_F_SPEED_DUPLEX has been negotiated, the driver
SHOULD re-read \field{speed} and \field{duplex} after a
configuration change notification.

A driver SHOULD NOT negotiate VIRTIO_NET_F_HASH_REPORT if it
does not negotiate VIRTIO_NET_F_CTRL_VQ.

A driver SHOULD NOT negotiate VIRTIO_NET_F_CTRL_RX_EXTRA if it
does not negotiate VIRTIO_NET_F_CTRL_VQ.

\subsubsection{Legacy Interface: Device configuration layout}\label{sec:Device Types / Network Device / Device configuration layout / Legacy Interface: Device configuration layout}
\label{sec:Device Types / Block Device / Feature bits / Device configuration layout / Legacy Interface: Device configuration layout}
When using the legacy interface, transitional devices and drivers
MUST format \field{status} and
\field{max_virtqueue_pairs} in struct virtio_net_config
according to the native endian of the guest rather than
(necessarily when not using the legacy interface) little-endian.

When using the legacy interface, \field{mac} is driver-writable
which provided a way for drivers to update the MAC without
negotiating VIRTIO_NET_F_CTRL_MAC_ADDR.

\subsection{Device Initialization}\label{sec:Device Types / Network Device / Device Initialization}

A driver would perform a typical initialization routine like so:

\begin{enumerate}
\item Identify and initialize the receive and
  transmission virtqueues, up to N of each kind. If
  VIRTIO_NET_F_MQ feature bit is negotiated,
  N=\field{max_virtqueue_pairs}, otherwise identify N=1.

\item If the VIRTIO_NET_F_CTRL_VQ feature bit is negotiated,
  identify the control virtqueue.

\item Fill the receive queues with buffers: see \ref{sec:Device Types / Network Device / Device Operation / Setting Up Receive Buffers}.

\item Even with VIRTIO_NET_F_MQ, only receiveq1, transmitq1 and
  controlq are used by default.  The driver would send the
  VIRTIO_NET_CTRL_MQ_VQ_PAIRS_SET command specifying the
  number of the transmit and receive queues to use.

\item If the VIRTIO_NET_F_MAC feature bit is set, the configuration
  space \field{mac} entry indicates the ``physical'' address of the
  device, otherwise the driver would typically generate a random
  local MAC address.

\item If the VIRTIO_NET_F_STATUS feature bit is negotiated, the link
  status comes from the bottom bit of \field{status}.
  Otherwise, the driver assumes it's active.

\item A performant driver would indicate that it will generate checksumless
  packets by negotiating the VIRTIO_NET_F_CSUM feature.

\item If that feature is negotiated, a driver can use TCP segmentation or UDP
  segmentation/fragmentation offload by negotiating the VIRTIO_NET_F_HOST_TSO4 (IPv4
  TCP), VIRTIO_NET_F_HOST_TSO6 (IPv6 TCP), VIRTIO_NET_F_HOST_UFO
  (UDP fragmentation) and VIRTIO_NET_F_HOST_USO (UDP segmentation) features.

\item The converse features are also available: a driver can save
  the virtual device some work by negotiating these features.\note{For example, a network packet transported between two guests on
the same system might not need checksumming at all, nor segmentation,
if both guests are amenable.}
   The VIRTIO_NET_F_GUEST_CSUM feature indicates that partially
  checksummed packets can be received, and if it can do that then
  the VIRTIO_NET_F_GUEST_TSO4, VIRTIO_NET_F_GUEST_TSO6,
  VIRTIO_NET_F_GUEST_UFO, VIRTIO_NET_F_GUEST_ECN, VIRTIO_NET_F_GUEST_USO4
  and VIRTIO_NET_F_GUEST_USO6 are the input equivalents of the features described above.
  See \ref{sec:Device Types / Network Device / Device Operation /
Setting Up Receive Buffers}~\nameref{sec:Device Types / Network
Device / Device Operation / Setting Up Receive Buffers} and
\ref{sec:Device Types / Network Device / Device Operation /
Processing of Incoming Packets}~\nameref{sec:Device Types /
Network Device / Device Operation / Processing of Incoming Packets} below.
\end{enumerate}

A truly minimal driver would only accept VIRTIO_NET_F_MAC and ignore
everything else.

\subsection{Device Operation}\label{sec:Device Types / Network Device / Device Operation}

Packets are transmitted by placing them in the
transmitq1\ldots transmitqN, and buffers for incoming packets are
placed in the receiveq1\ldots receiveqN. In each case, the packet
itself is preceded by a header:

\begin{lstlisting}
struct virtio_net_hdr {
#define VIRTIO_NET_HDR_F_NEEDS_CSUM    1
#define VIRTIO_NET_HDR_F_DATA_VALID    2
#define VIRTIO_NET_HDR_F_RSC_INFO      4
        u8 flags;
#define VIRTIO_NET_HDR_GSO_NONE        0
#define VIRTIO_NET_HDR_GSO_TCPV4       1
#define VIRTIO_NET_HDR_GSO_UDP         3
#define VIRTIO_NET_HDR_GSO_TCPV6       4
#define VIRTIO_NET_HDR_GSO_UDP_L4      5
#define VIRTIO_NET_HDR_GSO_ECN      0x80
        u8 gso_type;
        le16 hdr_len;
        le16 gso_size;
        le16 csum_start;
        le16 csum_offset;
        le16 num_buffers;
        le32 hash_value;        (Only if VIRTIO_NET_F_HASH_REPORT negotiated)
        le16 hash_report;       (Only if VIRTIO_NET_F_HASH_REPORT negotiated)
        le16 padding_reserved;  (Only if VIRTIO_NET_F_HASH_REPORT negotiated)
};
\end{lstlisting}

The controlq is used to control various device features described further in
section \ref{sec:Device Types / Network Device / Device Operation / Control Virtqueue}.

\subsubsection{Legacy Interface: Device Operation}\label{sec:Device Types / Network Device / Device Operation / Legacy Interface: Device Operation}
When using the legacy interface, transitional devices and drivers
MUST format the fields in \field{struct virtio_net_hdr}
according to the native endian of the guest rather than
(necessarily when not using the legacy interface) little-endian.

The legacy driver only presented \field{num_buffers} in the \field{struct virtio_net_hdr}
when VIRTIO_NET_F_MRG_RXBUF was negotiated; without that feature the
structure was 2 bytes shorter.

When using the legacy interface, the driver SHOULD ignore the
used length for the transmit queues
and the controlq queue.
\begin{note}
Historically, some devices put
the total descriptor length there, even though no data was
actually written.
\end{note}

\subsubsection{Packet Transmission}\label{sec:Device Types / Network Device / Device Operation / Packet Transmission}

Transmitting a single packet is simple, but varies depending on
the different features the driver negotiated.

\begin{enumerate}
\item The driver can send a completely checksummed packet.  In this case,
  \field{flags} will be zero, and \field{gso_type} will be VIRTIO_NET_HDR_GSO_NONE.

\item If the driver negotiated VIRTIO_NET_F_CSUM, it can skip
  checksumming the packet:
  \begin{itemize}
  \item \field{flags} has the VIRTIO_NET_HDR_F_NEEDS_CSUM set,

  \item \field{csum_start} is set to the offset within the packet to begin checksumming,
    and

  \item \field{csum_offset} indicates how many bytes after the csum_start the
    new (16 bit ones' complement) checksum is placed by the device.

  \item The TCP checksum field in the packet is set to the sum
    of the TCP pseudo header, so that replacing it by the ones'
    complement checksum of the TCP header and body will give the
    correct result.
  \end{itemize}

\begin{note}
For example, consider a partially checksummed TCP (IPv4) packet.
It will have a 14 byte ethernet header and 20 byte IP header
followed by the TCP header (with the TCP checksum field 16 bytes
into that header). \field{csum_start} will be 14+20 = 34 (the TCP
checksum includes the header), and \field{csum_offset} will be 16.
\end{note}

\item If the driver negotiated
  VIRTIO_NET_F_HOST_TSO4, TSO6, USO or UFO, and the packet requires
  TCP segmentation, UDP segmentation or fragmentation, then \field{gso_type}
  is set to VIRTIO_NET_HDR_GSO_TCPV4, TCPV6, UDP_L4 or UDP.
  (Otherwise, it is set to VIRTIO_NET_HDR_GSO_NONE). In this
  case, packets larger than 1514 bytes can be transmitted: the
  metadata indicates how to replicate the packet header to cut it
  into smaller packets. The other gso fields are set:

  \begin{itemize}
  \item If the VIRTIO_NET_F_GUEST_HDRLEN feature has been negotiated,
    \field{hdr_len} indicates the header length that needs to be replicated
    for each packet. It's the number of bytes from the beginning of the packet
    to the beginning of the transport payload.
    Otherwise, if the VIRTIO_NET_F_GUEST_HDRLEN feature has not been negotiated,
    \field{hdr_len} is a hint to the device as to how much of the header
    needs to be kept to copy into each packet, usually set to the
    length of the headers, including the transport header\footnote{Due to various bugs in implementations, this field is not useful
as a guarantee of the transport header size.
}.

  \begin{note}
  Some devices benefit from knowledge of the exact header length.
  \end{note}

  \item \field{gso_size} is the maximum size of each packet beyond that
    header (ie. MSS).

  \item If the driver negotiated the VIRTIO_NET_F_HOST_ECN feature,
    the VIRTIO_NET_HDR_GSO_ECN bit in \field{gso_type}
    indicates that the TCP packet has the ECN bit set\footnote{This case is not handled by some older hardware, so is called out
specifically in the protocol.}.
   \end{itemize}

\item \field{num_buffers} is set to zero.  This field is unused on transmitted packets.

\item The header and packet are added as one output descriptor to the
  transmitq, and the device is notified of the new entry
  (see \ref{sec:Device Types / Network Device / Device Initialization}~\nameref{sec:Device Types / Network Device / Device Initialization}).
\end{enumerate}

\drivernormative{\paragraph}{Packet Transmission}{Device Types / Network Device / Device Operation / Packet Transmission}

For the transmit packet buffer, the driver MUST use the size of the
structure \field{struct virtio_net_hdr} same as the receive packet buffer.

The driver MUST set \field{num_buffers} to zero.

If VIRTIO_NET_F_CSUM is not negotiated, the driver MUST set
\field{flags} to zero and SHOULD supply a fully checksummed
packet to the device.

If VIRTIO_NET_F_HOST_TSO4 is negotiated, the driver MAY set
\field{gso_type} to VIRTIO_NET_HDR_GSO_TCPV4 to request TCPv4
segmentation, otherwise the driver MUST NOT set
\field{gso_type} to VIRTIO_NET_HDR_GSO_TCPV4.

If VIRTIO_NET_F_HOST_TSO6 is negotiated, the driver MAY set
\field{gso_type} to VIRTIO_NET_HDR_GSO_TCPV6 to request TCPv6
segmentation, otherwise the driver MUST NOT set
\field{gso_type} to VIRTIO_NET_HDR_GSO_TCPV6.

If VIRTIO_NET_F_HOST_UFO is negotiated, the driver MAY set
\field{gso_type} to VIRTIO_NET_HDR_GSO_UDP to request UDP
fragmentation, otherwise the driver MUST NOT set
\field{gso_type} to VIRTIO_NET_HDR_GSO_UDP.

If VIRTIO_NET_F_HOST_USO is negotiated, the driver MAY set
\field{gso_type} to VIRTIO_NET_HDR_GSO_UDP_L4 to request UDP
segmentation, otherwise the driver MUST NOT set
\field{gso_type} to VIRTIO_NET_HDR_GSO_UDP_L4.

The driver SHOULD NOT send to the device TCP packets requiring segmentation offload
which have the Explicit Congestion Notification bit set, unless the
VIRTIO_NET_F_HOST_ECN feature is negotiated, in which case the
driver MUST set the VIRTIO_NET_HDR_GSO_ECN bit in
\field{gso_type}.

If the VIRTIO_NET_F_CSUM feature has been negotiated, the
driver MAY set the VIRTIO_NET_HDR_F_NEEDS_CSUM bit in
\field{flags}, if so:
\begin{enumerate}
\item the driver MUST validate the packet checksum at
	offset \field{csum_offset} from \field{csum_start} as well as all
	preceding offsets;
\begin{note}
If \field{gso_type} differs from VIRTIO_NET_HDR_GSO_NONE, \field{csum_offset}
points to the only transport header present in the packet, and there are no
additional preceding checksums validated by VIRTIO_NET_HDR_F_NEEDS_CSUM.
\end{note}
\item the driver MUST set the packet checksum stored in the
	buffer to the TCP/UDP pseudo header;
\item the driver MUST set \field{csum_start} and
	\field{csum_offset} such that calculating a ones'
	complement checksum from \field{csum_start} up until the end of
	the packet and storing the result at offset \field{csum_offset}
	from  \field{csum_start} will result in a fully checksummed
	packet;
\end{enumerate}

If none of the VIRTIO_NET_F_HOST_TSO4, TSO6, USO or UFO options have
been negotiated, the driver MUST set \field{gso_type} to
VIRTIO_NET_HDR_GSO_NONE.

If \field{gso_type} differs from VIRTIO_NET_HDR_GSO_NONE, then
the driver MUST also set the VIRTIO_NET_HDR_F_NEEDS_CSUM bit in
\field{flags} and MUST set \field{gso_size} to indicate the
desired MSS.

If one of the VIRTIO_NET_F_HOST_TSO4, TSO6, USO or UFO options have
been negotiated:
\begin{itemize}
\item If the VIRTIO_NET_F_GUEST_HDRLEN feature has been negotiated,
	and \field{gso_type} differs from VIRTIO_NET_HDR_GSO_NONE,
	the driver MUST set \field{hdr_len} to a value equal to the length
	of the headers, including the transport header.

\item If the VIRTIO_NET_F_GUEST_HDRLEN feature has not been negotiated,
	or \field{gso_type} is VIRTIO_NET_HDR_GSO_NONE,
	the driver SHOULD set \field{hdr_len} to a value
	not less than the length of the headers, including the transport
	header.
\end{itemize}

The driver SHOULD accept the VIRTIO_NET_F_GUEST_HDRLEN feature if it has
been offered, and if it's able to provide the exact header length.

The driver MUST NOT set the VIRTIO_NET_HDR_F_DATA_VALID and
VIRTIO_NET_HDR_F_RSC_INFO bits in \field{flags}.

\devicenormative{\paragraph}{Packet Transmission}{Device Types / Network Device / Device Operation / Packet Transmission}
The device MUST ignore \field{flag} bits that it does not recognize.

If VIRTIO_NET_HDR_F_NEEDS_CSUM bit in \field{flags} is not set, the
device MUST NOT use the \field{csum_start} and \field{csum_offset}.

If one of the VIRTIO_NET_F_HOST_TSO4, TSO6, USO or UFO options have
been negotiated:
\begin{itemize}
\item If the VIRTIO_NET_F_GUEST_HDRLEN feature has been negotiated,
	and \field{gso_type} differs from VIRTIO_NET_HDR_GSO_NONE,
	the device MAY use \field{hdr_len} as the transport header size.

	\begin{note}
	Caution should be taken by the implementation so as to prevent
	a malicious driver from attacking the device by setting an incorrect hdr_len.
	\end{note}

\item If the VIRTIO_NET_F_GUEST_HDRLEN feature has not been negotiated,
	or \field{gso_type} is VIRTIO_NET_HDR_GSO_NONE,
	the device MAY use \field{hdr_len} only as a hint about the
	transport header size.
	The device MUST NOT rely on \field{hdr_len} to be correct.

	\begin{note}
	This is due to various bugs in implementations.
	\end{note}
\end{itemize}

If VIRTIO_NET_HDR_F_NEEDS_CSUM is not set, the device MUST NOT
rely on the packet checksum being correct.
\paragraph{Packet Transmission Interrupt}\label{sec:Device Types / Network Device / Device Operation / Packet Transmission / Packet Transmission Interrupt}

Often a driver will suppress transmission virtqueue interrupts
and check for used packets in the transmit path of following
packets.

The normal behavior in this interrupt handler is to retrieve
used buffers from the virtqueue and free the corresponding
headers and packets.

\subsubsection{Setting Up Receive Buffers}\label{sec:Device Types / Network Device / Device Operation / Setting Up Receive Buffers}

It is generally a good idea to keep the receive virtqueue as
fully populated as possible: if it runs out, network performance
will suffer.

If the VIRTIO_NET_F_GUEST_TSO4, VIRTIO_NET_F_GUEST_TSO6,
VIRTIO_NET_F_GUEST_UFO, VIRTIO_NET_F_GUEST_USO4 or VIRTIO_NET_F_GUEST_USO6
features are used, the maximum incoming packet
will be 65589 bytes long (14 bytes of Ethernet header, plus 40 bytes of
the IPv6 header, plus 65535 bytes of maximum IPv6 payload including any
extension header), otherwise 1514 bytes.
When VIRTIO_NET_F_HASH_REPORT is not negotiated, the required receive buffer
size is either 65601 or 1526 bytes accounting for 20 bytes of
\field{struct virtio_net_hdr} followed by receive packet.
When VIRTIO_NET_F_HASH_REPORT is negotiated, the required receive buffer
size is either 65609 or 1534 bytes accounting for 12 bytes of
\field{struct virtio_net_hdr} followed by receive packet.

\drivernormative{\paragraph}{Setting Up Receive Buffers}{Device Types / Network Device / Device Operation / Setting Up Receive Buffers}

\begin{itemize}
\item If VIRTIO_NET_F_MRG_RXBUF is not negotiated:
  \begin{itemize}
    \item If VIRTIO_NET_F_GUEST_TSO4, VIRTIO_NET_F_GUEST_TSO6, VIRTIO_NET_F_GUEST_UFO,
	VIRTIO_NET_F_GUEST_USO4 or VIRTIO_NET_F_GUEST_USO6 are negotiated, the driver SHOULD populate
      the receive queue(s) with buffers of at least 65609 bytes if
      VIRTIO_NET_F_HASH_REPORT is negotiated, and of at least 65601 bytes if not.
    \item Otherwise, the driver SHOULD populate the receive queue(s)
      with buffers of at least 1534 bytes if VIRTIO_NET_F_HASH_REPORT
      is negotiated, and of at least 1526 bytes if not.
  \end{itemize}
\item If VIRTIO_NET_F_MRG_RXBUF is negotiated, each buffer MUST be at
least size of \field{struct virtio_net_hdr},
i.e. 20 bytes if VIRTIO_NET_F_HASH_REPORT is negotiated, and 12 bytes if not.
\end{itemize}

\begin{note}
Obviously each buffer can be split across multiple descriptor elements.
\end{note}

When calculating the size of \field{struct virtio_net_hdr}, the driver
MUST consider all the fields inclusive up to \field{padding_reserved},
i.e. 20 bytes if VIRTIO_NET_F_HASH_REPORT is negotiated, and 12 bytes if not.

If VIRTIO_NET_F_MQ is negotiated, each of receiveq1\ldots receiveqN
that will be used SHOULD be populated with receive buffers.

\devicenormative{\paragraph}{Setting Up Receive Buffers}{Device Types / Network Device / Device Operation / Setting Up Receive Buffers}

The device MUST set \field{num_buffers} to the number of descriptors used to
hold the incoming packet.

The device MUST use only a single descriptor if VIRTIO_NET_F_MRG_RXBUF
was not negotiated.
\begin{note}
{This means that \field{num_buffers} will always be 1
if VIRTIO_NET_F_MRG_RXBUF is not negotiated.}
\end{note}

\subsubsection{Processing of Incoming Packets}\label{sec:Device Types / Network Device / Device Operation / Processing of Incoming Packets}
\label{sec:Device Types / Network Device / Device Operation / Processing of Packets}%old label for latexdiff

When a packet is copied into a buffer in the receiveq, the
optimal path is to disable further used buffer notifications for the
receiveq and process packets until no more are found, then re-enable
them.

Processing incoming packets involves:

\begin{enumerate}
\item \field{num_buffers} indicates how many descriptors
  this packet is spread over (including this one): this will
  always be 1 if VIRTIO_NET_F_MRG_RXBUF was not negotiated.
  This allows receipt of large packets without having to allocate large
  buffers: a packet that does not fit in a single buffer can flow
  over to the next buffer, and so on. In this case, there will be
  at least \field{num_buffers} used buffers in the virtqueue, and the device
  chains them together to form a single packet in a way similar to
  how it would store it in a single buffer spread over multiple
  descriptors.
  The other buffers will not begin with a \field{struct virtio_net_hdr}.

\item If
  \field{num_buffers} is one, then the entire packet will be
  contained within this buffer, immediately following the struct
  virtio_net_hdr.
\item If the VIRTIO_NET_F_GUEST_CSUM feature was negotiated, the
  VIRTIO_NET_HDR_F_DATA_VALID bit in \field{flags} can be
  set: if so, device has validated the packet checksum.
  In case of multiple encapsulated protocols, one level of checksums
  has been validated.
\end{enumerate}

Additionally, VIRTIO_NET_F_GUEST_CSUM, TSO4, TSO6, UDP and ECN
features enable receive checksum, large receive offload and ECN
support which are the input equivalents of the transmit checksum,
transmit segmentation offloading and ECN features, as described
in \ref{sec:Device Types / Network Device / Device Operation /
Packet Transmission}:
\begin{enumerate}
\item If the VIRTIO_NET_F_GUEST_TSO4, TSO6, UFO, USO4 or USO6 options were
  negotiated, then \field{gso_type} MAY be something other than
  VIRTIO_NET_HDR_GSO_NONE, and \field{gso_size} field indicates the
  desired MSS (see Packet Transmission point 2).
\item If the VIRTIO_NET_F_RSC_EXT option was negotiated (this
  implies one of VIRTIO_NET_F_GUEST_TSO4, TSO6), the
  device processes also duplicated ACK segments, reports
  number of coalesced TCP segments in \field{csum_start} field and
  number of duplicated ACK segments in \field{csum_offset} field
  and sets bit VIRTIO_NET_HDR_F_RSC_INFO in \field{flags}.
\item If the VIRTIO_NET_F_GUEST_CSUM feature was negotiated, the
  VIRTIO_NET_HDR_F_NEEDS_CSUM bit in \field{flags} can be
  set: if so, the packet checksum at offset \field{csum_offset}
  from \field{csum_start} and any preceding checksums
  have been validated.  The checksum on the packet is incomplete and
  if bit VIRTIO_NET_HDR_F_RSC_INFO is not set in \field{flags},
  then \field{csum_start} and \field{csum_offset} indicate how to calculate it
  (see Packet Transmission point 1).
\begin{note}
If \field{gso_type} differs from VIRTIO_NET_HDR_GSO_NONE, \field{csum_offset}
points to the only transport header present in the packet, and there are no
additional preceding checksums validated by VIRTIO_NET_HDR_F_NEEDS_CSUM.
\end{note}
\end{enumerate}

If applicable, the device calculates per-packet hash for incoming packets as
defined in \ref{sec:Device Types / Network Device / Device Operation / Processing of Incoming Packets / Hash calculation for incoming packets}.

If applicable, the device reports hash information for incoming packets as
defined in \ref{sec:Device Types / Network Device / Device Operation / Processing of Incoming Packets / Hash reporting for incoming packets}.

\devicenormative{\paragraph}{Processing of Incoming Packets}{Device Types / Network Device / Device Operation / Processing of Incoming Packets}
\label{devicenormative:Device Types / Network Device / Device Operation / Processing of Packets}%old label for latexdiff

If VIRTIO_NET_F_MRG_RXBUF has not been negotiated, the device MUST set
\field{num_buffers} to 1.

If VIRTIO_NET_F_MRG_RXBUF has been negotiated, the device MUST set
\field{num_buffers} to indicate the number of buffers
the packet (including the header) is spread over.

If a receive packet is spread over multiple buffers, the device
MUST use all buffers but the last (i.e. the first \field{num_buffers} -
1 buffers) completely up to the full length of each buffer
supplied by the driver.

The device MUST use all buffers used by a single receive
packet together, such that at least \field{num_buffers} are
observed by driver as used.

If VIRTIO_NET_F_GUEST_CSUM is not negotiated, the device MUST set
\field{flags} to zero and SHOULD supply a fully checksummed
packet to the driver.

If VIRTIO_NET_F_GUEST_TSO4 is not negotiated, the device MUST NOT set
\field{gso_type} to VIRTIO_NET_HDR_GSO_TCPV4.

If VIRTIO_NET_F_GUEST_UDP is not negotiated, the device MUST NOT set
\field{gso_type} to VIRTIO_NET_HDR_GSO_UDP.

If VIRTIO_NET_F_GUEST_TSO6 is not negotiated, the device MUST NOT set
\field{gso_type} to VIRTIO_NET_HDR_GSO_TCPV6.

If none of VIRTIO_NET_F_GUEST_USO4 or VIRTIO_NET_F_GUEST_USO6 have been negotiated,
the device MUST NOT set \field{gso_type} to VIRTIO_NET_HDR_GSO_UDP_L4.

The device SHOULD NOT send to the driver TCP packets requiring segmentation offload
which have the Explicit Congestion Notification bit set, unless the
VIRTIO_NET_F_GUEST_ECN feature is negotiated, in which case the
device MUST set the VIRTIO_NET_HDR_GSO_ECN bit in
\field{gso_type}.

If the VIRTIO_NET_F_GUEST_CSUM feature has been negotiated, the
device MAY set the VIRTIO_NET_HDR_F_NEEDS_CSUM bit in
\field{flags}, if so:
\begin{enumerate}
\item the device MUST validate the packet checksum at
	offset \field{csum_offset} from \field{csum_start} as well as all
	preceding offsets;
\item the device MUST set the packet checksum stored in the
	receive buffer to the TCP/UDP pseudo header;
\item the device MUST set \field{csum_start} and
	\field{csum_offset} such that calculating a ones'
	complement checksum from \field{csum_start} up until the
	end of the packet and storing the result at offset
	\field{csum_offset} from  \field{csum_start} will result in a
	fully checksummed packet;
\end{enumerate}

If none of the VIRTIO_NET_F_GUEST_TSO4, TSO6, UFO, USO4 or USO6 options have
been negotiated, the device MUST set \field{gso_type} to
VIRTIO_NET_HDR_GSO_NONE.

If \field{gso_type} differs from VIRTIO_NET_HDR_GSO_NONE, then
the device MUST also set the VIRTIO_NET_HDR_F_NEEDS_CSUM bit in
\field{flags} MUST set \field{gso_size} to indicate the desired MSS.
If VIRTIO_NET_F_RSC_EXT was negotiated, the device MUST also
set VIRTIO_NET_HDR_F_RSC_INFO bit in \field{flags},
set \field{csum_start} to number of coalesced TCP segments and
set \field{csum_offset} to number of received duplicated ACK segments.

If VIRTIO_NET_F_RSC_EXT was not negotiated, the device MUST
not set VIRTIO_NET_HDR_F_RSC_INFO bit in \field{flags}.

If one of the VIRTIO_NET_F_GUEST_TSO4, TSO6, UFO, USO4 or USO6 options have
been negotiated, the device SHOULD set \field{hdr_len} to a value
not less than the length of the headers, including the transport
header.

If the VIRTIO_NET_F_GUEST_CSUM feature has been negotiated, the
device MAY set the VIRTIO_NET_HDR_F_DATA_VALID bit in
\field{flags}, if so, the device MUST validate the packet
checksum. One level of checksum is validated: in case of multiple
encapsulated protocols the outermost one.

\drivernormative{\paragraph}{Processing of Incoming
Packets}{Device Types / Network Device / Device Operation /
Processing of Incoming Packets}

The driver MUST ignore \field{flag} bits that it does not recognize.

If VIRTIO_NET_HDR_F_NEEDS_CSUM bit in \field{flags} is not set or
if VIRTIO_NET_HDR_F_RSC_INFO bit \field{flags} is set, the
driver MUST NOT use the \field{csum_start} and \field{csum_offset}.

If one of the VIRTIO_NET_F_GUEST_TSO4, TSO6, UFO, USO4 or USO6 options have
been negotiated, the driver MAY use \field{hdr_len} only as a hint about the
transport header size.
The driver MUST NOT rely on \field{hdr_len} to be correct.
\begin{note}
This is due to various bugs in implementations.
\end{note}

If neither VIRTIO_NET_HDR_F_NEEDS_CSUM nor
VIRTIO_NET_HDR_F_DATA_VALID is set, the driver MUST NOT
rely on the packet checksum being correct.

\paragraph{Hash calculation for incoming packets}
\label{sec:Device Types / Network Device / Device Operation / Processing of Incoming Packets / Hash calculation for incoming packets}

A device attempts to calculate a per-packet hash in the following cases:
\begin{itemize}
\item The feature VIRTIO_NET_F_RSS was negotiated. The device uses the hash to determine the receive virtqueue to place incoming packets.
\item The feature VIRTIO_NET_F_HASH_REPORT was negotiated. The device reports the hash value and the hash type with the packet.
\end{itemize}

If the feature VIRTIO_NET_F_RSS was negotiated:
\begin{itemize}
\item The device uses \field{hash_types} of the virtio_net_rss_config structure as 'Enabled hash types' bitmask.
\item If additionally the feature VIRTIO_NET_F_HASH_TUNNEL was negotiated, the device uses \field{enabled_tunnel_types} of the
      virtnet_hash_tunnel structure as 'Encapsulation types enabled for inner header hash' bitmask.
\item The device uses a key as defined in \field{hash_key_data} and \field{hash_key_length} of the virtio_net_rss_config structure (see
\ref{sec:Device Types / Network Device / Device Operation / Control Virtqueue / Receive-side scaling (RSS) / Setting RSS parameters}).
\end{itemize}

If the feature VIRTIO_NET_F_RSS was not negotiated:
\begin{itemize}
\item The device uses \field{hash_types} of the virtio_net_hash_config structure as 'Enabled hash types' bitmask.
\item If additionally the feature VIRTIO_NET_F_HASH_TUNNEL was negotiated, the device uses \field{enabled_tunnel_types} of the
      virtnet_hash_tunnel structure as 'Encapsulation types enabled for inner header hash' bitmask.
\item The device uses a key as defined in \field{hash_key_data} and \field{hash_key_length} of the virtio_net_hash_config structure (see
\ref{sec:Device Types / Network Device / Device Operation / Control Virtqueue / Automatic receive steering in multiqueue mode / Hash calculation}).
\end{itemize}

Note that if the device offers VIRTIO_NET_F_HASH_REPORT, even if it supports only one pair of virtqueues, it MUST support
at least one of commands of VIRTIO_NET_CTRL_MQ class to configure reported hash parameters:
\begin{itemize}
\item If the device offers VIRTIO_NET_F_RSS, it MUST support VIRTIO_NET_CTRL_MQ_RSS_CONFIG command per
 \ref{sec:Device Types / Network Device / Device Operation / Control Virtqueue / Receive-side scaling (RSS) / Setting RSS parameters}.
\item Otherwise the device MUST support VIRTIO_NET_CTRL_MQ_HASH_CONFIG command per
 \ref{sec:Device Types / Network Device / Device Operation / Control Virtqueue / Automatic receive steering in multiqueue mode / Hash calculation}.
\end{itemize}

The per-packet hash calculation can depend on the IP packet type. See
\hyperref[intro:IP]{[IP]}, \hyperref[intro:UDP]{[UDP]} and \hyperref[intro:TCP]{[TCP]}.

\subparagraph{Supported/enabled hash types}
\label{sec:Device Types / Network Device / Device Operation / Processing of Incoming Packets / Hash calculation for incoming packets / Supported/enabled hash types}
Hash types applicable for IPv4 packets:
\begin{lstlisting}
#define VIRTIO_NET_HASH_TYPE_IPv4              (1 << 0)
#define VIRTIO_NET_HASH_TYPE_TCPv4             (1 << 1)
#define VIRTIO_NET_HASH_TYPE_UDPv4             (1 << 2)
\end{lstlisting}
Hash types applicable for IPv6 packets without extension headers
\begin{lstlisting}
#define VIRTIO_NET_HASH_TYPE_IPv6              (1 << 3)
#define VIRTIO_NET_HASH_TYPE_TCPv6             (1 << 4)
#define VIRTIO_NET_HASH_TYPE_UDPv6             (1 << 5)
\end{lstlisting}
Hash types applicable for IPv6 packets with extension headers
\begin{lstlisting}
#define VIRTIO_NET_HASH_TYPE_IP_EX             (1 << 6)
#define VIRTIO_NET_HASH_TYPE_TCP_EX            (1 << 7)
#define VIRTIO_NET_HASH_TYPE_UDP_EX            (1 << 8)
\end{lstlisting}

\subparagraph{IPv4 packets}
\label{sec:Device Types / Network Device / Device Operation / Processing of Incoming Packets / Hash calculation for incoming packets / IPv4 packets}
The device calculates the hash on IPv4 packets according to 'Enabled hash types' bitmask as follows:
\begin{itemize}
\item If VIRTIO_NET_HASH_TYPE_TCPv4 is set and the packet has
a TCP header, the hash is calculated over the following fields:
\begin{itemize}
\item Source IP address
\item Destination IP address
\item Source TCP port
\item Destination TCP port
\end{itemize}
\item Else if VIRTIO_NET_HASH_TYPE_UDPv4 is set and the
packet has a UDP header, the hash is calculated over the following fields:
\begin{itemize}
\item Source IP address
\item Destination IP address
\item Source UDP port
\item Destination UDP port
\end{itemize}
\item Else if VIRTIO_NET_HASH_TYPE_IPv4 is set, the hash is
calculated over the following fields:
\begin{itemize}
\item Source IP address
\item Destination IP address
\end{itemize}
\item Else the device does not calculate the hash
\end{itemize}

\subparagraph{IPv6 packets without extension header}
\label{sec:Device Types / Network Device / Device Operation / Processing of Incoming Packets / Hash calculation for incoming packets / IPv6 packets without extension header}
The device calculates the hash on IPv6 packets without extension
headers according to 'Enabled hash types' bitmask as follows:
\begin{itemize}
\item If VIRTIO_NET_HASH_TYPE_TCPv6 is set and the packet has
a TCPv6 header, the hash is calculated over the following fields:
\begin{itemize}
\item Source IPv6 address
\item Destination IPv6 address
\item Source TCP port
\item Destination TCP port
\end{itemize}
\item Else if VIRTIO_NET_HASH_TYPE_UDPv6 is set and the
packet has a UDPv6 header, the hash is calculated over the following fields:
\begin{itemize}
\item Source IPv6 address
\item Destination IPv6 address
\item Source UDP port
\item Destination UDP port
\end{itemize}
\item Else if VIRTIO_NET_HASH_TYPE_IPv6 is set, the hash is
calculated over the following fields:
\begin{itemize}
\item Source IPv6 address
\item Destination IPv6 address
\end{itemize}
\item Else the device does not calculate the hash
\end{itemize}

\subparagraph{IPv6 packets with extension header}
\label{sec:Device Types / Network Device / Device Operation / Processing of Incoming Packets / Hash calculation for incoming packets / IPv6 packets with extension header}
The device calculates the hash on IPv6 packets with extension
headers according to 'Enabled hash types' bitmask as follows:
\begin{itemize}
\item If VIRTIO_NET_HASH_TYPE_TCP_EX is set and the packet
has a TCPv6 header, the hash is calculated over the following fields:
\begin{itemize}
\item Home address from the home address option in the IPv6 destination options header. If the extension header is not present, use the Source IPv6 address.
\item IPv6 address that is contained in the Routing-Header-Type-2 from the associated extension header. If the extension header is not present, use the Destination IPv6 address.
\item Source TCP port
\item Destination TCP port
\end{itemize}
\item Else if VIRTIO_NET_HASH_TYPE_UDP_EX is set and the
packet has a UDPv6 header, the hash is calculated over the following fields:
\begin{itemize}
\item Home address from the home address option in the IPv6 destination options header. If the extension header is not present, use the Source IPv6 address.
\item IPv6 address that is contained in the Routing-Header-Type-2 from the associated extension header. If the extension header is not present, use the Destination IPv6 address.
\item Source UDP port
\item Destination UDP port
\end{itemize}
\item Else if VIRTIO_NET_HASH_TYPE_IP_EX is set, the hash is
calculated over the following fields:
\begin{itemize}
\item Home address from the home address option in the IPv6 destination options header. If the extension header is not present, use the Source IPv6 address.
\item IPv6 address that is contained in the Routing-Header-Type-2 from the associated extension header. If the extension header is not present, use the Destination IPv6 address.
\end{itemize}
\item Else skip IPv6 extension headers and calculate the hash as
defined for an IPv6 packet without extension headers
(see \ref{sec:Device Types / Network Device / Device Operation / Processing of Incoming Packets / Hash calculation for incoming packets / IPv6 packets without extension header}).
\end{itemize}

\paragraph{Inner Header Hash}
\label{sec:Device Types / Network Device / Device Operation / Processing of Incoming Packets / Inner Header Hash}

If VIRTIO_NET_F_HASH_TUNNEL has been negotiated, the driver can send the command
VIRTIO_NET_CTRL_HASH_TUNNEL_SET to configure the calculation of the inner header hash.

\begin{lstlisting}
struct virtnet_hash_tunnel {
    le32 enabled_tunnel_types;
};

#define VIRTIO_NET_CTRL_HASH_TUNNEL 7
 #define VIRTIO_NET_CTRL_HASH_TUNNEL_SET 0
\end{lstlisting}

Field \field{enabled_tunnel_types} contains the bitmask of encapsulation types enabled for inner header hash.
See \ref{sec:Device Types / Network Device / Device Operation / Processing of Incoming Packets /
Hash calculation for incoming packets / Encapsulation types supported/enabled for inner header hash}.

The class VIRTIO_NET_CTRL_HASH_TUNNEL has one command:
VIRTIO_NET_CTRL_HASH_TUNNEL_SET sets \field{enabled_tunnel_types} for the device using the
virtnet_hash_tunnel structure, which is read-only for the device.

Inner header hash is disabled by VIRTIO_NET_CTRL_HASH_TUNNEL_SET with \field{enabled_tunnel_types} set to 0.

Initially (before the driver sends any VIRTIO_NET_CTRL_HASH_TUNNEL_SET command) all
encapsulation types are disabled for inner header hash.

\subparagraph{Encapsulated packet}
\label{sec:Device Types / Network Device / Device Operation / Processing of Incoming Packets / Hash calculation for incoming packets / Encapsulated packet}

Multiple tunneling protocols allow encapsulating an inner, payload packet in an outer, encapsulated packet.
The encapsulated packet thus contains an outer header and an inner header, and the device calculates the
hash over either the inner header or the outer header.

If VIRTIO_NET_F_HASH_TUNNEL is negotiated and a received encapsulated packet's outer header matches one of the
encapsulation types enabled in \field{enabled_tunnel_types}, then the device uses the inner header for hash
calculations (only a single level of encapsulation is currently supported).

If VIRTIO_NET_F_HASH_TUNNEL is negotiated and a received packet's (outer) header does not match any encapsulation
types enabled in \field{enabled_tunnel_types}, then the device uses the outer header for hash calculations.

\subparagraph{Encapsulation types supported/enabled for inner header hash}
\label{sec:Device Types / Network Device / Device Operation / Processing of Incoming Packets /
Hash calculation for incoming packets / Encapsulation types supported/enabled for inner header hash}

Encapsulation types applicable for inner header hash:
\begin{lstlisting}[escapechar=|]
#define VIRTIO_NET_HASH_TUNNEL_TYPE_GRE_2784    (1 << 0) /* |\hyperref[intro:rfc2784]{[RFC2784]}| */
#define VIRTIO_NET_HASH_TUNNEL_TYPE_GRE_2890    (1 << 1) /* |\hyperref[intro:rfc2890]{[RFC2890]}| */
#define VIRTIO_NET_HASH_TUNNEL_TYPE_GRE_7676    (1 << 2) /* |\hyperref[intro:rfc7676]{[RFC7676]}| */
#define VIRTIO_NET_HASH_TUNNEL_TYPE_GRE_UDP     (1 << 3) /* |\hyperref[intro:rfc8086]{[GRE-in-UDP]}| */
#define VIRTIO_NET_HASH_TUNNEL_TYPE_VXLAN       (1 << 4) /* |\hyperref[intro:vxlan]{[VXLAN]}| */
#define VIRTIO_NET_HASH_TUNNEL_TYPE_VXLAN_GPE   (1 << 5) /* |\hyperref[intro:vxlan-gpe]{[VXLAN-GPE]}| */
#define VIRTIO_NET_HASH_TUNNEL_TYPE_GENEVE      (1 << 6) /* |\hyperref[intro:geneve]{[GENEVE]}| */
#define VIRTIO_NET_HASH_TUNNEL_TYPE_IPIP        (1 << 7) /* |\hyperref[intro:ipip]{[IPIP]}| */
#define VIRTIO_NET_HASH_TUNNEL_TYPE_NVGRE       (1 << 8) /* |\hyperref[intro:nvgre]{[NVGRE]}| */
\end{lstlisting}

\subparagraph{Advice}
Example uses of the inner header hash:
\begin{itemize}
\item Legacy tunneling protocols, lacking the outer header entropy, can use RSS with the inner header hash to
      distribute flows with identical outer but different inner headers across various queues, improving performance.
\item Identify an inner flow distributed across multiple outer tunnels.
\end{itemize}

As using the inner header hash completely discards the outer header entropy, care must be taken
if the inner header is controlled by an adversary, as the adversary can then intentionally create
configurations with insufficient entropy.

Besides disabling the inner header hash, mitigations would depend on how the hash is used. When the hash
use is limited to the RSS queue selection, the inner header hash may have quality of service (QoS) limitations.

\devicenormative{\subparagraph}{Inner Header Hash}{Device Types / Network Device / Device Operation / Control Virtqueue / Inner Header Hash}

If the (outer) header of the received packet does not match any encapsulation types enabled
in \field{enabled_tunnel_types}, the device MUST calculate the hash on the outer header.

If the device receives any bits in \field{enabled_tunnel_types} which are not set in \field{supported_tunnel_types},
it SHOULD respond to the VIRTIO_NET_CTRL_HASH_TUNNEL_SET command with VIRTIO_NET_ERR.

If the driver sets \field{enabled_tunnel_types} to 0 through VIRTIO_NET_CTRL_HASH_TUNNEL_SET or upon the device reset,
the device MUST disable the inner header hash for all encapsulation types.

\drivernormative{\subparagraph}{Inner Header Hash}{Device Types / Network Device / Device Operation / Control Virtqueue / Inner Header Hash}

The driver MUST have negotiated the VIRTIO_NET_F_HASH_TUNNEL feature when issuing the VIRTIO_NET_CTRL_HASH_TUNNEL_SET command.

The driver MUST NOT set any bits in \field{enabled_tunnel_types} which are not set in \field{supported_tunnel_types}.

The driver MUST ignore bits in \field{supported_tunnel_types} which are not documented in this specification.

\paragraph{Hash reporting for incoming packets}
\label{sec:Device Types / Network Device / Device Operation / Processing of Incoming Packets / Hash reporting for incoming packets}

If VIRTIO_NET_F_HASH_REPORT was negotiated and
 the device has calculated the hash for the packet, the device fills \field{hash_report} with the report type of calculated hash
and \field{hash_value} with the value of calculated hash.

If VIRTIO_NET_F_HASH_REPORT was negotiated but due to any reason the
hash was not calculated, the device sets \field{hash_report} to VIRTIO_NET_HASH_REPORT_NONE.

Possible values that the device can report in \field{hash_report} are defined below.
They correspond to supported hash types defined in
\ref{sec:Device Types / Network Device / Device Operation / Processing of Incoming Packets / Hash calculation for incoming packets / Supported/enabled hash types}
as follows:

VIRTIO_NET_HASH_TYPE_XXX = 1 << (VIRTIO_NET_HASH_REPORT_XXX - 1)

\begin{lstlisting}
#define VIRTIO_NET_HASH_REPORT_NONE            0
#define VIRTIO_NET_HASH_REPORT_IPv4            1
#define VIRTIO_NET_HASH_REPORT_TCPv4           2
#define VIRTIO_NET_HASH_REPORT_UDPv4           3
#define VIRTIO_NET_HASH_REPORT_IPv6            4
#define VIRTIO_NET_HASH_REPORT_TCPv6           5
#define VIRTIO_NET_HASH_REPORT_UDPv6           6
#define VIRTIO_NET_HASH_REPORT_IPv6_EX         7
#define VIRTIO_NET_HASH_REPORT_TCPv6_EX        8
#define VIRTIO_NET_HASH_REPORT_UDPv6_EX        9
\end{lstlisting}

\subsubsection{Control Virtqueue}\label{sec:Device Types / Network Device / Device Operation / Control Virtqueue}

The driver uses the control virtqueue (if VIRTIO_NET_F_CTRL_VQ is
negotiated) to send commands to manipulate various features of
the device which would not easily map into the configuration
space.

All commands are of the following form:

\begin{lstlisting}
struct virtio_net_ctrl {
        u8 class;
        u8 command;
        u8 command-specific-data[];
        u8 ack;
};

/* ack values */
#define VIRTIO_NET_OK     0
#define VIRTIO_NET_ERR    1
\end{lstlisting}

The \field{class}, \field{command} and command-specific-data are set by the
driver, and the device sets the \field{ack} byte. There is little it can
do except issue a diagnostic if \field{ack} is not
VIRTIO_NET_OK.

\paragraph{Packet Receive Filtering}\label{sec:Device Types / Network Device / Device Operation / Control Virtqueue / Packet Receive Filtering}
\label{sec:Device Types / Network Device / Device Operation / Control Virtqueue / Setting Promiscuous Mode}%old label for latexdiff

If the VIRTIO_NET_F_CTRL_RX and VIRTIO_NET_F_CTRL_RX_EXTRA
features are negotiated, the driver can send control commands for
promiscuous mode, multicast, unicast and broadcast receiving.

\begin{note}
In general, these commands are best-effort: unwanted
packets could still arrive.
\end{note}

\begin{lstlisting}
#define VIRTIO_NET_CTRL_RX    0
 #define VIRTIO_NET_CTRL_RX_PROMISC      0
 #define VIRTIO_NET_CTRL_RX_ALLMULTI     1
 #define VIRTIO_NET_CTRL_RX_ALLUNI       2
 #define VIRTIO_NET_CTRL_RX_NOMULTI      3
 #define VIRTIO_NET_CTRL_RX_NOUNI        4
 #define VIRTIO_NET_CTRL_RX_NOBCAST      5
\end{lstlisting}


\devicenormative{\subparagraph}{Packet Receive Filtering}{Device Types / Network Device / Device Operation / Control Virtqueue / Packet Receive Filtering}

If the VIRTIO_NET_F_CTRL_RX feature has been negotiated,
the device MUST support the following VIRTIO_NET_CTRL_RX class
commands:
\begin{itemize}
\item VIRTIO_NET_CTRL_RX_PROMISC turns promiscuous mode on and
off. The command-specific-data is one byte containing 0 (off) or
1 (on). If promiscuous mode is on, the device SHOULD receive all
incoming packets.
This SHOULD take effect even if one of the other modes set by
a VIRTIO_NET_CTRL_RX class command is on.
\item VIRTIO_NET_CTRL_RX_ALLMULTI turns all-multicast receive on and
off. The command-specific-data is one byte containing 0 (off) or
1 (on). When all-multicast receive is on the device SHOULD allow
all incoming multicast packets.
\end{itemize}

If the VIRTIO_NET_F_CTRL_RX_EXTRA feature has been negotiated,
the device MUST support the following VIRTIO_NET_CTRL_RX class
commands:
\begin{itemize}
\item VIRTIO_NET_CTRL_RX_ALLUNI turns all-unicast receive on and
off. The command-specific-data is one byte containing 0 (off) or
1 (on). When all-unicast receive is on the device SHOULD allow
all incoming unicast packets.
\item VIRTIO_NET_CTRL_RX_NOMULTI suppresses multicast receive.
The command-specific-data is one byte containing 0 (multicast
receive allowed) or 1 (multicast receive suppressed).
When multicast receive is suppressed, the device SHOULD NOT
send multicast packets to the driver.
This SHOULD take effect even if VIRTIO_NET_CTRL_RX_ALLMULTI is on.
This filter SHOULD NOT apply to broadcast packets.
\item VIRTIO_NET_CTRL_RX_NOUNI suppresses unicast receive.
The command-specific-data is one byte containing 0 (unicast
receive allowed) or 1 (unicast receive suppressed).
When unicast receive is suppressed, the device SHOULD NOT
send unicast packets to the driver.
This SHOULD take effect even if VIRTIO_NET_CTRL_RX_ALLUNI is on.
\item VIRTIO_NET_CTRL_RX_NOBCAST suppresses broadcast receive.
The command-specific-data is one byte containing 0 (broadcast
receive allowed) or 1 (broadcast receive suppressed).
When broadcast receive is suppressed, the device SHOULD NOT
send broadcast packets to the driver.
This SHOULD take effect even if VIRTIO_NET_CTRL_RX_ALLMULTI is on.
\end{itemize}

\drivernormative{\subparagraph}{Packet Receive Filtering}{Device Types / Network Device / Device Operation / Control Virtqueue / Packet Receive Filtering}

If the VIRTIO_NET_F_CTRL_RX feature has not been negotiated,
the driver MUST NOT issue commands VIRTIO_NET_CTRL_RX_PROMISC or
VIRTIO_NET_CTRL_RX_ALLMULTI.

If the VIRTIO_NET_F_CTRL_RX_EXTRA feature has not been negotiated,
the driver MUST NOT issue commands
 VIRTIO_NET_CTRL_RX_ALLUNI,
 VIRTIO_NET_CTRL_RX_NOMULTI,
 VIRTIO_NET_CTRL_RX_NOUNI or
 VIRTIO_NET_CTRL_RX_NOBCAST.

\paragraph{Setting MAC Address Filtering}\label{sec:Device Types / Network Device / Device Operation / Control Virtqueue / Setting MAC Address Filtering}

If the VIRTIO_NET_F_CTRL_RX feature is negotiated, the driver can
send control commands for MAC address filtering.

\begin{lstlisting}
struct virtio_net_ctrl_mac {
        le32 entries;
        u8 macs[entries][6];
};

#define VIRTIO_NET_CTRL_MAC    1
 #define VIRTIO_NET_CTRL_MAC_TABLE_SET        0
 #define VIRTIO_NET_CTRL_MAC_ADDR_SET         1
\end{lstlisting}

The device can filter incoming packets by any number of destination
MAC addresses\footnote{Since there are no guarantees, it can use a hash filter or
silently switch to allmulti or promiscuous mode if it is given too
many addresses.
}. This table is set using the class
VIRTIO_NET_CTRL_MAC and the command VIRTIO_NET_CTRL_MAC_TABLE_SET. The
command-specific-data is two variable length tables of 6-byte MAC
addresses (as described in struct virtio_net_ctrl_mac). The first table contains unicast addresses, and the second
contains multicast addresses.

The VIRTIO_NET_CTRL_MAC_ADDR_SET command is used to set the
default MAC address which rx filtering
accepts (and if VIRTIO_NET_F_MAC has been negotiated,
this will be reflected in \field{mac} in config space).

The command-specific-data for VIRTIO_NET_CTRL_MAC_ADDR_SET is
the 6-byte MAC address.

\devicenormative{\subparagraph}{Setting MAC Address Filtering}{Device Types / Network Device / Device Operation / Control Virtqueue / Setting MAC Address Filtering}

The device MUST have an empty MAC filtering table on reset.

The device MUST update the MAC filtering table before it consumes
the VIRTIO_NET_CTRL_MAC_TABLE_SET command.

The device MUST update \field{mac} in config space before it consumes
the VIRTIO_NET_CTRL_MAC_ADDR_SET command, if VIRTIO_NET_F_MAC has
been negotiated.

The device SHOULD drop incoming packets which have a destination MAC which
matches neither the \field{mac} (or that set with VIRTIO_NET_CTRL_MAC_ADDR_SET)
nor the MAC filtering table.

\drivernormative{\subparagraph}{Setting MAC Address Filtering}{Device Types / Network Device / Device Operation / Control Virtqueue / Setting MAC Address Filtering}

If VIRTIO_NET_F_CTRL_RX has not been negotiated,
the driver MUST NOT issue VIRTIO_NET_CTRL_MAC class commands.

If VIRTIO_NET_F_CTRL_RX has been negotiated,
the driver SHOULD issue VIRTIO_NET_CTRL_MAC_ADDR_SET
to set the default mac if it is different from \field{mac}.

The driver MUST follow the VIRTIO_NET_CTRL_MAC_TABLE_SET command
by a le32 number, followed by that number of non-multicast
MAC addresses, followed by another le32 number, followed by
that number of multicast addresses.  Either number MAY be 0.

\subparagraph{Legacy Interface: Setting MAC Address Filtering}\label{sec:Device Types / Network Device / Device Operation / Control Virtqueue / Setting MAC Address Filtering / Legacy Interface: Setting MAC Address Filtering}
When using the legacy interface, transitional devices and drivers
MUST format \field{entries} in struct virtio_net_ctrl_mac
according to the native endian of the guest rather than
(necessarily when not using the legacy interface) little-endian.

Legacy drivers that didn't negotiate VIRTIO_NET_F_CTRL_MAC_ADDR
changed \field{mac} in config space when NIC is accepting
incoming packets. These drivers always wrote the mac value from
first to last byte, therefore after detecting such drivers,
a transitional device MAY defer MAC update, or MAY defer
processing incoming packets until driver writes the last byte
of \field{mac} in the config space.

\paragraph{VLAN Filtering}\label{sec:Device Types / Network Device / Device Operation / Control Virtqueue / VLAN Filtering}

If the driver negotiates the VIRTIO_NET_F_CTRL_VLAN feature, it
can control a VLAN filter table in the device. The VLAN filter
table applies only to VLAN tagged packets.

When VIRTIO_NET_F_CTRL_VLAN is negotiated, the device starts with
an empty VLAN filter table.

\begin{note}
Similar to the MAC address based filtering, the VLAN filtering
is also best-effort: unwanted packets could still arrive.
\end{note}

\begin{lstlisting}
#define VIRTIO_NET_CTRL_VLAN       2
 #define VIRTIO_NET_CTRL_VLAN_ADD             0
 #define VIRTIO_NET_CTRL_VLAN_DEL             1
\end{lstlisting}

Both the VIRTIO_NET_CTRL_VLAN_ADD and VIRTIO_NET_CTRL_VLAN_DEL
command take a little-endian 16-bit VLAN id as the command-specific-data.

VIRTIO_NET_CTRL_VLAN_ADD command adds the specified VLAN to the
VLAN filter table.

VIRTIO_NET_CTRL_VLAN_DEL command removes the specified VLAN from
the VLAN filter table.

\devicenormative{\subparagraph}{VLAN Filtering}{Device Types / Network Device / Device Operation / Control Virtqueue / VLAN Filtering}

When VIRTIO_NET_F_CTRL_VLAN is not negotiated, the device MUST
accept all VLAN tagged packets.

When VIRTIO_NET_F_CTRL_VLAN is negotiated, the device MUST
accept all VLAN tagged packets whose VLAN tag is present in
the VLAN filter table and SHOULD drop all VLAN tagged packets
whose VLAN tag is absent in the VLAN filter table.

\subparagraph{Legacy Interface: VLAN Filtering}\label{sec:Device Types / Network Device / Device Operation / Control Virtqueue / VLAN Filtering / Legacy Interface: VLAN Filtering}
When using the legacy interface, transitional devices and drivers
MUST format the VLAN id
according to the native endian of the guest rather than
(necessarily when not using the legacy interface) little-endian.

\paragraph{Gratuitous Packet Sending}\label{sec:Device Types / Network Device / Device Operation / Control Virtqueue / Gratuitous Packet Sending}

If the driver negotiates the VIRTIO_NET_F_GUEST_ANNOUNCE (depends
on VIRTIO_NET_F_CTRL_VQ), the device can ask the driver to send gratuitous
packets; this is usually done after the guest has been physically
migrated, and needs to announce its presence on the new network
links. (As hypervisor does not have the knowledge of guest
network configuration (eg. tagged vlan) it is simplest to prod
the guest in this way).

\begin{lstlisting}
#define VIRTIO_NET_CTRL_ANNOUNCE       3
 #define VIRTIO_NET_CTRL_ANNOUNCE_ACK             0
\end{lstlisting}

The driver checks VIRTIO_NET_S_ANNOUNCE bit in the device configuration \field{status} field
when it notices the changes of device configuration. The
command VIRTIO_NET_CTRL_ANNOUNCE_ACK is used to indicate that
driver has received the notification and device clears the
VIRTIO_NET_S_ANNOUNCE bit in \field{status}.

Processing this notification involves:

\begin{enumerate}
\item Sending the gratuitous packets (eg. ARP) or marking there are pending
  gratuitous packets to be sent and letting deferred routine to
  send them.

\item Sending VIRTIO_NET_CTRL_ANNOUNCE_ACK command through control
  vq.
\end{enumerate}

\drivernormative{\subparagraph}{Gratuitous Packet Sending}{Device Types / Network Device / Device Operation / Control Virtqueue / Gratuitous Packet Sending}

If the driver negotiates VIRTIO_NET_F_GUEST_ANNOUNCE, it SHOULD notify
network peers of its new location after it sees the VIRTIO_NET_S_ANNOUNCE bit
in \field{status}.  The driver MUST send a command on the command queue
with class VIRTIO_NET_CTRL_ANNOUNCE and command VIRTIO_NET_CTRL_ANNOUNCE_ACK.

\devicenormative{\subparagraph}{Gratuitous Packet Sending}{Device Types / Network Device / Device Operation / Control Virtqueue / Gratuitous Packet Sending}

If VIRTIO_NET_F_GUEST_ANNOUNCE is negotiated, the device MUST clear the
VIRTIO_NET_S_ANNOUNCE bit in \field{status} upon receipt of a command buffer
with class VIRTIO_NET_CTRL_ANNOUNCE and command VIRTIO_NET_CTRL_ANNOUNCE_ACK
before marking the buffer as used.

\paragraph{Device operation in multiqueue mode}\label{sec:Device Types / Network Device / Device Operation / Control Virtqueue / Device operation in multiqueue mode}

This specification defines the following modes that a device MAY implement for operation with multiple transmit/receive virtqueues:
\begin{itemize}
\item Automatic receive steering as defined in \ref{sec:Device Types / Network Device / Device Operation / Control Virtqueue / Automatic receive steering in multiqueue mode}.
 If a device supports this mode, it offers the VIRTIO_NET_F_MQ feature bit.
\item Receive-side scaling as defined in \ref{devicenormative:Device Types / Network Device / Device Operation / Control Virtqueue / Receive-side scaling (RSS) / RSS processing}.
 If a device supports this mode, it offers the VIRTIO_NET_F_RSS feature bit.
\end{itemize}

A device MAY support one of these features or both. The driver MAY negotiate any set of these features that the device supports.

Multiqueue is disabled by default.

The driver enables multiqueue by sending a command using \field{class} VIRTIO_NET_CTRL_MQ. The \field{command} selects the mode of multiqueue operation, as follows:
\begin{lstlisting}
#define VIRTIO_NET_CTRL_MQ    4
 #define VIRTIO_NET_CTRL_MQ_VQ_PAIRS_SET        0 (for automatic receive steering)
 #define VIRTIO_NET_CTRL_MQ_RSS_CONFIG          1 (for configurable receive steering)
 #define VIRTIO_NET_CTRL_MQ_HASH_CONFIG         2 (for configurable hash calculation)
\end{lstlisting}

If more than one multiqueue mode is negotiated, the resulting device configuration is defined by the last command sent by the driver.

\paragraph{Automatic receive steering in multiqueue mode}\label{sec:Device Types / Network Device / Device Operation / Control Virtqueue / Automatic receive steering in multiqueue mode}

If the driver negotiates the VIRTIO_NET_F_MQ feature bit (depends on VIRTIO_NET_F_CTRL_VQ), it MAY transmit outgoing packets on one
of the multiple transmitq1\ldots transmitqN and ask the device to
queue incoming packets into one of the multiple receiveq1\ldots receiveqN
depending on the packet flow.

The driver enables multiqueue by
sending the VIRTIO_NET_CTRL_MQ_VQ_PAIRS_SET command, specifying
the number of the transmit and receive queues to be used up to
\field{max_virtqueue_pairs}; subsequently,
transmitq1\ldots transmitqn and receiveq1\ldots receiveqn where
n=\field{virtqueue_pairs} MAY be used.
\begin{lstlisting}
struct virtio_net_ctrl_mq_pairs_set {
       le16 virtqueue_pairs;
};
#define VIRTIO_NET_CTRL_MQ_VQ_PAIRS_MIN        1
#define VIRTIO_NET_CTRL_MQ_VQ_PAIRS_MAX        0x8000

\end{lstlisting}

When multiqueue is enabled by VIRTIO_NET_CTRL_MQ_VQ_PAIRS_SET command, the device MUST use automatic receive steering
based on packet flow. Programming of the receive steering
classificator is implicit. After the driver transmitted a packet of a
flow on transmitqX, the device SHOULD cause incoming packets for that flow to
be steered to receiveqX. For uni-directional protocols, or where
no packets have been transmitted yet, the device MAY steer a packet
to a random queue out of the specified receiveq1\ldots receiveqn.

Multiqueue is disabled by VIRTIO_NET_CTRL_MQ_VQ_PAIRS_SET with \field{virtqueue_pairs} to 1 (this is
the default) and waiting for the device to use the command buffer.

\drivernormative{\subparagraph}{Automatic receive steering in multiqueue mode}{Device Types / Network Device / Device Operation / Control Virtqueue / Automatic receive steering in multiqueue mode}

The driver MUST configure the virtqueues before enabling them with the
VIRTIO_NET_CTRL_MQ_VQ_PAIRS_SET command.

The driver MUST NOT request a \field{virtqueue_pairs} of 0 or
greater than \field{max_virtqueue_pairs} in the device configuration space.

The driver MUST queue packets only on any transmitq1 before the
VIRTIO_NET_CTRL_MQ_VQ_PAIRS_SET command.

The driver MUST NOT queue packets on transmit queues greater than
\field{virtqueue_pairs} once it has placed the VIRTIO_NET_CTRL_MQ_VQ_PAIRS_SET command in the available ring.

\devicenormative{\subparagraph}{Automatic receive steering in multiqueue mode}{Device Types / Network Device / Device Operation / Control Virtqueue / Automatic receive steering in multiqueue mode}

After initialization of reset, the device MUST queue packets only on receiveq1.

The device MUST NOT queue packets on receive queues greater than
\field{virtqueue_pairs} once it has placed the
VIRTIO_NET_CTRL_MQ_VQ_PAIRS_SET command in a used buffer.

If the destination receive queue is being reset (See \ref{sec:Basic Facilities of a Virtio Device / Virtqueues / Virtqueue Reset}),
the device SHOULD re-select another random queue. If all receive queues are
being reset, the device MUST drop the packet.

\subparagraph{Legacy Interface: Automatic receive steering in multiqueue mode}\label{sec:Device Types / Network Device / Device Operation / Control Virtqueue / Automatic receive steering in multiqueue mode / Legacy Interface: Automatic receive steering in multiqueue mode}
When using the legacy interface, transitional devices and drivers
MUST format \field{virtqueue_pairs}
according to the native endian of the guest rather than
(necessarily when not using the legacy interface) little-endian.

\subparagraph{Hash calculation}\label{sec:Device Types / Network Device / Device Operation / Control Virtqueue / Automatic receive steering in multiqueue mode / Hash calculation}
If VIRTIO_NET_F_HASH_REPORT was negotiated and the device uses automatic receive steering,
the device MUST support a command to configure hash calculation parameters.

The driver provides parameters for hash calculation as follows:

\field{class} VIRTIO_NET_CTRL_MQ, \field{command} VIRTIO_NET_CTRL_MQ_HASH_CONFIG.

The \field{command-specific-data} has following format:
\begin{lstlisting}
struct virtio_net_hash_config {
    le32 hash_types;
    le16 reserved[4];
    u8 hash_key_length;
    u8 hash_key_data[hash_key_length];
};
\end{lstlisting}
Field \field{hash_types} contains a bitmask of allowed hash types as
defined in
\ref{sec:Device Types / Network Device / Device Operation / Processing of Incoming Packets / Hash calculation for incoming packets / Supported/enabled hash types}.
Initially the device has all hash types disabled and reports only VIRTIO_NET_HASH_REPORT_NONE.

Field \field{reserved} MUST contain zeroes. It is defined to make the structure to match the layout of virtio_net_rss_config structure,
defined in \ref{sec:Device Types / Network Device / Device Operation / Control Virtqueue / Receive-side scaling (RSS)}.

Fields \field{hash_key_length} and \field{hash_key_data} define the key to be used in hash calculation.

\paragraph{Receive-side scaling (RSS)}\label{sec:Device Types / Network Device / Device Operation / Control Virtqueue / Receive-side scaling (RSS)}
A device offers the feature VIRTIO_NET_F_RSS if it supports RSS receive steering with Toeplitz hash calculation and configurable parameters.

A driver queries RSS capabilities of the device by reading device configuration as defined in \ref{sec:Device Types / Network Device / Device configuration layout}

\subparagraph{Setting RSS parameters}\label{sec:Device Types / Network Device / Device Operation / Control Virtqueue / Receive-side scaling (RSS) / Setting RSS parameters}

Driver sends a VIRTIO_NET_CTRL_MQ_RSS_CONFIG command using the following format for \field{command-specific-data}:
\begin{lstlisting}
struct rss_rq_id {
   le16 vq_index_1_16: 15; /* Bits 1 to 16 of the virtqueue index */
   le16 reserved: 1; /* Set to zero */
};

struct virtio_net_rss_config {
    le32 hash_types;
    le16 indirection_table_mask;
    struct rss_rq_id unclassified_queue;
    struct rss_rq_id indirection_table[indirection_table_length];
    le16 max_tx_vq;
    u8 hash_key_length;
    u8 hash_key_data[hash_key_length];
};
\end{lstlisting}
Field \field{hash_types} contains a bitmask of allowed hash types as
defined in
\ref{sec:Device Types / Network Device / Device Operation / Processing of Incoming Packets / Hash calculation for incoming packets / Supported/enabled hash types}.

Field \field{indirection_table_mask} is a mask to be applied to
the calculated hash to produce an index in the
\field{indirection_table} array.
Number of entries in \field{indirection_table} is (\field{indirection_table_mask} + 1).

\field{rss_rq_id} is a receive virtqueue id. \field{vq_index_1_16}
consists of bits 1 to 16 of a virtqueue index. For example, a
\field{vq_index_1_16} value of 3 corresponds to virtqueue index 6,
which maps to receiveq4.

Field \field{unclassified_queue} specifies the receive virtqueue id in which to
place unclassified packets.

Field \field{indirection_table} is an array of receive virtqueues ids.

A driver sets \field{max_tx_vq} to inform a device how many transmit virtqueues it may use (transmitq1\ldots transmitq \field{max_tx_vq}).

Fields \field{hash_key_length} and \field{hash_key_data} define the key to be used in hash calculation.

\drivernormative{\subparagraph}{Setting RSS parameters}{Device Types / Network Device / Device Operation / Control Virtqueue / Receive-side scaling (RSS) }

A driver MUST NOT send the VIRTIO_NET_CTRL_MQ_RSS_CONFIG command if the feature VIRTIO_NET_F_RSS has not been negotiated.

A driver MUST fill the \field{indirection_table} array only with
enabled receive virtqueues ids.

The number of entries in \field{indirection_table} (\field{indirection_table_mask} + 1) MUST be a power of two.

A driver MUST use \field{indirection_table_mask} values that are less than \field{rss_max_indirection_table_length} reported by a device.

A driver MUST NOT set any VIRTIO_NET_HASH_TYPE_ flags that are not supported by a device.

\devicenormative{\subparagraph}{RSS processing}{Device Types / Network Device / Device Operation / Control Virtqueue / Receive-side scaling (RSS) / RSS processing}
The device MUST determine the destination queue for a network packet as follows:
\begin{itemize}
\item Calculate the hash of the packet as defined in \ref{sec:Device Types / Network Device / Device Operation / Processing of Incoming Packets / Hash calculation for incoming packets}.
\item If the device did not calculate the hash for the specific packet, the device directs the packet to the receiveq specified by \field{unclassified_queue} of virtio_net_rss_config structure.
\item Apply \field{indirection_table_mask} to the calculated hash
and use the result as the index in the indirection table to get
the destination receive virtqueue id.
\item If the destination receive queue is being reset (See \ref{sec:Basic Facilities of a Virtio Device / Virtqueues / Virtqueue Reset}), the device MUST drop the packet.
\end{itemize}

\paragraph{Offloads State Configuration}\label{sec:Device Types / Network Device / Device Operation / Control Virtqueue / Offloads State Configuration}

If the VIRTIO_NET_F_CTRL_GUEST_OFFLOADS feature is negotiated, the driver can
send control commands for dynamic offloads state configuration.

\subparagraph{Setting Offloads State}\label{sec:Device Types / Network Device / Device Operation / Control Virtqueue / Offloads State Configuration / Setting Offloads State}

To configure the offloads, the following layout structure and
definitions are used:

\begin{lstlisting}
le64 offloads;

#define VIRTIO_NET_F_GUEST_CSUM       1
#define VIRTIO_NET_F_GUEST_TSO4       7
#define VIRTIO_NET_F_GUEST_TSO6       8
#define VIRTIO_NET_F_GUEST_ECN        9
#define VIRTIO_NET_F_GUEST_UFO        10
#define VIRTIO_NET_F_GUEST_USO4       54
#define VIRTIO_NET_F_GUEST_USO6       55

#define VIRTIO_NET_CTRL_GUEST_OFFLOADS       5
 #define VIRTIO_NET_CTRL_GUEST_OFFLOADS_SET   0
\end{lstlisting}

The class VIRTIO_NET_CTRL_GUEST_OFFLOADS has one command:
VIRTIO_NET_CTRL_GUEST_OFFLOADS_SET applies the new offloads configuration.

le64 value passed as command data is a bitmask, bits set define
offloads to be enabled, bits cleared - offloads to be disabled.

There is a corresponding device feature for each offload. Upon feature
negotiation corresponding offload gets enabled to preserve backward
compatibility.

\drivernormative{\subparagraph}{Setting Offloads State}{Device Types / Network Device / Device Operation / Control Virtqueue / Offloads State Configuration / Setting Offloads State}

A driver MUST NOT enable an offload for which the appropriate feature
has not been negotiated.

\subparagraph{Legacy Interface: Setting Offloads State}\label{sec:Device Types / Network Device / Device Operation / Control Virtqueue / Offloads State Configuration / Setting Offloads State / Legacy Interface: Setting Offloads State}
When using the legacy interface, transitional devices and drivers
MUST format \field{offloads}
according to the native endian of the guest rather than
(necessarily when not using the legacy interface) little-endian.


\paragraph{Notifications Coalescing}\label{sec:Device Types / Network Device / Device Operation / Control Virtqueue / Notifications Coalescing}

If the VIRTIO_NET_F_NOTF_COAL feature is negotiated, the driver can
send commands VIRTIO_NET_CTRL_NOTF_COAL_TX_SET and VIRTIO_NET_CTRL_NOTF_COAL_RX_SET
for notification coalescing.

If the VIRTIO_NET_F_VQ_NOTF_COAL feature is negotiated, the driver can
send commands VIRTIO_NET_CTRL_NOTF_COAL_VQ_SET and VIRTIO_NET_CTRL_NOTF_COAL_VQ_GET
for virtqueue notification coalescing.

\begin{lstlisting}
struct virtio_net_ctrl_coal {
    le32 max_packets;
    le32 max_usecs;
};

struct virtio_net_ctrl_coal_vq {
    le16 vq_index;
    le16 reserved;
    struct virtio_net_ctrl_coal coal;
};

#define VIRTIO_NET_CTRL_NOTF_COAL 6
 #define VIRTIO_NET_CTRL_NOTF_COAL_TX_SET  0
 #define VIRTIO_NET_CTRL_NOTF_COAL_RX_SET 1
 #define VIRTIO_NET_CTRL_NOTF_COAL_VQ_SET 2
 #define VIRTIO_NET_CTRL_NOTF_COAL_VQ_GET 3
\end{lstlisting}

Coalescing parameters:
\begin{itemize}
\item \field{vq_index}: The virtqueue index of an enabled transmit or receive virtqueue.
\item \field{max_usecs} for RX: Maximum number of microseconds to delay a RX notification.
\item \field{max_usecs} for TX: Maximum number of microseconds to delay a TX notification.
\item \field{max_packets} for RX: Maximum number of packets to receive before a RX notification.
\item \field{max_packets} for TX: Maximum number of packets to send before a TX notification.
\end{itemize}

\field{reserved} is reserved and it is ignored by the device.

Read/Write attributes for coalescing parameters:
\begin{itemize}
\item For commands VIRTIO_NET_CTRL_NOTF_COAL_TX_SET and VIRTIO_NET_CTRL_NOTF_COAL_RX_SET, the structure virtio_net_ctrl_coal is write-only for the driver.
\item For the command VIRTIO_NET_CTRL_NOTF_COAL_VQ_SET, the structure virtio_net_ctrl_coal_vq is write-only for the driver.
\item For the command VIRTIO_NET_CTRL_NOTF_COAL_VQ_GET, \field{vq_index} and \field{reserved} are write-only
      for the driver, and the structure virtio_net_ctrl_coal is read-only for the driver.
\end{itemize}

The class VIRTIO_NET_CTRL_NOTF_COAL has the following commands:
\begin{enumerate}
\item VIRTIO_NET_CTRL_NOTF_COAL_TX_SET: use the structure virtio_net_ctrl_coal to set the \field{max_usecs} and \field{max_packets} parameters for all transmit virtqueues.
\item VIRTIO_NET_CTRL_NOTF_COAL_RX_SET: use the structure virtio_net_ctrl_coal to set the \field{max_usecs} and \field{max_packets} parameters for all receive virtqueues.
\item VIRTIO_NET_CTRL_NOTF_COAL_VQ_SET: use the structure virtio_net_ctrl_coal_vq to set the \field{max_usecs} and \field{max_packets} parameters
                                        for an enabled transmit/receive virtqueue whose index is \field{vq_index}.
\item VIRTIO_NET_CTRL_NOTF_COAL_VQ_GET: use the structure virtio_net_ctrl_coal_vq to get the \field{max_usecs} and \field{max_packets} parameters
                                        for an enabled transmit/receive virtqueue whose index is \field{vq_index}.
\end{enumerate}

The device may generate notifications more or less frequently than specified by set commands of the VIRTIO_NET_CTRL_NOTF_COAL class.

If coalescing parameters are being set, the device applies the last coalescing parameters set for a
virtqueue, regardless of the command used to set the parameters. Use the following command sequence
with two pairs of virtqueues as an example:
Each of the following commands sets \field{max_usecs} and \field{max_packets} parameters for virtqueues.
\begin{itemize}
\item Command1: VIRTIO_NET_CTRL_NOTF_COAL_RX_SET sets coalescing parameters for virtqueues having index 0 and index 2. Virtqueues having index 1 and index 3 retain their previous parameters.
\item Command2: VIRTIO_NET_CTRL_NOTF_COAL_VQ_SET with \field{vq_index} = 0 sets coalescing parameters for virtqueue having index 0. Virtqueue having index 2 retains the parameters from command1.
\item Command3: VIRTIO_NET_CTRL_NOTF_COAL_VQ_GET with \field{vq_index} = 0, the device responds with coalescing parameters of vq_index 0 set by command2.
\item Command4: VIRTIO_NET_CTRL_NOTF_COAL_VQ_SET with \field{vq_index} = 1 sets coalescing parameters for virtqueue having index 1. Virtqueue having index 3 retains its previous parameters.
\item Command5: VIRTIO_NET_CTRL_NOTF_COAL_TX_SET sets coalescing parameters for virtqueues having index 1 and index 3, and overrides the parameters set by command4.
\item Command6: VIRTIO_NET_CTRL_NOTF_COAL_VQ_GET with \field{vq_index} = 1, the device responds with coalescing parameters of index 1 set by command5.
\end{itemize}

\subparagraph{Operation}\label{sec:Device Types / Network Device / Device Operation / Control Virtqueue / Notifications Coalescing / Operation}

The device sends a used buffer notification once the notification conditions are met and if the notifications are not suppressed as explained in \ref{sec:Basic Facilities of a Virtio Device / Virtqueues / Used Buffer Notification Suppression}.

When the device has non-zero \field{max_usecs} and non-zero \field{max_packets}, it starts counting microseconds and packets upon receiving/sending a packet.
The device counts packets and microseconds for each receive virtqueue and transmit virtqueue separately.
In this case, the notification conditions are met when \field{max_usecs} microseconds elapse, or upon sending/receiving \field{max_packets} packets, whichever happens first.
Afterwards, the device waits for the next packet and starts counting packets and microseconds again.

When the device has \field{max_usecs} = 0 or \field{max_packets} = 0, the notification conditions are met after every packet received/sent.

\subparagraph{RX Example}\label{sec:Device Types / Network Device / Device Operation / Control Virtqueue / Notifications Coalescing / RX Example}

If, for example:
\begin{itemize}
\item \field{max_usecs} = 10.
\item \field{max_packets} = 15.
\end{itemize}
then each receive virtqueue of a device will operate as follows:
\begin{itemize}
\item The device will count packets received on each virtqueue until it accumulates 15, or until 10 microseconds elapsed since the first one was received.
\item If the notifications are not suppressed by the driver, the device will send an used buffer notification, otherwise, the device will not send an used buffer notification as long as the notifications are suppressed.
\end{itemize}

\subparagraph{TX Example}\label{sec:Device Types / Network Device / Device Operation / Control Virtqueue / Notifications Coalescing / TX Example}

If, for example:
\begin{itemize}
\item \field{max_usecs} = 10.
\item \field{max_packets} = 15.
\end{itemize}
then each transmit virtqueue of a device will operate as follows:
\begin{itemize}
\item The device will count packets sent on each virtqueue until it accumulates 15, or until 10 microseconds elapsed since the first one was sent.
\item If the notifications are not suppressed by the driver, the device will send an used buffer notification, otherwise, the device will not send an used buffer notification as long as the notifications are suppressed.
\end{itemize}

\subparagraph{Notifications When Coalescing Parameters Change}\label{sec:Device Types / Network Device / Device Operation / Control Virtqueue / Notifications Coalescing / Notifications When Coalescing Parameters Change}

When the coalescing parameters of a device change, the device needs to check if the new notification conditions are met and send a used buffer notification if so.

For example, \field{max_packets} = 15 for a device with a single transmit virtqueue: if the device sends 10 packets and afterwards receives a
VIRTIO_NET_CTRL_NOTF_COAL_TX_SET command with \field{max_packets} = 8, then the notification condition is immediately considered to be met;
the device needs to immediately send a used buffer notification, if the notifications are not suppressed by the driver.

\drivernormative{\subparagraph}{Notifications Coalescing}{Device Types / Network Device / Device Operation / Control Virtqueue / Notifications Coalescing}

The driver MUST set \field{vq_index} to the virtqueue index of an enabled transmit or receive virtqueue.

The driver MUST have negotiated the VIRTIO_NET_F_NOTF_COAL feature when issuing commands VIRTIO_NET_CTRL_NOTF_COAL_TX_SET and VIRTIO_NET_CTRL_NOTF_COAL_RX_SET.

The driver MUST have negotiated the VIRTIO_NET_F_VQ_NOTF_COAL feature when issuing commands VIRTIO_NET_CTRL_NOTF_COAL_VQ_SET and VIRTIO_NET_CTRL_NOTF_COAL_VQ_GET.

The driver MUST ignore the values of coalescing parameters received from the VIRTIO_NET_CTRL_NOTF_COAL_VQ_GET command if the device responds with VIRTIO_NET_ERR.

\devicenormative{\subparagraph}{Notifications Coalescing}{Device Types / Network Device / Device Operation / Control Virtqueue / Notifications Coalescing}

The device MUST ignore \field{reserved}.

The device SHOULD respond to VIRTIO_NET_CTRL_NOTF_COAL_TX_SET and VIRTIO_NET_CTRL_NOTF_COAL_RX_SET commands with VIRTIO_NET_ERR if it was not able to change the parameters.

The device MUST respond to the VIRTIO_NET_CTRL_NOTF_COAL_VQ_SET command with VIRTIO_NET_ERR if it was not able to change the parameters.

The device MUST respond to VIRTIO_NET_CTRL_NOTF_COAL_VQ_SET and VIRTIO_NET_CTRL_NOTF_COAL_VQ_GET commands with
VIRTIO_NET_ERR if the designated virtqueue is not an enabled transmit or receive virtqueue.

Upon disabling and re-enabling a transmit virtqueue, the device MUST set the coalescing parameters of the virtqueue
to those configured through the VIRTIO_NET_CTRL_NOTF_COAL_TX_SET command, or, if the driver did not set any TX coalescing parameters, to 0.

Upon disabling and re-enabling a receive virtqueue, the device MUST set the coalescing parameters of the virtqueue
to those configured through the VIRTIO_NET_CTRL_NOTF_COAL_RX_SET command, or, if the driver did not set any RX coalescing parameters, to 0.

The behavior of the device in response to set commands of the VIRTIO_NET_CTRL_NOTF_COAL class is best-effort:
the device MAY generate notifications more or less frequently than specified.

A device SHOULD NOT send used buffer notifications to the driver if the notifications are suppressed, even if the notification conditions are met.

Upon reset, a device MUST initialize all coalescing parameters to 0.

\subsubsection{Legacy Interface: Framing Requirements}\label{sec:Device
Types / Network Device / Legacy Interface: Framing Requirements}

When using legacy interfaces, transitional drivers which have not
negotiated VIRTIO_F_ANY_LAYOUT MUST use a single descriptor for the
\field{struct virtio_net_hdr} on both transmit and receive, with the
network data in the following descriptors.

Additionally, when using the control virtqueue (see \ref{sec:Device
Types / Network Device / Device Operation / Control Virtqueue})
, transitional drivers which have not
negotiated VIRTIO_F_ANY_LAYOUT MUST:
\begin{itemize}
\item for all commands, use a single 2-byte descriptor including the first two
fields: \field{class} and \field{command}
\item for all commands except VIRTIO_NET_CTRL_MAC_TABLE_SET
use a single descriptor including command-specific-data
with no padding.
\item for the VIRTIO_NET_CTRL_MAC_TABLE_SET command use exactly
two descriptors including command-specific-data with no padding:
the first of these descriptors MUST include the
virtio_net_ctrl_mac table structure for the unicast addresses with no padding,
the second of these descriptors MUST include the
virtio_net_ctrl_mac table structure for the multicast addresses
with no padding.
\item for all commands, use a single 1-byte descriptor for the
\field{ack} field
\end{itemize}

See \ref{sec:Basic
Facilities of a Virtio Device / Virtqueues / Message Framing}.

\section{Block Device}\label{sec:Device Types / Block Device}

The virtio block device is a simple virtual block device (ie.
disk). Read and write requests (and other exotic requests) are
placed in one of its queues, and serviced (probably out of order) by the
device except where noted.

\subsection{Device ID}\label{sec:Device Types / Block Device / Device ID}
  2

\subsection{Virtqueues}\label{sec:Device Types / Block Device / Virtqueues}
\begin{description}
\item[0] requestq1
\item[\ldots]
\item[N-1] requestqN
\end{description}

 N=1 if VIRTIO_BLK_F_MQ is not negotiated, otherwise N is set by
 \field{num_queues}.

\subsection{Feature bits}\label{sec:Device Types / Block Device / Feature bits}

\begin{description}
\item[VIRTIO_BLK_F_SIZE_MAX (1)] Maximum size of any single segment is
    in \field{size_max}.

\item[VIRTIO_BLK_F_SEG_MAX (2)] Maximum number of segments in a
    request is in \field{seg_max}.

\item[VIRTIO_BLK_F_GEOMETRY (4)] Disk-style geometry specified in
    \field{geometry}.

\item[VIRTIO_BLK_F_RO (5)] Device is read-only.

\item[VIRTIO_BLK_F_BLK_SIZE (6)] Block size of disk is in \field{blk_size}.

\item[VIRTIO_BLK_F_FLUSH (9)] Cache flush command support.

\item[VIRTIO_BLK_F_TOPOLOGY (10)] Device exports information on optimal I/O
    alignment.

\item[VIRTIO_BLK_F_CONFIG_WCE (11)] Device can toggle its cache between writeback
    and writethrough modes.

\item[VIRTIO_BLK_F_MQ (12)] Device supports multiqueue.

\item[VIRTIO_BLK_F_DISCARD (13)] Device can support discard command, maximum
    discard sectors size in \field{max_discard_sectors} and maximum discard
    segment number in \field{max_discard_seg}.

\item[VIRTIO_BLK_F_WRITE_ZEROES (14)] Device can support write zeroes command,
     maximum write zeroes sectors size in \field{max_write_zeroes_sectors} and
     maximum write zeroes segment number in \field{max_write_zeroes_seg}.

\item[VIRTIO_BLK_F_LIFETIME (15)] Device supports providing storage lifetime
     information.

\item[VIRTIO_BLK_F_SECURE_ERASE (16)] Device supports secure erase command,
     maximum erase sectors count in \field{max_secure_erase_sectors} and
     maximum erase segment number in \field{max_secure_erase_seg}.

\item[VIRTIO_BLK_F_ZONED(17)] Device is a Zoned Block Device, that is, a device
	that follows the zoned storage device behavior that is also supported by
	industry standards such as the T10 Zoned Block Command standard (ZBC r05) or
	the NVMe(TM) NVM Express Zoned Namespace Command Set Specification 1.1b
	(ZNS). For brevity, these standard documents are referred as "ZBD standards"
	from this point on in the text.

\end{description}

\subsubsection{Legacy Interface: Feature bits}\label{sec:Device Types / Block Device / Feature bits / Legacy Interface: Feature bits}

\begin{description}
\item[VIRTIO_BLK_F_BARRIER (0)] Device supports request barriers.

\item[VIRTIO_BLK_F_SCSI (7)] Device supports scsi packet commands.
\end{description}

\begin{note}
  In the legacy interface, VIRTIO_BLK_F_FLUSH was also
  called VIRTIO_BLK_F_WCE.
\end{note}

\subsection{Device configuration layout}\label{sec:Device Types / Block Device / Device configuration layout}

The block device has the following device configuration layout.

\begin{lstlisting}
struct virtio_blk_config {
        le64 capacity;
        le32 size_max;
        le32 seg_max;
        struct virtio_blk_geometry {
                le16 cylinders;
                u8 heads;
                u8 sectors;
        } geometry;
        le32 blk_size;
        struct virtio_blk_topology {
                // # of logical blocks per physical block (log2)
                u8 physical_block_exp;
                // offset of first aligned logical block
                u8 alignment_offset;
                // suggested minimum I/O size in blocks
                le16 min_io_size;
                // optimal (suggested maximum) I/O size in blocks
                le32 opt_io_size;
        } topology;
        u8 writeback;
        u8 unused0;
        le16 num_queues;
        le32 max_discard_sectors;
        le32 max_discard_seg;
        le32 discard_sector_alignment;
        le32 max_write_zeroes_sectors;
        le32 max_write_zeroes_seg;
        u8 write_zeroes_may_unmap;
        u8 unused1[3];
        le32 max_secure_erase_sectors;
        le32 max_secure_erase_seg;
        le32 secure_erase_sector_alignment;
        struct virtio_blk_zoned_characteristics {
                le32 zone_sectors;
                le32 max_open_zones;
                le32 max_active_zones;
                le32 max_append_sectors;
                le32 write_granularity;
                u8 model;
                u8 unused2[3];
        } zoned;
};
\end{lstlisting}

The \field{capacity} of the device (expressed in 512-byte sectors) is always
present. The availability of the others all depend on various feature
bits as indicated above.

The field \field{num_queues} only exists if VIRTIO_BLK_F_MQ is set. This field specifies
the number of queues.

The parameters in the configuration space of the device \field{max_discard_sectors}
\field{discard_sector_alignment} are expressed in 512-byte units if the
VIRTIO_BLK_F_DISCARD feature bit is negotiated. The \field{max_write_zeroes_sectors}
is expressed in 512-byte units if the VIRTIO_BLK_F_WRITE_ZEROES feature
bit is negotiated. The parameters in the configuration space of the device
\field{max_secure_erase_sectors} \field{secure_erase_sector_alignment} are expressed
in 512-byte units if the VIRTIO_BLK_F_SECURE_ERASE feature bit is negotiated.

If the VIRTIO_BLK_F_ZONED feature is negotiated, then in
\field{virtio_blk_zoned_characteristics},
\begin{itemize}
\item \field{zone_sectors} value is expressed in 512-byte sectors.
\item \field{max_append_sectors} value is expressed in 512-byte sectors.
\item \field{write_granularity} value is expressed in bytes.
\end{itemize}

The \field{model} field in \field{zoned} may have the following values:

\begin{lstlisting}
#define VIRTIO_BLK_Z_NONE      0
#define VIRTIO_BLK_Z_HM        1
#define VIRTIO_BLK_Z_HA        2
\end{lstlisting}

Depending on their design, zoned block devices may follow several possible
models of operation. The three models that are standardized for ZBDs are
drive-managed, host-managed and host-aware.

While being zoned internally, drive-managed ZBDs behave exactly like regular,
non-zoned block devices. For the purposes of virtio standardization,
drive-managed ZBDs can always be treated as non-zoned devices. These devices
have the VIRTIO_BLK_Z_NONE model value set in the \field{model} field in
\field{zoned}.

Devices that offer the VIRTIO_BLK_F_ZONED feature while reporting the
VIRTIO_BLK_Z_NONE zoned model are drive-managed zoned block devices. In this
case, the driver treats the device as a regular non-zoned block device.

Host-managed zoned block devices have their LBA range divided into Sequential
Write Required (SWR) zones that require some additional handling by the host
for correct operation. All write requests to SWR zones are required be
sequential and zones containing some written data need to be reset before that
data can be rewritten. Host-managed devices support a set of ZBD-specific I/O
requests that can be used by the host to manage device zones. Host-managed
devices report VIRTIO_BLK_Z_HM in the \field{model} field in \field{zoned}.

Host-aware zoned block devices have their LBA range divided to Sequential
Write Preferred (SWP) zones that support random write access, similar to
regular non-zoned devices. However, the device I/O performance might not be
optimal if SWP zones are used in a random I/O pattern. SWP zones also support
the same set of ZBD-specific I/O requests as host-managed devices that allow
host-aware devices to be managed by any host that supports zoned block devices
to achieve its optimum performance. Host-aware devices report VIRTIO_BLK_Z_HA
in the \field{model} field in \field{zoned}.

Both SWR zones and SWP zones are sometimes referred as sequential zones.

During device operation, sequential zones can be in one of the following states:
empty, implicitly-open, explicitly-open, closed and full. The state machine that
governs the transitions between these states is described later in this document.

SWR and SWP zones consume volatile device resources while being in certain
states and the device may set limits on the number of zones that can be in these
states simultaneously.

Zoned block devices use two internal counters to account for the device
resources in use, the number of currently open zones and the number of currently
active zones.

Any zone state transition from a state that doesn't consume a zone resource to a
state that consumes the same resource increments the internal device counter for
that resource. Any zone transition out of a state that consumes a zone resource
to a state that doesn't consume the same resource decrements the counter. Any
request that causes the device to exceed the reported zone resource limits is
terminated by the device with a "zone resources exceeded" error as defined for
specific commands later.

\subsubsection{Legacy Interface: Device configuration layout}\label{sec:Device Types / Block Device / Device configuration layout / Legacy Interface: Device configuration layout}
When using the legacy interface, transitional devices and drivers
MUST format the fields in struct virtio_blk_config
according to the native endian of the guest rather than
(necessarily when not using the legacy interface) little-endian.


\subsection{Device Initialization}\label{sec:Device Types / Block Device / Device Initialization}

\begin{enumerate}
\item The device size can be read from \field{capacity}.

\item If the VIRTIO_BLK_F_BLK_SIZE feature is negotiated,
  \field{blk_size} can be read to determine the optimal sector size
  for the driver to use. This does not affect the units used in
  the protocol (always 512 bytes), but awareness of the correct
  value can affect performance.

\item If the VIRTIO_BLK_F_RO feature is set by the device, any write
  requests will fail.

\item If the VIRTIO_BLK_F_TOPOLOGY feature is negotiated, the fields in the
  \field{topology} struct can be read to determine the physical block size and optimal
  I/O lengths for the driver to use. This also does not affect the units
  in the protocol, only performance.

\item If the VIRTIO_BLK_F_CONFIG_WCE feature is negotiated, the cache
  mode can be read or set through the \field{writeback} field.  0 corresponds
  to a writethrough cache, 1 to a writeback cache\footnote{Consistent with
    \ref{devicenormative:Device Types / Block Device / Device Operation},
    a writethrough cache can be defined broadly as a cache that commits
    writes to persistent device backend storage before reporting their
    completion. For example, a battery-backed writeback cache actually
    counts as writethrough according to this definition.}.  The cache mode
  after reset can be either writeback or writethrough.  The actual
  mode can be determined by reading \field{writeback} after feature
  negotiation.

\item If the VIRTIO_BLK_F_DISCARD feature is negotiated,
    \field{max_discard_sectors} and \field{max_discard_seg} can be read
    to determine the maximum discard sectors and maximum number of discard
    segments for the block driver to use. \field{discard_sector_alignment}
    can be used by OS when splitting a request based on alignment.

\item If the VIRTIO_BLK_F_WRITE_ZEROES feature is negotiated,
    \field{max_write_zeroes_sectors} and \field{max_write_zeroes_seg} can
    be read to determine the maximum write zeroes sectors and maximum
    number of write zeroes segments for the block driver to use.

\item If the VIRTIO_BLK_F_MQ feature is negotiated, \field{num_queues} field
    can be read to determine the number of queues.

\item If the VIRTIO_BLK_F_SECURE_ERASE feature is negotiated,
    \field{max_secure_erase_sectors} and \field{max_secure_erase_seg} can be read
    to determine the maximum secure erase sectors and maximum number of
    secure erase segments for the block driver to use.
    \field{secure_erase_sector_alignment} can be used by OS when splitting a
    request based on alignment.

\item If the VIRTIO_BLK_F_ZONED feature is negotiated, the fields in
    \field{zoned} can be read by the driver to determine the zone
    characteristics of the device. All \field{zoned} fields are read-only.

\end{enumerate}

\drivernormative{\subsubsection}{Device Initialization}{Device Types / Block Device / Device Initialization}

Drivers SHOULD NOT negotiate VIRTIO_BLK_F_FLUSH if they are incapable of
sending VIRTIO_BLK_T_FLUSH commands.

If neither VIRTIO_BLK_F_CONFIG_WCE nor VIRTIO_BLK_F_FLUSH are
negotiated, the driver MAY deduce the presence of a writethrough cache.
If VIRTIO_BLK_F_CONFIG_WCE was not negotiated but VIRTIO_BLK_F_FLUSH was,
the driver SHOULD assume presence of a writeback cache.

The driver MUST NOT read \field{writeback} before setting
the FEATURES_OK \field{device status} bit.

Drivers MUST NOT negotiate the VIRTIO_BLK_F_ZONED feature if they are incapable
of supporting devices with the VIRTIO_BLK_Z_HM, VIRTIO_BLK_Z_HA or
VIRTIO_BLK_Z_NONE zoned model.

If the VIRTIO_BLK_F_ZONED feature is offered by the device with the
VIRTIO_BLK_Z_HM zone model, then the VIRTIO_BLK_F_DISCARD feature MUST NOT be
offered by the driver.

If the VIRTIO_BLK_F_ZONED feature and VIRTIO_BLK_F_DISCARD feature are both
offered by the device with the VIRTIO_BLK_Z_HA or VIRTIO_BLK_Z_NONE zone model,
then the driver MAY negotiate these two bits independently.

If the VIRTIO_BLK_F_ZONED feature is negotiated, then
\begin{itemize}
\item if the driver that can not support host-managed zoned devices
    reads VIRTIO_BLK_Z_HM from the \field{model} field of \field{zoned}, the
    driver MUST NOT set FEATURES_OK flag and instead set the FAILED bit.

\item if the driver that can not support zoned devices reads VIRTIO_BLK_Z_HA
    from the \field{model} field of \field{zoned}, the driver
    MAY handle the device as a non-zoned device. In this case, the
    driver SHOULD ignore all other fields in \field{zoned}.
\end{itemize}

\devicenormative{\subsubsection}{Device Initialization}{Device Types / Block Device / Device Initialization}

Devices SHOULD always offer VIRTIO_BLK_F_FLUSH, and MUST offer it
if they offer VIRTIO_BLK_F_CONFIG_WCE.

If VIRTIO_BLK_F_CONFIG_WCE is negotiated but VIRTIO_BLK_F_FLUSH
is not, the device MUST initialize \field{writeback} to 0.

The device MUST initialize padding bytes \field{unused0} and
\field{unused1} to 0.

If the device that is being initialized is a not a zoned device, the device
SHOULD NOT offer the VIRTIO_BLK_F_ZONED feature.

The VIRTIO_BLK_F_ZONED feature cannot be properly negotiated without
FEATURES_OK bit. Legacy devices MUST NOT offer VIRTIO_BLK_F_ZONED feature bit.

If the VIRTIO_BLK_F_ZONED feature is not accepted by the driver,
\begin{itemize}
\item the device with the VIRTIO_BLK_Z_HA or VIRTIO_BLK_Z_NONE zone model SHOULD
    proceed with the initialization while setting all zoned characteristics
    fields to zero.

\item the device with the VIRTIO_BLK_Z_HM zone model MUST fail to set the
    FEATURES_OK device status bit when the driver writes the Device Status
    field.
\end{itemize}

If the VIRTIO_BLK_F_ZONED feature is negotiated, then the \field{model} field in
\field{zoned} struct in the configuration space MUST be set by the device
\begin{itemize}
\item to the value of VIRTIO_BLK_Z_NONE if it operates as a drive-managed
    zoned block device or a non-zoned block device.

\item to the value of VIRTIO_BLK_Z_HM if it operates as a host-managed zoned
    block device.

\item to the value of VIRTIO_BLK_Z_HA if it operates as a host-aware zoned
    block device.
\end{itemize}

If the VIRTIO_BLK_F_ZONED feature is negotiated and the device \field{model}
field in \field{zoned} struct is VIRTIO_BLK_Z_HM or VIRTIO_BLK_Z_HA,

\begin{itemize}
\item the \field{zone_sectors} field of \field{zoned} MUST be set by the device
    to the size of a single zone on the device. All zones of the device have the
    same size indicated by \field{zone_sectors} except for the last zone that
    MAY be smaller than all other zones. The driver can calculate the number of
    zones on the device as
    \begin{lstlisting}
        nr_zones = (capacity + zone_sectors - 1) / zone_sectors;
    \end{lstlisting}
    and the size of the last zone as
    \begin{lstlisting}
        zs_last = capacity - (nr_zones - 1) * zone_sectors;
    \end{lstlisting}

\item The \field{max_open_zones} field of the \field{zoned} structure MUST be
    set by the device to the maximum number of zones that can be open on the
    device (zones in the implicit open or explicit open state). A value
    of zero indicates that the device does not have any limit on the number of
    open zones.

\item The \field{max_active_zones} field of the \field{zoned} structure MUST
    be set by the device to the maximum number zones that can be active on the
    device (zones in the implicit open, explicit open or closed state). A value
    of zero indicates that the device does not have any limit on the number of
    active zones.

\item the \field{max_append_sectors} field of \field{zoned} MUST be set by
    the device to the maximum data size of a VIRTIO_BLK_T_ZONE_APPEND request
    that can be successfully issued to the device. The value of this field MUST
    NOT exceed the \field{seg_max} * \field{size_max} value. A device MAY set
    the \field{max_append_sectors} to zero if it doesn't support
    VIRTIO_BLK_T_ZONE_APPEND requests.

\item the \field{write_granularity} field of \field{zoned} MUST be set by the
    device to the offset and size alignment constraint for VIRTIO_BLK_T_OUT
    and VIRTIO_BLK_T_ZONE_APPEND requests issued to a sequential zone of the
    device.

\item the device MUST initialize padding bytes \field{unused2} to 0.
\end{itemize}

\subsubsection{Legacy Interface: Device Initialization}\label{sec:Device Types / Block Device / Device Initialization / Legacy Interface: Device Initialization}

Because legacy devices do not have FEATURES_OK, transitional devices
MUST implement slightly different behavior around feature negotiation
when used through the legacy interface.  In particular, when using the
legacy interface:

\begin{itemize}
\item the driver MAY read or write \field{writeback} before setting
  the DRIVER or DRIVER_OK \field{device status} bit

\item the device MUST NOT modify the cache mode (and \field{writeback})
  as a result of a driver setting a status bit, unless
  the DRIVER_OK bit is being set and the driver has not set the
  VIRTIO_BLK_F_CONFIG_WCE driver feature bit.

\item the device MUST NOT modify the cache mode (and \field{writeback})
  as a result of a driver modifying the driver feature bits, for example
  if the driver sets the VIRTIO_BLK_F_CONFIG_WCE driver feature bit but
  does not set the VIRTIO_BLK_F_FLUSH bit.
\end{itemize}


\subsection{Device Operation}\label{sec:Device Types / Block Device / Device Operation}

The driver enqueues requests to the virtqueues, and they are used by
the device (not necessarily in order). Each request except
VIRTIO_BLK_T_ZONE_APPEND is of form:

\begin{lstlisting}
struct virtio_blk_req {
        le32 type;
        le32 reserved;
        le64 sector;
        u8 data[];
        u8 status;
};
\end{lstlisting}

The type of the request is either a read (VIRTIO_BLK_T_IN), a write
(VIRTIO_BLK_T_OUT), a discard (VIRTIO_BLK_T_DISCARD), a write zeroes
(VIRTIO_BLK_T_WRITE_ZEROES), a flush (VIRTIO_BLK_T_FLUSH), a get device ID
string command (VIRTIO_BLK_T_GET_ID), a secure erase
(VIRTIO_BLK_T_SECURE_ERASE), or a get device lifetime command
(VIRTIO_BLK_T_GET_LIFETIME).

\begin{lstlisting}
#define VIRTIO_BLK_T_IN           0
#define VIRTIO_BLK_T_OUT          1
#define VIRTIO_BLK_T_FLUSH        4
#define VIRTIO_BLK_T_GET_ID       8
#define VIRTIO_BLK_T_GET_LIFETIME 10
#define VIRTIO_BLK_T_DISCARD      11
#define VIRTIO_BLK_T_WRITE_ZEROES 13
#define VIRTIO_BLK_T_SECURE_ERASE   14
\end{lstlisting}

The \field{sector} number indicates the offset (multiplied by 512) where
the read or write is to occur. This field is unused and set to 0 for
commands other than read, write and some zone operations.

VIRTIO_BLK_T_IN requests populate \field{data} with the contents of sectors
read from the block device (in multiples of 512 bytes).  VIRTIO_BLK_T_OUT
requests write the contents of \field{data} to the block device (in multiples
of 512 bytes).

The \field{data} used for discard, secure erase or write zeroes commands
consists of one or more segments. The maximum number of segments is
\field{max_discard_seg} for discard commands, \field{max_secure_erase_seg} for
secure erase commands and \field{max_write_zeroes_seg} for write zeroes
commands.
Each segment is of form:

\begin{lstlisting}
struct virtio_blk_discard_write_zeroes {
       le64 sector;
       le32 num_sectors;
       struct {
               le32 unmap:1;
               le32 reserved:31;
       } flags;
};
\end{lstlisting}

\field{sector} indicates the starting offset (in 512-byte units) of the
segment, while \field{num_sectors} indicates the number of sectors in each
discarded range. \field{unmap} is only used in write zeroes commands and allows
the device to discard the specified range, provided that following reads return
zeroes.

VIRTIO_BLK_T_GET_ID requests fetch the device ID string from the device into
\field{data}.  The device ID string is a NUL-padded ASCII string up to 20 bytes
long.  If the string is 20 bytes long then there is no NUL terminator.

The \field{data} used for VIRTIO_BLK_T_GET_LIFETIME requests is populated
by the device, and is of the form

\begin{lstlisting}
struct virtio_blk_lifetime {
  le16 pre_eol_info;
  le16 device_lifetime_est_typ_a;
  le16 device_lifetime_est_typ_b;
};
\end{lstlisting}

The \field{pre_eol_info} specifies the percentage of reserved blocks
that are consumed and will have one of these values:

\begin{lstlisting}
/* Value not available */
#define VIRTIO_BLK_PRE_EOL_INFO_UNDEFINED    0
/* < 80% of reserved blocks are consumed */
#define VIRTIO_BLK_PRE_EOL_INFO_NORMAL       1
/* 80% of reserved blocks are consumed */
#define VIRTIO_BLK_PRE_EOL_INFO_WARNING      2
/* 90% of reserved blocks are consumed */
#define VIRTIO_BLK_PRE_EOL_INFO_URGENT       3
/* All others values are reserved */
\end{lstlisting}

The \field{device_lifetime_est_typ_a} refers to wear of SLC cells and is provided
in increments of 10%, with 0 meaning undefined, 1 meaning up-to 10% of lifetime
used, and so on, thru to 11 meaning estimated lifetime exceeded.
All values above 11 are reserved.

The \field{device_lifetime_est_typ_b} refers to wear of MLC cells and is provided
with the same semantics as \field{device_lifetime_est_typ_a}.

The final \field{status} byte is written by the device: either
VIRTIO_BLK_S_OK for success, VIRTIO_BLK_S_IOERR for device or driver
error or VIRTIO_BLK_S_UNSUPP for a request unsupported by device:

\begin{lstlisting}
#define VIRTIO_BLK_S_OK        0
#define VIRTIO_BLK_S_IOERR     1
#define VIRTIO_BLK_S_UNSUPP    2
\end{lstlisting}

The status of individual segments is indeterminate when a discard or write zero
command produces VIRTIO_BLK_S_IOERR.  A segment may have completed
successfully, failed, or not been processed by the device.

The following requirements only apply if the VIRTIO_BLK_F_ZONED feature is
negotiated.

In addition to the request types defined for non-zoned devices, the type of the
request can be a zone report (VIRTIO_BLK_T_ZONE_REPORT), an explicit zone open
(VIRTIO_BLK_T_ZONE_OPEN), a zone close (VIRTIO_BLK_T_ZONE_CLOSE), a zone finish
(VIRTIO_BLK_T_ZONE_FINISH), a zone_append (VIRTIO_BLK_T_ZONE_APPEND), a zone
reset (VIRTIO_BLK_T_ZONE_RESET) or a zone reset all
(VIRTIO_BLK_T_ZONE_RESET_ALL).

\begin{lstlisting}
#define VIRTIO_BLK_T_ZONE_APPEND    15
#define VIRTIO_BLK_T_ZONE_REPORT    16
#define VIRTIO_BLK_T_ZONE_OPEN      18
#define VIRTIO_BLK_T_ZONE_CLOSE     20
#define VIRTIO_BLK_T_ZONE_FINISH    22
#define VIRTIO_BLK_T_ZONE_RESET     24
#define VIRTIO_BLK_T_ZONE_RESET_ALL 26
\end{lstlisting}

Requests of type VIRTIO_BLK_T_OUT, VIRTIO_BLK_T_ZONE_OPEN,
VIRTIO_BLK_T_ZONE_CLOSE, VIRTIO_BLK_T_ZONE_FINISH, VIRTIO_BLK_T_ZONE_APPEND,
VIRTIO_BLK_T_ZONE_RESET or VIRTIO_BLK_T_ZONE_RESET_ALL may be completed by the
device with VIRTIO_BLK_S_OK, VIRTIO_BLK_S_IOERR or VIRTIO_BLK_S_UNSUPP
\field{status}, or, additionally, with  VIRTIO_BLK_S_ZONE_INVALID_CMD,
VIRTIO_BLK_S_ZONE_UNALIGNED_WP, VIRTIO_BLK_S_ZONE_OPEN_RESOURCE or
VIRTIO_BLK_S_ZONE_ACTIVE_RESOURCE ZBD-specific status codes.

Besides the request status, VIRTIO_BLK_T_ZONE_APPEND requests return the
starting sector of the appended data back to the driver. For this reason,
the VIRTIO_BLK_T_ZONE_APPEND request has the layout that is extended to have
the \field{append_sector} field to carry this value:

\begin{lstlisting}
struct virtio_blk_req_za {
        le32 type;
        le32 reserved;
        le64 sector;
        u8 data[];
        le64 append_sector;
        u8 status;
};
\end{lstlisting}

\begin{lstlisting}
#define VIRTIO_BLK_S_ZONE_INVALID_CMD     3
#define VIRTIO_BLK_S_ZONE_UNALIGNED_WP    4
#define VIRTIO_BLK_S_ZONE_OPEN_RESOURCE   5
#define VIRTIO_BLK_S_ZONE_ACTIVE_RESOURCE 6
\end{lstlisting}

Requests of the type VIRTIO_BLK_T_ZONE_REPORT are reads and requests of the type
VIRTIO_BLK_T_ZONE_APPEND are writes. VIRTIO_BLK_T_ZONE_OPEN,
VIRTIO_BLK_T_ZONE_CLOSE, VIRTIO_BLK_T_ZONE_FINISH, VIRTIO_BLK_T_ZONE_RESET and
VIRTIO_BLK_T_ZONE_RESET_ALL are non-data requests.

Zone sector address is a 64-bit address of the first 512-byte sector of the
zone.

VIRTIO_BLK_T_ZONE_OPEN, VIRTIO_BLK_T_ZONE_CLOSE, VIRTIO_BLK_T_ZONE_FINISH and
VIRTIO_BLK_T_ZONE_RESET requests make the zone operation to act on a particular
zone specified by the zone sector address in the \field{sector} of the request.

VIRTIO_BLK_T_ZONE_RESET_ALL request acts upon all applicable zones of the
device. The \field{sector} value is not used for this request.

In ZBD standards, the VIRTIO_BLK_T_ZONE_REPORT request belongs to "Zone
Management Receive" command category and VIRTIO_BLK_T_ZONE_OPEN,
VIRTIO_BLK_T_ZONE_CLOSE, VIRTIO_BLK_T_ZONE_FINISH and
VIRTIO_BLK_T_ZONE_RESET/VIRTIO_BLK_T_ZONE_RESET_ALL requests are categorized as
"Zone Management Send" commands. VIRTIO_BLK_T_ZONE_APPEND is categorized
separately from zone management commands and is the only request that uses
the \field{append_secctor} field \field{virtio_blk_req_za} to return
to the driver the sector at which the data has been appended to the zone.

VIRTIO_BLK_T_ZONE_REPORT is a read request that returns the information about
the current state of zones on the device starting from the zone containing the
\field{sector} of the request. The report consists of a header followed by zero
or more zone descriptors.

A zone report reply has the following structure:

\begin{lstlisting}
struct virtio_blk_zone_report {
        le64   nr_zones;
        u8     reserved[56];
        struct virtio_blk_zone_descriptor zones[];
};
\end{lstlisting}

The device sets the \field{nr_zones} field in the report header to the number of
fully transferred zone descriptors in the data buffer.

A zone descriptor has the following structure:

\begin{lstlisting}
struct virtio_blk_zone_descriptor {
        le64   z_cap;
        le64   z_start;
        le64   z_wp;
        u8     z_type;
        u8     z_state;
        u8     reserved[38];
};
\end{lstlisting}

The zone descriptor field \field{z_type} \field{virtio_blk_zone_descriptor}
indicates the type of the zone.

The following zone types are available:

\begin{lstlisting}
#define VIRTIO_BLK_ZT_CONV     1
#define VIRTIO_BLK_ZT_SWR      2
#define VIRTIO_BLK_ZT_SWP      3
\end{lstlisting}

Read and write operations into zones with the VIRTIO_BLK_ZT_CONV (Conventional)
type have the same behavior as read and write operations on a regular block
device. Any block in a conventional zone can be read or written at any time and
in any order.

Zones with VIRTIO_BLK_ZT_SWR can be read randomly, but must be written
sequentially at a certain point in the zone called the Write Pointer (WP). With
every write, the Write Pointer is incremented by the number of sectors written.

Zones with VIRTIO_BLK_ZT_SWP can be read randomly and should be written
sequentially, similarly to SWR zones. However, SWP zones can accept random write
operations, that is, VIRTIO_BLK_T_OUT requests with a start sector different
from the zone write pointer position.

The field \field{z_state} of \field{virtio_blk_zone_descriptor} indicates the
state of the device zone.

The following zone states are available:

\begin{lstlisting}
#define VIRTIO_BLK_ZS_NOT_WP   0
#define VIRTIO_BLK_ZS_EMPTY    1
#define VIRTIO_BLK_ZS_IOPEN    2
#define VIRTIO_BLK_ZS_EOPEN    3
#define VIRTIO_BLK_ZS_CLOSED   4
#define VIRTIO_BLK_ZS_RDONLY   13
#define VIRTIO_BLK_ZS_FULL     14
#define VIRTIO_BLK_ZS_OFFLINE  15
\end{lstlisting}

Zones of the type VIRTIO_BLK_ZT_CONV are always reported by the device to be in
the VIRTIO_BLK_ZS_NOT_WP state. Zones of the types VIRTIO_BLK_ZT_SWR and
VIRTIO_BLK_ZT_SWP can not transition to the VIRTIO_BLK_ZS_NOT_WP state.

Zones in VIRTIO_BLK_ZS_EMPTY (Empty), VIRTIO_BLK_ZS_IOPEN (Implicitly Open),
VIRTIO_BLK_ZS_EOPEN (Explicitly Open) and VIRTIO_BLK_ZS_CLOSED (Closed) state
are writable, but zones in VIRTIO_BLK_ZS_RDONLY (Read-Only), VIRTIO_BLK_ZS_FULL
(Full) and VIRTIO_BLK_ZS_OFFLINE (Offline) state are not. The write pointer
value (\field{z_wp}) is not valid for Read-Only, Full and Offline zones.

The zone descriptor field \field{z_cap} contains the maximum number of 512-byte
sectors that are available to be written with user data when the zone is in the
Empty state. This value shall be less than or equal to the \field{zone_sectors}
value in \field{virtio_blk_zoned_characteristics} structure in the device
configuration space.

The zone descriptor field \field{z_start} contains the zone sector address.

The zone descriptor field \field{z_wp} contains the sector address where the
next write operation for this zone should be issued. This value is undefined
for conventional zones and for zones in VIRTIO_BLK_ZS_RDONLY,
VIRTIO_BLK_ZS_FULL and VIRTIO_BLK_ZS_OFFLINE state.

Depending on their state, zones consume resources as follows:
\begin{itemize}
\item a zone in VIRTIO_BLK_ZS_IOPEN and VIRTIO_BLK_ZS_EOPEN state consumes one
    open zone resource and, additionally,

\item a zone in VIRTIO_BLK_ZS_IOPEN, VIRTIO_BLK_ZS_EOPEN and
    VIRTIO_BLK_ZS_CLOSED state consumes one active resource.
\end{itemize}

Attempts for zone transitions that violate zone resource limits must fail with
VIRTIO_BLK_S_ZONE_OPEN_RESOURCE or VIRTIO_BLK_S_ZONE_ACTIVE_RESOURCE
\field{status}.

Zones in the VIRTIO_BLK_ZS_EMPTY (Empty) state have the write pointer value
equal to the sector address of the zone. In this state, the entire capacity of
the zone is available for writing. A zone can transition from this state to
\begin{itemize}
\item VIRTIO_BLK_ZS_IOPEN when a successful VIRTIO_BLK_T_OUT request or
    VIRTIO_BLK_T_ZONE_APPEND with a non-zero data size is received for the zone.

\item VIRTIO_BLK_ZS_EOPEN when a successful VIRTIO_BLK_T_ZONE_OPEN request is
    received for the zone
\end{itemize}

When a VIRTIO_BLK_T_ZONE_RESET request is issued to an Empty zone, the request
is completed successfully and the zone stays in the VIRTIO_BLK_ZS_EMPTY state.

Zones in the VIRTIO_BLK_ZS_IOPEN (Implicitly Open) state transition from
this state to
\begin{itemize}
\item VIRTIO_BLK_ZS_EMPTY when a successful VIRTIO_BLK_T_ZONE_RESET request is
    received for the zone,

\item VIRTIO_BLK_ZS_EMPTY when a successful VIRTIO_BLK_T_ZONE_RESET_ALL request
    is received by the device,

\item VIRTIO_BLK_ZS_EOPEN when a successful VIRTIO_BLK_T_ZONE_OPEN request is
    received for the zone,

\item VIRTIO_BLK_ZS_CLOSED when a successful VIRTIO_BLK_T_ZONE_CLOSE request is
    received for the zone,

\item VIRTIO_BLK_ZS_CLOSED implicitly by the device when another zone is
    entering the VIRTIO_BLK_ZS_IOPEN or VIRTIO_BLK_ZS_EOPEN state and the number
    of currently open zones is at \field{max_open_zones} limit,

\item VIRTIO_BLK_ZS_FULL when a successful VIRTIO_BLK_T_ZONE_FINISH request is
    received for the zone.

\item VIRTIO_BLK_ZS_FULL when a successful VIRTIO_BLK_T_OUT or
    VIRTIO_BLK_T_ZONE_APPEND request that causes the zone to reach its writable
    capacity is received for the zone.
\end{itemize}

Zones in the VIRTIO_BLK_ZS_EOPEN (Explicitly Open) state transition from
this state to
\begin{itemize}
\item VIRTIO_BLK_ZS_EMPTY when a successful VIRTIO_BLK_T_ZONE_RESET request is
    received for the zone,

\item VIRTIO_BLK_ZS_EMPTY when a successful VIRTIO_BLK_T_ZONE_RESET_ALL request
    is received by the device,

\item VIRTIO_BLK_ZS_EMPTY when a successful VIRTIO_BLK_T_ZONE_CLOSE request is
    received for the zone and the write pointer of the zone has the value equal
    to the start sector of the zone,

\item VIRTIO_BLK_ZS_CLOSED when a successful VIRTIO_BLK_T_ZONE_CLOSE request is
    received for the zone and the zone write pointer is larger then the start
    sector of the zone,

\item VIRTIO_BLK_ZS_FULL when a successful VIRTIO_BLK_T_ZONE_FINISH request is
    received for the zone,

\item VIRTIO_BLK_ZS_FULL when a successful VIRTIO_BLK_T_OUT or
    VIRTIO_BLK_T_ZONE_APPEND request that causes the zone to reach its writable
    capacity is received for the zone.
\end{itemize}

When a VIRTIO_BLK_T_ZONE_EOPEN request is issued to an Explicitly Open zone, the
request is completed successfully and the zone stays in the VIRTIO_BLK_ZS_EOPEN
state.

Zones in the VIRTIO_BLK_ZS_CLOSED (Closed) state transition from this state
to
\begin{itemize}
\item VIRTIO_BLK_ZS_EMPTY when a successful VIRTIO_BLK_T_ZONE_RESET request is
    received for the zone,

\item VIRTIO_BLK_ZS_EMPTY when a successful VIRTIO_BLK_T_ZONE_RESET_ALL request
    is received by the device,

\item VIRTIO_BLK_ZS_IOPEN when a successful VIRTIO_BLK_T_OUT request or
    VIRTIO_BLK_T_ZONE_APPEND with a non-zero data size is received for the zone.

\item VIRTIO_BLK_ZS_EOPEN when a successful VIRTIO_BLK_T_ZONE_OPEN request is
    received for the zone,
\end{itemize}

When a VIRTIO_BLK_T_ZONE_CLOSE request is issued to a Closed zone, the request
is completed successfully and the zone stays in the VIRTIO_BLK_ZS_CLOSED state.

Zones in the VIRTIO_BLK_ZS_FULL (Full) state transition from this state to
VIRTIO_BLK_ZS_EMPTY when a successful VIRTIO_BLK_T_ZONE_RESET request is
received for the zone or a successful VIRTIO_BLK_T_ZONE_RESET_ALL request is
received by the device.

When a VIRTIO_BLK_T_ZONE_FINISH request is issued to a Full zone, the request
is completed successfully and the zone stays in the VIRTIO_BLK_ZS_FULL state.

The device may automatically transition zones to VIRTIO_BLK_ZS_RDONLY
(Read-Only) or VIRTIO_BLK_ZS_OFFLINE (Offline) state from any other state. The
device may also automatically transition zones in the Read-Only state to the
Offline state. Zones in the Offline state may not transition to any other state.
Such automatic transitions usually indicate hardware failures. The previously
written data may only be read from zones in the Read-Only state. Zones in the
Offline state can not be read or written.

VIRTIO_BLK_S_ZONE_UNALIGNED_WP is set by the device when the request received
from the driver attempts to perform a write to an SWR zone and at least one of
the following conditions is met:

\begin{itemize}
\item the starting sector of the request is not equal to the current value of
    the zone write pointer.

\item the ending sector of the request data multiplied by 512 is not a multiple
    of the value reported by the device in the field \field{write_granularity}
    in the device configuration space.
\end{itemize}

VIRTIO_BLK_S_ZONE_OPEN_RESOURCE is set by the device when a zone operation or
write request received from the driver can not be handled without exceeding the
\field{max_open_zones} limit value reported by the device in the configuration
space.

VIRTIO_BLK_S_ZONE_ACTIVE_RESOURCE is set by the device when a zone operation or
write request received from the driver can not be handled without exceeding the
\field{max_active_zones} limit value reported by the device in the configuration
space.

A zone transition request that leads to both the \field{max_open_zones} and the
\field{max_active_zones} limits to be exceeded is terminated by the device with
VIRTIO_BLK_S_ZONE_ACTIVE_RESOURCE \field{status} value.

The device reports all other error conditions related to zoned block model
operation by setting the VIRTIO_BLK_S_ZONE_INVALID_CMD value in
\field{status} of \field{virtio_blk_req} structure.

\drivernormative{\subsubsection}{Device Operation}{Device Types / Block Device / Device Operation}

The driver SHOULD check if the content of the \field{capacity} field has
changed upon receiving a configuration change notification.

A driver MUST NOT submit a request which would cause a read or write
beyond \field{capacity}.

A driver SHOULD accept the VIRTIO_BLK_F_RO feature if offered.

A driver MUST set \field{sector} to 0 for a VIRTIO_BLK_T_FLUSH request.
A driver SHOULD NOT include any data in a VIRTIO_BLK_T_FLUSH request.

The length of \field{data} MUST be a multiple of 512 bytes for VIRTIO_BLK_T_IN
and VIRTIO_BLK_T_OUT requests.

The length of \field{data} MUST be a multiple of the size of struct
virtio_blk_discard_write_zeroes for VIRTIO_BLK_T_DISCARD,
VIRTIO_BLK_T_SECURE_ERASE and VIRTIO_BLK_T_WRITE_ZEROES requests.

The length of \field{data} MUST be 20 bytes for VIRTIO_BLK_T_GET_ID requests.

VIRTIO_BLK_T_DISCARD requests MUST NOT contain more than
\field{max_discard_seg} struct virtio_blk_discard_write_zeroes segments in
\field{data}.

VIRTIO_BLK_T_SECURE_ERASE requests MUST NOT contain more than
\field{max_secure_erase_seg} struct virtio_blk_discard_write_zeroes segments in
\field{data}.

VIRTIO_BLK_T_WRITE_ZEROES requests MUST NOT contain more than
\field{max_write_zeroes_seg} struct virtio_blk_discard_write_zeroes segments in
\field{data}.

If the VIRTIO_BLK_F_CONFIG_WCE feature is negotiated, the driver MAY
switch to writethrough or writeback mode by writing respectively 0 and
1 to the \field{writeback} field.  After writing a 0 to \field{writeback},
the driver MUST NOT assume that any volatile writes have been committed
to persistent device backend storage.

The \field{unmap} bit MUST be zero for discard commands.  The driver
MUST NOT assume anything about the data returned by read requests after
a range of sectors has been discarded.

A driver MUST NOT assume that individual segments in a multi-segment
VIRTIO_BLK_T_DISCARD or VIRTIO_BLK_T_WRITE_ZEROES request completed
successfully, failed, or were processed by the device at all if the request
failed with VIRTIO_BLK_S_IOERR.

The following requirements only apply if the VIRTIO_BLK_F_ZONED feature is
negotiated.

A zone sector address provided by the driver MUST be a multiple of 512 bytes.

When forming a VIRTIO_BLK_T_ZONE_REPORT request, the driver MUST set a sector
within the sector range of the starting zone to report to \field{sector} field.
It MAY be a sector that is different from the zone sector address.

In VIRTIO_BLK_T_ZONE_OPEN, VIRTIO_BLK_T_ZONE_CLOSE, VIRTIO_BLK_T_ZONE_FINISH and
VIRTIO_BLK_T_ZONE_RESET requests, the driver MUST set \field{sector} field to
point at the first sector in the target zone.

In VIRTIO_BLK_T_ZONE_RESET_ALL request, the driver MUST set the field
\field{sector} to zero value.

The \field{sector} field of the VIRTIO_BLK_T_ZONE_APPEND request MUST specify
the zone sector address of the zone to which data is to be appended at the
position of the write pointer. The size of the data that is appended MUST be a
multiple of \field{write_granularity} bytes and MUST NOT exceed the
\field{max_append_sectors} value provided by the device in
\field{virtio_blk_zoned_characteristics} configuration space structure.

Upon a successful completion of a VIRTIO_BLK_T_ZONE_APPEND request, the driver
MAY read the starting sector location of the written data from the request
field \field{append_sector}.

All VIRTIO_BLK_T_OUT requests issued by the driver to sequential zones and
VIRTIO_BLK_T_ZONE_APPEND requests MUST have:

\begin{enumerate}
\item the data size that is a multiple of the number of bytes reported
    by the device in the field \field{write_granularity} in the
    \field{virtio_blk_zoned_characteristics} configuration space structure.

\item the value of the field \field{sector} that is a multiple of the number of
    bytes reported by the device in the field \field{write_granularity} in the
    \field{virtio_blk_zoned_characteristics} configuration space structure.

\item the data size that will not exceed the writable zone capacity when its
    value is added to the current value of the write pointer of the zone.

\end{enumerate}

\devicenormative{\subsubsection}{Device Operation}{Device Types / Block Device / Device Operation}

The device MAY change the content of the \field{capacity} field during
operation of the device. When this happens, the device SHOULD trigger a
configuration change notification.

A device MUST set the \field{status} byte to VIRTIO_BLK_S_IOERR
for a write request if the VIRTIO_BLK_F_RO feature if offered, and MUST NOT
write any data.

The device MUST set the \field{status} byte to VIRTIO_BLK_S_UNSUPP for
discard, secure erase and write zeroes commands if any unknown flag is set.
Furthermore, the device MUST set the \field{status} byte to
VIRTIO_BLK_S_UNSUPP for discard commands if the \field{unmap} flag is set.

For discard commands, the device MAY deallocate the specified range of
sectors in the device backend storage.

For write zeroes commands, if the \field{unmap} is set, the device MAY
deallocate the specified range of sectors in the device backend storage,
as if the discard command had been sent.  After a write zeroes command
is completed, reads of the specified ranges of sectors MUST return
zeroes.  This is true independent of whether \field{unmap} was set or clear.

The device SHOULD clear the \field{write_zeroes_may_unmap} field of the
virtio configuration space if and only if a write zeroes request cannot
result in deallocating one or more sectors.  The device MAY change the
content of the field during operation of the device; when this happens,
the device SHOULD trigger a configuration change notification.

A write is considered volatile when it is submitted; the contents of
sectors covered by a volatile write are undefined in persistent device
backend storage until the write becomes stable.  A write becomes stable
once it is completed and one or more of the following conditions is true:

\begin{enumerate}
\item\label{item:flush1} neither VIRTIO_BLK_F_CONFIG_WCE nor
  VIRTIO_BLK_F_FLUSH feature were negotiated, but VIRTIO_BLK_F_FLUSH was
  offered by the device;

\item\label{item:flush2} the VIRTIO_BLK_F_CONFIG_WCE feature was negotiated and the
  \field{writeback} field in configuration space was 0 \textbf{all the time between
  the submission of the write and its completion};

\item\label{item:flush3} a VIRTIO_BLK_T_FLUSH request is sent \textbf{after the write is
  completed} and is completed itself.
\end{enumerate}

If the device is backed by persistent storage, the device MUST ensure that
stable writes are committed to it, before reporting completion of the write
(cases~\ref{item:flush1} and~\ref{item:flush2}) or the flush
(case~\ref{item:flush3}).  Failure to do so can cause data loss
in case of a crash.

If the driver changes \field{writeback} between the submission of the write
and its completion, the write could be either volatile or stable when
its completion is reported; in other words, the exact behavior is undefined.

% According to the device requirements for device initialization:
%   Offer(CONFIG_WCE) => Offer(FLUSH).
%
% After reversing the implication:
%   not Offer(FLUSH) => not Offer(CONFIG_WCE).

If VIRTIO_BLK_F_FLUSH was not offered by the
  device\footnote{Note that in this case, according to
  \ref{devicenormative:Device Types / Block Device / Device Initialization},
  the device will not have offered VIRTIO_BLK_F_CONFIG_WCE either.}, the
device MAY also commit writes to persistent device backend storage before
reporting their completion.  Unlike case~\ref{item:flush1}, however, this
is not an absolute requirement of the specification.

\begin{note}
  An implementation that does not offer VIRTIO_BLK_F_FLUSH and does not commit
  completed writes will not be resilient to data loss in case of crashes.
  Not offering VIRTIO_BLK_F_FLUSH is an absolute requirement
  for implementations that do not wish to be safe against such data losses.
\end{note}

If the device is backed by storage providing lifetime metrics (such as eMMC
or UFS persistent storage), the device SHOULD offer the VIRTIO_BLK_F_LIFETIME
flag. The flag MUST NOT be offered if the device is backed by storage for which
the lifetime metrics described in this document cannot be obtained or for which
such metrics have no useful meaning. If the metrics are offered, the device MUST NOT
send any reserved values, as defined in this specification.

\begin{note}
  The device lifetime metrics \field{pre_eol_info}, \field{device_lifetime_est_a}
  and \field{device_lifetime_est_b} are discussed in the JESD84-B50 specification.

  The complete JESD84-B50 is available at the JEDEC website (https://www.jedec.org)
  pursuant to JEDEC's licensing terms and conditions. This information is provided to
  simplfy passthrough implementations from eMMC devices.
\end{note}

If the VIRTIO_BLK_F_ZONED feature is not negotiated, the device MUST reject
VIRTIO_BLK_T_ZONE_REPORT, VIRTIO_BLK_T_ZONE_OPEN, VIRTIO_BLK_T_ZONE_CLOSE,
VIRTIO_BLK_T_ZONE_FINISH, VIRTIO_BLK_T_ZONE_APPEND, VIRTIO_BLK_T_ZONE_RESET and
VIRTIO_BLK_T_ZONE_RESET_ALL requests with VIRTIO_BLK_S_UNSUPP status.

The following device requirements only apply if the VIRTIO_BLK_F_ZONED feature
is negotiated.

If a request of type VIRTIO_BLK_T_ZONE_OPEN, VIRTIO_BLK_T_ZONE_CLOSE,
VIRTIO_BLK_T_ZONE_FINISH or VIRTIO_BLK_T_ZONE_RESET is issued for a Conventional
zone (type VIRTIO_BLK_ZT_CONV), the device MUST complete the request with
VIRTIO_BLK_S_ZONE_INVALID_CMD \field{status}.

If the zone specified by the VIRTIO_BLK_T_ZONE_APPEND request is not a SWR zone,
then the request SHALL be completed with VIRTIO_BLK_S_ZONE_INVALID_CMD
\field{status}.

The device handles a VIRTIO_BLK_T_ZONE_OPEN request by attempting to change the
state of the zone with the \field{sector} address to VIRTIO_BLK_ZS_EOPEN. If the
transition to this state can not be performed, the request MUST be completed
with VIRTIO_BLK_S_ZONE_INVALID_CMD \field{status}. If, while processing this
request, the available zone resources are insufficient, then the zone state does
not change and the request MUST be completed with
VIRTIO_BLK_S_ZONE_OPEN_RESOURCE or VIRTIO_BLK_S_ZONE_ACTIVE_RESOURCE value in
the field \field{status}.

The device handles a VIRTIO_BLK_T_ZONE_CLOSE request by attempting to change the
state of the zone with the \field{sector} address to VIRTIO_BLK_ZS_CLOSED. If
the transition to this state can not be performed, the request MUST be completed
with VIRTIO_BLK_S_ZONE_INVALID_CMD value in the field \field{status}.

The device handles a VIRTIO_BLK_T_ZONE_FINISH request by attempting to change
the state of the zone with the \field{sector} address to VIRTIO_BLK_ZS_FULL. If
the transition to this state can not be performed, the zone state does not
change and the request MUST be completed with VIRTIO_BLK_S_ZONE_INVALID_CMD
value in the field \field{status}.

The device handles a VIRTIO_BLK_T_ZONE_RESET request by attempting to change the
state of the zone with the \field{sector} address to VIRTIO_BLK_ZS_EMPTY state.
If the transition to this state can not be performed, the zone state does not
change and the request MUST be completed with VIRTIO_BLK_S_ZONE_INVALID_CMD
value in the field \field{status}.

The device handles a VIRTIO_BLK_T_ZONE_RESET_ALL request by transitioning all
sequential device zones in VIRTIO_BLK_ZS_IOPEN, VIRTIO_BLK_ZS_EOPEN,
VIRTIO_BLK_ZS_CLOSED and VIRTIO_BLK_ZS_FULL state to VIRTIO_BLK_ZS_EMPTY state.

Upon receiving a VIRTIO_BLK_T_ZONE_APPEND request or a VIRTIO_BLK_T_OUT
request issued to a SWR zone in VIRTIO_BLK_ZS_EMPTY or VIRTIO_BLK_ZS_CLOSED
state, the device attempts to perform the transition of the zone to
VIRTIO_BLK_ZS_IOPEN state before writing data. This transition may fail due to
insufficient open and/or active zone resources available on the device. In this
case, the request MUST be completed with VIRTIO_BLK_S_ZONE_OPEN_RESOURCE or
VIRTIO_BLK_S_ZONE_ACTIVE_RESOURCE value in the \field{status}.

If the \field{sector} field in the VIRTIO_BLK_T_ZONE_APPEND request does not
specify the lowest sector for a zone, then the request SHALL be completed with
VIRTIO_BLK_S_ZONE_INVALID_CMD value in \field{status}.

A VIRTIO_BLK_T_ZONE_APPEND request or a VIRTIO_BLK_T_OUT request that has the
data range that exceeds the remaining writable capacity for the zone, then the
request SHALL be completed with VIRTIO_BLK_S_ZONE_INVALID_CMD value in
\field{status}.

If a request of the type VIRTIO_BLK_T_ZONE_APPEND is completed with
VIRTIO_BLK_S_OK status, the field \field{append_sector} in
\field{virtio_blk_req_za} MUST be set by the device to contain the first sector
of the data written to the zone.

If a request of the type VIRTIO_BLK_T_ZONE_APPEND is completed with a status
other than VIRTIO_BLK_S_OK, the value of \field{append_sector} field in
\field{virtio_blk_req_za} is undefined.

A VIRTIO_BLK_T_ZONE_APPEND request that has the data size that exceeds
\field{max_append_sectors} configuration space value, then,
\begin{itemize}
\item if \field{max_append_sectors} configuration space value is reported as
    zero by the device, the request SHALL be completed with VIRTIO_BLK_S_UNSUPP
    \field{status}.

\item if \field{max_append_sectors} configuration space value is reported as
    a non-zero value by the device, the request SHALL be completed with
    VIRTIO_BLK_S_ZONE_INVALID_CMD \field{status}.
\end{itemize}

If a VIRTIO_BLK_T_ZONE_APPEND request, a VIRTIO_BLK_T_IN request or a
VIRTIO_BLK_T_OUT request issued to a SWR zone has the range that has sectors in
more than one zone, then the request SHALL be completed with
VIRTIO_BLK_S_ZONE_INVALID_CMD value in the field \field{status}.

A VIRTIO_BLK_T_OUT request that has the \field{sector} value that is not aligned
with the write pointer for the zone, then the request SHALL be completed with
VIRTIO_BLK_S_ZONE_UNALIGNED_WP value in the field \field{status}.

In order to avoid resource-related errors while opening zones implicitly, the
device MAY automatically transition zones in VIRTIO_BLK_ZS_IOPEN state to
VIRTIO_BLK_ZS_CLOSED state.

All VIRTIO_BLK_T_OUT requests or VIRTIO_BLK_T_ZONE_APPEND requests issued
to a zone in the VIRTIO_BLK_ZS_RDONLY state SHALL be completed with
VIRTIO_BLK_S_ZONE_INVALID_CMD \field{status}.

All requests issued to a zone in the VIRTIO_BLK_ZS_OFFLINE state SHALL be
completed with VIRTIO_BLK_S_ZONE_INVALID_CMD value in the field \field{status}.

The device MUST consider the sectors that are read between the write pointer
position of a zone and the end of the last sector of the zone as unwritten data.
The sectors between the write pointer position and the end of the last sector
within the zone capacity during VIRTIO_BLK_T_ZONE_FINISH request processing are
also considered unwritten data.

When unwritten data is present in the sector range of a read request, the device
MUST process this data in one of the following ways -

\begin{enumerate}
\item Fill the unwritten data with a device-specific byte pattern. The
configuration, control and reporting of this byte pattern is beyond the scope
of this standard. This is the preferred approach.

\item Fail the request. Depending on the driver implementation, this may prevent
the device from becoming operational.
\end{enumerate}

If both the VIRTIO_BLK_F_ZONED and VIRTIO_BLK_F_SECURE_ERASE features are
negotiated, then

\begin{enumerate}
\item the field \field{secure_erase_sector_alignment} in the configuration space
of the device MUST be a multiple of \field{zone_sectors} value reported in the
device configuration space.

\item the data size in VIRTIO_BLK_T_SECURE_ERASE requests MUST be a multiple of
\field{zone_sectors} value in the device configuration space.
\end{enumerate}

The device MUST handle a VIRTIO_BLK_T_SECURE_ERASE request in the same way it
handles VIRTIO_BLK_T_ZONE_RESET request for the zone range specified in the
VIRTIO_BLK_T_SECURE_ERASE request.

\subsubsection{Legacy Interface: Device Operation}\label{sec:Device Types / Block Device / Device Operation / Legacy Interface: Device Operation}
When using the legacy interface, transitional devices and drivers
MUST format the fields in struct virtio_blk_req
according to the native endian of the guest rather than
(necessarily when not using the legacy interface) little-endian.

When using the legacy interface, transitional drivers
SHOULD ignore the used length values.
\begin{note}
Historically, some devices put the total descriptor length,
or the total length of device-writable buffers there,
even when only the status byte was actually written.
\end{note}

The \field{reserved} field was previously called \field{ioprio}.  \field{ioprio}
is a hint about the relative priorities of requests to the device:
higher numbers indicate more important requests.

\begin{lstlisting}
#define VIRTIO_BLK_T_FLUSH_OUT    5
\end{lstlisting}

The command VIRTIO_BLK_T_FLUSH_OUT was a synonym for VIRTIO_BLK_T_FLUSH;
a driver MUST treat it as a VIRTIO_BLK_T_FLUSH command.

\begin{lstlisting}
#define VIRTIO_BLK_T_BARRIER     0x80000000
\end{lstlisting}

If the device has VIRTIO_BLK_F_BARRIER
feature the high bit (VIRTIO_BLK_T_BARRIER) indicates that this
request acts as a barrier and that all preceding requests SHOULD be
complete before this one, and all following requests SHOULD NOT be
started until this is complete.

\begin{note} A barrier does not flush
caches in the underlying backend device in host, and thus does not
serve as data consistency guarantee.  Only a VIRTIO_BLK_T_FLUSH request
does that.
\end{note}

Some older legacy devices did not commit completed writes to persistent
device backend storage when VIRTIO_BLK_F_FLUSH was offered but not
negotiated.  In order to work around this, the driver MAY set the
\field{writeback} to 0 (if available) or it MAY send an explicit flush
request after every completed write.

If the device has VIRTIO_BLK_F_SCSI feature, it can also support
scsi packet command requests, each of these requests is of form:

\begin{lstlisting}
/* All fields are in guest's native endian. */
struct virtio_scsi_pc_req {
        u32 type;
        u32 ioprio;
        u64 sector;
        u8 cmd[];
        u8 data[][512];
#define SCSI_SENSE_BUFFERSIZE   96
        u8 sense[SCSI_SENSE_BUFFERSIZE];
        u32 errors;
        u32 data_len;
        u32 sense_len;
        u32 residual;
        u8 status;
};
\end{lstlisting}

A request type can also be a scsi packet command (VIRTIO_BLK_T_SCSI_CMD or
VIRTIO_BLK_T_SCSI_CMD_OUT).  The two types are equivalent, the device
does not distinguish between them:

\begin{lstlisting}
#define VIRTIO_BLK_T_SCSI_CMD     2
#define VIRTIO_BLK_T_SCSI_CMD_OUT 3
\end{lstlisting}

The \field{cmd} field is only present for scsi packet command requests,
and indicates the command to perform. This field MUST reside in a
single, separate device-readable buffer; command length can be derived
from the length of this buffer.

Note that these first three (four for scsi packet commands)
fields are always device-readable: \field{data} is either device-readable
or device-writable, depending on the request. The size of the read or
write can be derived from the total size of the request buffers.

\field{sense} is only present for scsi packet command requests,
and indicates the buffer for scsi sense data.

\field{data_len} is only present for scsi packet command
requests, this field is deprecated, and SHOULD be ignored by the
driver. Historically, devices copied data length there.

\field{sense_len} is only present for scsi packet command
requests and indicates the number of bytes actually written to
the \field{sense} buffer.

\field{residual} field is only present for scsi packet command
requests and indicates the residual size, calculated as data
length - number of bytes actually transferred.

\subsubsection{Legacy Interface: Framing Requirements}\label{sec:Device Types / Block Device / Legacy Interface: Framing Requirements}

When using legacy interfaces, transitional drivers which have not
negotiated VIRTIO_F_ANY_LAYOUT:

\begin{itemize}
\item MUST use a single 8-byte descriptor containing \field{type},
  \field{reserved} and \field{sector}, followed by descriptors
  for \field{data}, then finally a separate 1-byte descriptor
  for \field{status}.

\item For SCSI commands there are additional constraints.
  \field{sense} MUST reside in a
  single separate device-writable descriptor of size 96 bytes,
  and \field{errors}, \field{data_len}, \field{sense_len} and
  \field{residual} MUST reside a single separate
  device-writable descriptor.
\end{itemize}

See \ref{sec:Basic Facilities of a Virtio Device / Virtqueues / Message Framing}.

\section{GPU Device}\label{sec:Device Types / GPU Device}

virtio-gpu is a virtio based graphics adapter.  It can operate in 2D
mode and in 3D mode.  3D mode will offload rendering ops to
the host gpu and therefore requires a gpu with 3D support on the host
machine.

In 2D mode the virtio-gpu device provides support for ARGB Hardware
cursors and multiple scanouts (aka heads).

\subsection{Device ID}\label{sec:Device Types / GPU Device / Device ID}

16

\subsection{Virtqueues}\label{sec:Device Types / GPU Device / Virtqueues}

\begin{description}
\item[0] controlq - queue for sending control commands
\item[1] cursorq - queue for sending cursor updates
\end{description}

Both queues have the same format.  Each request and each response have
a fixed header, followed by command specific data fields.  The
separate cursor queue is the "fast track" for cursor commands
(VIRTIO_GPU_CMD_UPDATE_CURSOR and VIRTIO_GPU_CMD_MOVE_CURSOR), so they
go through without being delayed by time-consuming commands in the
control queue.

\subsection{Feature bits}\label{sec:Device Types / GPU Device / Feature bits}

\begin{description}
\item[VIRTIO_GPU_F_VIRGL (0)] virgl 3D mode is supported.
\item[VIRTIO_GPU_F_EDID  (1)] EDID is supported.
\item[VIRTIO_GPU_F_RESOURCE_UUID (2)] assigning resources UUIDs for export
  to other virtio devices is supported.
\item[VIRTIO_GPU_F_RESOURCE_BLOB (3)] creating and using size-based blob
  resources is supported.
\item[VIRTIO_GPU_F_CONTEXT_INIT (4)] multiple context types and
  synchronization timelines supported.  Requires VIRTIO_GPU_F_VIRGL.
\end{description}

\subsection{Device configuration layout}\label{sec:Device Types / GPU Device / Device configuration layout}

GPU device configuration uses the following layout structure and
definitions:

\begin{lstlisting}
#define VIRTIO_GPU_EVENT_DISPLAY (1 << 0)

struct virtio_gpu_config {
        le32 events_read;
        le32 events_clear;
        le32 num_scanouts;
        le32 num_capsets;
};
\end{lstlisting}

\subsubsection{Device configuration fields}

\begin{description}
\item[\field{events_read}] signals pending events to the driver.  The
  driver MUST NOT write to this field.
\item[\field{events_clear}] clears pending events in the device.
  Writing a '1' into a bit will clear the corresponding bit in
  \field{events_read}, mimicking write-to-clear behavior.
\item[\field{num_scanouts}] specifies the maximum number of scanouts
  supported by the device.  Minimum value is 1, maximum value is 16.
\item[\field{num_capsets}] specifies the maximum number of capability
  sets supported by the device.  The minimum value is zero.
\end{description}

\subsubsection{Events}

\begin{description}
\item[VIRTIO_GPU_EVENT_DISPLAY] Display configuration has changed.
  The driver SHOULD use the VIRTIO_GPU_CMD_GET_DISPLAY_INFO command to
  fetch the information from the device.  In case EDID support is
  negotiated (VIRTIO_GPU_F_EDID feature flag) the device SHOULD also
  fetch the updated EDID blobs using the VIRTIO_GPU_CMD_GET_EDID
  command.
\end{description}

\devicenormative{\subsection}{Device Initialization}{Device Types / GPU Device / Device Initialization}

The driver SHOULD query the display information from the device using
the VIRTIO_GPU_CMD_GET_DISPLAY_INFO command and use that information
for the initial scanout setup.  In case EDID support is negotiated
(VIRTIO_GPU_F_EDID feature flag) the device SHOULD also fetch the EDID
information using the VIRTIO_GPU_CMD_GET_EDID command.  If no
information is available or all displays are disabled the driver MAY
choose to use a fallback, such as 1024x768 at display 0.

The driver SHOULD query all shared memory regions supported by the device.
If the device supports shared memory, the \field{shmid} of a region MUST
(see \ref{sec:Basic Facilities of a Virtio Device /
Shared Memory Regions}~\nameref{sec:Basic Facilities of a Virtio Device /
Shared Memory Regions}) be one of the following:

\begin{lstlisting}
enum virtio_gpu_shm_id {
        VIRTIO_GPU_SHM_ID_UNDEFINED = 0,
        VIRTIO_GPU_SHM_ID_HOST_VISIBLE = 1,
};
\end{lstlisting}

The shared memory region with VIRTIO_GPU_SHM_ID_HOST_VISIBLE is referred as
the "host visible memory region".  The device MUST support the
VIRTIO_GPU_CMD_RESOURCE_MAP_BLOB and VIRTIO_GPU_CMD_RESOURCE_UNMAP_BLOB
if the host visible memory region is available.

\subsection{Device Operation}\label{sec:Device Types / GPU Device / Device Operation}

The virtio-gpu is based around the concept of resources private to the
host.  The guest must DMA transfer into these resources, unless shared memory
regions are supported. This is a design requirement in order to interface with
future 3D rendering. In the unaccelerated 2D mode there is no support for DMA
transfers from resources, just to them.

Resources are initially simple 2D resources, consisting of a width,
height and format along with an identifier. The guest must then attach
backing store to the resources in order for DMA transfers to
work. This is like a GART in a real GPU.

\subsubsection{Device Operation: Create a framebuffer and configure scanout}

\begin{itemize*}
\item Create a host resource using VIRTIO_GPU_CMD_RESOURCE_CREATE_2D.
\item Allocate a framebuffer from guest ram, and attach it as backing
  storage to the resource just created, using
  VIRTIO_GPU_CMD_RESOURCE_ATTACH_BACKING.  Scatter lists are
  supported, so the framebuffer doesn't need to be contignous in guest
  physical memory.
\item Use VIRTIO_GPU_CMD_SET_SCANOUT to link the framebuffer to
  a display scanout.
\end{itemize*}

\subsubsection{Device Operation: Update a framebuffer and scanout}

\begin{itemize*}
\item Render to your framebuffer memory.
\item Use VIRTIO_GPU_CMD_TRANSFER_TO_HOST_2D to update the host resource
  from guest memory.
\item Use VIRTIO_GPU_CMD_RESOURCE_FLUSH to flush the updated resource
  to the display.
\end{itemize*}

\subsubsection{Device Operation: Using pageflip}

It is possible to create multiple framebuffers, flip between them
using VIRTIO_GPU_CMD_SET_SCANOUT and VIRTIO_GPU_CMD_RESOURCE_FLUSH,
and update the invisible framebuffer using
VIRTIO_GPU_CMD_TRANSFER_TO_HOST_2D.

\subsubsection{Device Operation: Multihead setup}

In case two or more displays are present there are different ways to
configure things:

\begin{itemize*}
\item Create a single framebuffer, link it to all displays
  (mirroring).
\item Create an framebuffer for each display.
\item Create one big framebuffer, configure scanouts to display a
  different rectangle of that framebuffer each.
\end{itemize*}

\devicenormative{\subsubsection}{Device Operation: Command lifecycle and fencing}{Device Types / GPU Device / Device Operation / Device Operation: Command lifecycle and fencing}

The device MAY process controlq commands asyncronously and return them
to the driver before the processing is complete.  If the driver needs
to know when the processing is finished it can set the
VIRTIO_GPU_FLAG_FENCE flag in the request.  The device MUST finish the
processing before returning the command then.

Note: current qemu implementation does asyncrounous processing only in
3d mode, when offloading the processing to the host gpu.

\subsubsection{Device Operation: Configure mouse cursor}

The mouse cursor image is a normal resource, except that it must be
64x64 in size.  The driver MUST create and populate the resource
(using the usual VIRTIO_GPU_CMD_RESOURCE_CREATE_2D,
VIRTIO_GPU_CMD_RESOURCE_ATTACH_BACKING and
VIRTIO_GPU_CMD_TRANSFER_TO_HOST_2D controlq commands) and make sure they
are completed (using VIRTIO_GPU_FLAG_FENCE).

Then VIRTIO_GPU_CMD_UPDATE_CURSOR can be sent to the cursorq to set
the pointer shape and position.  To move the pointer without updating
the shape use VIRTIO_GPU_CMD_MOVE_CURSOR instead.

\subsubsection{Device Operation: Request header}\label{sec:Device Types / GPU Device / Device Operation / Device Operation: Request header}

All requests and responses on the virtqueues have a fixed header
using the following layout structure and definitions:

\begin{lstlisting}
enum virtio_gpu_ctrl_type {

        /* 2d commands */
        VIRTIO_GPU_CMD_GET_DISPLAY_INFO = 0x0100,
        VIRTIO_GPU_CMD_RESOURCE_CREATE_2D,
        VIRTIO_GPU_CMD_RESOURCE_UNREF,
        VIRTIO_GPU_CMD_SET_SCANOUT,
        VIRTIO_GPU_CMD_RESOURCE_FLUSH,
        VIRTIO_GPU_CMD_TRANSFER_TO_HOST_2D,
        VIRTIO_GPU_CMD_RESOURCE_ATTACH_BACKING,
        VIRTIO_GPU_CMD_RESOURCE_DETACH_BACKING,
        VIRTIO_GPU_CMD_GET_CAPSET_INFO,
        VIRTIO_GPU_CMD_GET_CAPSET,
        VIRTIO_GPU_CMD_GET_EDID,
        VIRTIO_GPU_CMD_RESOURCE_ASSIGN_UUID,
        VIRTIO_GPU_CMD_RESOURCE_CREATE_BLOB,
        VIRTIO_GPU_CMD_SET_SCANOUT_BLOB,

        /* 3d commands */
        VIRTIO_GPU_CMD_CTX_CREATE = 0x0200,
        VIRTIO_GPU_CMD_CTX_DESTROY,
        VIRTIO_GPU_CMD_CTX_ATTACH_RESOURCE,
        VIRTIO_GPU_CMD_CTX_DETACH_RESOURCE,
        VIRTIO_GPU_CMD_RESOURCE_CREATE_3D,
        VIRTIO_GPU_CMD_TRANSFER_TO_HOST_3D,
        VIRTIO_GPU_CMD_TRANSFER_FROM_HOST_3D,
        VIRTIO_GPU_CMD_SUBMIT_3D,
        VIRTIO_GPU_CMD_RESOURCE_MAP_BLOB,
        VIRTIO_GPU_CMD_RESOURCE_UNMAP_BLOB,

        /* cursor commands */
        VIRTIO_GPU_CMD_UPDATE_CURSOR = 0x0300,
        VIRTIO_GPU_CMD_MOVE_CURSOR,

        /* success responses */
        VIRTIO_GPU_RESP_OK_NODATA = 0x1100,
        VIRTIO_GPU_RESP_OK_DISPLAY_INFO,
        VIRTIO_GPU_RESP_OK_CAPSET_INFO,
        VIRTIO_GPU_RESP_OK_CAPSET,
        VIRTIO_GPU_RESP_OK_EDID,
        VIRTIO_GPU_RESP_OK_RESOURCE_UUID,
        VIRTIO_GPU_RESP_OK_MAP_INFO,

        /* error responses */
        VIRTIO_GPU_RESP_ERR_UNSPEC = 0x1200,
        VIRTIO_GPU_RESP_ERR_OUT_OF_MEMORY,
        VIRTIO_GPU_RESP_ERR_INVALID_SCANOUT_ID,
        VIRTIO_GPU_RESP_ERR_INVALID_RESOURCE_ID,
        VIRTIO_GPU_RESP_ERR_INVALID_CONTEXT_ID,
        VIRTIO_GPU_RESP_ERR_INVALID_PARAMETER,
};

#define VIRTIO_GPU_FLAG_FENCE (1 << 0)
#define VIRTIO_GPU_FLAG_INFO_RING_IDX (1 << 1)

struct virtio_gpu_ctrl_hdr {
        le32 type;
        le32 flags;
        le64 fence_id;
        le32 ctx_id;
        u8 ring_idx;
        u8 padding[3];
};
\end{lstlisting}

The fixed header \field{struct virtio_gpu_ctrl_hdr} in each
request includes the following fields:

\begin{description}
\item[\field{type}] specifies the type of the driver request
  (VIRTIO_GPU_CMD_*) or device response (VIRTIO_GPU_RESP_*).
\item[\field{flags}] request / response flags.
\item[\field{fence_id}] If the driver sets the VIRTIO_GPU_FLAG_FENCE
  bit in the request \field{flags} field the device MUST:
  \begin{itemize*}
  \item set VIRTIO_GPU_FLAG_FENCE bit in the response,
  \item copy the content of the \field{fence_id} field from the
    request to the response, and
  \item send the response only after command processing is complete.
  \end{itemize*}
\item[\field{ctx_id}] Rendering context (used in 3D mode only).
\item[\field{ring_idx}] If VIRTIO_GPU_F_CONTEXT_INIT is supported, then
  the driver MAY set VIRTIO_GPU_FLAG_INFO_RING_IDX bit in the request
  \field{flags}.  In that case:
  \begin{itemize*}
  \item \field{ring_idx} indicates the value of a context-specific ring
   index.  The minimum value is 0 and maximum value is 63 (inclusive).
  \item If VIRTIO_GPU_FLAG_FENCE is set, \field{fence_id} acts as a
   sequence number on the synchronization timeline defined by
   \field{ctx_idx} and the ring index.
  \item If VIRTIO_GPU_FLAG_FENCE is set and when the command associated
   with \field{fence_id} is complete, the device MUST send a response for
   all outstanding commands with a sequence number less than or equal to
   \field{fence_id} on the same synchronization timeline.
  \end{itemize*}
\end{description}

On success the device will return VIRTIO_GPU_RESP_OK_NODATA in
case there is no payload.  Otherwise the \field{type} field will
indicate the kind of payload.

On error the device will return one of the
VIRTIO_GPU_RESP_ERR_* error codes.

\subsubsection{Device Operation: controlq}\label{sec:Device Types / GPU Device / Device Operation / Device Operation: controlq}

For any coordinates given 0,0 is top left, larger x moves right,
larger y moves down.

\begin{description}

\item[VIRTIO_GPU_CMD_GET_DISPLAY_INFO] Retrieve the current output
  configuration.  No request data (just bare \field{struct
    virtio_gpu_ctrl_hdr}).  Response type is
  VIRTIO_GPU_RESP_OK_DISPLAY_INFO, response data is \field{struct
    virtio_gpu_resp_display_info}.

\begin{lstlisting}
#define VIRTIO_GPU_MAX_SCANOUTS 16

struct virtio_gpu_rect {
        le32 x;
        le32 y;
        le32 width;
        le32 height;
};

struct virtio_gpu_resp_display_info {
        struct virtio_gpu_ctrl_hdr hdr;
        struct virtio_gpu_display_one {
                struct virtio_gpu_rect r;
                le32 enabled;
                le32 flags;
        } pmodes[VIRTIO_GPU_MAX_SCANOUTS];
};
\end{lstlisting}

The response contains a list of per-scanout information.  The info
contains whether the scanout is enabled and what its preferred
position and size is.

The size (fields \field{width} and \field{height}) is similar to the
native panel resolution in EDID display information, except that in
the virtual machine case the size can change when the host window
representing the guest display is gets resized.

The position (fields \field{x} and \field{y}) describe how the
displays are arranged (i.e. which is -- for example -- the left
display).

The \field{enabled} field is set when the user enabled the display.
It is roughly the same as the connected state of a phyiscal display
connector.

\item[VIRTIO_GPU_CMD_GET_EDID] Retrieve the EDID data for a given
  scanout.  Request data is \field{struct virtio_gpu_get_edid}).
  Response type is VIRTIO_GPU_RESP_OK_EDID, response data is
  \field{struct virtio_gpu_resp_edid}.  Support is optional and
  negotiated using the VIRTIO_GPU_F_EDID feature flag.

\begin{lstlisting}
struct virtio_gpu_get_edid {
        struct virtio_gpu_ctrl_hdr hdr;
        le32 scanout;
        le32 padding;
};

struct virtio_gpu_resp_edid {
        struct virtio_gpu_ctrl_hdr hdr;
        le32 size;
        le32 padding;
        u8 edid[1024];
};
\end{lstlisting}

The response contains the EDID display data blob (as specified by
VESA) for the scanout.

\item[VIRTIO_GPU_CMD_RESOURCE_CREATE_2D] Create a 2D resource on the
  host.  Request data is \field{struct virtio_gpu_resource_create_2d}.
  Response type is VIRTIO_GPU_RESP_OK_NODATA.

\begin{lstlisting}
enum virtio_gpu_formats {
        VIRTIO_GPU_FORMAT_B8G8R8A8_UNORM  = 1,
        VIRTIO_GPU_FORMAT_B8G8R8X8_UNORM  = 2,
        VIRTIO_GPU_FORMAT_A8R8G8B8_UNORM  = 3,
        VIRTIO_GPU_FORMAT_X8R8G8B8_UNORM  = 4,

        VIRTIO_GPU_FORMAT_R8G8B8A8_UNORM  = 67,
        VIRTIO_GPU_FORMAT_X8B8G8R8_UNORM  = 68,

        VIRTIO_GPU_FORMAT_A8B8G8R8_UNORM  = 121,
        VIRTIO_GPU_FORMAT_R8G8B8X8_UNORM  = 134,
};

struct virtio_gpu_resource_create_2d {
        struct virtio_gpu_ctrl_hdr hdr;
        le32 resource_id;
        le32 format;
        le32 width;
        le32 height;
};
\end{lstlisting}

This creates a 2D resource on the host with the specified width,
height and format.  The resource ids are generated by the guest.

\item[VIRTIO_GPU_CMD_RESOURCE_UNREF] Destroy a resource.  Request data
  is \field{struct virtio_gpu_resource_unref}.  Response type is
  VIRTIO_GPU_RESP_OK_NODATA.

\begin{lstlisting}
struct virtio_gpu_resource_unref {
        struct virtio_gpu_ctrl_hdr hdr;
        le32 resource_id;
        le32 padding;
};
\end{lstlisting}

This informs the host that a resource is no longer required by the
guest.

\item[VIRTIO_GPU_CMD_SET_SCANOUT] Set the scanout parameters for a
  single output.  Request data is \field{struct
    virtio_gpu_set_scanout}.  Response type is
  VIRTIO_GPU_RESP_OK_NODATA.

\begin{lstlisting}
struct virtio_gpu_set_scanout {
        struct virtio_gpu_ctrl_hdr hdr;
        struct virtio_gpu_rect r;
        le32 scanout_id;
        le32 resource_id;
};
\end{lstlisting}

This sets the scanout parameters for a single scanout. The resource_id
is the resource to be scanned out from, along with a rectangle.

Scanout rectangles must be completely covered by the underlying
resource.  Overlapping (or identical) scanouts are allowed, typical
use case is screen mirroring.

The driver can use resource_id = 0 to disable a scanout.

\item[VIRTIO_GPU_CMD_RESOURCE_FLUSH] Flush a scanout resource Request
  data is \field{struct virtio_gpu_resource_flush}.  Response type is
  VIRTIO_GPU_RESP_OK_NODATA.

\begin{lstlisting}
struct virtio_gpu_resource_flush {
        struct virtio_gpu_ctrl_hdr hdr;
        struct virtio_gpu_rect r;
        le32 resource_id;
        le32 padding;
};
\end{lstlisting}

This flushes a resource to screen.  It takes a rectangle and a
resource id, and flushes any scanouts the resource is being used on.

\item[VIRTIO_GPU_CMD_TRANSFER_TO_HOST_2D] Transfer from guest memory
  to host resource.  Request data is \field{struct
    virtio_gpu_transfer_to_host_2d}.  Response type is
  VIRTIO_GPU_RESP_OK_NODATA.

\begin{lstlisting}
struct virtio_gpu_transfer_to_host_2d {
        struct virtio_gpu_ctrl_hdr hdr;
        struct virtio_gpu_rect r;
        le64 offset;
        le32 resource_id;
        le32 padding;
};
\end{lstlisting}

This takes a resource id along with an destination offset into the
resource, and a box to transfer to the host backing for the resource.

\item[VIRTIO_GPU_CMD_RESOURCE_ATTACH_BACKING] Assign backing pages to
  a resource.  Request data is \field{struct
    virtio_gpu_resource_attach_backing}, followed by \field{struct
    virtio_gpu_mem_entry} entries.  Response type is
  VIRTIO_GPU_RESP_OK_NODATA.

\begin{lstlisting}
struct virtio_gpu_resource_attach_backing {
        struct virtio_gpu_ctrl_hdr hdr;
        le32 resource_id;
        le32 nr_entries;
};

struct virtio_gpu_mem_entry {
        le64 addr;
        le32 length;
        le32 padding;
};
\end{lstlisting}

This assign an array of guest pages as the backing store for a
resource. These pages are then used for the transfer operations for
that resource from that point on.

\item[VIRTIO_GPU_CMD_RESOURCE_DETACH_BACKING] Detach backing pages
  from a resource.  Request data is \field{struct
    virtio_gpu_resource_detach_backing}.  Response type is
  VIRTIO_GPU_RESP_OK_NODATA.

\begin{lstlisting}
struct virtio_gpu_resource_detach_backing {
        struct virtio_gpu_ctrl_hdr hdr;
        le32 resource_id;
        le32 padding;
};
\end{lstlisting}

This detaches any backing pages from a resource, to be used in case of
guest swapping or object destruction.

\item[VIRTIO_GPU_CMD_GET_CAPSET_INFO] Gets the information associated with
  a particular \field{capset_index}, which MUST less than \field{num_capsets}
  defined in the device configuration.  Request data is
  \field{struct virtio_gpu_get_capset_info}.  Response type is
  VIRTIO_GPU_RESP_OK_CAPSET_INFO.

  On success, \field{struct virtio_gpu_resp_capset_info} contains the
  \field{capset_id}, \field{capset_max_version}, \field{capset_max_size}
  associated with capset at the specified {capset_idex}.  field{capset_id} MUST
  be one of the following (see listing for values):

  \begin{itemize*}
  \item \href{https://gitlab.freedesktop.org/virgl/virglrenderer/-/blob/master/src/virgl_hw.h#L526}{VIRTIO_GPU_CAPSET_VIRGL} --
	the first edition of Virgl (Gallium OpenGL) protocol.
  \item \href{https://gitlab.freedesktop.org/virgl/virglrenderer/-/blob/master/src/virgl_hw.h#L550}{VIRTIO_GPU_CAPSET_VIRGL2} --
	the second edition of Virgl (Gallium OpenGL) protocol after the capset fix.
  \item \href{https://android.googlesource.com/device/generic/vulkan-cereal/+/refs/heads/master/protocols/}{VIRTIO_GPU_CAPSET_GFXSTREAM} --
	gfxtream's (mostly) autogenerated GLES and Vulkan streaming protocols.
  \item \href{https://gitlab.freedesktop.org/olv/venus-protocol}{VIRTIO_GPU_CAPSET_VENUS} --
	Mesa's (mostly) autogenerated Vulkan protocol.
  \item \href{https://chromium.googlesource.com/chromiumos/platform/crosvm/+/refs/heads/main/rutabaga_gfx/src/cross_domain/cross_domain_protocol.rs}{VIRTIO_GPU_CAPSET_CROSS_DOMAIN} --
	protocol for display virtualization via Wayland proxying.
  \end{itemize*}

\begin{lstlisting}
struct virtio_gpu_get_capset_info {
        struct virtio_gpu_ctrl_hdr hdr;
        le32 capset_index;
        le32 padding;
};

#define VIRTIO_GPU_CAPSET_VIRGL 1
#define VIRTIO_GPU_CAPSET_VIRGL2 2
#define VIRTIO_GPU_CAPSET_GFXSTREAM 3
#define VIRTIO_GPU_CAPSET_VENUS 4
#define VIRTIO_GPU_CAPSET_CROSS_DOMAIN 5
struct virtio_gpu_resp_capset_info {
        struct virtio_gpu_ctrl_hdr hdr;
        le32 capset_id;
        le32 capset_max_version;
        le32 capset_max_size;
        le32 padding;
};
\end{lstlisting}

\item[VIRTIO_GPU_CMD_GET_CAPSET] Gets the capset associated with a
  particular \field{capset_id} and \field{capset_version}.  Request data is
  \field{struct virtio_gpu_get_capset}.  Response type is
  VIRTIO_GPU_RESP_OK_CAPSET.

\begin{lstlisting}
struct virtio_gpu_get_capset {
        struct virtio_gpu_ctrl_hdr hdr;
        le32 capset_id;
        le32 capset_version;
};

struct virtio_gpu_resp_capset {
        struct virtio_gpu_ctrl_hdr hdr;
        u8 capset_data[];
};
\end{lstlisting}

\item[VIRTIO_GPU_CMD_RESOURCE_ASSIGN_UUID] Creates an exported object from
  a resource. Request data is \field{struct
    virtio_gpu_resource_assign_uuid}.  Response type is
  VIRTIO_GPU_RESP_OK_RESOURCE_UUID, response data is \field{struct
    virtio_gpu_resp_resource_uuid}. Support is optional and negotiated
    using the VIRTIO_GPU_F_RESOURCE_UUID feature flag.

\begin{lstlisting}
struct virtio_gpu_resource_assign_uuid {
        struct virtio_gpu_ctrl_hdr hdr;
        le32 resource_id;
        le32 padding;
};

struct virtio_gpu_resp_resource_uuid {
        struct virtio_gpu_ctrl_hdr hdr;
        u8 uuid[16];
};
\end{lstlisting}

The response contains a UUID which identifies the exported object created from
the host private resource. Note that if the resource has an attached backing,
modifications made to the host private resource through the exported object by
other devices are not visible in the attached backing until they are transferred
into the backing.

\item[VIRTIO_GPU_CMD_RESOURCE_CREATE_BLOB] Creates a virtio-gpu blob
  resource. Request data is \field{struct
  virtio_gpu_resource_create_blob}, followed by \field{struct
  virtio_gpu_mem_entry} entries. Response type is
  VIRTIO_GPU_RESP_OK_NODATA. Support is optional and negotiated
  using the VIRTIO_GPU_F_RESOURCE_BLOB feature flag.

\begin{lstlisting}
#define VIRTIO_GPU_BLOB_MEM_GUEST             0x0001
#define VIRTIO_GPU_BLOB_MEM_HOST3D            0x0002
#define VIRTIO_GPU_BLOB_MEM_HOST3D_GUEST      0x0003

#define VIRTIO_GPU_BLOB_FLAG_USE_MAPPABLE     0x0001
#define VIRTIO_GPU_BLOB_FLAG_USE_SHAREABLE    0x0002
#define VIRTIO_GPU_BLOB_FLAG_USE_CROSS_DEVICE 0x0004

struct virtio_gpu_resource_create_blob {
       struct virtio_gpu_ctrl_hdr hdr;
       le32 resource_id;
       le32 blob_mem;
       le32 blob_flags;
       le32 nr_entries;
       le64 blob_id;
       le64 size;
};

\end{lstlisting}

A blob resource is a container for:

  \begin{itemize*}
  \item a guest memory allocation (referred to as a
  "guest-only blob resource").
  \item a host memory allocation (referred to as a
  "host-only blob resource").
  \item a guest memory and host memory allocation (referred
  to as a "default blob resource").
  \end{itemize*}

The memory properties of the blob resource MUST be described by
\field{blob_mem}, which MUST be non-zero.

For default and guest-only blob resources, \field{nr_entries} guest
memory entries may be assigned to the resource.  For default blob resources
(i.e, when \field{blob_mem} is VIRTIO_GPU_BLOB_MEM_HOST3D_GUEST), these
memory entries are used as a shadow buffer for the host memory. To
facilitate drivers that support swap-in and swap-out, \field{nr_entries} may
be zero and VIRTIO_GPU_CMD_RESOURCE_ATTACH_BACKING may be subsequently used.
VIRTIO_GPU_CMD_RESOURCE_DETACH_BACKING may be used to unassign memory entries.

\field{blob_mem} can only be VIRTIO_GPU_BLOB_MEM_HOST3D and
VIRTIO_GPU_BLOB_MEM_HOST3D_GUEST if VIRTIO_GPU_F_VIRGL is supported.
VIRTIO_GPU_BLOB_MEM_GUEST is valid regardless whether VIRTIO_GPU_F_VIRGL
is supported or not.

For VIRTIO_GPU_BLOB_MEM_HOST3D and VIRTIO_GPU_BLOB_MEM_HOST3D_GUEST, the
virtio-gpu resource MUST be created from the rendering context local object
identified by the \field{blob_id}. The actual allocation is done via
VIRTIO_GPU_CMD_SUBMIT_3D.

The driver MUST inform the device if the blob resource is used for
memory access, sharing between driver instances and/or sharing with
other devices. This is done via the \field{blob_flags} field.

If VIRTIO_GPU_F_VIRGL is set, both VIRTIO_GPU_CMD_TRANSFER_TO_HOST_3D
and VIRTIO_GPU_CMD_TRANSFER_FROM_HOST_3D may be used to update the
resource. There is no restriction on the image/buffer view the driver
has on the blob resource.

\item[VIRTIO_GPU_CMD_SET_SCANOUT_BLOB] sets scanout parameters for a
   blob resource. Request data is
  \field{struct virtio_gpu_set_scanout_blob}. Response type is
  VIRTIO_GPU_RESP_OK_NODATA. Support is optional and negotiated
  using the VIRTIO_GPU_F_RESOURCE_BLOB feature flag.

\begin{lstlisting}
struct virtio_gpu_set_scanout_blob {
       struct virtio_gpu_ctrl_hdr hdr;
       struct virtio_gpu_rect r;
       le32 scanout_id;
       le32 resource_id;
       le32 width;
       le32 height;
       le32 format;
       le32 padding;
       le32 strides[4];
       le32 offsets[4];
};
\end{lstlisting}

The rectangle \field{r} represents the portion of the blob resource being
displayed. The rest is the metadata associated with the blob resource. The
format MUST be one of \field{enum virtio_gpu_formats}.  The format MAY be
compressed with header and data planes.

\end{description}

\subsubsection{Device Operation: controlq (3d)}\label{sec:Device Types / GPU Device / Device Operation / Device Operation: controlq (3d)}

These commands are supported by the device if the VIRTIO_GPU_F_VIRGL
feature flag is set.

\begin{description}

\item[VIRTIO_GPU_CMD_CTX_CREATE] creates a context for submitting an opaque
  command stream.  Request data is \field{struct virtio_gpu_ctx_create}.
  Response type is VIRTIO_GPU_RESP_OK_NODATA.

\begin{lstlisting}
#define VIRTIO_GPU_CONTEXT_INIT_CAPSET_ID_MASK 0x000000ff;
struct virtio_gpu_ctx_create {
       struct virtio_gpu_ctrl_hdr hdr;
       le32 nlen;
       le32 context_init;
       char debug_name[64];
};
\end{lstlisting}

The implementation MUST create a context for the given \field{ctx_id} in
the \field{hdr}.  For debugging purposes, a \field{debug_name} and it's
length \field{nlen} is provided by the driver.  If
VIRTIO_GPU_F_CONTEXT_INIT is supported, then lower 8 bits of
\field{context_init} MAY contain the \field{capset_id} associated with
context.  In that case, then the device MUST create a context that can
handle the specified command stream.

If the lower 8-bits of the \field{context_init} are zero, then the type of
the context is determined by the device.

\item[VIRTIO_GPU_CMD_CTX_DESTROY]
\item[VIRTIO_GPU_CMD_CTX_ATTACH_RESOURCE]
\item[VIRTIO_GPU_CMD_CTX_DETACH_RESOURCE]
  Manage virtio-gpu 3d contexts.

\item[VIRTIO_GPU_CMD_RESOURCE_CREATE_3D]
  Create virtio-gpu 3d resources.

\item[VIRTIO_GPU_CMD_TRANSFER_TO_HOST_3D]
\item[VIRTIO_GPU_CMD_TRANSFER_FROM_HOST_3D]
  Transfer data from and to virtio-gpu 3d resources.

\item[VIRTIO_GPU_CMD_SUBMIT_3D]
  Submit an opaque command stream.  The type of the command stream is
  determined when creating a context.

\item[VIRTIO_GPU_CMD_RESOURCE_MAP_BLOB] maps a host-only
  blob resource into an offset in the host visible memory region. Request
  data is \field{struct virtio_gpu_resource_map_blob}.  The driver MUST
  not map a blob resource that is already mapped.  Response type is
  VIRTIO_GPU_RESP_OK_MAP_INFO. Support is optional and negotiated
  using the VIRTIO_GPU_F_RESOURCE_BLOB feature flag and checking for
  the presence of the host visible memory region.

\begin{lstlisting}
struct virtio_gpu_resource_map_blob {
        struct virtio_gpu_ctrl_hdr hdr;
        le32 resource_id;
        le32 padding;
        le64 offset;
};

#define VIRTIO_GPU_MAP_CACHE_MASK      0x0f
#define VIRTIO_GPU_MAP_CACHE_NONE      0x00
#define VIRTIO_GPU_MAP_CACHE_CACHED    0x01
#define VIRTIO_GPU_MAP_CACHE_UNCACHED  0x02
#define VIRTIO_GPU_MAP_CACHE_WC        0x03
struct virtio_gpu_resp_map_info {
        struct virtio_gpu_ctrl_hdr hdr;
        u32 map_info;
        u32 padding;
};
\end{lstlisting}

\item[VIRTIO_GPU_CMD_RESOURCE_UNMAP_BLOB] unmaps a
  host-only blob resource from the host visible memory region. Request data
  is \field{struct virtio_gpu_resource_unmap_blob}.  Response type is
  VIRTIO_GPU_RESP_OK_NODATA.  Support is optional and negotiated
  using the VIRTIO_GPU_F_RESOURCE_BLOB feature flag and checking for
  the presence of the host visible memory region.

\begin{lstlisting}
struct virtio_gpu_resource_unmap_blob {
        struct virtio_gpu_ctrl_hdr hdr;
        le32 resource_id;
        le32 padding;
};
\end{lstlisting}

\end{description}

\subsubsection{Device Operation: cursorq}\label{sec:Device Types / GPU Device / Device Operation / Device Operation: cursorq}

Both cursorq commands use the same command struct.

\begin{lstlisting}
struct virtio_gpu_cursor_pos {
        le32 scanout_id;
        le32 x;
        le32 y;
        le32 padding;
};

struct virtio_gpu_update_cursor {
        struct virtio_gpu_ctrl_hdr hdr;
        struct virtio_gpu_cursor_pos pos;
        le32 resource_id;
        le32 hot_x;
        le32 hot_y;
        le32 padding;
};
\end{lstlisting}

\begin{description}

\item[VIRTIO_GPU_CMD_UPDATE_CURSOR]
Update cursor.
Request data is \field{struct virtio_gpu_update_cursor}.
Response type is VIRTIO_GPU_RESP_OK_NODATA.

Full cursor update.  Cursor will be loaded from the specified
\field{resource_id} and will be moved to \field{pos}.  The driver must
transfer the cursor into the resource beforehand (using control queue
commands) and make sure the commands to fill the resource are actually
processed (using fencing).

\item[VIRTIO_GPU_CMD_MOVE_CURSOR]
Move cursor.
Request data is \field{struct virtio_gpu_update_cursor}.
Response type is VIRTIO_GPU_RESP_OK_NODATA.

Move cursor to the place specified in \field{pos}.  The other fields
are not used and will be ignored by the device.

\end{description}

\subsection{VGA Compatibility}\label{sec:Device Types / GPU Device / VGA Compatibility}

Applies to Virtio Over PCI only.  The GPU device can come with and
without VGA compatibility.  The PCI class should be DISPLAY_VGA if VGA
compatibility is present and DISPLAY_OTHER otherwise.

VGA compatibility: PCI region 0 has the linear framebuffer, standard
vga registers are present.  Configuring a scanout
(VIRTIO_GPU_CMD_SET_SCANOUT) switches the device from vga
compatibility mode into native virtio mode.  A reset switches it back
into vga compatibility mode.

Note: qemu implementation also provides bochs dispi interface io ports
and mmio bar at pci region 1 and is therefore fully compatible with
the qemu stdvga (see \href{https://git.qemu-project.org/?p=qemu.git;a=blob;f=docs/specs/standard-vga.txt;hb=HEAD}{docs/specs/standard-vga.txt} in the qemu source tree).

\section{GPU Device}\label{sec:Device Types / GPU Device}

virtio-gpu is a virtio based graphics adapter.  It can operate in 2D
mode and in 3D mode.  3D mode will offload rendering ops to
the host gpu and therefore requires a gpu with 3D support on the host
machine.

In 2D mode the virtio-gpu device provides support for ARGB Hardware
cursors and multiple scanouts (aka heads).

\subsection{Device ID}\label{sec:Device Types / GPU Device / Device ID}

16

\subsection{Virtqueues}\label{sec:Device Types / GPU Device / Virtqueues}

\begin{description}
\item[0] controlq - queue for sending control commands
\item[1] cursorq - queue for sending cursor updates
\end{description}

Both queues have the same format.  Each request and each response have
a fixed header, followed by command specific data fields.  The
separate cursor queue is the "fast track" for cursor commands
(VIRTIO_GPU_CMD_UPDATE_CURSOR and VIRTIO_GPU_CMD_MOVE_CURSOR), so they
go through without being delayed by time-consuming commands in the
control queue.

\subsection{Feature bits}\label{sec:Device Types / GPU Device / Feature bits}

\begin{description}
\item[VIRTIO_GPU_F_VIRGL (0)] virgl 3D mode is supported.
\item[VIRTIO_GPU_F_EDID  (1)] EDID is supported.
\item[VIRTIO_GPU_F_RESOURCE_UUID (2)] assigning resources UUIDs for export
  to other virtio devices is supported.
\item[VIRTIO_GPU_F_RESOURCE_BLOB (3)] creating and using size-based blob
  resources is supported.
\item[VIRTIO_GPU_F_CONTEXT_INIT (4)] multiple context types and
  synchronization timelines supported.  Requires VIRTIO_GPU_F_VIRGL.
\end{description}

\subsection{Device configuration layout}\label{sec:Device Types / GPU Device / Device configuration layout}

GPU device configuration uses the following layout structure and
definitions:

\begin{lstlisting}
#define VIRTIO_GPU_EVENT_DISPLAY (1 << 0)

struct virtio_gpu_config {
        le32 events_read;
        le32 events_clear;
        le32 num_scanouts;
        le32 num_capsets;
};
\end{lstlisting}

\subsubsection{Device configuration fields}

\begin{description}
\item[\field{events_read}] signals pending events to the driver.  The
  driver MUST NOT write to this field.
\item[\field{events_clear}] clears pending events in the device.
  Writing a '1' into a bit will clear the corresponding bit in
  \field{events_read}, mimicking write-to-clear behavior.
\item[\field{num_scanouts}] specifies the maximum number of scanouts
  supported by the device.  Minimum value is 1, maximum value is 16.
\item[\field{num_capsets}] specifies the maximum number of capability
  sets supported by the device.  The minimum value is zero.
\end{description}

\subsubsection{Events}

\begin{description}
\item[VIRTIO_GPU_EVENT_DISPLAY] Display configuration has changed.
  The driver SHOULD use the VIRTIO_GPU_CMD_GET_DISPLAY_INFO command to
  fetch the information from the device.  In case EDID support is
  negotiated (VIRTIO_GPU_F_EDID feature flag) the device SHOULD also
  fetch the updated EDID blobs using the VIRTIO_GPU_CMD_GET_EDID
  command.
\end{description}

\devicenormative{\subsection}{Device Initialization}{Device Types / GPU Device / Device Initialization}

The driver SHOULD query the display information from the device using
the VIRTIO_GPU_CMD_GET_DISPLAY_INFO command and use that information
for the initial scanout setup.  In case EDID support is negotiated
(VIRTIO_GPU_F_EDID feature flag) the device SHOULD also fetch the EDID
information using the VIRTIO_GPU_CMD_GET_EDID command.  If no
information is available or all displays are disabled the driver MAY
choose to use a fallback, such as 1024x768 at display 0.

The driver SHOULD query all shared memory regions supported by the device.
If the device supports shared memory, the \field{shmid} of a region MUST
(see \ref{sec:Basic Facilities of a Virtio Device /
Shared Memory Regions}~\nameref{sec:Basic Facilities of a Virtio Device /
Shared Memory Regions}) be one of the following:

\begin{lstlisting}
enum virtio_gpu_shm_id {
        VIRTIO_GPU_SHM_ID_UNDEFINED = 0,
        VIRTIO_GPU_SHM_ID_HOST_VISIBLE = 1,
};
\end{lstlisting}

The shared memory region with VIRTIO_GPU_SHM_ID_HOST_VISIBLE is referred as
the "host visible memory region".  The device MUST support the
VIRTIO_GPU_CMD_RESOURCE_MAP_BLOB and VIRTIO_GPU_CMD_RESOURCE_UNMAP_BLOB
if the host visible memory region is available.

\subsection{Device Operation}\label{sec:Device Types / GPU Device / Device Operation}

The virtio-gpu is based around the concept of resources private to the
host.  The guest must DMA transfer into these resources, unless shared memory
regions are supported. This is a design requirement in order to interface with
future 3D rendering. In the unaccelerated 2D mode there is no support for DMA
transfers from resources, just to them.

Resources are initially simple 2D resources, consisting of a width,
height and format along with an identifier. The guest must then attach
backing store to the resources in order for DMA transfers to
work. This is like a GART in a real GPU.

\subsubsection{Device Operation: Create a framebuffer and configure scanout}

\begin{itemize*}
\item Create a host resource using VIRTIO_GPU_CMD_RESOURCE_CREATE_2D.
\item Allocate a framebuffer from guest ram, and attach it as backing
  storage to the resource just created, using
  VIRTIO_GPU_CMD_RESOURCE_ATTACH_BACKING.  Scatter lists are
  supported, so the framebuffer doesn't need to be contignous in guest
  physical memory.
\item Use VIRTIO_GPU_CMD_SET_SCANOUT to link the framebuffer to
  a display scanout.
\end{itemize*}

\subsubsection{Device Operation: Update a framebuffer and scanout}

\begin{itemize*}
\item Render to your framebuffer memory.
\item Use VIRTIO_GPU_CMD_TRANSFER_TO_HOST_2D to update the host resource
  from guest memory.
\item Use VIRTIO_GPU_CMD_RESOURCE_FLUSH to flush the updated resource
  to the display.
\end{itemize*}

\subsubsection{Device Operation: Using pageflip}

It is possible to create multiple framebuffers, flip between them
using VIRTIO_GPU_CMD_SET_SCANOUT and VIRTIO_GPU_CMD_RESOURCE_FLUSH,
and update the invisible framebuffer using
VIRTIO_GPU_CMD_TRANSFER_TO_HOST_2D.

\subsubsection{Device Operation: Multihead setup}

In case two or more displays are present there are different ways to
configure things:

\begin{itemize*}
\item Create a single framebuffer, link it to all displays
  (mirroring).
\item Create an framebuffer for each display.
\item Create one big framebuffer, configure scanouts to display a
  different rectangle of that framebuffer each.
\end{itemize*}

\devicenormative{\subsubsection}{Device Operation: Command lifecycle and fencing}{Device Types / GPU Device / Device Operation / Device Operation: Command lifecycle and fencing}

The device MAY process controlq commands asyncronously and return them
to the driver before the processing is complete.  If the driver needs
to know when the processing is finished it can set the
VIRTIO_GPU_FLAG_FENCE flag in the request.  The device MUST finish the
processing before returning the command then.

Note: current qemu implementation does asyncrounous processing only in
3d mode, when offloading the processing to the host gpu.

\subsubsection{Device Operation: Configure mouse cursor}

The mouse cursor image is a normal resource, except that it must be
64x64 in size.  The driver MUST create and populate the resource
(using the usual VIRTIO_GPU_CMD_RESOURCE_CREATE_2D,
VIRTIO_GPU_CMD_RESOURCE_ATTACH_BACKING and
VIRTIO_GPU_CMD_TRANSFER_TO_HOST_2D controlq commands) and make sure they
are completed (using VIRTIO_GPU_FLAG_FENCE).

Then VIRTIO_GPU_CMD_UPDATE_CURSOR can be sent to the cursorq to set
the pointer shape and position.  To move the pointer without updating
the shape use VIRTIO_GPU_CMD_MOVE_CURSOR instead.

\subsubsection{Device Operation: Request header}\label{sec:Device Types / GPU Device / Device Operation / Device Operation: Request header}

All requests and responses on the virtqueues have a fixed header
using the following layout structure and definitions:

\begin{lstlisting}
enum virtio_gpu_ctrl_type {

        /* 2d commands */
        VIRTIO_GPU_CMD_GET_DISPLAY_INFO = 0x0100,
        VIRTIO_GPU_CMD_RESOURCE_CREATE_2D,
        VIRTIO_GPU_CMD_RESOURCE_UNREF,
        VIRTIO_GPU_CMD_SET_SCANOUT,
        VIRTIO_GPU_CMD_RESOURCE_FLUSH,
        VIRTIO_GPU_CMD_TRANSFER_TO_HOST_2D,
        VIRTIO_GPU_CMD_RESOURCE_ATTACH_BACKING,
        VIRTIO_GPU_CMD_RESOURCE_DETACH_BACKING,
        VIRTIO_GPU_CMD_GET_CAPSET_INFO,
        VIRTIO_GPU_CMD_GET_CAPSET,
        VIRTIO_GPU_CMD_GET_EDID,
        VIRTIO_GPU_CMD_RESOURCE_ASSIGN_UUID,
        VIRTIO_GPU_CMD_RESOURCE_CREATE_BLOB,
        VIRTIO_GPU_CMD_SET_SCANOUT_BLOB,

        /* 3d commands */
        VIRTIO_GPU_CMD_CTX_CREATE = 0x0200,
        VIRTIO_GPU_CMD_CTX_DESTROY,
        VIRTIO_GPU_CMD_CTX_ATTACH_RESOURCE,
        VIRTIO_GPU_CMD_CTX_DETACH_RESOURCE,
        VIRTIO_GPU_CMD_RESOURCE_CREATE_3D,
        VIRTIO_GPU_CMD_TRANSFER_TO_HOST_3D,
        VIRTIO_GPU_CMD_TRANSFER_FROM_HOST_3D,
        VIRTIO_GPU_CMD_SUBMIT_3D,
        VIRTIO_GPU_CMD_RESOURCE_MAP_BLOB,
        VIRTIO_GPU_CMD_RESOURCE_UNMAP_BLOB,

        /* cursor commands */
        VIRTIO_GPU_CMD_UPDATE_CURSOR = 0x0300,
        VIRTIO_GPU_CMD_MOVE_CURSOR,

        /* success responses */
        VIRTIO_GPU_RESP_OK_NODATA = 0x1100,
        VIRTIO_GPU_RESP_OK_DISPLAY_INFO,
        VIRTIO_GPU_RESP_OK_CAPSET_INFO,
        VIRTIO_GPU_RESP_OK_CAPSET,
        VIRTIO_GPU_RESP_OK_EDID,
        VIRTIO_GPU_RESP_OK_RESOURCE_UUID,
        VIRTIO_GPU_RESP_OK_MAP_INFO,

        /* error responses */
        VIRTIO_GPU_RESP_ERR_UNSPEC = 0x1200,
        VIRTIO_GPU_RESP_ERR_OUT_OF_MEMORY,
        VIRTIO_GPU_RESP_ERR_INVALID_SCANOUT_ID,
        VIRTIO_GPU_RESP_ERR_INVALID_RESOURCE_ID,
        VIRTIO_GPU_RESP_ERR_INVALID_CONTEXT_ID,
        VIRTIO_GPU_RESP_ERR_INVALID_PARAMETER,
};

#define VIRTIO_GPU_FLAG_FENCE (1 << 0)
#define VIRTIO_GPU_FLAG_INFO_RING_IDX (1 << 1)

struct virtio_gpu_ctrl_hdr {
        le32 type;
        le32 flags;
        le64 fence_id;
        le32 ctx_id;
        u8 ring_idx;
        u8 padding[3];
};
\end{lstlisting}

The fixed header \field{struct virtio_gpu_ctrl_hdr} in each
request includes the following fields:

\begin{description}
\item[\field{type}] specifies the type of the driver request
  (VIRTIO_GPU_CMD_*) or device response (VIRTIO_GPU_RESP_*).
\item[\field{flags}] request / response flags.
\item[\field{fence_id}] If the driver sets the VIRTIO_GPU_FLAG_FENCE
  bit in the request \field{flags} field the device MUST:
  \begin{itemize*}
  \item set VIRTIO_GPU_FLAG_FENCE bit in the response,
  \item copy the content of the \field{fence_id} field from the
    request to the response, and
  \item send the response only after command processing is complete.
  \end{itemize*}
\item[\field{ctx_id}] Rendering context (used in 3D mode only).
\item[\field{ring_idx}] If VIRTIO_GPU_F_CONTEXT_INIT is supported, then
  the driver MAY set VIRTIO_GPU_FLAG_INFO_RING_IDX bit in the request
  \field{flags}.  In that case:
  \begin{itemize*}
  \item \field{ring_idx} indicates the value of a context-specific ring
   index.  The minimum value is 0 and maximum value is 63 (inclusive).
  \item If VIRTIO_GPU_FLAG_FENCE is set, \field{fence_id} acts as a
   sequence number on the synchronization timeline defined by
   \field{ctx_idx} and the ring index.
  \item If VIRTIO_GPU_FLAG_FENCE is set and when the command associated
   with \field{fence_id} is complete, the device MUST send a response for
   all outstanding commands with a sequence number less than or equal to
   \field{fence_id} on the same synchronization timeline.
  \end{itemize*}
\end{description}

On success the device will return VIRTIO_GPU_RESP_OK_NODATA in
case there is no payload.  Otherwise the \field{type} field will
indicate the kind of payload.

On error the device will return one of the
VIRTIO_GPU_RESP_ERR_* error codes.

\subsubsection{Device Operation: controlq}\label{sec:Device Types / GPU Device / Device Operation / Device Operation: controlq}

For any coordinates given 0,0 is top left, larger x moves right,
larger y moves down.

\begin{description}

\item[VIRTIO_GPU_CMD_GET_DISPLAY_INFO] Retrieve the current output
  configuration.  No request data (just bare \field{struct
    virtio_gpu_ctrl_hdr}).  Response type is
  VIRTIO_GPU_RESP_OK_DISPLAY_INFO, response data is \field{struct
    virtio_gpu_resp_display_info}.

\begin{lstlisting}
#define VIRTIO_GPU_MAX_SCANOUTS 16

struct virtio_gpu_rect {
        le32 x;
        le32 y;
        le32 width;
        le32 height;
};

struct virtio_gpu_resp_display_info {
        struct virtio_gpu_ctrl_hdr hdr;
        struct virtio_gpu_display_one {
                struct virtio_gpu_rect r;
                le32 enabled;
                le32 flags;
        } pmodes[VIRTIO_GPU_MAX_SCANOUTS];
};
\end{lstlisting}

The response contains a list of per-scanout information.  The info
contains whether the scanout is enabled and what its preferred
position and size is.

The size (fields \field{width} and \field{height}) is similar to the
native panel resolution in EDID display information, except that in
the virtual machine case the size can change when the host window
representing the guest display is gets resized.

The position (fields \field{x} and \field{y}) describe how the
displays are arranged (i.e. which is -- for example -- the left
display).

The \field{enabled} field is set when the user enabled the display.
It is roughly the same as the connected state of a phyiscal display
connector.

\item[VIRTIO_GPU_CMD_GET_EDID] Retrieve the EDID data for a given
  scanout.  Request data is \field{struct virtio_gpu_get_edid}).
  Response type is VIRTIO_GPU_RESP_OK_EDID, response data is
  \field{struct virtio_gpu_resp_edid}.  Support is optional and
  negotiated using the VIRTIO_GPU_F_EDID feature flag.

\begin{lstlisting}
struct virtio_gpu_get_edid {
        struct virtio_gpu_ctrl_hdr hdr;
        le32 scanout;
        le32 padding;
};

struct virtio_gpu_resp_edid {
        struct virtio_gpu_ctrl_hdr hdr;
        le32 size;
        le32 padding;
        u8 edid[1024];
};
\end{lstlisting}

The response contains the EDID display data blob (as specified by
VESA) for the scanout.

\item[VIRTIO_GPU_CMD_RESOURCE_CREATE_2D] Create a 2D resource on the
  host.  Request data is \field{struct virtio_gpu_resource_create_2d}.
  Response type is VIRTIO_GPU_RESP_OK_NODATA.

\begin{lstlisting}
enum virtio_gpu_formats {
        VIRTIO_GPU_FORMAT_B8G8R8A8_UNORM  = 1,
        VIRTIO_GPU_FORMAT_B8G8R8X8_UNORM  = 2,
        VIRTIO_GPU_FORMAT_A8R8G8B8_UNORM  = 3,
        VIRTIO_GPU_FORMAT_X8R8G8B8_UNORM  = 4,

        VIRTIO_GPU_FORMAT_R8G8B8A8_UNORM  = 67,
        VIRTIO_GPU_FORMAT_X8B8G8R8_UNORM  = 68,

        VIRTIO_GPU_FORMAT_A8B8G8R8_UNORM  = 121,
        VIRTIO_GPU_FORMAT_R8G8B8X8_UNORM  = 134,
};

struct virtio_gpu_resource_create_2d {
        struct virtio_gpu_ctrl_hdr hdr;
        le32 resource_id;
        le32 format;
        le32 width;
        le32 height;
};
\end{lstlisting}

This creates a 2D resource on the host with the specified width,
height and format.  The resource ids are generated by the guest.

\item[VIRTIO_GPU_CMD_RESOURCE_UNREF] Destroy a resource.  Request data
  is \field{struct virtio_gpu_resource_unref}.  Response type is
  VIRTIO_GPU_RESP_OK_NODATA.

\begin{lstlisting}
struct virtio_gpu_resource_unref {
        struct virtio_gpu_ctrl_hdr hdr;
        le32 resource_id;
        le32 padding;
};
\end{lstlisting}

This informs the host that a resource is no longer required by the
guest.

\item[VIRTIO_GPU_CMD_SET_SCANOUT] Set the scanout parameters for a
  single output.  Request data is \field{struct
    virtio_gpu_set_scanout}.  Response type is
  VIRTIO_GPU_RESP_OK_NODATA.

\begin{lstlisting}
struct virtio_gpu_set_scanout {
        struct virtio_gpu_ctrl_hdr hdr;
        struct virtio_gpu_rect r;
        le32 scanout_id;
        le32 resource_id;
};
\end{lstlisting}

This sets the scanout parameters for a single scanout. The resource_id
is the resource to be scanned out from, along with a rectangle.

Scanout rectangles must be completely covered by the underlying
resource.  Overlapping (or identical) scanouts are allowed, typical
use case is screen mirroring.

The driver can use resource_id = 0 to disable a scanout.

\item[VIRTIO_GPU_CMD_RESOURCE_FLUSH] Flush a scanout resource Request
  data is \field{struct virtio_gpu_resource_flush}.  Response type is
  VIRTIO_GPU_RESP_OK_NODATA.

\begin{lstlisting}
struct virtio_gpu_resource_flush {
        struct virtio_gpu_ctrl_hdr hdr;
        struct virtio_gpu_rect r;
        le32 resource_id;
        le32 padding;
};
\end{lstlisting}

This flushes a resource to screen.  It takes a rectangle and a
resource id, and flushes any scanouts the resource is being used on.

\item[VIRTIO_GPU_CMD_TRANSFER_TO_HOST_2D] Transfer from guest memory
  to host resource.  Request data is \field{struct
    virtio_gpu_transfer_to_host_2d}.  Response type is
  VIRTIO_GPU_RESP_OK_NODATA.

\begin{lstlisting}
struct virtio_gpu_transfer_to_host_2d {
        struct virtio_gpu_ctrl_hdr hdr;
        struct virtio_gpu_rect r;
        le64 offset;
        le32 resource_id;
        le32 padding;
};
\end{lstlisting}

This takes a resource id along with an destination offset into the
resource, and a box to transfer to the host backing for the resource.

\item[VIRTIO_GPU_CMD_RESOURCE_ATTACH_BACKING] Assign backing pages to
  a resource.  Request data is \field{struct
    virtio_gpu_resource_attach_backing}, followed by \field{struct
    virtio_gpu_mem_entry} entries.  Response type is
  VIRTIO_GPU_RESP_OK_NODATA.

\begin{lstlisting}
struct virtio_gpu_resource_attach_backing {
        struct virtio_gpu_ctrl_hdr hdr;
        le32 resource_id;
        le32 nr_entries;
};

struct virtio_gpu_mem_entry {
        le64 addr;
        le32 length;
        le32 padding;
};
\end{lstlisting}

This assign an array of guest pages as the backing store for a
resource. These pages are then used for the transfer operations for
that resource from that point on.

\item[VIRTIO_GPU_CMD_RESOURCE_DETACH_BACKING] Detach backing pages
  from a resource.  Request data is \field{struct
    virtio_gpu_resource_detach_backing}.  Response type is
  VIRTIO_GPU_RESP_OK_NODATA.

\begin{lstlisting}
struct virtio_gpu_resource_detach_backing {
        struct virtio_gpu_ctrl_hdr hdr;
        le32 resource_id;
        le32 padding;
};
\end{lstlisting}

This detaches any backing pages from a resource, to be used in case of
guest swapping or object destruction.

\item[VIRTIO_GPU_CMD_GET_CAPSET_INFO] Gets the information associated with
  a particular \field{capset_index}, which MUST less than \field{num_capsets}
  defined in the device configuration.  Request data is
  \field{struct virtio_gpu_get_capset_info}.  Response type is
  VIRTIO_GPU_RESP_OK_CAPSET_INFO.

  On success, \field{struct virtio_gpu_resp_capset_info} contains the
  \field{capset_id}, \field{capset_max_version}, \field{capset_max_size}
  associated with capset at the specified {capset_idex}.  field{capset_id} MUST
  be one of the following (see listing for values):

  \begin{itemize*}
  \item \href{https://gitlab.freedesktop.org/virgl/virglrenderer/-/blob/master/src/virgl_hw.h#L526}{VIRTIO_GPU_CAPSET_VIRGL} --
	the first edition of Virgl (Gallium OpenGL) protocol.
  \item \href{https://gitlab.freedesktop.org/virgl/virglrenderer/-/blob/master/src/virgl_hw.h#L550}{VIRTIO_GPU_CAPSET_VIRGL2} --
	the second edition of Virgl (Gallium OpenGL) protocol after the capset fix.
  \item \href{https://android.googlesource.com/device/generic/vulkan-cereal/+/refs/heads/master/protocols/}{VIRTIO_GPU_CAPSET_GFXSTREAM} --
	gfxtream's (mostly) autogenerated GLES and Vulkan streaming protocols.
  \item \href{https://gitlab.freedesktop.org/olv/venus-protocol}{VIRTIO_GPU_CAPSET_VENUS} --
	Mesa's (mostly) autogenerated Vulkan protocol.
  \item \href{https://chromium.googlesource.com/chromiumos/platform/crosvm/+/refs/heads/main/rutabaga_gfx/src/cross_domain/cross_domain_protocol.rs}{VIRTIO_GPU_CAPSET_CROSS_DOMAIN} --
	protocol for display virtualization via Wayland proxying.
  \end{itemize*}

\begin{lstlisting}
struct virtio_gpu_get_capset_info {
        struct virtio_gpu_ctrl_hdr hdr;
        le32 capset_index;
        le32 padding;
};

#define VIRTIO_GPU_CAPSET_VIRGL 1
#define VIRTIO_GPU_CAPSET_VIRGL2 2
#define VIRTIO_GPU_CAPSET_GFXSTREAM 3
#define VIRTIO_GPU_CAPSET_VENUS 4
#define VIRTIO_GPU_CAPSET_CROSS_DOMAIN 5
struct virtio_gpu_resp_capset_info {
        struct virtio_gpu_ctrl_hdr hdr;
        le32 capset_id;
        le32 capset_max_version;
        le32 capset_max_size;
        le32 padding;
};
\end{lstlisting}

\item[VIRTIO_GPU_CMD_GET_CAPSET] Gets the capset associated with a
  particular \field{capset_id} and \field{capset_version}.  Request data is
  \field{struct virtio_gpu_get_capset}.  Response type is
  VIRTIO_GPU_RESP_OK_CAPSET.

\begin{lstlisting}
struct virtio_gpu_get_capset {
        struct virtio_gpu_ctrl_hdr hdr;
        le32 capset_id;
        le32 capset_version;
};

struct virtio_gpu_resp_capset {
        struct virtio_gpu_ctrl_hdr hdr;
        u8 capset_data[];
};
\end{lstlisting}

\item[VIRTIO_GPU_CMD_RESOURCE_ASSIGN_UUID] Creates an exported object from
  a resource. Request data is \field{struct
    virtio_gpu_resource_assign_uuid}.  Response type is
  VIRTIO_GPU_RESP_OK_RESOURCE_UUID, response data is \field{struct
    virtio_gpu_resp_resource_uuid}. Support is optional and negotiated
    using the VIRTIO_GPU_F_RESOURCE_UUID feature flag.

\begin{lstlisting}
struct virtio_gpu_resource_assign_uuid {
        struct virtio_gpu_ctrl_hdr hdr;
        le32 resource_id;
        le32 padding;
};

struct virtio_gpu_resp_resource_uuid {
        struct virtio_gpu_ctrl_hdr hdr;
        u8 uuid[16];
};
\end{lstlisting}

The response contains a UUID which identifies the exported object created from
the host private resource. Note that if the resource has an attached backing,
modifications made to the host private resource through the exported object by
other devices are not visible in the attached backing until they are transferred
into the backing.

\item[VIRTIO_GPU_CMD_RESOURCE_CREATE_BLOB] Creates a virtio-gpu blob
  resource. Request data is \field{struct
  virtio_gpu_resource_create_blob}, followed by \field{struct
  virtio_gpu_mem_entry} entries. Response type is
  VIRTIO_GPU_RESP_OK_NODATA. Support is optional and negotiated
  using the VIRTIO_GPU_F_RESOURCE_BLOB feature flag.

\begin{lstlisting}
#define VIRTIO_GPU_BLOB_MEM_GUEST             0x0001
#define VIRTIO_GPU_BLOB_MEM_HOST3D            0x0002
#define VIRTIO_GPU_BLOB_MEM_HOST3D_GUEST      0x0003

#define VIRTIO_GPU_BLOB_FLAG_USE_MAPPABLE     0x0001
#define VIRTIO_GPU_BLOB_FLAG_USE_SHAREABLE    0x0002
#define VIRTIO_GPU_BLOB_FLAG_USE_CROSS_DEVICE 0x0004

struct virtio_gpu_resource_create_blob {
       struct virtio_gpu_ctrl_hdr hdr;
       le32 resource_id;
       le32 blob_mem;
       le32 blob_flags;
       le32 nr_entries;
       le64 blob_id;
       le64 size;
};

\end{lstlisting}

A blob resource is a container for:

  \begin{itemize*}
  \item a guest memory allocation (referred to as a
  "guest-only blob resource").
  \item a host memory allocation (referred to as a
  "host-only blob resource").
  \item a guest memory and host memory allocation (referred
  to as a "default blob resource").
  \end{itemize*}

The memory properties of the blob resource MUST be described by
\field{blob_mem}, which MUST be non-zero.

For default and guest-only blob resources, \field{nr_entries} guest
memory entries may be assigned to the resource.  For default blob resources
(i.e, when \field{blob_mem} is VIRTIO_GPU_BLOB_MEM_HOST3D_GUEST), these
memory entries are used as a shadow buffer for the host memory. To
facilitate drivers that support swap-in and swap-out, \field{nr_entries} may
be zero and VIRTIO_GPU_CMD_RESOURCE_ATTACH_BACKING may be subsequently used.
VIRTIO_GPU_CMD_RESOURCE_DETACH_BACKING may be used to unassign memory entries.

\field{blob_mem} can only be VIRTIO_GPU_BLOB_MEM_HOST3D and
VIRTIO_GPU_BLOB_MEM_HOST3D_GUEST if VIRTIO_GPU_F_VIRGL is supported.
VIRTIO_GPU_BLOB_MEM_GUEST is valid regardless whether VIRTIO_GPU_F_VIRGL
is supported or not.

For VIRTIO_GPU_BLOB_MEM_HOST3D and VIRTIO_GPU_BLOB_MEM_HOST3D_GUEST, the
virtio-gpu resource MUST be created from the rendering context local object
identified by the \field{blob_id}. The actual allocation is done via
VIRTIO_GPU_CMD_SUBMIT_3D.

The driver MUST inform the device if the blob resource is used for
memory access, sharing between driver instances and/or sharing with
other devices. This is done via the \field{blob_flags} field.

If VIRTIO_GPU_F_VIRGL is set, both VIRTIO_GPU_CMD_TRANSFER_TO_HOST_3D
and VIRTIO_GPU_CMD_TRANSFER_FROM_HOST_3D may be used to update the
resource. There is no restriction on the image/buffer view the driver
has on the blob resource.

\item[VIRTIO_GPU_CMD_SET_SCANOUT_BLOB] sets scanout parameters for a
   blob resource. Request data is
  \field{struct virtio_gpu_set_scanout_blob}. Response type is
  VIRTIO_GPU_RESP_OK_NODATA. Support is optional and negotiated
  using the VIRTIO_GPU_F_RESOURCE_BLOB feature flag.

\begin{lstlisting}
struct virtio_gpu_set_scanout_blob {
       struct virtio_gpu_ctrl_hdr hdr;
       struct virtio_gpu_rect r;
       le32 scanout_id;
       le32 resource_id;
       le32 width;
       le32 height;
       le32 format;
       le32 padding;
       le32 strides[4];
       le32 offsets[4];
};
\end{lstlisting}

The rectangle \field{r} represents the portion of the blob resource being
displayed. The rest is the metadata associated with the blob resource. The
format MUST be one of \field{enum virtio_gpu_formats}.  The format MAY be
compressed with header and data planes.

\end{description}

\subsubsection{Device Operation: controlq (3d)}\label{sec:Device Types / GPU Device / Device Operation / Device Operation: controlq (3d)}

These commands are supported by the device if the VIRTIO_GPU_F_VIRGL
feature flag is set.

\begin{description}

\item[VIRTIO_GPU_CMD_CTX_CREATE] creates a context for submitting an opaque
  command stream.  Request data is \field{struct virtio_gpu_ctx_create}.
  Response type is VIRTIO_GPU_RESP_OK_NODATA.

\begin{lstlisting}
#define VIRTIO_GPU_CONTEXT_INIT_CAPSET_ID_MASK 0x000000ff;
struct virtio_gpu_ctx_create {
       struct virtio_gpu_ctrl_hdr hdr;
       le32 nlen;
       le32 context_init;
       char debug_name[64];
};
\end{lstlisting}

The implementation MUST create a context for the given \field{ctx_id} in
the \field{hdr}.  For debugging purposes, a \field{debug_name} and it's
length \field{nlen} is provided by the driver.  If
VIRTIO_GPU_F_CONTEXT_INIT is supported, then lower 8 bits of
\field{context_init} MAY contain the \field{capset_id} associated with
context.  In that case, then the device MUST create a context that can
handle the specified command stream.

If the lower 8-bits of the \field{context_init} are zero, then the type of
the context is determined by the device.

\item[VIRTIO_GPU_CMD_CTX_DESTROY]
\item[VIRTIO_GPU_CMD_CTX_ATTACH_RESOURCE]
\item[VIRTIO_GPU_CMD_CTX_DETACH_RESOURCE]
  Manage virtio-gpu 3d contexts.

\item[VIRTIO_GPU_CMD_RESOURCE_CREATE_3D]
  Create virtio-gpu 3d resources.

\item[VIRTIO_GPU_CMD_TRANSFER_TO_HOST_3D]
\item[VIRTIO_GPU_CMD_TRANSFER_FROM_HOST_3D]
  Transfer data from and to virtio-gpu 3d resources.

\item[VIRTIO_GPU_CMD_SUBMIT_3D]
  Submit an opaque command stream.  The type of the command stream is
  determined when creating a context.

\item[VIRTIO_GPU_CMD_RESOURCE_MAP_BLOB] maps a host-only
  blob resource into an offset in the host visible memory region. Request
  data is \field{struct virtio_gpu_resource_map_blob}.  The driver MUST
  not map a blob resource that is already mapped.  Response type is
  VIRTIO_GPU_RESP_OK_MAP_INFO. Support is optional and negotiated
  using the VIRTIO_GPU_F_RESOURCE_BLOB feature flag and checking for
  the presence of the host visible memory region.

\begin{lstlisting}
struct virtio_gpu_resource_map_blob {
        struct virtio_gpu_ctrl_hdr hdr;
        le32 resource_id;
        le32 padding;
        le64 offset;
};

#define VIRTIO_GPU_MAP_CACHE_MASK      0x0f
#define VIRTIO_GPU_MAP_CACHE_NONE      0x00
#define VIRTIO_GPU_MAP_CACHE_CACHED    0x01
#define VIRTIO_GPU_MAP_CACHE_UNCACHED  0x02
#define VIRTIO_GPU_MAP_CACHE_WC        0x03
struct virtio_gpu_resp_map_info {
        struct virtio_gpu_ctrl_hdr hdr;
        u32 map_info;
        u32 padding;
};
\end{lstlisting}

\item[VIRTIO_GPU_CMD_RESOURCE_UNMAP_BLOB] unmaps a
  host-only blob resource from the host visible memory region. Request data
  is \field{struct virtio_gpu_resource_unmap_blob}.  Response type is
  VIRTIO_GPU_RESP_OK_NODATA.  Support is optional and negotiated
  using the VIRTIO_GPU_F_RESOURCE_BLOB feature flag and checking for
  the presence of the host visible memory region.

\begin{lstlisting}
struct virtio_gpu_resource_unmap_blob {
        struct virtio_gpu_ctrl_hdr hdr;
        le32 resource_id;
        le32 padding;
};
\end{lstlisting}

\end{description}

\subsubsection{Device Operation: cursorq}\label{sec:Device Types / GPU Device / Device Operation / Device Operation: cursorq}

Both cursorq commands use the same command struct.

\begin{lstlisting}
struct virtio_gpu_cursor_pos {
        le32 scanout_id;
        le32 x;
        le32 y;
        le32 padding;
};

struct virtio_gpu_update_cursor {
        struct virtio_gpu_ctrl_hdr hdr;
        struct virtio_gpu_cursor_pos pos;
        le32 resource_id;
        le32 hot_x;
        le32 hot_y;
        le32 padding;
};
\end{lstlisting}

\begin{description}

\item[VIRTIO_GPU_CMD_UPDATE_CURSOR]
Update cursor.
Request data is \field{struct virtio_gpu_update_cursor}.
Response type is VIRTIO_GPU_RESP_OK_NODATA.

Full cursor update.  Cursor will be loaded from the specified
\field{resource_id} and will be moved to \field{pos}.  The driver must
transfer the cursor into the resource beforehand (using control queue
commands) and make sure the commands to fill the resource are actually
processed (using fencing).

\item[VIRTIO_GPU_CMD_MOVE_CURSOR]
Move cursor.
Request data is \field{struct virtio_gpu_update_cursor}.
Response type is VIRTIO_GPU_RESP_OK_NODATA.

Move cursor to the place specified in \field{pos}.  The other fields
are not used and will be ignored by the device.

\end{description}

\subsection{VGA Compatibility}\label{sec:Device Types / GPU Device / VGA Compatibility}

Applies to Virtio Over PCI only.  The GPU device can come with and
without VGA compatibility.  The PCI class should be DISPLAY_VGA if VGA
compatibility is present and DISPLAY_OTHER otherwise.

VGA compatibility: PCI region 0 has the linear framebuffer, standard
vga registers are present.  Configuring a scanout
(VIRTIO_GPU_CMD_SET_SCANOUT) switches the device from vga
compatibility mode into native virtio mode.  A reset switches it back
into vga compatibility mode.

Note: qemu implementation also provides bochs dispi interface io ports
and mmio bar at pci region 1 and is therefore fully compatible with
the qemu stdvga (see \href{https://git.qemu-project.org/?p=qemu.git;a=blob;f=docs/specs/standard-vga.txt;hb=HEAD}{docs/specs/standard-vga.txt} in the qemu source tree).

\section{GPU Device}\label{sec:Device Types / GPU Device}

virtio-gpu is a virtio based graphics adapter.  It can operate in 2D
mode and in 3D mode.  3D mode will offload rendering ops to
the host gpu and therefore requires a gpu with 3D support on the host
machine.

In 2D mode the virtio-gpu device provides support for ARGB Hardware
cursors and multiple scanouts (aka heads).

\subsection{Device ID}\label{sec:Device Types / GPU Device / Device ID}

16

\subsection{Virtqueues}\label{sec:Device Types / GPU Device / Virtqueues}

\begin{description}
\item[0] controlq - queue for sending control commands
\item[1] cursorq - queue for sending cursor updates
\end{description}

Both queues have the same format.  Each request and each response have
a fixed header, followed by command specific data fields.  The
separate cursor queue is the "fast track" for cursor commands
(VIRTIO_GPU_CMD_UPDATE_CURSOR and VIRTIO_GPU_CMD_MOVE_CURSOR), so they
go through without being delayed by time-consuming commands in the
control queue.

\subsection{Feature bits}\label{sec:Device Types / GPU Device / Feature bits}

\begin{description}
\item[VIRTIO_GPU_F_VIRGL (0)] virgl 3D mode is supported.
\item[VIRTIO_GPU_F_EDID  (1)] EDID is supported.
\item[VIRTIO_GPU_F_RESOURCE_UUID (2)] assigning resources UUIDs for export
  to other virtio devices is supported.
\item[VIRTIO_GPU_F_RESOURCE_BLOB (3)] creating and using size-based blob
  resources is supported.
\item[VIRTIO_GPU_F_CONTEXT_INIT (4)] multiple context types and
  synchronization timelines supported.  Requires VIRTIO_GPU_F_VIRGL.
\end{description}

\subsection{Device configuration layout}\label{sec:Device Types / GPU Device / Device configuration layout}

GPU device configuration uses the following layout structure and
definitions:

\begin{lstlisting}
#define VIRTIO_GPU_EVENT_DISPLAY (1 << 0)

struct virtio_gpu_config {
        le32 events_read;
        le32 events_clear;
        le32 num_scanouts;
        le32 num_capsets;
};
\end{lstlisting}

\subsubsection{Device configuration fields}

\begin{description}
\item[\field{events_read}] signals pending events to the driver.  The
  driver MUST NOT write to this field.
\item[\field{events_clear}] clears pending events in the device.
  Writing a '1' into a bit will clear the corresponding bit in
  \field{events_read}, mimicking write-to-clear behavior.
\item[\field{num_scanouts}] specifies the maximum number of scanouts
  supported by the device.  Minimum value is 1, maximum value is 16.
\item[\field{num_capsets}] specifies the maximum number of capability
  sets supported by the device.  The minimum value is zero.
\end{description}

\subsubsection{Events}

\begin{description}
\item[VIRTIO_GPU_EVENT_DISPLAY] Display configuration has changed.
  The driver SHOULD use the VIRTIO_GPU_CMD_GET_DISPLAY_INFO command to
  fetch the information from the device.  In case EDID support is
  negotiated (VIRTIO_GPU_F_EDID feature flag) the device SHOULD also
  fetch the updated EDID blobs using the VIRTIO_GPU_CMD_GET_EDID
  command.
\end{description}

\devicenormative{\subsection}{Device Initialization}{Device Types / GPU Device / Device Initialization}

The driver SHOULD query the display information from the device using
the VIRTIO_GPU_CMD_GET_DISPLAY_INFO command and use that information
for the initial scanout setup.  In case EDID support is negotiated
(VIRTIO_GPU_F_EDID feature flag) the device SHOULD also fetch the EDID
information using the VIRTIO_GPU_CMD_GET_EDID command.  If no
information is available or all displays are disabled the driver MAY
choose to use a fallback, such as 1024x768 at display 0.

The driver SHOULD query all shared memory regions supported by the device.
If the device supports shared memory, the \field{shmid} of a region MUST
(see \ref{sec:Basic Facilities of a Virtio Device /
Shared Memory Regions}~\nameref{sec:Basic Facilities of a Virtio Device /
Shared Memory Regions}) be one of the following:

\begin{lstlisting}
enum virtio_gpu_shm_id {
        VIRTIO_GPU_SHM_ID_UNDEFINED = 0,
        VIRTIO_GPU_SHM_ID_HOST_VISIBLE = 1,
};
\end{lstlisting}

The shared memory region with VIRTIO_GPU_SHM_ID_HOST_VISIBLE is referred as
the "host visible memory region".  The device MUST support the
VIRTIO_GPU_CMD_RESOURCE_MAP_BLOB and VIRTIO_GPU_CMD_RESOURCE_UNMAP_BLOB
if the host visible memory region is available.

\subsection{Device Operation}\label{sec:Device Types / GPU Device / Device Operation}

The virtio-gpu is based around the concept of resources private to the
host.  The guest must DMA transfer into these resources, unless shared memory
regions are supported. This is a design requirement in order to interface with
future 3D rendering. In the unaccelerated 2D mode there is no support for DMA
transfers from resources, just to them.

Resources are initially simple 2D resources, consisting of a width,
height and format along with an identifier. The guest must then attach
backing store to the resources in order for DMA transfers to
work. This is like a GART in a real GPU.

\subsubsection{Device Operation: Create a framebuffer and configure scanout}

\begin{itemize*}
\item Create a host resource using VIRTIO_GPU_CMD_RESOURCE_CREATE_2D.
\item Allocate a framebuffer from guest ram, and attach it as backing
  storage to the resource just created, using
  VIRTIO_GPU_CMD_RESOURCE_ATTACH_BACKING.  Scatter lists are
  supported, so the framebuffer doesn't need to be contignous in guest
  physical memory.
\item Use VIRTIO_GPU_CMD_SET_SCANOUT to link the framebuffer to
  a display scanout.
\end{itemize*}

\subsubsection{Device Operation: Update a framebuffer and scanout}

\begin{itemize*}
\item Render to your framebuffer memory.
\item Use VIRTIO_GPU_CMD_TRANSFER_TO_HOST_2D to update the host resource
  from guest memory.
\item Use VIRTIO_GPU_CMD_RESOURCE_FLUSH to flush the updated resource
  to the display.
\end{itemize*}

\subsubsection{Device Operation: Using pageflip}

It is possible to create multiple framebuffers, flip between them
using VIRTIO_GPU_CMD_SET_SCANOUT and VIRTIO_GPU_CMD_RESOURCE_FLUSH,
and update the invisible framebuffer using
VIRTIO_GPU_CMD_TRANSFER_TO_HOST_2D.

\subsubsection{Device Operation: Multihead setup}

In case two or more displays are present there are different ways to
configure things:

\begin{itemize*}
\item Create a single framebuffer, link it to all displays
  (mirroring).
\item Create an framebuffer for each display.
\item Create one big framebuffer, configure scanouts to display a
  different rectangle of that framebuffer each.
\end{itemize*}

\devicenormative{\subsubsection}{Device Operation: Command lifecycle and fencing}{Device Types / GPU Device / Device Operation / Device Operation: Command lifecycle and fencing}

The device MAY process controlq commands asyncronously and return them
to the driver before the processing is complete.  If the driver needs
to know when the processing is finished it can set the
VIRTIO_GPU_FLAG_FENCE flag in the request.  The device MUST finish the
processing before returning the command then.

Note: current qemu implementation does asyncrounous processing only in
3d mode, when offloading the processing to the host gpu.

\subsubsection{Device Operation: Configure mouse cursor}

The mouse cursor image is a normal resource, except that it must be
64x64 in size.  The driver MUST create and populate the resource
(using the usual VIRTIO_GPU_CMD_RESOURCE_CREATE_2D,
VIRTIO_GPU_CMD_RESOURCE_ATTACH_BACKING and
VIRTIO_GPU_CMD_TRANSFER_TO_HOST_2D controlq commands) and make sure they
are completed (using VIRTIO_GPU_FLAG_FENCE).

Then VIRTIO_GPU_CMD_UPDATE_CURSOR can be sent to the cursorq to set
the pointer shape and position.  To move the pointer without updating
the shape use VIRTIO_GPU_CMD_MOVE_CURSOR instead.

\subsubsection{Device Operation: Request header}\label{sec:Device Types / GPU Device / Device Operation / Device Operation: Request header}

All requests and responses on the virtqueues have a fixed header
using the following layout structure and definitions:

\begin{lstlisting}
enum virtio_gpu_ctrl_type {

        /* 2d commands */
        VIRTIO_GPU_CMD_GET_DISPLAY_INFO = 0x0100,
        VIRTIO_GPU_CMD_RESOURCE_CREATE_2D,
        VIRTIO_GPU_CMD_RESOURCE_UNREF,
        VIRTIO_GPU_CMD_SET_SCANOUT,
        VIRTIO_GPU_CMD_RESOURCE_FLUSH,
        VIRTIO_GPU_CMD_TRANSFER_TO_HOST_2D,
        VIRTIO_GPU_CMD_RESOURCE_ATTACH_BACKING,
        VIRTIO_GPU_CMD_RESOURCE_DETACH_BACKING,
        VIRTIO_GPU_CMD_GET_CAPSET_INFO,
        VIRTIO_GPU_CMD_GET_CAPSET,
        VIRTIO_GPU_CMD_GET_EDID,
        VIRTIO_GPU_CMD_RESOURCE_ASSIGN_UUID,
        VIRTIO_GPU_CMD_RESOURCE_CREATE_BLOB,
        VIRTIO_GPU_CMD_SET_SCANOUT_BLOB,

        /* 3d commands */
        VIRTIO_GPU_CMD_CTX_CREATE = 0x0200,
        VIRTIO_GPU_CMD_CTX_DESTROY,
        VIRTIO_GPU_CMD_CTX_ATTACH_RESOURCE,
        VIRTIO_GPU_CMD_CTX_DETACH_RESOURCE,
        VIRTIO_GPU_CMD_RESOURCE_CREATE_3D,
        VIRTIO_GPU_CMD_TRANSFER_TO_HOST_3D,
        VIRTIO_GPU_CMD_TRANSFER_FROM_HOST_3D,
        VIRTIO_GPU_CMD_SUBMIT_3D,
        VIRTIO_GPU_CMD_RESOURCE_MAP_BLOB,
        VIRTIO_GPU_CMD_RESOURCE_UNMAP_BLOB,

        /* cursor commands */
        VIRTIO_GPU_CMD_UPDATE_CURSOR = 0x0300,
        VIRTIO_GPU_CMD_MOVE_CURSOR,

        /* success responses */
        VIRTIO_GPU_RESP_OK_NODATA = 0x1100,
        VIRTIO_GPU_RESP_OK_DISPLAY_INFO,
        VIRTIO_GPU_RESP_OK_CAPSET_INFO,
        VIRTIO_GPU_RESP_OK_CAPSET,
        VIRTIO_GPU_RESP_OK_EDID,
        VIRTIO_GPU_RESP_OK_RESOURCE_UUID,
        VIRTIO_GPU_RESP_OK_MAP_INFO,

        /* error responses */
        VIRTIO_GPU_RESP_ERR_UNSPEC = 0x1200,
        VIRTIO_GPU_RESP_ERR_OUT_OF_MEMORY,
        VIRTIO_GPU_RESP_ERR_INVALID_SCANOUT_ID,
        VIRTIO_GPU_RESP_ERR_INVALID_RESOURCE_ID,
        VIRTIO_GPU_RESP_ERR_INVALID_CONTEXT_ID,
        VIRTIO_GPU_RESP_ERR_INVALID_PARAMETER,
};

#define VIRTIO_GPU_FLAG_FENCE (1 << 0)
#define VIRTIO_GPU_FLAG_INFO_RING_IDX (1 << 1)

struct virtio_gpu_ctrl_hdr {
        le32 type;
        le32 flags;
        le64 fence_id;
        le32 ctx_id;
        u8 ring_idx;
        u8 padding[3];
};
\end{lstlisting}

The fixed header \field{struct virtio_gpu_ctrl_hdr} in each
request includes the following fields:

\begin{description}
\item[\field{type}] specifies the type of the driver request
  (VIRTIO_GPU_CMD_*) or device response (VIRTIO_GPU_RESP_*).
\item[\field{flags}] request / response flags.
\item[\field{fence_id}] If the driver sets the VIRTIO_GPU_FLAG_FENCE
  bit in the request \field{flags} field the device MUST:
  \begin{itemize*}
  \item set VIRTIO_GPU_FLAG_FENCE bit in the response,
  \item copy the content of the \field{fence_id} field from the
    request to the response, and
  \item send the response only after command processing is complete.
  \end{itemize*}
\item[\field{ctx_id}] Rendering context (used in 3D mode only).
\item[\field{ring_idx}] If VIRTIO_GPU_F_CONTEXT_INIT is supported, then
  the driver MAY set VIRTIO_GPU_FLAG_INFO_RING_IDX bit in the request
  \field{flags}.  In that case:
  \begin{itemize*}
  \item \field{ring_idx} indicates the value of a context-specific ring
   index.  The minimum value is 0 and maximum value is 63 (inclusive).
  \item If VIRTIO_GPU_FLAG_FENCE is set, \field{fence_id} acts as a
   sequence number on the synchronization timeline defined by
   \field{ctx_idx} and the ring index.
  \item If VIRTIO_GPU_FLAG_FENCE is set and when the command associated
   with \field{fence_id} is complete, the device MUST send a response for
   all outstanding commands with a sequence number less than or equal to
   \field{fence_id} on the same synchronization timeline.
  \end{itemize*}
\end{description}

On success the device will return VIRTIO_GPU_RESP_OK_NODATA in
case there is no payload.  Otherwise the \field{type} field will
indicate the kind of payload.

On error the device will return one of the
VIRTIO_GPU_RESP_ERR_* error codes.

\subsubsection{Device Operation: controlq}\label{sec:Device Types / GPU Device / Device Operation / Device Operation: controlq}

For any coordinates given 0,0 is top left, larger x moves right,
larger y moves down.

\begin{description}

\item[VIRTIO_GPU_CMD_GET_DISPLAY_INFO] Retrieve the current output
  configuration.  No request data (just bare \field{struct
    virtio_gpu_ctrl_hdr}).  Response type is
  VIRTIO_GPU_RESP_OK_DISPLAY_INFO, response data is \field{struct
    virtio_gpu_resp_display_info}.

\begin{lstlisting}
#define VIRTIO_GPU_MAX_SCANOUTS 16

struct virtio_gpu_rect {
        le32 x;
        le32 y;
        le32 width;
        le32 height;
};

struct virtio_gpu_resp_display_info {
        struct virtio_gpu_ctrl_hdr hdr;
        struct virtio_gpu_display_one {
                struct virtio_gpu_rect r;
                le32 enabled;
                le32 flags;
        } pmodes[VIRTIO_GPU_MAX_SCANOUTS];
};
\end{lstlisting}

The response contains a list of per-scanout information.  The info
contains whether the scanout is enabled and what its preferred
position and size is.

The size (fields \field{width} and \field{height}) is similar to the
native panel resolution in EDID display information, except that in
the virtual machine case the size can change when the host window
representing the guest display is gets resized.

The position (fields \field{x} and \field{y}) describe how the
displays are arranged (i.e. which is -- for example -- the left
display).

The \field{enabled} field is set when the user enabled the display.
It is roughly the same as the connected state of a phyiscal display
connector.

\item[VIRTIO_GPU_CMD_GET_EDID] Retrieve the EDID data for a given
  scanout.  Request data is \field{struct virtio_gpu_get_edid}).
  Response type is VIRTIO_GPU_RESP_OK_EDID, response data is
  \field{struct virtio_gpu_resp_edid}.  Support is optional and
  negotiated using the VIRTIO_GPU_F_EDID feature flag.

\begin{lstlisting}
struct virtio_gpu_get_edid {
        struct virtio_gpu_ctrl_hdr hdr;
        le32 scanout;
        le32 padding;
};

struct virtio_gpu_resp_edid {
        struct virtio_gpu_ctrl_hdr hdr;
        le32 size;
        le32 padding;
        u8 edid[1024];
};
\end{lstlisting}

The response contains the EDID display data blob (as specified by
VESA) for the scanout.

\item[VIRTIO_GPU_CMD_RESOURCE_CREATE_2D] Create a 2D resource on the
  host.  Request data is \field{struct virtio_gpu_resource_create_2d}.
  Response type is VIRTIO_GPU_RESP_OK_NODATA.

\begin{lstlisting}
enum virtio_gpu_formats {
        VIRTIO_GPU_FORMAT_B8G8R8A8_UNORM  = 1,
        VIRTIO_GPU_FORMAT_B8G8R8X8_UNORM  = 2,
        VIRTIO_GPU_FORMAT_A8R8G8B8_UNORM  = 3,
        VIRTIO_GPU_FORMAT_X8R8G8B8_UNORM  = 4,

        VIRTIO_GPU_FORMAT_R8G8B8A8_UNORM  = 67,
        VIRTIO_GPU_FORMAT_X8B8G8R8_UNORM  = 68,

        VIRTIO_GPU_FORMAT_A8B8G8R8_UNORM  = 121,
        VIRTIO_GPU_FORMAT_R8G8B8X8_UNORM  = 134,
};

struct virtio_gpu_resource_create_2d {
        struct virtio_gpu_ctrl_hdr hdr;
        le32 resource_id;
        le32 format;
        le32 width;
        le32 height;
};
\end{lstlisting}

This creates a 2D resource on the host with the specified width,
height and format.  The resource ids are generated by the guest.

\item[VIRTIO_GPU_CMD_RESOURCE_UNREF] Destroy a resource.  Request data
  is \field{struct virtio_gpu_resource_unref}.  Response type is
  VIRTIO_GPU_RESP_OK_NODATA.

\begin{lstlisting}
struct virtio_gpu_resource_unref {
        struct virtio_gpu_ctrl_hdr hdr;
        le32 resource_id;
        le32 padding;
};
\end{lstlisting}

This informs the host that a resource is no longer required by the
guest.

\item[VIRTIO_GPU_CMD_SET_SCANOUT] Set the scanout parameters for a
  single output.  Request data is \field{struct
    virtio_gpu_set_scanout}.  Response type is
  VIRTIO_GPU_RESP_OK_NODATA.

\begin{lstlisting}
struct virtio_gpu_set_scanout {
        struct virtio_gpu_ctrl_hdr hdr;
        struct virtio_gpu_rect r;
        le32 scanout_id;
        le32 resource_id;
};
\end{lstlisting}

This sets the scanout parameters for a single scanout. The resource_id
is the resource to be scanned out from, along with a rectangle.

Scanout rectangles must be completely covered by the underlying
resource.  Overlapping (or identical) scanouts are allowed, typical
use case is screen mirroring.

The driver can use resource_id = 0 to disable a scanout.

\item[VIRTIO_GPU_CMD_RESOURCE_FLUSH] Flush a scanout resource Request
  data is \field{struct virtio_gpu_resource_flush}.  Response type is
  VIRTIO_GPU_RESP_OK_NODATA.

\begin{lstlisting}
struct virtio_gpu_resource_flush {
        struct virtio_gpu_ctrl_hdr hdr;
        struct virtio_gpu_rect r;
        le32 resource_id;
        le32 padding;
};
\end{lstlisting}

This flushes a resource to screen.  It takes a rectangle and a
resource id, and flushes any scanouts the resource is being used on.

\item[VIRTIO_GPU_CMD_TRANSFER_TO_HOST_2D] Transfer from guest memory
  to host resource.  Request data is \field{struct
    virtio_gpu_transfer_to_host_2d}.  Response type is
  VIRTIO_GPU_RESP_OK_NODATA.

\begin{lstlisting}
struct virtio_gpu_transfer_to_host_2d {
        struct virtio_gpu_ctrl_hdr hdr;
        struct virtio_gpu_rect r;
        le64 offset;
        le32 resource_id;
        le32 padding;
};
\end{lstlisting}

This takes a resource id along with an destination offset into the
resource, and a box to transfer to the host backing for the resource.

\item[VIRTIO_GPU_CMD_RESOURCE_ATTACH_BACKING] Assign backing pages to
  a resource.  Request data is \field{struct
    virtio_gpu_resource_attach_backing}, followed by \field{struct
    virtio_gpu_mem_entry} entries.  Response type is
  VIRTIO_GPU_RESP_OK_NODATA.

\begin{lstlisting}
struct virtio_gpu_resource_attach_backing {
        struct virtio_gpu_ctrl_hdr hdr;
        le32 resource_id;
        le32 nr_entries;
};

struct virtio_gpu_mem_entry {
        le64 addr;
        le32 length;
        le32 padding;
};
\end{lstlisting}

This assign an array of guest pages as the backing store for a
resource. These pages are then used for the transfer operations for
that resource from that point on.

\item[VIRTIO_GPU_CMD_RESOURCE_DETACH_BACKING] Detach backing pages
  from a resource.  Request data is \field{struct
    virtio_gpu_resource_detach_backing}.  Response type is
  VIRTIO_GPU_RESP_OK_NODATA.

\begin{lstlisting}
struct virtio_gpu_resource_detach_backing {
        struct virtio_gpu_ctrl_hdr hdr;
        le32 resource_id;
        le32 padding;
};
\end{lstlisting}

This detaches any backing pages from a resource, to be used in case of
guest swapping or object destruction.

\item[VIRTIO_GPU_CMD_GET_CAPSET_INFO] Gets the information associated with
  a particular \field{capset_index}, which MUST less than \field{num_capsets}
  defined in the device configuration.  Request data is
  \field{struct virtio_gpu_get_capset_info}.  Response type is
  VIRTIO_GPU_RESP_OK_CAPSET_INFO.

  On success, \field{struct virtio_gpu_resp_capset_info} contains the
  \field{capset_id}, \field{capset_max_version}, \field{capset_max_size}
  associated with capset at the specified {capset_idex}.  field{capset_id} MUST
  be one of the following (see listing for values):

  \begin{itemize*}
  \item \href{https://gitlab.freedesktop.org/virgl/virglrenderer/-/blob/master/src/virgl_hw.h#L526}{VIRTIO_GPU_CAPSET_VIRGL} --
	the first edition of Virgl (Gallium OpenGL) protocol.
  \item \href{https://gitlab.freedesktop.org/virgl/virglrenderer/-/blob/master/src/virgl_hw.h#L550}{VIRTIO_GPU_CAPSET_VIRGL2} --
	the second edition of Virgl (Gallium OpenGL) protocol after the capset fix.
  \item \href{https://android.googlesource.com/device/generic/vulkan-cereal/+/refs/heads/master/protocols/}{VIRTIO_GPU_CAPSET_GFXSTREAM} --
	gfxtream's (mostly) autogenerated GLES and Vulkan streaming protocols.
  \item \href{https://gitlab.freedesktop.org/olv/venus-protocol}{VIRTIO_GPU_CAPSET_VENUS} --
	Mesa's (mostly) autogenerated Vulkan protocol.
  \item \href{https://chromium.googlesource.com/chromiumos/platform/crosvm/+/refs/heads/main/rutabaga_gfx/src/cross_domain/cross_domain_protocol.rs}{VIRTIO_GPU_CAPSET_CROSS_DOMAIN} --
	protocol for display virtualization via Wayland proxying.
  \end{itemize*}

\begin{lstlisting}
struct virtio_gpu_get_capset_info {
        struct virtio_gpu_ctrl_hdr hdr;
        le32 capset_index;
        le32 padding;
};

#define VIRTIO_GPU_CAPSET_VIRGL 1
#define VIRTIO_GPU_CAPSET_VIRGL2 2
#define VIRTIO_GPU_CAPSET_GFXSTREAM 3
#define VIRTIO_GPU_CAPSET_VENUS 4
#define VIRTIO_GPU_CAPSET_CROSS_DOMAIN 5
struct virtio_gpu_resp_capset_info {
        struct virtio_gpu_ctrl_hdr hdr;
        le32 capset_id;
        le32 capset_max_version;
        le32 capset_max_size;
        le32 padding;
};
\end{lstlisting}

\item[VIRTIO_GPU_CMD_GET_CAPSET] Gets the capset associated with a
  particular \field{capset_id} and \field{capset_version}.  Request data is
  \field{struct virtio_gpu_get_capset}.  Response type is
  VIRTIO_GPU_RESP_OK_CAPSET.

\begin{lstlisting}
struct virtio_gpu_get_capset {
        struct virtio_gpu_ctrl_hdr hdr;
        le32 capset_id;
        le32 capset_version;
};

struct virtio_gpu_resp_capset {
        struct virtio_gpu_ctrl_hdr hdr;
        u8 capset_data[];
};
\end{lstlisting}

\item[VIRTIO_GPU_CMD_RESOURCE_ASSIGN_UUID] Creates an exported object from
  a resource. Request data is \field{struct
    virtio_gpu_resource_assign_uuid}.  Response type is
  VIRTIO_GPU_RESP_OK_RESOURCE_UUID, response data is \field{struct
    virtio_gpu_resp_resource_uuid}. Support is optional and negotiated
    using the VIRTIO_GPU_F_RESOURCE_UUID feature flag.

\begin{lstlisting}
struct virtio_gpu_resource_assign_uuid {
        struct virtio_gpu_ctrl_hdr hdr;
        le32 resource_id;
        le32 padding;
};

struct virtio_gpu_resp_resource_uuid {
        struct virtio_gpu_ctrl_hdr hdr;
        u8 uuid[16];
};
\end{lstlisting}

The response contains a UUID which identifies the exported object created from
the host private resource. Note that if the resource has an attached backing,
modifications made to the host private resource through the exported object by
other devices are not visible in the attached backing until they are transferred
into the backing.

\item[VIRTIO_GPU_CMD_RESOURCE_CREATE_BLOB] Creates a virtio-gpu blob
  resource. Request data is \field{struct
  virtio_gpu_resource_create_blob}, followed by \field{struct
  virtio_gpu_mem_entry} entries. Response type is
  VIRTIO_GPU_RESP_OK_NODATA. Support is optional and negotiated
  using the VIRTIO_GPU_F_RESOURCE_BLOB feature flag.

\begin{lstlisting}
#define VIRTIO_GPU_BLOB_MEM_GUEST             0x0001
#define VIRTIO_GPU_BLOB_MEM_HOST3D            0x0002
#define VIRTIO_GPU_BLOB_MEM_HOST3D_GUEST      0x0003

#define VIRTIO_GPU_BLOB_FLAG_USE_MAPPABLE     0x0001
#define VIRTIO_GPU_BLOB_FLAG_USE_SHAREABLE    0x0002
#define VIRTIO_GPU_BLOB_FLAG_USE_CROSS_DEVICE 0x0004

struct virtio_gpu_resource_create_blob {
       struct virtio_gpu_ctrl_hdr hdr;
       le32 resource_id;
       le32 blob_mem;
       le32 blob_flags;
       le32 nr_entries;
       le64 blob_id;
       le64 size;
};

\end{lstlisting}

A blob resource is a container for:

  \begin{itemize*}
  \item a guest memory allocation (referred to as a
  "guest-only blob resource").
  \item a host memory allocation (referred to as a
  "host-only blob resource").
  \item a guest memory and host memory allocation (referred
  to as a "default blob resource").
  \end{itemize*}

The memory properties of the blob resource MUST be described by
\field{blob_mem}, which MUST be non-zero.

For default and guest-only blob resources, \field{nr_entries} guest
memory entries may be assigned to the resource.  For default blob resources
(i.e, when \field{blob_mem} is VIRTIO_GPU_BLOB_MEM_HOST3D_GUEST), these
memory entries are used as a shadow buffer for the host memory. To
facilitate drivers that support swap-in and swap-out, \field{nr_entries} may
be zero and VIRTIO_GPU_CMD_RESOURCE_ATTACH_BACKING may be subsequently used.
VIRTIO_GPU_CMD_RESOURCE_DETACH_BACKING may be used to unassign memory entries.

\field{blob_mem} can only be VIRTIO_GPU_BLOB_MEM_HOST3D and
VIRTIO_GPU_BLOB_MEM_HOST3D_GUEST if VIRTIO_GPU_F_VIRGL is supported.
VIRTIO_GPU_BLOB_MEM_GUEST is valid regardless whether VIRTIO_GPU_F_VIRGL
is supported or not.

For VIRTIO_GPU_BLOB_MEM_HOST3D and VIRTIO_GPU_BLOB_MEM_HOST3D_GUEST, the
virtio-gpu resource MUST be created from the rendering context local object
identified by the \field{blob_id}. The actual allocation is done via
VIRTIO_GPU_CMD_SUBMIT_3D.

The driver MUST inform the device if the blob resource is used for
memory access, sharing between driver instances and/or sharing with
other devices. This is done via the \field{blob_flags} field.

If VIRTIO_GPU_F_VIRGL is set, both VIRTIO_GPU_CMD_TRANSFER_TO_HOST_3D
and VIRTIO_GPU_CMD_TRANSFER_FROM_HOST_3D may be used to update the
resource. There is no restriction on the image/buffer view the driver
has on the blob resource.

\item[VIRTIO_GPU_CMD_SET_SCANOUT_BLOB] sets scanout parameters for a
   blob resource. Request data is
  \field{struct virtio_gpu_set_scanout_blob}. Response type is
  VIRTIO_GPU_RESP_OK_NODATA. Support is optional and negotiated
  using the VIRTIO_GPU_F_RESOURCE_BLOB feature flag.

\begin{lstlisting}
struct virtio_gpu_set_scanout_blob {
       struct virtio_gpu_ctrl_hdr hdr;
       struct virtio_gpu_rect r;
       le32 scanout_id;
       le32 resource_id;
       le32 width;
       le32 height;
       le32 format;
       le32 padding;
       le32 strides[4];
       le32 offsets[4];
};
\end{lstlisting}

The rectangle \field{r} represents the portion of the blob resource being
displayed. The rest is the metadata associated with the blob resource. The
format MUST be one of \field{enum virtio_gpu_formats}.  The format MAY be
compressed with header and data planes.

\end{description}

\subsubsection{Device Operation: controlq (3d)}\label{sec:Device Types / GPU Device / Device Operation / Device Operation: controlq (3d)}

These commands are supported by the device if the VIRTIO_GPU_F_VIRGL
feature flag is set.

\begin{description}

\item[VIRTIO_GPU_CMD_CTX_CREATE] creates a context for submitting an opaque
  command stream.  Request data is \field{struct virtio_gpu_ctx_create}.
  Response type is VIRTIO_GPU_RESP_OK_NODATA.

\begin{lstlisting}
#define VIRTIO_GPU_CONTEXT_INIT_CAPSET_ID_MASK 0x000000ff;
struct virtio_gpu_ctx_create {
       struct virtio_gpu_ctrl_hdr hdr;
       le32 nlen;
       le32 context_init;
       char debug_name[64];
};
\end{lstlisting}

The implementation MUST create a context for the given \field{ctx_id} in
the \field{hdr}.  For debugging purposes, a \field{debug_name} and it's
length \field{nlen} is provided by the driver.  If
VIRTIO_GPU_F_CONTEXT_INIT is supported, then lower 8 bits of
\field{context_init} MAY contain the \field{capset_id} associated with
context.  In that case, then the device MUST create a context that can
handle the specified command stream.

If the lower 8-bits of the \field{context_init} are zero, then the type of
the context is determined by the device.

\item[VIRTIO_GPU_CMD_CTX_DESTROY]
\item[VIRTIO_GPU_CMD_CTX_ATTACH_RESOURCE]
\item[VIRTIO_GPU_CMD_CTX_DETACH_RESOURCE]
  Manage virtio-gpu 3d contexts.

\item[VIRTIO_GPU_CMD_RESOURCE_CREATE_3D]
  Create virtio-gpu 3d resources.

\item[VIRTIO_GPU_CMD_TRANSFER_TO_HOST_3D]
\item[VIRTIO_GPU_CMD_TRANSFER_FROM_HOST_3D]
  Transfer data from and to virtio-gpu 3d resources.

\item[VIRTIO_GPU_CMD_SUBMIT_3D]
  Submit an opaque command stream.  The type of the command stream is
  determined when creating a context.

\item[VIRTIO_GPU_CMD_RESOURCE_MAP_BLOB] maps a host-only
  blob resource into an offset in the host visible memory region. Request
  data is \field{struct virtio_gpu_resource_map_blob}.  The driver MUST
  not map a blob resource that is already mapped.  Response type is
  VIRTIO_GPU_RESP_OK_MAP_INFO. Support is optional and negotiated
  using the VIRTIO_GPU_F_RESOURCE_BLOB feature flag and checking for
  the presence of the host visible memory region.

\begin{lstlisting}
struct virtio_gpu_resource_map_blob {
        struct virtio_gpu_ctrl_hdr hdr;
        le32 resource_id;
        le32 padding;
        le64 offset;
};

#define VIRTIO_GPU_MAP_CACHE_MASK      0x0f
#define VIRTIO_GPU_MAP_CACHE_NONE      0x00
#define VIRTIO_GPU_MAP_CACHE_CACHED    0x01
#define VIRTIO_GPU_MAP_CACHE_UNCACHED  0x02
#define VIRTIO_GPU_MAP_CACHE_WC        0x03
struct virtio_gpu_resp_map_info {
        struct virtio_gpu_ctrl_hdr hdr;
        u32 map_info;
        u32 padding;
};
\end{lstlisting}

\item[VIRTIO_GPU_CMD_RESOURCE_UNMAP_BLOB] unmaps a
  host-only blob resource from the host visible memory region. Request data
  is \field{struct virtio_gpu_resource_unmap_blob}.  Response type is
  VIRTIO_GPU_RESP_OK_NODATA.  Support is optional and negotiated
  using the VIRTIO_GPU_F_RESOURCE_BLOB feature flag and checking for
  the presence of the host visible memory region.

\begin{lstlisting}
struct virtio_gpu_resource_unmap_blob {
        struct virtio_gpu_ctrl_hdr hdr;
        le32 resource_id;
        le32 padding;
};
\end{lstlisting}

\end{description}

\subsubsection{Device Operation: cursorq}\label{sec:Device Types / GPU Device / Device Operation / Device Operation: cursorq}

Both cursorq commands use the same command struct.

\begin{lstlisting}
struct virtio_gpu_cursor_pos {
        le32 scanout_id;
        le32 x;
        le32 y;
        le32 padding;
};

struct virtio_gpu_update_cursor {
        struct virtio_gpu_ctrl_hdr hdr;
        struct virtio_gpu_cursor_pos pos;
        le32 resource_id;
        le32 hot_x;
        le32 hot_y;
        le32 padding;
};
\end{lstlisting}

\begin{description}

\item[VIRTIO_GPU_CMD_UPDATE_CURSOR]
Update cursor.
Request data is \field{struct virtio_gpu_update_cursor}.
Response type is VIRTIO_GPU_RESP_OK_NODATA.

Full cursor update.  Cursor will be loaded from the specified
\field{resource_id} and will be moved to \field{pos}.  The driver must
transfer the cursor into the resource beforehand (using control queue
commands) and make sure the commands to fill the resource are actually
processed (using fencing).

\item[VIRTIO_GPU_CMD_MOVE_CURSOR]
Move cursor.
Request data is \field{struct virtio_gpu_update_cursor}.
Response type is VIRTIO_GPU_RESP_OK_NODATA.

Move cursor to the place specified in \field{pos}.  The other fields
are not used and will be ignored by the device.

\end{description}

\subsection{VGA Compatibility}\label{sec:Device Types / GPU Device / VGA Compatibility}

Applies to Virtio Over PCI only.  The GPU device can come with and
without VGA compatibility.  The PCI class should be DISPLAY_VGA if VGA
compatibility is present and DISPLAY_OTHER otherwise.

VGA compatibility: PCI region 0 has the linear framebuffer, standard
vga registers are present.  Configuring a scanout
(VIRTIO_GPU_CMD_SET_SCANOUT) switches the device from vga
compatibility mode into native virtio mode.  A reset switches it back
into vga compatibility mode.

Note: qemu implementation also provides bochs dispi interface io ports
and mmio bar at pci region 1 and is therefore fully compatible with
the qemu stdvga (see \href{https://git.qemu-project.org/?p=qemu.git;a=blob;f=docs/specs/standard-vga.txt;hb=HEAD}{docs/specs/standard-vga.txt} in the qemu source tree).

\section{GPU Device}\label{sec:Device Types / GPU Device}

virtio-gpu is a virtio based graphics adapter.  It can operate in 2D
mode and in 3D mode.  3D mode will offload rendering ops to
the host gpu and therefore requires a gpu with 3D support on the host
machine.

In 2D mode the virtio-gpu device provides support for ARGB Hardware
cursors and multiple scanouts (aka heads).

\subsection{Device ID}\label{sec:Device Types / GPU Device / Device ID}

16

\subsection{Virtqueues}\label{sec:Device Types / GPU Device / Virtqueues}

\begin{description}
\item[0] controlq - queue for sending control commands
\item[1] cursorq - queue for sending cursor updates
\end{description}

Both queues have the same format.  Each request and each response have
a fixed header, followed by command specific data fields.  The
separate cursor queue is the "fast track" for cursor commands
(VIRTIO_GPU_CMD_UPDATE_CURSOR and VIRTIO_GPU_CMD_MOVE_CURSOR), so they
go through without being delayed by time-consuming commands in the
control queue.

\subsection{Feature bits}\label{sec:Device Types / GPU Device / Feature bits}

\begin{description}
\item[VIRTIO_GPU_F_VIRGL (0)] virgl 3D mode is supported.
\item[VIRTIO_GPU_F_EDID  (1)] EDID is supported.
\item[VIRTIO_GPU_F_RESOURCE_UUID (2)] assigning resources UUIDs for export
  to other virtio devices is supported.
\item[VIRTIO_GPU_F_RESOURCE_BLOB (3)] creating and using size-based blob
  resources is supported.
\item[VIRTIO_GPU_F_CONTEXT_INIT (4)] multiple context types and
  synchronization timelines supported.  Requires VIRTIO_GPU_F_VIRGL.
\end{description}

\subsection{Device configuration layout}\label{sec:Device Types / GPU Device / Device configuration layout}

GPU device configuration uses the following layout structure and
definitions:

\begin{lstlisting}
#define VIRTIO_GPU_EVENT_DISPLAY (1 << 0)

struct virtio_gpu_config {
        le32 events_read;
        le32 events_clear;
        le32 num_scanouts;
        le32 num_capsets;
};
\end{lstlisting}

\subsubsection{Device configuration fields}

\begin{description}
\item[\field{events_read}] signals pending events to the driver.  The
  driver MUST NOT write to this field.
\item[\field{events_clear}] clears pending events in the device.
  Writing a '1' into a bit will clear the corresponding bit in
  \field{events_read}, mimicking write-to-clear behavior.
\item[\field{num_scanouts}] specifies the maximum number of scanouts
  supported by the device.  Minimum value is 1, maximum value is 16.
\item[\field{num_capsets}] specifies the maximum number of capability
  sets supported by the device.  The minimum value is zero.
\end{description}

\subsubsection{Events}

\begin{description}
\item[VIRTIO_GPU_EVENT_DISPLAY] Display configuration has changed.
  The driver SHOULD use the VIRTIO_GPU_CMD_GET_DISPLAY_INFO command to
  fetch the information from the device.  In case EDID support is
  negotiated (VIRTIO_GPU_F_EDID feature flag) the device SHOULD also
  fetch the updated EDID blobs using the VIRTIO_GPU_CMD_GET_EDID
  command.
\end{description}

\devicenormative{\subsection}{Device Initialization}{Device Types / GPU Device / Device Initialization}

The driver SHOULD query the display information from the device using
the VIRTIO_GPU_CMD_GET_DISPLAY_INFO command and use that information
for the initial scanout setup.  In case EDID support is negotiated
(VIRTIO_GPU_F_EDID feature flag) the device SHOULD also fetch the EDID
information using the VIRTIO_GPU_CMD_GET_EDID command.  If no
information is available or all displays are disabled the driver MAY
choose to use a fallback, such as 1024x768 at display 0.

The driver SHOULD query all shared memory regions supported by the device.
If the device supports shared memory, the \field{shmid} of a region MUST
(see \ref{sec:Basic Facilities of a Virtio Device /
Shared Memory Regions}~\nameref{sec:Basic Facilities of a Virtio Device /
Shared Memory Regions}) be one of the following:

\begin{lstlisting}
enum virtio_gpu_shm_id {
        VIRTIO_GPU_SHM_ID_UNDEFINED = 0,
        VIRTIO_GPU_SHM_ID_HOST_VISIBLE = 1,
};
\end{lstlisting}

The shared memory region with VIRTIO_GPU_SHM_ID_HOST_VISIBLE is referred as
the "host visible memory region".  The device MUST support the
VIRTIO_GPU_CMD_RESOURCE_MAP_BLOB and VIRTIO_GPU_CMD_RESOURCE_UNMAP_BLOB
if the host visible memory region is available.

\subsection{Device Operation}\label{sec:Device Types / GPU Device / Device Operation}

The virtio-gpu is based around the concept of resources private to the
host.  The guest must DMA transfer into these resources, unless shared memory
regions are supported. This is a design requirement in order to interface with
future 3D rendering. In the unaccelerated 2D mode there is no support for DMA
transfers from resources, just to them.

Resources are initially simple 2D resources, consisting of a width,
height and format along with an identifier. The guest must then attach
backing store to the resources in order for DMA transfers to
work. This is like a GART in a real GPU.

\subsubsection{Device Operation: Create a framebuffer and configure scanout}

\begin{itemize*}
\item Create a host resource using VIRTIO_GPU_CMD_RESOURCE_CREATE_2D.
\item Allocate a framebuffer from guest ram, and attach it as backing
  storage to the resource just created, using
  VIRTIO_GPU_CMD_RESOURCE_ATTACH_BACKING.  Scatter lists are
  supported, so the framebuffer doesn't need to be contignous in guest
  physical memory.
\item Use VIRTIO_GPU_CMD_SET_SCANOUT to link the framebuffer to
  a display scanout.
\end{itemize*}

\subsubsection{Device Operation: Update a framebuffer and scanout}

\begin{itemize*}
\item Render to your framebuffer memory.
\item Use VIRTIO_GPU_CMD_TRANSFER_TO_HOST_2D to update the host resource
  from guest memory.
\item Use VIRTIO_GPU_CMD_RESOURCE_FLUSH to flush the updated resource
  to the display.
\end{itemize*}

\subsubsection{Device Operation: Using pageflip}

It is possible to create multiple framebuffers, flip between them
using VIRTIO_GPU_CMD_SET_SCANOUT and VIRTIO_GPU_CMD_RESOURCE_FLUSH,
and update the invisible framebuffer using
VIRTIO_GPU_CMD_TRANSFER_TO_HOST_2D.

\subsubsection{Device Operation: Multihead setup}

In case two or more displays are present there are different ways to
configure things:

\begin{itemize*}
\item Create a single framebuffer, link it to all displays
  (mirroring).
\item Create an framebuffer for each display.
\item Create one big framebuffer, configure scanouts to display a
  different rectangle of that framebuffer each.
\end{itemize*}

\devicenormative{\subsubsection}{Device Operation: Command lifecycle and fencing}{Device Types / GPU Device / Device Operation / Device Operation: Command lifecycle and fencing}

The device MAY process controlq commands asyncronously and return them
to the driver before the processing is complete.  If the driver needs
to know when the processing is finished it can set the
VIRTIO_GPU_FLAG_FENCE flag in the request.  The device MUST finish the
processing before returning the command then.

Note: current qemu implementation does asyncrounous processing only in
3d mode, when offloading the processing to the host gpu.

\subsubsection{Device Operation: Configure mouse cursor}

The mouse cursor image is a normal resource, except that it must be
64x64 in size.  The driver MUST create and populate the resource
(using the usual VIRTIO_GPU_CMD_RESOURCE_CREATE_2D,
VIRTIO_GPU_CMD_RESOURCE_ATTACH_BACKING and
VIRTIO_GPU_CMD_TRANSFER_TO_HOST_2D controlq commands) and make sure they
are completed (using VIRTIO_GPU_FLAG_FENCE).

Then VIRTIO_GPU_CMD_UPDATE_CURSOR can be sent to the cursorq to set
the pointer shape and position.  To move the pointer without updating
the shape use VIRTIO_GPU_CMD_MOVE_CURSOR instead.

\subsubsection{Device Operation: Request header}\label{sec:Device Types / GPU Device / Device Operation / Device Operation: Request header}

All requests and responses on the virtqueues have a fixed header
using the following layout structure and definitions:

\begin{lstlisting}
enum virtio_gpu_ctrl_type {

        /* 2d commands */
        VIRTIO_GPU_CMD_GET_DISPLAY_INFO = 0x0100,
        VIRTIO_GPU_CMD_RESOURCE_CREATE_2D,
        VIRTIO_GPU_CMD_RESOURCE_UNREF,
        VIRTIO_GPU_CMD_SET_SCANOUT,
        VIRTIO_GPU_CMD_RESOURCE_FLUSH,
        VIRTIO_GPU_CMD_TRANSFER_TO_HOST_2D,
        VIRTIO_GPU_CMD_RESOURCE_ATTACH_BACKING,
        VIRTIO_GPU_CMD_RESOURCE_DETACH_BACKING,
        VIRTIO_GPU_CMD_GET_CAPSET_INFO,
        VIRTIO_GPU_CMD_GET_CAPSET,
        VIRTIO_GPU_CMD_GET_EDID,
        VIRTIO_GPU_CMD_RESOURCE_ASSIGN_UUID,
        VIRTIO_GPU_CMD_RESOURCE_CREATE_BLOB,
        VIRTIO_GPU_CMD_SET_SCANOUT_BLOB,

        /* 3d commands */
        VIRTIO_GPU_CMD_CTX_CREATE = 0x0200,
        VIRTIO_GPU_CMD_CTX_DESTROY,
        VIRTIO_GPU_CMD_CTX_ATTACH_RESOURCE,
        VIRTIO_GPU_CMD_CTX_DETACH_RESOURCE,
        VIRTIO_GPU_CMD_RESOURCE_CREATE_3D,
        VIRTIO_GPU_CMD_TRANSFER_TO_HOST_3D,
        VIRTIO_GPU_CMD_TRANSFER_FROM_HOST_3D,
        VIRTIO_GPU_CMD_SUBMIT_3D,
        VIRTIO_GPU_CMD_RESOURCE_MAP_BLOB,
        VIRTIO_GPU_CMD_RESOURCE_UNMAP_BLOB,

        /* cursor commands */
        VIRTIO_GPU_CMD_UPDATE_CURSOR = 0x0300,
        VIRTIO_GPU_CMD_MOVE_CURSOR,

        /* success responses */
        VIRTIO_GPU_RESP_OK_NODATA = 0x1100,
        VIRTIO_GPU_RESP_OK_DISPLAY_INFO,
        VIRTIO_GPU_RESP_OK_CAPSET_INFO,
        VIRTIO_GPU_RESP_OK_CAPSET,
        VIRTIO_GPU_RESP_OK_EDID,
        VIRTIO_GPU_RESP_OK_RESOURCE_UUID,
        VIRTIO_GPU_RESP_OK_MAP_INFO,

        /* error responses */
        VIRTIO_GPU_RESP_ERR_UNSPEC = 0x1200,
        VIRTIO_GPU_RESP_ERR_OUT_OF_MEMORY,
        VIRTIO_GPU_RESP_ERR_INVALID_SCANOUT_ID,
        VIRTIO_GPU_RESP_ERR_INVALID_RESOURCE_ID,
        VIRTIO_GPU_RESP_ERR_INVALID_CONTEXT_ID,
        VIRTIO_GPU_RESP_ERR_INVALID_PARAMETER,
};

#define VIRTIO_GPU_FLAG_FENCE (1 << 0)
#define VIRTIO_GPU_FLAG_INFO_RING_IDX (1 << 1)

struct virtio_gpu_ctrl_hdr {
        le32 type;
        le32 flags;
        le64 fence_id;
        le32 ctx_id;
        u8 ring_idx;
        u8 padding[3];
};
\end{lstlisting}

The fixed header \field{struct virtio_gpu_ctrl_hdr} in each
request includes the following fields:

\begin{description}
\item[\field{type}] specifies the type of the driver request
  (VIRTIO_GPU_CMD_*) or device response (VIRTIO_GPU_RESP_*).
\item[\field{flags}] request / response flags.
\item[\field{fence_id}] If the driver sets the VIRTIO_GPU_FLAG_FENCE
  bit in the request \field{flags} field the device MUST:
  \begin{itemize*}
  \item set VIRTIO_GPU_FLAG_FENCE bit in the response,
  \item copy the content of the \field{fence_id} field from the
    request to the response, and
  \item send the response only after command processing is complete.
  \end{itemize*}
\item[\field{ctx_id}] Rendering context (used in 3D mode only).
\item[\field{ring_idx}] If VIRTIO_GPU_F_CONTEXT_INIT is supported, then
  the driver MAY set VIRTIO_GPU_FLAG_INFO_RING_IDX bit in the request
  \field{flags}.  In that case:
  \begin{itemize*}
  \item \field{ring_idx} indicates the value of a context-specific ring
   index.  The minimum value is 0 and maximum value is 63 (inclusive).
  \item If VIRTIO_GPU_FLAG_FENCE is set, \field{fence_id} acts as a
   sequence number on the synchronization timeline defined by
   \field{ctx_idx} and the ring index.
  \item If VIRTIO_GPU_FLAG_FENCE is set and when the command associated
   with \field{fence_id} is complete, the device MUST send a response for
   all outstanding commands with a sequence number less than or equal to
   \field{fence_id} on the same synchronization timeline.
  \end{itemize*}
\end{description}

On success the device will return VIRTIO_GPU_RESP_OK_NODATA in
case there is no payload.  Otherwise the \field{type} field will
indicate the kind of payload.

On error the device will return one of the
VIRTIO_GPU_RESP_ERR_* error codes.

\subsubsection{Device Operation: controlq}\label{sec:Device Types / GPU Device / Device Operation / Device Operation: controlq}

For any coordinates given 0,0 is top left, larger x moves right,
larger y moves down.

\begin{description}

\item[VIRTIO_GPU_CMD_GET_DISPLAY_INFO] Retrieve the current output
  configuration.  No request data (just bare \field{struct
    virtio_gpu_ctrl_hdr}).  Response type is
  VIRTIO_GPU_RESP_OK_DISPLAY_INFO, response data is \field{struct
    virtio_gpu_resp_display_info}.

\begin{lstlisting}
#define VIRTIO_GPU_MAX_SCANOUTS 16

struct virtio_gpu_rect {
        le32 x;
        le32 y;
        le32 width;
        le32 height;
};

struct virtio_gpu_resp_display_info {
        struct virtio_gpu_ctrl_hdr hdr;
        struct virtio_gpu_display_one {
                struct virtio_gpu_rect r;
                le32 enabled;
                le32 flags;
        } pmodes[VIRTIO_GPU_MAX_SCANOUTS];
};
\end{lstlisting}

The response contains a list of per-scanout information.  The info
contains whether the scanout is enabled and what its preferred
position and size is.

The size (fields \field{width} and \field{height}) is similar to the
native panel resolution in EDID display information, except that in
the virtual machine case the size can change when the host window
representing the guest display is gets resized.

The position (fields \field{x} and \field{y}) describe how the
displays are arranged (i.e. which is -- for example -- the left
display).

The \field{enabled} field is set when the user enabled the display.
It is roughly the same as the connected state of a phyiscal display
connector.

\item[VIRTIO_GPU_CMD_GET_EDID] Retrieve the EDID data for a given
  scanout.  Request data is \field{struct virtio_gpu_get_edid}).
  Response type is VIRTIO_GPU_RESP_OK_EDID, response data is
  \field{struct virtio_gpu_resp_edid}.  Support is optional and
  negotiated using the VIRTIO_GPU_F_EDID feature flag.

\begin{lstlisting}
struct virtio_gpu_get_edid {
        struct virtio_gpu_ctrl_hdr hdr;
        le32 scanout;
        le32 padding;
};

struct virtio_gpu_resp_edid {
        struct virtio_gpu_ctrl_hdr hdr;
        le32 size;
        le32 padding;
        u8 edid[1024];
};
\end{lstlisting}

The response contains the EDID display data blob (as specified by
VESA) for the scanout.

\item[VIRTIO_GPU_CMD_RESOURCE_CREATE_2D] Create a 2D resource on the
  host.  Request data is \field{struct virtio_gpu_resource_create_2d}.
  Response type is VIRTIO_GPU_RESP_OK_NODATA.

\begin{lstlisting}
enum virtio_gpu_formats {
        VIRTIO_GPU_FORMAT_B8G8R8A8_UNORM  = 1,
        VIRTIO_GPU_FORMAT_B8G8R8X8_UNORM  = 2,
        VIRTIO_GPU_FORMAT_A8R8G8B8_UNORM  = 3,
        VIRTIO_GPU_FORMAT_X8R8G8B8_UNORM  = 4,

        VIRTIO_GPU_FORMAT_R8G8B8A8_UNORM  = 67,
        VIRTIO_GPU_FORMAT_X8B8G8R8_UNORM  = 68,

        VIRTIO_GPU_FORMAT_A8B8G8R8_UNORM  = 121,
        VIRTIO_GPU_FORMAT_R8G8B8X8_UNORM  = 134,
};

struct virtio_gpu_resource_create_2d {
        struct virtio_gpu_ctrl_hdr hdr;
        le32 resource_id;
        le32 format;
        le32 width;
        le32 height;
};
\end{lstlisting}

This creates a 2D resource on the host with the specified width,
height and format.  The resource ids are generated by the guest.

\item[VIRTIO_GPU_CMD_RESOURCE_UNREF] Destroy a resource.  Request data
  is \field{struct virtio_gpu_resource_unref}.  Response type is
  VIRTIO_GPU_RESP_OK_NODATA.

\begin{lstlisting}
struct virtio_gpu_resource_unref {
        struct virtio_gpu_ctrl_hdr hdr;
        le32 resource_id;
        le32 padding;
};
\end{lstlisting}

This informs the host that a resource is no longer required by the
guest.

\item[VIRTIO_GPU_CMD_SET_SCANOUT] Set the scanout parameters for a
  single output.  Request data is \field{struct
    virtio_gpu_set_scanout}.  Response type is
  VIRTIO_GPU_RESP_OK_NODATA.

\begin{lstlisting}
struct virtio_gpu_set_scanout {
        struct virtio_gpu_ctrl_hdr hdr;
        struct virtio_gpu_rect r;
        le32 scanout_id;
        le32 resource_id;
};
\end{lstlisting}

This sets the scanout parameters for a single scanout. The resource_id
is the resource to be scanned out from, along with a rectangle.

Scanout rectangles must be completely covered by the underlying
resource.  Overlapping (or identical) scanouts are allowed, typical
use case is screen mirroring.

The driver can use resource_id = 0 to disable a scanout.

\item[VIRTIO_GPU_CMD_RESOURCE_FLUSH] Flush a scanout resource Request
  data is \field{struct virtio_gpu_resource_flush}.  Response type is
  VIRTIO_GPU_RESP_OK_NODATA.

\begin{lstlisting}
struct virtio_gpu_resource_flush {
        struct virtio_gpu_ctrl_hdr hdr;
        struct virtio_gpu_rect r;
        le32 resource_id;
        le32 padding;
};
\end{lstlisting}

This flushes a resource to screen.  It takes a rectangle and a
resource id, and flushes any scanouts the resource is being used on.

\item[VIRTIO_GPU_CMD_TRANSFER_TO_HOST_2D] Transfer from guest memory
  to host resource.  Request data is \field{struct
    virtio_gpu_transfer_to_host_2d}.  Response type is
  VIRTIO_GPU_RESP_OK_NODATA.

\begin{lstlisting}
struct virtio_gpu_transfer_to_host_2d {
        struct virtio_gpu_ctrl_hdr hdr;
        struct virtio_gpu_rect r;
        le64 offset;
        le32 resource_id;
        le32 padding;
};
\end{lstlisting}

This takes a resource id along with an destination offset into the
resource, and a box to transfer to the host backing for the resource.

\item[VIRTIO_GPU_CMD_RESOURCE_ATTACH_BACKING] Assign backing pages to
  a resource.  Request data is \field{struct
    virtio_gpu_resource_attach_backing}, followed by \field{struct
    virtio_gpu_mem_entry} entries.  Response type is
  VIRTIO_GPU_RESP_OK_NODATA.

\begin{lstlisting}
struct virtio_gpu_resource_attach_backing {
        struct virtio_gpu_ctrl_hdr hdr;
        le32 resource_id;
        le32 nr_entries;
};

struct virtio_gpu_mem_entry {
        le64 addr;
        le32 length;
        le32 padding;
};
\end{lstlisting}

This assign an array of guest pages as the backing store for a
resource. These pages are then used for the transfer operations for
that resource from that point on.

\item[VIRTIO_GPU_CMD_RESOURCE_DETACH_BACKING] Detach backing pages
  from a resource.  Request data is \field{struct
    virtio_gpu_resource_detach_backing}.  Response type is
  VIRTIO_GPU_RESP_OK_NODATA.

\begin{lstlisting}
struct virtio_gpu_resource_detach_backing {
        struct virtio_gpu_ctrl_hdr hdr;
        le32 resource_id;
        le32 padding;
};
\end{lstlisting}

This detaches any backing pages from a resource, to be used in case of
guest swapping or object destruction.

\item[VIRTIO_GPU_CMD_GET_CAPSET_INFO] Gets the information associated with
  a particular \field{capset_index}, which MUST less than \field{num_capsets}
  defined in the device configuration.  Request data is
  \field{struct virtio_gpu_get_capset_info}.  Response type is
  VIRTIO_GPU_RESP_OK_CAPSET_INFO.

  On success, \field{struct virtio_gpu_resp_capset_info} contains the
  \field{capset_id}, \field{capset_max_version}, \field{capset_max_size}
  associated with capset at the specified {capset_idex}.  field{capset_id} MUST
  be one of the following (see listing for values):

  \begin{itemize*}
  \item \href{https://gitlab.freedesktop.org/virgl/virglrenderer/-/blob/master/src/virgl_hw.h#L526}{VIRTIO_GPU_CAPSET_VIRGL} --
	the first edition of Virgl (Gallium OpenGL) protocol.
  \item \href{https://gitlab.freedesktop.org/virgl/virglrenderer/-/blob/master/src/virgl_hw.h#L550}{VIRTIO_GPU_CAPSET_VIRGL2} --
	the second edition of Virgl (Gallium OpenGL) protocol after the capset fix.
  \item \href{https://android.googlesource.com/device/generic/vulkan-cereal/+/refs/heads/master/protocols/}{VIRTIO_GPU_CAPSET_GFXSTREAM} --
	gfxtream's (mostly) autogenerated GLES and Vulkan streaming protocols.
  \item \href{https://gitlab.freedesktop.org/olv/venus-protocol}{VIRTIO_GPU_CAPSET_VENUS} --
	Mesa's (mostly) autogenerated Vulkan protocol.
  \item \href{https://chromium.googlesource.com/chromiumos/platform/crosvm/+/refs/heads/main/rutabaga_gfx/src/cross_domain/cross_domain_protocol.rs}{VIRTIO_GPU_CAPSET_CROSS_DOMAIN} --
	protocol for display virtualization via Wayland proxying.
  \end{itemize*}

\begin{lstlisting}
struct virtio_gpu_get_capset_info {
        struct virtio_gpu_ctrl_hdr hdr;
        le32 capset_index;
        le32 padding;
};

#define VIRTIO_GPU_CAPSET_VIRGL 1
#define VIRTIO_GPU_CAPSET_VIRGL2 2
#define VIRTIO_GPU_CAPSET_GFXSTREAM 3
#define VIRTIO_GPU_CAPSET_VENUS 4
#define VIRTIO_GPU_CAPSET_CROSS_DOMAIN 5
struct virtio_gpu_resp_capset_info {
        struct virtio_gpu_ctrl_hdr hdr;
        le32 capset_id;
        le32 capset_max_version;
        le32 capset_max_size;
        le32 padding;
};
\end{lstlisting}

\item[VIRTIO_GPU_CMD_GET_CAPSET] Gets the capset associated with a
  particular \field{capset_id} and \field{capset_version}.  Request data is
  \field{struct virtio_gpu_get_capset}.  Response type is
  VIRTIO_GPU_RESP_OK_CAPSET.

\begin{lstlisting}
struct virtio_gpu_get_capset {
        struct virtio_gpu_ctrl_hdr hdr;
        le32 capset_id;
        le32 capset_version;
};

struct virtio_gpu_resp_capset {
        struct virtio_gpu_ctrl_hdr hdr;
        u8 capset_data[];
};
\end{lstlisting}

\item[VIRTIO_GPU_CMD_RESOURCE_ASSIGN_UUID] Creates an exported object from
  a resource. Request data is \field{struct
    virtio_gpu_resource_assign_uuid}.  Response type is
  VIRTIO_GPU_RESP_OK_RESOURCE_UUID, response data is \field{struct
    virtio_gpu_resp_resource_uuid}. Support is optional and negotiated
    using the VIRTIO_GPU_F_RESOURCE_UUID feature flag.

\begin{lstlisting}
struct virtio_gpu_resource_assign_uuid {
        struct virtio_gpu_ctrl_hdr hdr;
        le32 resource_id;
        le32 padding;
};

struct virtio_gpu_resp_resource_uuid {
        struct virtio_gpu_ctrl_hdr hdr;
        u8 uuid[16];
};
\end{lstlisting}

The response contains a UUID which identifies the exported object created from
the host private resource. Note that if the resource has an attached backing,
modifications made to the host private resource through the exported object by
other devices are not visible in the attached backing until they are transferred
into the backing.

\item[VIRTIO_GPU_CMD_RESOURCE_CREATE_BLOB] Creates a virtio-gpu blob
  resource. Request data is \field{struct
  virtio_gpu_resource_create_blob}, followed by \field{struct
  virtio_gpu_mem_entry} entries. Response type is
  VIRTIO_GPU_RESP_OK_NODATA. Support is optional and negotiated
  using the VIRTIO_GPU_F_RESOURCE_BLOB feature flag.

\begin{lstlisting}
#define VIRTIO_GPU_BLOB_MEM_GUEST             0x0001
#define VIRTIO_GPU_BLOB_MEM_HOST3D            0x0002
#define VIRTIO_GPU_BLOB_MEM_HOST3D_GUEST      0x0003

#define VIRTIO_GPU_BLOB_FLAG_USE_MAPPABLE     0x0001
#define VIRTIO_GPU_BLOB_FLAG_USE_SHAREABLE    0x0002
#define VIRTIO_GPU_BLOB_FLAG_USE_CROSS_DEVICE 0x0004

struct virtio_gpu_resource_create_blob {
       struct virtio_gpu_ctrl_hdr hdr;
       le32 resource_id;
       le32 blob_mem;
       le32 blob_flags;
       le32 nr_entries;
       le64 blob_id;
       le64 size;
};

\end{lstlisting}

A blob resource is a container for:

  \begin{itemize*}
  \item a guest memory allocation (referred to as a
  "guest-only blob resource").
  \item a host memory allocation (referred to as a
  "host-only blob resource").
  \item a guest memory and host memory allocation (referred
  to as a "default blob resource").
  \end{itemize*}

The memory properties of the blob resource MUST be described by
\field{blob_mem}, which MUST be non-zero.

For default and guest-only blob resources, \field{nr_entries} guest
memory entries may be assigned to the resource.  For default blob resources
(i.e, when \field{blob_mem} is VIRTIO_GPU_BLOB_MEM_HOST3D_GUEST), these
memory entries are used as a shadow buffer for the host memory. To
facilitate drivers that support swap-in and swap-out, \field{nr_entries} may
be zero and VIRTIO_GPU_CMD_RESOURCE_ATTACH_BACKING may be subsequently used.
VIRTIO_GPU_CMD_RESOURCE_DETACH_BACKING may be used to unassign memory entries.

\field{blob_mem} can only be VIRTIO_GPU_BLOB_MEM_HOST3D and
VIRTIO_GPU_BLOB_MEM_HOST3D_GUEST if VIRTIO_GPU_F_VIRGL is supported.
VIRTIO_GPU_BLOB_MEM_GUEST is valid regardless whether VIRTIO_GPU_F_VIRGL
is supported or not.

For VIRTIO_GPU_BLOB_MEM_HOST3D and VIRTIO_GPU_BLOB_MEM_HOST3D_GUEST, the
virtio-gpu resource MUST be created from the rendering context local object
identified by the \field{blob_id}. The actual allocation is done via
VIRTIO_GPU_CMD_SUBMIT_3D.

The driver MUST inform the device if the blob resource is used for
memory access, sharing between driver instances and/or sharing with
other devices. This is done via the \field{blob_flags} field.

If VIRTIO_GPU_F_VIRGL is set, both VIRTIO_GPU_CMD_TRANSFER_TO_HOST_3D
and VIRTIO_GPU_CMD_TRANSFER_FROM_HOST_3D may be used to update the
resource. There is no restriction on the image/buffer view the driver
has on the blob resource.

\item[VIRTIO_GPU_CMD_SET_SCANOUT_BLOB] sets scanout parameters for a
   blob resource. Request data is
  \field{struct virtio_gpu_set_scanout_blob}. Response type is
  VIRTIO_GPU_RESP_OK_NODATA. Support is optional and negotiated
  using the VIRTIO_GPU_F_RESOURCE_BLOB feature flag.

\begin{lstlisting}
struct virtio_gpu_set_scanout_blob {
       struct virtio_gpu_ctrl_hdr hdr;
       struct virtio_gpu_rect r;
       le32 scanout_id;
       le32 resource_id;
       le32 width;
       le32 height;
       le32 format;
       le32 padding;
       le32 strides[4];
       le32 offsets[4];
};
\end{lstlisting}

The rectangle \field{r} represents the portion of the blob resource being
displayed. The rest is the metadata associated with the blob resource. The
format MUST be one of \field{enum virtio_gpu_formats}.  The format MAY be
compressed with header and data planes.

\end{description}

\subsubsection{Device Operation: controlq (3d)}\label{sec:Device Types / GPU Device / Device Operation / Device Operation: controlq (3d)}

These commands are supported by the device if the VIRTIO_GPU_F_VIRGL
feature flag is set.

\begin{description}

\item[VIRTIO_GPU_CMD_CTX_CREATE] creates a context for submitting an opaque
  command stream.  Request data is \field{struct virtio_gpu_ctx_create}.
  Response type is VIRTIO_GPU_RESP_OK_NODATA.

\begin{lstlisting}
#define VIRTIO_GPU_CONTEXT_INIT_CAPSET_ID_MASK 0x000000ff;
struct virtio_gpu_ctx_create {
       struct virtio_gpu_ctrl_hdr hdr;
       le32 nlen;
       le32 context_init;
       char debug_name[64];
};
\end{lstlisting}

The implementation MUST create a context for the given \field{ctx_id} in
the \field{hdr}.  For debugging purposes, a \field{debug_name} and it's
length \field{nlen} is provided by the driver.  If
VIRTIO_GPU_F_CONTEXT_INIT is supported, then lower 8 bits of
\field{context_init} MAY contain the \field{capset_id} associated with
context.  In that case, then the device MUST create a context that can
handle the specified command stream.

If the lower 8-bits of the \field{context_init} are zero, then the type of
the context is determined by the device.

\item[VIRTIO_GPU_CMD_CTX_DESTROY]
\item[VIRTIO_GPU_CMD_CTX_ATTACH_RESOURCE]
\item[VIRTIO_GPU_CMD_CTX_DETACH_RESOURCE]
  Manage virtio-gpu 3d contexts.

\item[VIRTIO_GPU_CMD_RESOURCE_CREATE_3D]
  Create virtio-gpu 3d resources.

\item[VIRTIO_GPU_CMD_TRANSFER_TO_HOST_3D]
\item[VIRTIO_GPU_CMD_TRANSFER_FROM_HOST_3D]
  Transfer data from and to virtio-gpu 3d resources.

\item[VIRTIO_GPU_CMD_SUBMIT_3D]
  Submit an opaque command stream.  The type of the command stream is
  determined when creating a context.

\item[VIRTIO_GPU_CMD_RESOURCE_MAP_BLOB] maps a host-only
  blob resource into an offset in the host visible memory region. Request
  data is \field{struct virtio_gpu_resource_map_blob}.  The driver MUST
  not map a blob resource that is already mapped.  Response type is
  VIRTIO_GPU_RESP_OK_MAP_INFO. Support is optional and negotiated
  using the VIRTIO_GPU_F_RESOURCE_BLOB feature flag and checking for
  the presence of the host visible memory region.

\begin{lstlisting}
struct virtio_gpu_resource_map_blob {
        struct virtio_gpu_ctrl_hdr hdr;
        le32 resource_id;
        le32 padding;
        le64 offset;
};

#define VIRTIO_GPU_MAP_CACHE_MASK      0x0f
#define VIRTIO_GPU_MAP_CACHE_NONE      0x00
#define VIRTIO_GPU_MAP_CACHE_CACHED    0x01
#define VIRTIO_GPU_MAP_CACHE_UNCACHED  0x02
#define VIRTIO_GPU_MAP_CACHE_WC        0x03
struct virtio_gpu_resp_map_info {
        struct virtio_gpu_ctrl_hdr hdr;
        u32 map_info;
        u32 padding;
};
\end{lstlisting}

\item[VIRTIO_GPU_CMD_RESOURCE_UNMAP_BLOB] unmaps a
  host-only blob resource from the host visible memory region. Request data
  is \field{struct virtio_gpu_resource_unmap_blob}.  Response type is
  VIRTIO_GPU_RESP_OK_NODATA.  Support is optional and negotiated
  using the VIRTIO_GPU_F_RESOURCE_BLOB feature flag and checking for
  the presence of the host visible memory region.

\begin{lstlisting}
struct virtio_gpu_resource_unmap_blob {
        struct virtio_gpu_ctrl_hdr hdr;
        le32 resource_id;
        le32 padding;
};
\end{lstlisting}

\end{description}

\subsubsection{Device Operation: cursorq}\label{sec:Device Types / GPU Device / Device Operation / Device Operation: cursorq}

Both cursorq commands use the same command struct.

\begin{lstlisting}
struct virtio_gpu_cursor_pos {
        le32 scanout_id;
        le32 x;
        le32 y;
        le32 padding;
};

struct virtio_gpu_update_cursor {
        struct virtio_gpu_ctrl_hdr hdr;
        struct virtio_gpu_cursor_pos pos;
        le32 resource_id;
        le32 hot_x;
        le32 hot_y;
        le32 padding;
};
\end{lstlisting}

\begin{description}

\item[VIRTIO_GPU_CMD_UPDATE_CURSOR]
Update cursor.
Request data is \field{struct virtio_gpu_update_cursor}.
Response type is VIRTIO_GPU_RESP_OK_NODATA.

Full cursor update.  Cursor will be loaded from the specified
\field{resource_id} and will be moved to \field{pos}.  The driver must
transfer the cursor into the resource beforehand (using control queue
commands) and make sure the commands to fill the resource are actually
processed (using fencing).

\item[VIRTIO_GPU_CMD_MOVE_CURSOR]
Move cursor.
Request data is \field{struct virtio_gpu_update_cursor}.
Response type is VIRTIO_GPU_RESP_OK_NODATA.

Move cursor to the place specified in \field{pos}.  The other fields
are not used and will be ignored by the device.

\end{description}

\subsection{VGA Compatibility}\label{sec:Device Types / GPU Device / VGA Compatibility}

Applies to Virtio Over PCI only.  The GPU device can come with and
without VGA compatibility.  The PCI class should be DISPLAY_VGA if VGA
compatibility is present and DISPLAY_OTHER otherwise.

VGA compatibility: PCI region 0 has the linear framebuffer, standard
vga registers are present.  Configuring a scanout
(VIRTIO_GPU_CMD_SET_SCANOUT) switches the device from vga
compatibility mode into native virtio mode.  A reset switches it back
into vga compatibility mode.

Note: qemu implementation also provides bochs dispi interface io ports
and mmio bar at pci region 1 and is therefore fully compatible with
the qemu stdvga (see \href{https://git.qemu-project.org/?p=qemu.git;a=blob;f=docs/specs/standard-vga.txt;hb=HEAD}{docs/specs/standard-vga.txt} in the qemu source tree).

\section{GPU Device}\label{sec:Device Types / GPU Device}

virtio-gpu is a virtio based graphics adapter.  It can operate in 2D
mode and in 3D mode.  3D mode will offload rendering ops to
the host gpu and therefore requires a gpu with 3D support on the host
machine.

In 2D mode the virtio-gpu device provides support for ARGB Hardware
cursors and multiple scanouts (aka heads).

\subsection{Device ID}\label{sec:Device Types / GPU Device / Device ID}

16

\subsection{Virtqueues}\label{sec:Device Types / GPU Device / Virtqueues}

\begin{description}
\item[0] controlq - queue for sending control commands
\item[1] cursorq - queue for sending cursor updates
\end{description}

Both queues have the same format.  Each request and each response have
a fixed header, followed by command specific data fields.  The
separate cursor queue is the "fast track" for cursor commands
(VIRTIO_GPU_CMD_UPDATE_CURSOR and VIRTIO_GPU_CMD_MOVE_CURSOR), so they
go through without being delayed by time-consuming commands in the
control queue.

\subsection{Feature bits}\label{sec:Device Types / GPU Device / Feature bits}

\begin{description}
\item[VIRTIO_GPU_F_VIRGL (0)] virgl 3D mode is supported.
\item[VIRTIO_GPU_F_EDID  (1)] EDID is supported.
\item[VIRTIO_GPU_F_RESOURCE_UUID (2)] assigning resources UUIDs for export
  to other virtio devices is supported.
\item[VIRTIO_GPU_F_RESOURCE_BLOB (3)] creating and using size-based blob
  resources is supported.
\item[VIRTIO_GPU_F_CONTEXT_INIT (4)] multiple context types and
  synchronization timelines supported.  Requires VIRTIO_GPU_F_VIRGL.
\end{description}

\subsection{Device configuration layout}\label{sec:Device Types / GPU Device / Device configuration layout}

GPU device configuration uses the following layout structure and
definitions:

\begin{lstlisting}
#define VIRTIO_GPU_EVENT_DISPLAY (1 << 0)

struct virtio_gpu_config {
        le32 events_read;
        le32 events_clear;
        le32 num_scanouts;
        le32 num_capsets;
};
\end{lstlisting}

\subsubsection{Device configuration fields}

\begin{description}
\item[\field{events_read}] signals pending events to the driver.  The
  driver MUST NOT write to this field.
\item[\field{events_clear}] clears pending events in the device.
  Writing a '1' into a bit will clear the corresponding bit in
  \field{events_read}, mimicking write-to-clear behavior.
\item[\field{num_scanouts}] specifies the maximum number of scanouts
  supported by the device.  Minimum value is 1, maximum value is 16.
\item[\field{num_capsets}] specifies the maximum number of capability
  sets supported by the device.  The minimum value is zero.
\end{description}

\subsubsection{Events}

\begin{description}
\item[VIRTIO_GPU_EVENT_DISPLAY] Display configuration has changed.
  The driver SHOULD use the VIRTIO_GPU_CMD_GET_DISPLAY_INFO command to
  fetch the information from the device.  In case EDID support is
  negotiated (VIRTIO_GPU_F_EDID feature flag) the device SHOULD also
  fetch the updated EDID blobs using the VIRTIO_GPU_CMD_GET_EDID
  command.
\end{description}

\devicenormative{\subsection}{Device Initialization}{Device Types / GPU Device / Device Initialization}

The driver SHOULD query the display information from the device using
the VIRTIO_GPU_CMD_GET_DISPLAY_INFO command and use that information
for the initial scanout setup.  In case EDID support is negotiated
(VIRTIO_GPU_F_EDID feature flag) the device SHOULD also fetch the EDID
information using the VIRTIO_GPU_CMD_GET_EDID command.  If no
information is available or all displays are disabled the driver MAY
choose to use a fallback, such as 1024x768 at display 0.

The driver SHOULD query all shared memory regions supported by the device.
If the device supports shared memory, the \field{shmid} of a region MUST
(see \ref{sec:Basic Facilities of a Virtio Device /
Shared Memory Regions}~\nameref{sec:Basic Facilities of a Virtio Device /
Shared Memory Regions}) be one of the following:

\begin{lstlisting}
enum virtio_gpu_shm_id {
        VIRTIO_GPU_SHM_ID_UNDEFINED = 0,
        VIRTIO_GPU_SHM_ID_HOST_VISIBLE = 1,
};
\end{lstlisting}

The shared memory region with VIRTIO_GPU_SHM_ID_HOST_VISIBLE is referred as
the "host visible memory region".  The device MUST support the
VIRTIO_GPU_CMD_RESOURCE_MAP_BLOB and VIRTIO_GPU_CMD_RESOURCE_UNMAP_BLOB
if the host visible memory region is available.

\subsection{Device Operation}\label{sec:Device Types / GPU Device / Device Operation}

The virtio-gpu is based around the concept of resources private to the
host.  The guest must DMA transfer into these resources, unless shared memory
regions are supported. This is a design requirement in order to interface with
future 3D rendering. In the unaccelerated 2D mode there is no support for DMA
transfers from resources, just to them.

Resources are initially simple 2D resources, consisting of a width,
height and format along with an identifier. The guest must then attach
backing store to the resources in order for DMA transfers to
work. This is like a GART in a real GPU.

\subsubsection{Device Operation: Create a framebuffer and configure scanout}

\begin{itemize*}
\item Create a host resource using VIRTIO_GPU_CMD_RESOURCE_CREATE_2D.
\item Allocate a framebuffer from guest ram, and attach it as backing
  storage to the resource just created, using
  VIRTIO_GPU_CMD_RESOURCE_ATTACH_BACKING.  Scatter lists are
  supported, so the framebuffer doesn't need to be contignous in guest
  physical memory.
\item Use VIRTIO_GPU_CMD_SET_SCANOUT to link the framebuffer to
  a display scanout.
\end{itemize*}

\subsubsection{Device Operation: Update a framebuffer and scanout}

\begin{itemize*}
\item Render to your framebuffer memory.
\item Use VIRTIO_GPU_CMD_TRANSFER_TO_HOST_2D to update the host resource
  from guest memory.
\item Use VIRTIO_GPU_CMD_RESOURCE_FLUSH to flush the updated resource
  to the display.
\end{itemize*}

\subsubsection{Device Operation: Using pageflip}

It is possible to create multiple framebuffers, flip between them
using VIRTIO_GPU_CMD_SET_SCANOUT and VIRTIO_GPU_CMD_RESOURCE_FLUSH,
and update the invisible framebuffer using
VIRTIO_GPU_CMD_TRANSFER_TO_HOST_2D.

\subsubsection{Device Operation: Multihead setup}

In case two or more displays are present there are different ways to
configure things:

\begin{itemize*}
\item Create a single framebuffer, link it to all displays
  (mirroring).
\item Create an framebuffer for each display.
\item Create one big framebuffer, configure scanouts to display a
  different rectangle of that framebuffer each.
\end{itemize*}

\devicenormative{\subsubsection}{Device Operation: Command lifecycle and fencing}{Device Types / GPU Device / Device Operation / Device Operation: Command lifecycle and fencing}

The device MAY process controlq commands asyncronously and return them
to the driver before the processing is complete.  If the driver needs
to know when the processing is finished it can set the
VIRTIO_GPU_FLAG_FENCE flag in the request.  The device MUST finish the
processing before returning the command then.

Note: current qemu implementation does asyncrounous processing only in
3d mode, when offloading the processing to the host gpu.

\subsubsection{Device Operation: Configure mouse cursor}

The mouse cursor image is a normal resource, except that it must be
64x64 in size.  The driver MUST create and populate the resource
(using the usual VIRTIO_GPU_CMD_RESOURCE_CREATE_2D,
VIRTIO_GPU_CMD_RESOURCE_ATTACH_BACKING and
VIRTIO_GPU_CMD_TRANSFER_TO_HOST_2D controlq commands) and make sure they
are completed (using VIRTIO_GPU_FLAG_FENCE).

Then VIRTIO_GPU_CMD_UPDATE_CURSOR can be sent to the cursorq to set
the pointer shape and position.  To move the pointer without updating
the shape use VIRTIO_GPU_CMD_MOVE_CURSOR instead.

\subsubsection{Device Operation: Request header}\label{sec:Device Types / GPU Device / Device Operation / Device Operation: Request header}

All requests and responses on the virtqueues have a fixed header
using the following layout structure and definitions:

\begin{lstlisting}
enum virtio_gpu_ctrl_type {

        /* 2d commands */
        VIRTIO_GPU_CMD_GET_DISPLAY_INFO = 0x0100,
        VIRTIO_GPU_CMD_RESOURCE_CREATE_2D,
        VIRTIO_GPU_CMD_RESOURCE_UNREF,
        VIRTIO_GPU_CMD_SET_SCANOUT,
        VIRTIO_GPU_CMD_RESOURCE_FLUSH,
        VIRTIO_GPU_CMD_TRANSFER_TO_HOST_2D,
        VIRTIO_GPU_CMD_RESOURCE_ATTACH_BACKING,
        VIRTIO_GPU_CMD_RESOURCE_DETACH_BACKING,
        VIRTIO_GPU_CMD_GET_CAPSET_INFO,
        VIRTIO_GPU_CMD_GET_CAPSET,
        VIRTIO_GPU_CMD_GET_EDID,
        VIRTIO_GPU_CMD_RESOURCE_ASSIGN_UUID,
        VIRTIO_GPU_CMD_RESOURCE_CREATE_BLOB,
        VIRTIO_GPU_CMD_SET_SCANOUT_BLOB,

        /* 3d commands */
        VIRTIO_GPU_CMD_CTX_CREATE = 0x0200,
        VIRTIO_GPU_CMD_CTX_DESTROY,
        VIRTIO_GPU_CMD_CTX_ATTACH_RESOURCE,
        VIRTIO_GPU_CMD_CTX_DETACH_RESOURCE,
        VIRTIO_GPU_CMD_RESOURCE_CREATE_3D,
        VIRTIO_GPU_CMD_TRANSFER_TO_HOST_3D,
        VIRTIO_GPU_CMD_TRANSFER_FROM_HOST_3D,
        VIRTIO_GPU_CMD_SUBMIT_3D,
        VIRTIO_GPU_CMD_RESOURCE_MAP_BLOB,
        VIRTIO_GPU_CMD_RESOURCE_UNMAP_BLOB,

        /* cursor commands */
        VIRTIO_GPU_CMD_UPDATE_CURSOR = 0x0300,
        VIRTIO_GPU_CMD_MOVE_CURSOR,

        /* success responses */
        VIRTIO_GPU_RESP_OK_NODATA = 0x1100,
        VIRTIO_GPU_RESP_OK_DISPLAY_INFO,
        VIRTIO_GPU_RESP_OK_CAPSET_INFO,
        VIRTIO_GPU_RESP_OK_CAPSET,
        VIRTIO_GPU_RESP_OK_EDID,
        VIRTIO_GPU_RESP_OK_RESOURCE_UUID,
        VIRTIO_GPU_RESP_OK_MAP_INFO,

        /* error responses */
        VIRTIO_GPU_RESP_ERR_UNSPEC = 0x1200,
        VIRTIO_GPU_RESP_ERR_OUT_OF_MEMORY,
        VIRTIO_GPU_RESP_ERR_INVALID_SCANOUT_ID,
        VIRTIO_GPU_RESP_ERR_INVALID_RESOURCE_ID,
        VIRTIO_GPU_RESP_ERR_INVALID_CONTEXT_ID,
        VIRTIO_GPU_RESP_ERR_INVALID_PARAMETER,
};

#define VIRTIO_GPU_FLAG_FENCE (1 << 0)
#define VIRTIO_GPU_FLAG_INFO_RING_IDX (1 << 1)

struct virtio_gpu_ctrl_hdr {
        le32 type;
        le32 flags;
        le64 fence_id;
        le32 ctx_id;
        u8 ring_idx;
        u8 padding[3];
};
\end{lstlisting}

The fixed header \field{struct virtio_gpu_ctrl_hdr} in each
request includes the following fields:

\begin{description}
\item[\field{type}] specifies the type of the driver request
  (VIRTIO_GPU_CMD_*) or device response (VIRTIO_GPU_RESP_*).
\item[\field{flags}] request / response flags.
\item[\field{fence_id}] If the driver sets the VIRTIO_GPU_FLAG_FENCE
  bit in the request \field{flags} field the device MUST:
  \begin{itemize*}
  \item set VIRTIO_GPU_FLAG_FENCE bit in the response,
  \item copy the content of the \field{fence_id} field from the
    request to the response, and
  \item send the response only after command processing is complete.
  \end{itemize*}
\item[\field{ctx_id}] Rendering context (used in 3D mode only).
\item[\field{ring_idx}] If VIRTIO_GPU_F_CONTEXT_INIT is supported, then
  the driver MAY set VIRTIO_GPU_FLAG_INFO_RING_IDX bit in the request
  \field{flags}.  In that case:
  \begin{itemize*}
  \item \field{ring_idx} indicates the value of a context-specific ring
   index.  The minimum value is 0 and maximum value is 63 (inclusive).
  \item If VIRTIO_GPU_FLAG_FENCE is set, \field{fence_id} acts as a
   sequence number on the synchronization timeline defined by
   \field{ctx_idx} and the ring index.
  \item If VIRTIO_GPU_FLAG_FENCE is set and when the command associated
   with \field{fence_id} is complete, the device MUST send a response for
   all outstanding commands with a sequence number less than or equal to
   \field{fence_id} on the same synchronization timeline.
  \end{itemize*}
\end{description}

On success the device will return VIRTIO_GPU_RESP_OK_NODATA in
case there is no payload.  Otherwise the \field{type} field will
indicate the kind of payload.

On error the device will return one of the
VIRTIO_GPU_RESP_ERR_* error codes.

\subsubsection{Device Operation: controlq}\label{sec:Device Types / GPU Device / Device Operation / Device Operation: controlq}

For any coordinates given 0,0 is top left, larger x moves right,
larger y moves down.

\begin{description}

\item[VIRTIO_GPU_CMD_GET_DISPLAY_INFO] Retrieve the current output
  configuration.  No request data (just bare \field{struct
    virtio_gpu_ctrl_hdr}).  Response type is
  VIRTIO_GPU_RESP_OK_DISPLAY_INFO, response data is \field{struct
    virtio_gpu_resp_display_info}.

\begin{lstlisting}
#define VIRTIO_GPU_MAX_SCANOUTS 16

struct virtio_gpu_rect {
        le32 x;
        le32 y;
        le32 width;
        le32 height;
};

struct virtio_gpu_resp_display_info {
        struct virtio_gpu_ctrl_hdr hdr;
        struct virtio_gpu_display_one {
                struct virtio_gpu_rect r;
                le32 enabled;
                le32 flags;
        } pmodes[VIRTIO_GPU_MAX_SCANOUTS];
};
\end{lstlisting}

The response contains a list of per-scanout information.  The info
contains whether the scanout is enabled and what its preferred
position and size is.

The size (fields \field{width} and \field{height}) is similar to the
native panel resolution in EDID display information, except that in
the virtual machine case the size can change when the host window
representing the guest display is gets resized.

The position (fields \field{x} and \field{y}) describe how the
displays are arranged (i.e. which is -- for example -- the left
display).

The \field{enabled} field is set when the user enabled the display.
It is roughly the same as the connected state of a phyiscal display
connector.

\item[VIRTIO_GPU_CMD_GET_EDID] Retrieve the EDID data for a given
  scanout.  Request data is \field{struct virtio_gpu_get_edid}).
  Response type is VIRTIO_GPU_RESP_OK_EDID, response data is
  \field{struct virtio_gpu_resp_edid}.  Support is optional and
  negotiated using the VIRTIO_GPU_F_EDID feature flag.

\begin{lstlisting}
struct virtio_gpu_get_edid {
        struct virtio_gpu_ctrl_hdr hdr;
        le32 scanout;
        le32 padding;
};

struct virtio_gpu_resp_edid {
        struct virtio_gpu_ctrl_hdr hdr;
        le32 size;
        le32 padding;
        u8 edid[1024];
};
\end{lstlisting}

The response contains the EDID display data blob (as specified by
VESA) for the scanout.

\item[VIRTIO_GPU_CMD_RESOURCE_CREATE_2D] Create a 2D resource on the
  host.  Request data is \field{struct virtio_gpu_resource_create_2d}.
  Response type is VIRTIO_GPU_RESP_OK_NODATA.

\begin{lstlisting}
enum virtio_gpu_formats {
        VIRTIO_GPU_FORMAT_B8G8R8A8_UNORM  = 1,
        VIRTIO_GPU_FORMAT_B8G8R8X8_UNORM  = 2,
        VIRTIO_GPU_FORMAT_A8R8G8B8_UNORM  = 3,
        VIRTIO_GPU_FORMAT_X8R8G8B8_UNORM  = 4,

        VIRTIO_GPU_FORMAT_R8G8B8A8_UNORM  = 67,
        VIRTIO_GPU_FORMAT_X8B8G8R8_UNORM  = 68,

        VIRTIO_GPU_FORMAT_A8B8G8R8_UNORM  = 121,
        VIRTIO_GPU_FORMAT_R8G8B8X8_UNORM  = 134,
};

struct virtio_gpu_resource_create_2d {
        struct virtio_gpu_ctrl_hdr hdr;
        le32 resource_id;
        le32 format;
        le32 width;
        le32 height;
};
\end{lstlisting}

This creates a 2D resource on the host with the specified width,
height and format.  The resource ids are generated by the guest.

\item[VIRTIO_GPU_CMD_RESOURCE_UNREF] Destroy a resource.  Request data
  is \field{struct virtio_gpu_resource_unref}.  Response type is
  VIRTIO_GPU_RESP_OK_NODATA.

\begin{lstlisting}
struct virtio_gpu_resource_unref {
        struct virtio_gpu_ctrl_hdr hdr;
        le32 resource_id;
        le32 padding;
};
\end{lstlisting}

This informs the host that a resource is no longer required by the
guest.

\item[VIRTIO_GPU_CMD_SET_SCANOUT] Set the scanout parameters for a
  single output.  Request data is \field{struct
    virtio_gpu_set_scanout}.  Response type is
  VIRTIO_GPU_RESP_OK_NODATA.

\begin{lstlisting}
struct virtio_gpu_set_scanout {
        struct virtio_gpu_ctrl_hdr hdr;
        struct virtio_gpu_rect r;
        le32 scanout_id;
        le32 resource_id;
};
\end{lstlisting}

This sets the scanout parameters for a single scanout. The resource_id
is the resource to be scanned out from, along with a rectangle.

Scanout rectangles must be completely covered by the underlying
resource.  Overlapping (or identical) scanouts are allowed, typical
use case is screen mirroring.

The driver can use resource_id = 0 to disable a scanout.

\item[VIRTIO_GPU_CMD_RESOURCE_FLUSH] Flush a scanout resource Request
  data is \field{struct virtio_gpu_resource_flush}.  Response type is
  VIRTIO_GPU_RESP_OK_NODATA.

\begin{lstlisting}
struct virtio_gpu_resource_flush {
        struct virtio_gpu_ctrl_hdr hdr;
        struct virtio_gpu_rect r;
        le32 resource_id;
        le32 padding;
};
\end{lstlisting}

This flushes a resource to screen.  It takes a rectangle and a
resource id, and flushes any scanouts the resource is being used on.

\item[VIRTIO_GPU_CMD_TRANSFER_TO_HOST_2D] Transfer from guest memory
  to host resource.  Request data is \field{struct
    virtio_gpu_transfer_to_host_2d}.  Response type is
  VIRTIO_GPU_RESP_OK_NODATA.

\begin{lstlisting}
struct virtio_gpu_transfer_to_host_2d {
        struct virtio_gpu_ctrl_hdr hdr;
        struct virtio_gpu_rect r;
        le64 offset;
        le32 resource_id;
        le32 padding;
};
\end{lstlisting}

This takes a resource id along with an destination offset into the
resource, and a box to transfer to the host backing for the resource.

\item[VIRTIO_GPU_CMD_RESOURCE_ATTACH_BACKING] Assign backing pages to
  a resource.  Request data is \field{struct
    virtio_gpu_resource_attach_backing}, followed by \field{struct
    virtio_gpu_mem_entry} entries.  Response type is
  VIRTIO_GPU_RESP_OK_NODATA.

\begin{lstlisting}
struct virtio_gpu_resource_attach_backing {
        struct virtio_gpu_ctrl_hdr hdr;
        le32 resource_id;
        le32 nr_entries;
};

struct virtio_gpu_mem_entry {
        le64 addr;
        le32 length;
        le32 padding;
};
\end{lstlisting}

This assign an array of guest pages as the backing store for a
resource. These pages are then used for the transfer operations for
that resource from that point on.

\item[VIRTIO_GPU_CMD_RESOURCE_DETACH_BACKING] Detach backing pages
  from a resource.  Request data is \field{struct
    virtio_gpu_resource_detach_backing}.  Response type is
  VIRTIO_GPU_RESP_OK_NODATA.

\begin{lstlisting}
struct virtio_gpu_resource_detach_backing {
        struct virtio_gpu_ctrl_hdr hdr;
        le32 resource_id;
        le32 padding;
};
\end{lstlisting}

This detaches any backing pages from a resource, to be used in case of
guest swapping or object destruction.

\item[VIRTIO_GPU_CMD_GET_CAPSET_INFO] Gets the information associated with
  a particular \field{capset_index}, which MUST less than \field{num_capsets}
  defined in the device configuration.  Request data is
  \field{struct virtio_gpu_get_capset_info}.  Response type is
  VIRTIO_GPU_RESP_OK_CAPSET_INFO.

  On success, \field{struct virtio_gpu_resp_capset_info} contains the
  \field{capset_id}, \field{capset_max_version}, \field{capset_max_size}
  associated with capset at the specified {capset_idex}.  field{capset_id} MUST
  be one of the following (see listing for values):

  \begin{itemize*}
  \item \href{https://gitlab.freedesktop.org/virgl/virglrenderer/-/blob/master/src/virgl_hw.h#L526}{VIRTIO_GPU_CAPSET_VIRGL} --
	the first edition of Virgl (Gallium OpenGL) protocol.
  \item \href{https://gitlab.freedesktop.org/virgl/virglrenderer/-/blob/master/src/virgl_hw.h#L550}{VIRTIO_GPU_CAPSET_VIRGL2} --
	the second edition of Virgl (Gallium OpenGL) protocol after the capset fix.
  \item \href{https://android.googlesource.com/device/generic/vulkan-cereal/+/refs/heads/master/protocols/}{VIRTIO_GPU_CAPSET_GFXSTREAM} --
	gfxtream's (mostly) autogenerated GLES and Vulkan streaming protocols.
  \item \href{https://gitlab.freedesktop.org/olv/venus-protocol}{VIRTIO_GPU_CAPSET_VENUS} --
	Mesa's (mostly) autogenerated Vulkan protocol.
  \item \href{https://chromium.googlesource.com/chromiumos/platform/crosvm/+/refs/heads/main/rutabaga_gfx/src/cross_domain/cross_domain_protocol.rs}{VIRTIO_GPU_CAPSET_CROSS_DOMAIN} --
	protocol for display virtualization via Wayland proxying.
  \end{itemize*}

\begin{lstlisting}
struct virtio_gpu_get_capset_info {
        struct virtio_gpu_ctrl_hdr hdr;
        le32 capset_index;
        le32 padding;
};

#define VIRTIO_GPU_CAPSET_VIRGL 1
#define VIRTIO_GPU_CAPSET_VIRGL2 2
#define VIRTIO_GPU_CAPSET_GFXSTREAM 3
#define VIRTIO_GPU_CAPSET_VENUS 4
#define VIRTIO_GPU_CAPSET_CROSS_DOMAIN 5
struct virtio_gpu_resp_capset_info {
        struct virtio_gpu_ctrl_hdr hdr;
        le32 capset_id;
        le32 capset_max_version;
        le32 capset_max_size;
        le32 padding;
};
\end{lstlisting}

\item[VIRTIO_GPU_CMD_GET_CAPSET] Gets the capset associated with a
  particular \field{capset_id} and \field{capset_version}.  Request data is
  \field{struct virtio_gpu_get_capset}.  Response type is
  VIRTIO_GPU_RESP_OK_CAPSET.

\begin{lstlisting}
struct virtio_gpu_get_capset {
        struct virtio_gpu_ctrl_hdr hdr;
        le32 capset_id;
        le32 capset_version;
};

struct virtio_gpu_resp_capset {
        struct virtio_gpu_ctrl_hdr hdr;
        u8 capset_data[];
};
\end{lstlisting}

\item[VIRTIO_GPU_CMD_RESOURCE_ASSIGN_UUID] Creates an exported object from
  a resource. Request data is \field{struct
    virtio_gpu_resource_assign_uuid}.  Response type is
  VIRTIO_GPU_RESP_OK_RESOURCE_UUID, response data is \field{struct
    virtio_gpu_resp_resource_uuid}. Support is optional and negotiated
    using the VIRTIO_GPU_F_RESOURCE_UUID feature flag.

\begin{lstlisting}
struct virtio_gpu_resource_assign_uuid {
        struct virtio_gpu_ctrl_hdr hdr;
        le32 resource_id;
        le32 padding;
};

struct virtio_gpu_resp_resource_uuid {
        struct virtio_gpu_ctrl_hdr hdr;
        u8 uuid[16];
};
\end{lstlisting}

The response contains a UUID which identifies the exported object created from
the host private resource. Note that if the resource has an attached backing,
modifications made to the host private resource through the exported object by
other devices are not visible in the attached backing until they are transferred
into the backing.

\item[VIRTIO_GPU_CMD_RESOURCE_CREATE_BLOB] Creates a virtio-gpu blob
  resource. Request data is \field{struct
  virtio_gpu_resource_create_blob}, followed by \field{struct
  virtio_gpu_mem_entry} entries. Response type is
  VIRTIO_GPU_RESP_OK_NODATA. Support is optional and negotiated
  using the VIRTIO_GPU_F_RESOURCE_BLOB feature flag.

\begin{lstlisting}
#define VIRTIO_GPU_BLOB_MEM_GUEST             0x0001
#define VIRTIO_GPU_BLOB_MEM_HOST3D            0x0002
#define VIRTIO_GPU_BLOB_MEM_HOST3D_GUEST      0x0003

#define VIRTIO_GPU_BLOB_FLAG_USE_MAPPABLE     0x0001
#define VIRTIO_GPU_BLOB_FLAG_USE_SHAREABLE    0x0002
#define VIRTIO_GPU_BLOB_FLAG_USE_CROSS_DEVICE 0x0004

struct virtio_gpu_resource_create_blob {
       struct virtio_gpu_ctrl_hdr hdr;
       le32 resource_id;
       le32 blob_mem;
       le32 blob_flags;
       le32 nr_entries;
       le64 blob_id;
       le64 size;
};

\end{lstlisting}

A blob resource is a container for:

  \begin{itemize*}
  \item a guest memory allocation (referred to as a
  "guest-only blob resource").
  \item a host memory allocation (referred to as a
  "host-only blob resource").
  \item a guest memory and host memory allocation (referred
  to as a "default blob resource").
  \end{itemize*}

The memory properties of the blob resource MUST be described by
\field{blob_mem}, which MUST be non-zero.

For default and guest-only blob resources, \field{nr_entries} guest
memory entries may be assigned to the resource.  For default blob resources
(i.e, when \field{blob_mem} is VIRTIO_GPU_BLOB_MEM_HOST3D_GUEST), these
memory entries are used as a shadow buffer for the host memory. To
facilitate drivers that support swap-in and swap-out, \field{nr_entries} may
be zero and VIRTIO_GPU_CMD_RESOURCE_ATTACH_BACKING may be subsequently used.
VIRTIO_GPU_CMD_RESOURCE_DETACH_BACKING may be used to unassign memory entries.

\field{blob_mem} can only be VIRTIO_GPU_BLOB_MEM_HOST3D and
VIRTIO_GPU_BLOB_MEM_HOST3D_GUEST if VIRTIO_GPU_F_VIRGL is supported.
VIRTIO_GPU_BLOB_MEM_GUEST is valid regardless whether VIRTIO_GPU_F_VIRGL
is supported or not.

For VIRTIO_GPU_BLOB_MEM_HOST3D and VIRTIO_GPU_BLOB_MEM_HOST3D_GUEST, the
virtio-gpu resource MUST be created from the rendering context local object
identified by the \field{blob_id}. The actual allocation is done via
VIRTIO_GPU_CMD_SUBMIT_3D.

The driver MUST inform the device if the blob resource is used for
memory access, sharing between driver instances and/or sharing with
other devices. This is done via the \field{blob_flags} field.

If VIRTIO_GPU_F_VIRGL is set, both VIRTIO_GPU_CMD_TRANSFER_TO_HOST_3D
and VIRTIO_GPU_CMD_TRANSFER_FROM_HOST_3D may be used to update the
resource. There is no restriction on the image/buffer view the driver
has on the blob resource.

\item[VIRTIO_GPU_CMD_SET_SCANOUT_BLOB] sets scanout parameters for a
   blob resource. Request data is
  \field{struct virtio_gpu_set_scanout_blob}. Response type is
  VIRTIO_GPU_RESP_OK_NODATA. Support is optional and negotiated
  using the VIRTIO_GPU_F_RESOURCE_BLOB feature flag.

\begin{lstlisting}
struct virtio_gpu_set_scanout_blob {
       struct virtio_gpu_ctrl_hdr hdr;
       struct virtio_gpu_rect r;
       le32 scanout_id;
       le32 resource_id;
       le32 width;
       le32 height;
       le32 format;
       le32 padding;
       le32 strides[4];
       le32 offsets[4];
};
\end{lstlisting}

The rectangle \field{r} represents the portion of the blob resource being
displayed. The rest is the metadata associated with the blob resource. The
format MUST be one of \field{enum virtio_gpu_formats}.  The format MAY be
compressed with header and data planes.

\end{description}

\subsubsection{Device Operation: controlq (3d)}\label{sec:Device Types / GPU Device / Device Operation / Device Operation: controlq (3d)}

These commands are supported by the device if the VIRTIO_GPU_F_VIRGL
feature flag is set.

\begin{description}

\item[VIRTIO_GPU_CMD_CTX_CREATE] creates a context for submitting an opaque
  command stream.  Request data is \field{struct virtio_gpu_ctx_create}.
  Response type is VIRTIO_GPU_RESP_OK_NODATA.

\begin{lstlisting}
#define VIRTIO_GPU_CONTEXT_INIT_CAPSET_ID_MASK 0x000000ff;
struct virtio_gpu_ctx_create {
       struct virtio_gpu_ctrl_hdr hdr;
       le32 nlen;
       le32 context_init;
       char debug_name[64];
};
\end{lstlisting}

The implementation MUST create a context for the given \field{ctx_id} in
the \field{hdr}.  For debugging purposes, a \field{debug_name} and it's
length \field{nlen} is provided by the driver.  If
VIRTIO_GPU_F_CONTEXT_INIT is supported, then lower 8 bits of
\field{context_init} MAY contain the \field{capset_id} associated with
context.  In that case, then the device MUST create a context that can
handle the specified command stream.

If the lower 8-bits of the \field{context_init} are zero, then the type of
the context is determined by the device.

\item[VIRTIO_GPU_CMD_CTX_DESTROY]
\item[VIRTIO_GPU_CMD_CTX_ATTACH_RESOURCE]
\item[VIRTIO_GPU_CMD_CTX_DETACH_RESOURCE]
  Manage virtio-gpu 3d contexts.

\item[VIRTIO_GPU_CMD_RESOURCE_CREATE_3D]
  Create virtio-gpu 3d resources.

\item[VIRTIO_GPU_CMD_TRANSFER_TO_HOST_3D]
\item[VIRTIO_GPU_CMD_TRANSFER_FROM_HOST_3D]
  Transfer data from and to virtio-gpu 3d resources.

\item[VIRTIO_GPU_CMD_SUBMIT_3D]
  Submit an opaque command stream.  The type of the command stream is
  determined when creating a context.

\item[VIRTIO_GPU_CMD_RESOURCE_MAP_BLOB] maps a host-only
  blob resource into an offset in the host visible memory region. Request
  data is \field{struct virtio_gpu_resource_map_blob}.  The driver MUST
  not map a blob resource that is already mapped.  Response type is
  VIRTIO_GPU_RESP_OK_MAP_INFO. Support is optional and negotiated
  using the VIRTIO_GPU_F_RESOURCE_BLOB feature flag and checking for
  the presence of the host visible memory region.

\begin{lstlisting}
struct virtio_gpu_resource_map_blob {
        struct virtio_gpu_ctrl_hdr hdr;
        le32 resource_id;
        le32 padding;
        le64 offset;
};

#define VIRTIO_GPU_MAP_CACHE_MASK      0x0f
#define VIRTIO_GPU_MAP_CACHE_NONE      0x00
#define VIRTIO_GPU_MAP_CACHE_CACHED    0x01
#define VIRTIO_GPU_MAP_CACHE_UNCACHED  0x02
#define VIRTIO_GPU_MAP_CACHE_WC        0x03
struct virtio_gpu_resp_map_info {
        struct virtio_gpu_ctrl_hdr hdr;
        u32 map_info;
        u32 padding;
};
\end{lstlisting}

\item[VIRTIO_GPU_CMD_RESOURCE_UNMAP_BLOB] unmaps a
  host-only blob resource from the host visible memory region. Request data
  is \field{struct virtio_gpu_resource_unmap_blob}.  Response type is
  VIRTIO_GPU_RESP_OK_NODATA.  Support is optional and negotiated
  using the VIRTIO_GPU_F_RESOURCE_BLOB feature flag and checking for
  the presence of the host visible memory region.

\begin{lstlisting}
struct virtio_gpu_resource_unmap_blob {
        struct virtio_gpu_ctrl_hdr hdr;
        le32 resource_id;
        le32 padding;
};
\end{lstlisting}

\end{description}

\subsubsection{Device Operation: cursorq}\label{sec:Device Types / GPU Device / Device Operation / Device Operation: cursorq}

Both cursorq commands use the same command struct.

\begin{lstlisting}
struct virtio_gpu_cursor_pos {
        le32 scanout_id;
        le32 x;
        le32 y;
        le32 padding;
};

struct virtio_gpu_update_cursor {
        struct virtio_gpu_ctrl_hdr hdr;
        struct virtio_gpu_cursor_pos pos;
        le32 resource_id;
        le32 hot_x;
        le32 hot_y;
        le32 padding;
};
\end{lstlisting}

\begin{description}

\item[VIRTIO_GPU_CMD_UPDATE_CURSOR]
Update cursor.
Request data is \field{struct virtio_gpu_update_cursor}.
Response type is VIRTIO_GPU_RESP_OK_NODATA.

Full cursor update.  Cursor will be loaded from the specified
\field{resource_id} and will be moved to \field{pos}.  The driver must
transfer the cursor into the resource beforehand (using control queue
commands) and make sure the commands to fill the resource are actually
processed (using fencing).

\item[VIRTIO_GPU_CMD_MOVE_CURSOR]
Move cursor.
Request data is \field{struct virtio_gpu_update_cursor}.
Response type is VIRTIO_GPU_RESP_OK_NODATA.

Move cursor to the place specified in \field{pos}.  The other fields
are not used and will be ignored by the device.

\end{description}

\subsection{VGA Compatibility}\label{sec:Device Types / GPU Device / VGA Compatibility}

Applies to Virtio Over PCI only.  The GPU device can come with and
without VGA compatibility.  The PCI class should be DISPLAY_VGA if VGA
compatibility is present and DISPLAY_OTHER otherwise.

VGA compatibility: PCI region 0 has the linear framebuffer, standard
vga registers are present.  Configuring a scanout
(VIRTIO_GPU_CMD_SET_SCANOUT) switches the device from vga
compatibility mode into native virtio mode.  A reset switches it back
into vga compatibility mode.

Note: qemu implementation also provides bochs dispi interface io ports
and mmio bar at pci region 1 and is therefore fully compatible with
the qemu stdvga (see \href{https://git.qemu-project.org/?p=qemu.git;a=blob;f=docs/specs/standard-vga.txt;hb=HEAD}{docs/specs/standard-vga.txt} in the qemu source tree).

\section{GPU Device}\label{sec:Device Types / GPU Device}

virtio-gpu is a virtio based graphics adapter.  It can operate in 2D
mode and in 3D mode.  3D mode will offload rendering ops to
the host gpu and therefore requires a gpu with 3D support on the host
machine.

In 2D mode the virtio-gpu device provides support for ARGB Hardware
cursors and multiple scanouts (aka heads).

\subsection{Device ID}\label{sec:Device Types / GPU Device / Device ID}

16

\subsection{Virtqueues}\label{sec:Device Types / GPU Device / Virtqueues}

\begin{description}
\item[0] controlq - queue for sending control commands
\item[1] cursorq - queue for sending cursor updates
\end{description}

Both queues have the same format.  Each request and each response have
a fixed header, followed by command specific data fields.  The
separate cursor queue is the "fast track" for cursor commands
(VIRTIO_GPU_CMD_UPDATE_CURSOR and VIRTIO_GPU_CMD_MOVE_CURSOR), so they
go through without being delayed by time-consuming commands in the
control queue.

\subsection{Feature bits}\label{sec:Device Types / GPU Device / Feature bits}

\begin{description}
\item[VIRTIO_GPU_F_VIRGL (0)] virgl 3D mode is supported.
\item[VIRTIO_GPU_F_EDID  (1)] EDID is supported.
\item[VIRTIO_GPU_F_RESOURCE_UUID (2)] assigning resources UUIDs for export
  to other virtio devices is supported.
\item[VIRTIO_GPU_F_RESOURCE_BLOB (3)] creating and using size-based blob
  resources is supported.
\item[VIRTIO_GPU_F_CONTEXT_INIT (4)] multiple context types and
  synchronization timelines supported.  Requires VIRTIO_GPU_F_VIRGL.
\end{description}

\subsection{Device configuration layout}\label{sec:Device Types / GPU Device / Device configuration layout}

GPU device configuration uses the following layout structure and
definitions:

\begin{lstlisting}
#define VIRTIO_GPU_EVENT_DISPLAY (1 << 0)

struct virtio_gpu_config {
        le32 events_read;
        le32 events_clear;
        le32 num_scanouts;
        le32 num_capsets;
};
\end{lstlisting}

\subsubsection{Device configuration fields}

\begin{description}
\item[\field{events_read}] signals pending events to the driver.  The
  driver MUST NOT write to this field.
\item[\field{events_clear}] clears pending events in the device.
  Writing a '1' into a bit will clear the corresponding bit in
  \field{events_read}, mimicking write-to-clear behavior.
\item[\field{num_scanouts}] specifies the maximum number of scanouts
  supported by the device.  Minimum value is 1, maximum value is 16.
\item[\field{num_capsets}] specifies the maximum number of capability
  sets supported by the device.  The minimum value is zero.
\end{description}

\subsubsection{Events}

\begin{description}
\item[VIRTIO_GPU_EVENT_DISPLAY] Display configuration has changed.
  The driver SHOULD use the VIRTIO_GPU_CMD_GET_DISPLAY_INFO command to
  fetch the information from the device.  In case EDID support is
  negotiated (VIRTIO_GPU_F_EDID feature flag) the device SHOULD also
  fetch the updated EDID blobs using the VIRTIO_GPU_CMD_GET_EDID
  command.
\end{description}

\devicenormative{\subsection}{Device Initialization}{Device Types / GPU Device / Device Initialization}

The driver SHOULD query the display information from the device using
the VIRTIO_GPU_CMD_GET_DISPLAY_INFO command and use that information
for the initial scanout setup.  In case EDID support is negotiated
(VIRTIO_GPU_F_EDID feature flag) the device SHOULD also fetch the EDID
information using the VIRTIO_GPU_CMD_GET_EDID command.  If no
information is available or all displays are disabled the driver MAY
choose to use a fallback, such as 1024x768 at display 0.

The driver SHOULD query all shared memory regions supported by the device.
If the device supports shared memory, the \field{shmid} of a region MUST
(see \ref{sec:Basic Facilities of a Virtio Device /
Shared Memory Regions}~\nameref{sec:Basic Facilities of a Virtio Device /
Shared Memory Regions}) be one of the following:

\begin{lstlisting}
enum virtio_gpu_shm_id {
        VIRTIO_GPU_SHM_ID_UNDEFINED = 0,
        VIRTIO_GPU_SHM_ID_HOST_VISIBLE = 1,
};
\end{lstlisting}

The shared memory region with VIRTIO_GPU_SHM_ID_HOST_VISIBLE is referred as
the "host visible memory region".  The device MUST support the
VIRTIO_GPU_CMD_RESOURCE_MAP_BLOB and VIRTIO_GPU_CMD_RESOURCE_UNMAP_BLOB
if the host visible memory region is available.

\subsection{Device Operation}\label{sec:Device Types / GPU Device / Device Operation}

The virtio-gpu is based around the concept of resources private to the
host.  The guest must DMA transfer into these resources, unless shared memory
regions are supported. This is a design requirement in order to interface with
future 3D rendering. In the unaccelerated 2D mode there is no support for DMA
transfers from resources, just to them.

Resources are initially simple 2D resources, consisting of a width,
height and format along with an identifier. The guest must then attach
backing store to the resources in order for DMA transfers to
work. This is like a GART in a real GPU.

\subsubsection{Device Operation: Create a framebuffer and configure scanout}

\begin{itemize*}
\item Create a host resource using VIRTIO_GPU_CMD_RESOURCE_CREATE_2D.
\item Allocate a framebuffer from guest ram, and attach it as backing
  storage to the resource just created, using
  VIRTIO_GPU_CMD_RESOURCE_ATTACH_BACKING.  Scatter lists are
  supported, so the framebuffer doesn't need to be contignous in guest
  physical memory.
\item Use VIRTIO_GPU_CMD_SET_SCANOUT to link the framebuffer to
  a display scanout.
\end{itemize*}

\subsubsection{Device Operation: Update a framebuffer and scanout}

\begin{itemize*}
\item Render to your framebuffer memory.
\item Use VIRTIO_GPU_CMD_TRANSFER_TO_HOST_2D to update the host resource
  from guest memory.
\item Use VIRTIO_GPU_CMD_RESOURCE_FLUSH to flush the updated resource
  to the display.
\end{itemize*}

\subsubsection{Device Operation: Using pageflip}

It is possible to create multiple framebuffers, flip between them
using VIRTIO_GPU_CMD_SET_SCANOUT and VIRTIO_GPU_CMD_RESOURCE_FLUSH,
and update the invisible framebuffer using
VIRTIO_GPU_CMD_TRANSFER_TO_HOST_2D.

\subsubsection{Device Operation: Multihead setup}

In case two or more displays are present there are different ways to
configure things:

\begin{itemize*}
\item Create a single framebuffer, link it to all displays
  (mirroring).
\item Create an framebuffer for each display.
\item Create one big framebuffer, configure scanouts to display a
  different rectangle of that framebuffer each.
\end{itemize*}

\devicenormative{\subsubsection}{Device Operation: Command lifecycle and fencing}{Device Types / GPU Device / Device Operation / Device Operation: Command lifecycle and fencing}

The device MAY process controlq commands asyncronously and return them
to the driver before the processing is complete.  If the driver needs
to know when the processing is finished it can set the
VIRTIO_GPU_FLAG_FENCE flag in the request.  The device MUST finish the
processing before returning the command then.

Note: current qemu implementation does asyncrounous processing only in
3d mode, when offloading the processing to the host gpu.

\subsubsection{Device Operation: Configure mouse cursor}

The mouse cursor image is a normal resource, except that it must be
64x64 in size.  The driver MUST create and populate the resource
(using the usual VIRTIO_GPU_CMD_RESOURCE_CREATE_2D,
VIRTIO_GPU_CMD_RESOURCE_ATTACH_BACKING and
VIRTIO_GPU_CMD_TRANSFER_TO_HOST_2D controlq commands) and make sure they
are completed (using VIRTIO_GPU_FLAG_FENCE).

Then VIRTIO_GPU_CMD_UPDATE_CURSOR can be sent to the cursorq to set
the pointer shape and position.  To move the pointer without updating
the shape use VIRTIO_GPU_CMD_MOVE_CURSOR instead.

\subsubsection{Device Operation: Request header}\label{sec:Device Types / GPU Device / Device Operation / Device Operation: Request header}

All requests and responses on the virtqueues have a fixed header
using the following layout structure and definitions:

\begin{lstlisting}
enum virtio_gpu_ctrl_type {

        /* 2d commands */
        VIRTIO_GPU_CMD_GET_DISPLAY_INFO = 0x0100,
        VIRTIO_GPU_CMD_RESOURCE_CREATE_2D,
        VIRTIO_GPU_CMD_RESOURCE_UNREF,
        VIRTIO_GPU_CMD_SET_SCANOUT,
        VIRTIO_GPU_CMD_RESOURCE_FLUSH,
        VIRTIO_GPU_CMD_TRANSFER_TO_HOST_2D,
        VIRTIO_GPU_CMD_RESOURCE_ATTACH_BACKING,
        VIRTIO_GPU_CMD_RESOURCE_DETACH_BACKING,
        VIRTIO_GPU_CMD_GET_CAPSET_INFO,
        VIRTIO_GPU_CMD_GET_CAPSET,
        VIRTIO_GPU_CMD_GET_EDID,
        VIRTIO_GPU_CMD_RESOURCE_ASSIGN_UUID,
        VIRTIO_GPU_CMD_RESOURCE_CREATE_BLOB,
        VIRTIO_GPU_CMD_SET_SCANOUT_BLOB,

        /* 3d commands */
        VIRTIO_GPU_CMD_CTX_CREATE = 0x0200,
        VIRTIO_GPU_CMD_CTX_DESTROY,
        VIRTIO_GPU_CMD_CTX_ATTACH_RESOURCE,
        VIRTIO_GPU_CMD_CTX_DETACH_RESOURCE,
        VIRTIO_GPU_CMD_RESOURCE_CREATE_3D,
        VIRTIO_GPU_CMD_TRANSFER_TO_HOST_3D,
        VIRTIO_GPU_CMD_TRANSFER_FROM_HOST_3D,
        VIRTIO_GPU_CMD_SUBMIT_3D,
        VIRTIO_GPU_CMD_RESOURCE_MAP_BLOB,
        VIRTIO_GPU_CMD_RESOURCE_UNMAP_BLOB,

        /* cursor commands */
        VIRTIO_GPU_CMD_UPDATE_CURSOR = 0x0300,
        VIRTIO_GPU_CMD_MOVE_CURSOR,

        /* success responses */
        VIRTIO_GPU_RESP_OK_NODATA = 0x1100,
        VIRTIO_GPU_RESP_OK_DISPLAY_INFO,
        VIRTIO_GPU_RESP_OK_CAPSET_INFO,
        VIRTIO_GPU_RESP_OK_CAPSET,
        VIRTIO_GPU_RESP_OK_EDID,
        VIRTIO_GPU_RESP_OK_RESOURCE_UUID,
        VIRTIO_GPU_RESP_OK_MAP_INFO,

        /* error responses */
        VIRTIO_GPU_RESP_ERR_UNSPEC = 0x1200,
        VIRTIO_GPU_RESP_ERR_OUT_OF_MEMORY,
        VIRTIO_GPU_RESP_ERR_INVALID_SCANOUT_ID,
        VIRTIO_GPU_RESP_ERR_INVALID_RESOURCE_ID,
        VIRTIO_GPU_RESP_ERR_INVALID_CONTEXT_ID,
        VIRTIO_GPU_RESP_ERR_INVALID_PARAMETER,
};

#define VIRTIO_GPU_FLAG_FENCE (1 << 0)
#define VIRTIO_GPU_FLAG_INFO_RING_IDX (1 << 1)

struct virtio_gpu_ctrl_hdr {
        le32 type;
        le32 flags;
        le64 fence_id;
        le32 ctx_id;
        u8 ring_idx;
        u8 padding[3];
};
\end{lstlisting}

The fixed header \field{struct virtio_gpu_ctrl_hdr} in each
request includes the following fields:

\begin{description}
\item[\field{type}] specifies the type of the driver request
  (VIRTIO_GPU_CMD_*) or device response (VIRTIO_GPU_RESP_*).
\item[\field{flags}] request / response flags.
\item[\field{fence_id}] If the driver sets the VIRTIO_GPU_FLAG_FENCE
  bit in the request \field{flags} field the device MUST:
  \begin{itemize*}
  \item set VIRTIO_GPU_FLAG_FENCE bit in the response,
  \item copy the content of the \field{fence_id} field from the
    request to the response, and
  \item send the response only after command processing is complete.
  \end{itemize*}
\item[\field{ctx_id}] Rendering context (used in 3D mode only).
\item[\field{ring_idx}] If VIRTIO_GPU_F_CONTEXT_INIT is supported, then
  the driver MAY set VIRTIO_GPU_FLAG_INFO_RING_IDX bit in the request
  \field{flags}.  In that case:
  \begin{itemize*}
  \item \field{ring_idx} indicates the value of a context-specific ring
   index.  The minimum value is 0 and maximum value is 63 (inclusive).
  \item If VIRTIO_GPU_FLAG_FENCE is set, \field{fence_id} acts as a
   sequence number on the synchronization timeline defined by
   \field{ctx_idx} and the ring index.
  \item If VIRTIO_GPU_FLAG_FENCE is set and when the command associated
   with \field{fence_id} is complete, the device MUST send a response for
   all outstanding commands with a sequence number less than or equal to
   \field{fence_id} on the same synchronization timeline.
  \end{itemize*}
\end{description}

On success the device will return VIRTIO_GPU_RESP_OK_NODATA in
case there is no payload.  Otherwise the \field{type} field will
indicate the kind of payload.

On error the device will return one of the
VIRTIO_GPU_RESP_ERR_* error codes.

\subsubsection{Device Operation: controlq}\label{sec:Device Types / GPU Device / Device Operation / Device Operation: controlq}

For any coordinates given 0,0 is top left, larger x moves right,
larger y moves down.

\begin{description}

\item[VIRTIO_GPU_CMD_GET_DISPLAY_INFO] Retrieve the current output
  configuration.  No request data (just bare \field{struct
    virtio_gpu_ctrl_hdr}).  Response type is
  VIRTIO_GPU_RESP_OK_DISPLAY_INFO, response data is \field{struct
    virtio_gpu_resp_display_info}.

\begin{lstlisting}
#define VIRTIO_GPU_MAX_SCANOUTS 16

struct virtio_gpu_rect {
        le32 x;
        le32 y;
        le32 width;
        le32 height;
};

struct virtio_gpu_resp_display_info {
        struct virtio_gpu_ctrl_hdr hdr;
        struct virtio_gpu_display_one {
                struct virtio_gpu_rect r;
                le32 enabled;
                le32 flags;
        } pmodes[VIRTIO_GPU_MAX_SCANOUTS];
};
\end{lstlisting}

The response contains a list of per-scanout information.  The info
contains whether the scanout is enabled and what its preferred
position and size is.

The size (fields \field{width} and \field{height}) is similar to the
native panel resolution in EDID display information, except that in
the virtual machine case the size can change when the host window
representing the guest display is gets resized.

The position (fields \field{x} and \field{y}) describe how the
displays are arranged (i.e. which is -- for example -- the left
display).

The \field{enabled} field is set when the user enabled the display.
It is roughly the same as the connected state of a phyiscal display
connector.

\item[VIRTIO_GPU_CMD_GET_EDID] Retrieve the EDID data for a given
  scanout.  Request data is \field{struct virtio_gpu_get_edid}).
  Response type is VIRTIO_GPU_RESP_OK_EDID, response data is
  \field{struct virtio_gpu_resp_edid}.  Support is optional and
  negotiated using the VIRTIO_GPU_F_EDID feature flag.

\begin{lstlisting}
struct virtio_gpu_get_edid {
        struct virtio_gpu_ctrl_hdr hdr;
        le32 scanout;
        le32 padding;
};

struct virtio_gpu_resp_edid {
        struct virtio_gpu_ctrl_hdr hdr;
        le32 size;
        le32 padding;
        u8 edid[1024];
};
\end{lstlisting}

The response contains the EDID display data blob (as specified by
VESA) for the scanout.

\item[VIRTIO_GPU_CMD_RESOURCE_CREATE_2D] Create a 2D resource on the
  host.  Request data is \field{struct virtio_gpu_resource_create_2d}.
  Response type is VIRTIO_GPU_RESP_OK_NODATA.

\begin{lstlisting}
enum virtio_gpu_formats {
        VIRTIO_GPU_FORMAT_B8G8R8A8_UNORM  = 1,
        VIRTIO_GPU_FORMAT_B8G8R8X8_UNORM  = 2,
        VIRTIO_GPU_FORMAT_A8R8G8B8_UNORM  = 3,
        VIRTIO_GPU_FORMAT_X8R8G8B8_UNORM  = 4,

        VIRTIO_GPU_FORMAT_R8G8B8A8_UNORM  = 67,
        VIRTIO_GPU_FORMAT_X8B8G8R8_UNORM  = 68,

        VIRTIO_GPU_FORMAT_A8B8G8R8_UNORM  = 121,
        VIRTIO_GPU_FORMAT_R8G8B8X8_UNORM  = 134,
};

struct virtio_gpu_resource_create_2d {
        struct virtio_gpu_ctrl_hdr hdr;
        le32 resource_id;
        le32 format;
        le32 width;
        le32 height;
};
\end{lstlisting}

This creates a 2D resource on the host with the specified width,
height and format.  The resource ids are generated by the guest.

\item[VIRTIO_GPU_CMD_RESOURCE_UNREF] Destroy a resource.  Request data
  is \field{struct virtio_gpu_resource_unref}.  Response type is
  VIRTIO_GPU_RESP_OK_NODATA.

\begin{lstlisting}
struct virtio_gpu_resource_unref {
        struct virtio_gpu_ctrl_hdr hdr;
        le32 resource_id;
        le32 padding;
};
\end{lstlisting}

This informs the host that a resource is no longer required by the
guest.

\item[VIRTIO_GPU_CMD_SET_SCANOUT] Set the scanout parameters for a
  single output.  Request data is \field{struct
    virtio_gpu_set_scanout}.  Response type is
  VIRTIO_GPU_RESP_OK_NODATA.

\begin{lstlisting}
struct virtio_gpu_set_scanout {
        struct virtio_gpu_ctrl_hdr hdr;
        struct virtio_gpu_rect r;
        le32 scanout_id;
        le32 resource_id;
};
\end{lstlisting}

This sets the scanout parameters for a single scanout. The resource_id
is the resource to be scanned out from, along with a rectangle.

Scanout rectangles must be completely covered by the underlying
resource.  Overlapping (or identical) scanouts are allowed, typical
use case is screen mirroring.

The driver can use resource_id = 0 to disable a scanout.

\item[VIRTIO_GPU_CMD_RESOURCE_FLUSH] Flush a scanout resource Request
  data is \field{struct virtio_gpu_resource_flush}.  Response type is
  VIRTIO_GPU_RESP_OK_NODATA.

\begin{lstlisting}
struct virtio_gpu_resource_flush {
        struct virtio_gpu_ctrl_hdr hdr;
        struct virtio_gpu_rect r;
        le32 resource_id;
        le32 padding;
};
\end{lstlisting}

This flushes a resource to screen.  It takes a rectangle and a
resource id, and flushes any scanouts the resource is being used on.

\item[VIRTIO_GPU_CMD_TRANSFER_TO_HOST_2D] Transfer from guest memory
  to host resource.  Request data is \field{struct
    virtio_gpu_transfer_to_host_2d}.  Response type is
  VIRTIO_GPU_RESP_OK_NODATA.

\begin{lstlisting}
struct virtio_gpu_transfer_to_host_2d {
        struct virtio_gpu_ctrl_hdr hdr;
        struct virtio_gpu_rect r;
        le64 offset;
        le32 resource_id;
        le32 padding;
};
\end{lstlisting}

This takes a resource id along with an destination offset into the
resource, and a box to transfer to the host backing for the resource.

\item[VIRTIO_GPU_CMD_RESOURCE_ATTACH_BACKING] Assign backing pages to
  a resource.  Request data is \field{struct
    virtio_gpu_resource_attach_backing}, followed by \field{struct
    virtio_gpu_mem_entry} entries.  Response type is
  VIRTIO_GPU_RESP_OK_NODATA.

\begin{lstlisting}
struct virtio_gpu_resource_attach_backing {
        struct virtio_gpu_ctrl_hdr hdr;
        le32 resource_id;
        le32 nr_entries;
};

struct virtio_gpu_mem_entry {
        le64 addr;
        le32 length;
        le32 padding;
};
\end{lstlisting}

This assign an array of guest pages as the backing store for a
resource. These pages are then used for the transfer operations for
that resource from that point on.

\item[VIRTIO_GPU_CMD_RESOURCE_DETACH_BACKING] Detach backing pages
  from a resource.  Request data is \field{struct
    virtio_gpu_resource_detach_backing}.  Response type is
  VIRTIO_GPU_RESP_OK_NODATA.

\begin{lstlisting}
struct virtio_gpu_resource_detach_backing {
        struct virtio_gpu_ctrl_hdr hdr;
        le32 resource_id;
        le32 padding;
};
\end{lstlisting}

This detaches any backing pages from a resource, to be used in case of
guest swapping or object destruction.

\item[VIRTIO_GPU_CMD_GET_CAPSET_INFO] Gets the information associated with
  a particular \field{capset_index}, which MUST less than \field{num_capsets}
  defined in the device configuration.  Request data is
  \field{struct virtio_gpu_get_capset_info}.  Response type is
  VIRTIO_GPU_RESP_OK_CAPSET_INFO.

  On success, \field{struct virtio_gpu_resp_capset_info} contains the
  \field{capset_id}, \field{capset_max_version}, \field{capset_max_size}
  associated with capset at the specified {capset_idex}.  field{capset_id} MUST
  be one of the following (see listing for values):

  \begin{itemize*}
  \item \href{https://gitlab.freedesktop.org/virgl/virglrenderer/-/blob/master/src/virgl_hw.h#L526}{VIRTIO_GPU_CAPSET_VIRGL} --
	the first edition of Virgl (Gallium OpenGL) protocol.
  \item \href{https://gitlab.freedesktop.org/virgl/virglrenderer/-/blob/master/src/virgl_hw.h#L550}{VIRTIO_GPU_CAPSET_VIRGL2} --
	the second edition of Virgl (Gallium OpenGL) protocol after the capset fix.
  \item \href{https://android.googlesource.com/device/generic/vulkan-cereal/+/refs/heads/master/protocols/}{VIRTIO_GPU_CAPSET_GFXSTREAM} --
	gfxtream's (mostly) autogenerated GLES and Vulkan streaming protocols.
  \item \href{https://gitlab.freedesktop.org/olv/venus-protocol}{VIRTIO_GPU_CAPSET_VENUS} --
	Mesa's (mostly) autogenerated Vulkan protocol.
  \item \href{https://chromium.googlesource.com/chromiumos/platform/crosvm/+/refs/heads/main/rutabaga_gfx/src/cross_domain/cross_domain_protocol.rs}{VIRTIO_GPU_CAPSET_CROSS_DOMAIN} --
	protocol for display virtualization via Wayland proxying.
  \end{itemize*}

\begin{lstlisting}
struct virtio_gpu_get_capset_info {
        struct virtio_gpu_ctrl_hdr hdr;
        le32 capset_index;
        le32 padding;
};

#define VIRTIO_GPU_CAPSET_VIRGL 1
#define VIRTIO_GPU_CAPSET_VIRGL2 2
#define VIRTIO_GPU_CAPSET_GFXSTREAM 3
#define VIRTIO_GPU_CAPSET_VENUS 4
#define VIRTIO_GPU_CAPSET_CROSS_DOMAIN 5
struct virtio_gpu_resp_capset_info {
        struct virtio_gpu_ctrl_hdr hdr;
        le32 capset_id;
        le32 capset_max_version;
        le32 capset_max_size;
        le32 padding;
};
\end{lstlisting}

\item[VIRTIO_GPU_CMD_GET_CAPSET] Gets the capset associated with a
  particular \field{capset_id} and \field{capset_version}.  Request data is
  \field{struct virtio_gpu_get_capset}.  Response type is
  VIRTIO_GPU_RESP_OK_CAPSET.

\begin{lstlisting}
struct virtio_gpu_get_capset {
        struct virtio_gpu_ctrl_hdr hdr;
        le32 capset_id;
        le32 capset_version;
};

struct virtio_gpu_resp_capset {
        struct virtio_gpu_ctrl_hdr hdr;
        u8 capset_data[];
};
\end{lstlisting}

\item[VIRTIO_GPU_CMD_RESOURCE_ASSIGN_UUID] Creates an exported object from
  a resource. Request data is \field{struct
    virtio_gpu_resource_assign_uuid}.  Response type is
  VIRTIO_GPU_RESP_OK_RESOURCE_UUID, response data is \field{struct
    virtio_gpu_resp_resource_uuid}. Support is optional and negotiated
    using the VIRTIO_GPU_F_RESOURCE_UUID feature flag.

\begin{lstlisting}
struct virtio_gpu_resource_assign_uuid {
        struct virtio_gpu_ctrl_hdr hdr;
        le32 resource_id;
        le32 padding;
};

struct virtio_gpu_resp_resource_uuid {
        struct virtio_gpu_ctrl_hdr hdr;
        u8 uuid[16];
};
\end{lstlisting}

The response contains a UUID which identifies the exported object created from
the host private resource. Note that if the resource has an attached backing,
modifications made to the host private resource through the exported object by
other devices are not visible in the attached backing until they are transferred
into the backing.

\item[VIRTIO_GPU_CMD_RESOURCE_CREATE_BLOB] Creates a virtio-gpu blob
  resource. Request data is \field{struct
  virtio_gpu_resource_create_blob}, followed by \field{struct
  virtio_gpu_mem_entry} entries. Response type is
  VIRTIO_GPU_RESP_OK_NODATA. Support is optional and negotiated
  using the VIRTIO_GPU_F_RESOURCE_BLOB feature flag.

\begin{lstlisting}
#define VIRTIO_GPU_BLOB_MEM_GUEST             0x0001
#define VIRTIO_GPU_BLOB_MEM_HOST3D            0x0002
#define VIRTIO_GPU_BLOB_MEM_HOST3D_GUEST      0x0003

#define VIRTIO_GPU_BLOB_FLAG_USE_MAPPABLE     0x0001
#define VIRTIO_GPU_BLOB_FLAG_USE_SHAREABLE    0x0002
#define VIRTIO_GPU_BLOB_FLAG_USE_CROSS_DEVICE 0x0004

struct virtio_gpu_resource_create_blob {
       struct virtio_gpu_ctrl_hdr hdr;
       le32 resource_id;
       le32 blob_mem;
       le32 blob_flags;
       le32 nr_entries;
       le64 blob_id;
       le64 size;
};

\end{lstlisting}

A blob resource is a container for:

  \begin{itemize*}
  \item a guest memory allocation (referred to as a
  "guest-only blob resource").
  \item a host memory allocation (referred to as a
  "host-only blob resource").
  \item a guest memory and host memory allocation (referred
  to as a "default blob resource").
  \end{itemize*}

The memory properties of the blob resource MUST be described by
\field{blob_mem}, which MUST be non-zero.

For default and guest-only blob resources, \field{nr_entries} guest
memory entries may be assigned to the resource.  For default blob resources
(i.e, when \field{blob_mem} is VIRTIO_GPU_BLOB_MEM_HOST3D_GUEST), these
memory entries are used as a shadow buffer for the host memory. To
facilitate drivers that support swap-in and swap-out, \field{nr_entries} may
be zero and VIRTIO_GPU_CMD_RESOURCE_ATTACH_BACKING may be subsequently used.
VIRTIO_GPU_CMD_RESOURCE_DETACH_BACKING may be used to unassign memory entries.

\field{blob_mem} can only be VIRTIO_GPU_BLOB_MEM_HOST3D and
VIRTIO_GPU_BLOB_MEM_HOST3D_GUEST if VIRTIO_GPU_F_VIRGL is supported.
VIRTIO_GPU_BLOB_MEM_GUEST is valid regardless whether VIRTIO_GPU_F_VIRGL
is supported or not.

For VIRTIO_GPU_BLOB_MEM_HOST3D and VIRTIO_GPU_BLOB_MEM_HOST3D_GUEST, the
virtio-gpu resource MUST be created from the rendering context local object
identified by the \field{blob_id}. The actual allocation is done via
VIRTIO_GPU_CMD_SUBMIT_3D.

The driver MUST inform the device if the blob resource is used for
memory access, sharing between driver instances and/or sharing with
other devices. This is done via the \field{blob_flags} field.

If VIRTIO_GPU_F_VIRGL is set, both VIRTIO_GPU_CMD_TRANSFER_TO_HOST_3D
and VIRTIO_GPU_CMD_TRANSFER_FROM_HOST_3D may be used to update the
resource. There is no restriction on the image/buffer view the driver
has on the blob resource.

\item[VIRTIO_GPU_CMD_SET_SCANOUT_BLOB] sets scanout parameters for a
   blob resource. Request data is
  \field{struct virtio_gpu_set_scanout_blob}. Response type is
  VIRTIO_GPU_RESP_OK_NODATA. Support is optional and negotiated
  using the VIRTIO_GPU_F_RESOURCE_BLOB feature flag.

\begin{lstlisting}
struct virtio_gpu_set_scanout_blob {
       struct virtio_gpu_ctrl_hdr hdr;
       struct virtio_gpu_rect r;
       le32 scanout_id;
       le32 resource_id;
       le32 width;
       le32 height;
       le32 format;
       le32 padding;
       le32 strides[4];
       le32 offsets[4];
};
\end{lstlisting}

The rectangle \field{r} represents the portion of the blob resource being
displayed. The rest is the metadata associated with the blob resource. The
format MUST be one of \field{enum virtio_gpu_formats}.  The format MAY be
compressed with header and data planes.

\end{description}

\subsubsection{Device Operation: controlq (3d)}\label{sec:Device Types / GPU Device / Device Operation / Device Operation: controlq (3d)}

These commands are supported by the device if the VIRTIO_GPU_F_VIRGL
feature flag is set.

\begin{description}

\item[VIRTIO_GPU_CMD_CTX_CREATE] creates a context for submitting an opaque
  command stream.  Request data is \field{struct virtio_gpu_ctx_create}.
  Response type is VIRTIO_GPU_RESP_OK_NODATA.

\begin{lstlisting}
#define VIRTIO_GPU_CONTEXT_INIT_CAPSET_ID_MASK 0x000000ff;
struct virtio_gpu_ctx_create {
       struct virtio_gpu_ctrl_hdr hdr;
       le32 nlen;
       le32 context_init;
       char debug_name[64];
};
\end{lstlisting}

The implementation MUST create a context for the given \field{ctx_id} in
the \field{hdr}.  For debugging purposes, a \field{debug_name} and it's
length \field{nlen} is provided by the driver.  If
VIRTIO_GPU_F_CONTEXT_INIT is supported, then lower 8 bits of
\field{context_init} MAY contain the \field{capset_id} associated with
context.  In that case, then the device MUST create a context that can
handle the specified command stream.

If the lower 8-bits of the \field{context_init} are zero, then the type of
the context is determined by the device.

\item[VIRTIO_GPU_CMD_CTX_DESTROY]
\item[VIRTIO_GPU_CMD_CTX_ATTACH_RESOURCE]
\item[VIRTIO_GPU_CMD_CTX_DETACH_RESOURCE]
  Manage virtio-gpu 3d contexts.

\item[VIRTIO_GPU_CMD_RESOURCE_CREATE_3D]
  Create virtio-gpu 3d resources.

\item[VIRTIO_GPU_CMD_TRANSFER_TO_HOST_3D]
\item[VIRTIO_GPU_CMD_TRANSFER_FROM_HOST_3D]
  Transfer data from and to virtio-gpu 3d resources.

\item[VIRTIO_GPU_CMD_SUBMIT_3D]
  Submit an opaque command stream.  The type of the command stream is
  determined when creating a context.

\item[VIRTIO_GPU_CMD_RESOURCE_MAP_BLOB] maps a host-only
  blob resource into an offset in the host visible memory region. Request
  data is \field{struct virtio_gpu_resource_map_blob}.  The driver MUST
  not map a blob resource that is already mapped.  Response type is
  VIRTIO_GPU_RESP_OK_MAP_INFO. Support is optional and negotiated
  using the VIRTIO_GPU_F_RESOURCE_BLOB feature flag and checking for
  the presence of the host visible memory region.

\begin{lstlisting}
struct virtio_gpu_resource_map_blob {
        struct virtio_gpu_ctrl_hdr hdr;
        le32 resource_id;
        le32 padding;
        le64 offset;
};

#define VIRTIO_GPU_MAP_CACHE_MASK      0x0f
#define VIRTIO_GPU_MAP_CACHE_NONE      0x00
#define VIRTIO_GPU_MAP_CACHE_CACHED    0x01
#define VIRTIO_GPU_MAP_CACHE_UNCACHED  0x02
#define VIRTIO_GPU_MAP_CACHE_WC        0x03
struct virtio_gpu_resp_map_info {
        struct virtio_gpu_ctrl_hdr hdr;
        u32 map_info;
        u32 padding;
};
\end{lstlisting}

\item[VIRTIO_GPU_CMD_RESOURCE_UNMAP_BLOB] unmaps a
  host-only blob resource from the host visible memory region. Request data
  is \field{struct virtio_gpu_resource_unmap_blob}.  Response type is
  VIRTIO_GPU_RESP_OK_NODATA.  Support is optional and negotiated
  using the VIRTIO_GPU_F_RESOURCE_BLOB feature flag and checking for
  the presence of the host visible memory region.

\begin{lstlisting}
struct virtio_gpu_resource_unmap_blob {
        struct virtio_gpu_ctrl_hdr hdr;
        le32 resource_id;
        le32 padding;
};
\end{lstlisting}

\end{description}

\subsubsection{Device Operation: cursorq}\label{sec:Device Types / GPU Device / Device Operation / Device Operation: cursorq}

Both cursorq commands use the same command struct.

\begin{lstlisting}
struct virtio_gpu_cursor_pos {
        le32 scanout_id;
        le32 x;
        le32 y;
        le32 padding;
};

struct virtio_gpu_update_cursor {
        struct virtio_gpu_ctrl_hdr hdr;
        struct virtio_gpu_cursor_pos pos;
        le32 resource_id;
        le32 hot_x;
        le32 hot_y;
        le32 padding;
};
\end{lstlisting}

\begin{description}

\item[VIRTIO_GPU_CMD_UPDATE_CURSOR]
Update cursor.
Request data is \field{struct virtio_gpu_update_cursor}.
Response type is VIRTIO_GPU_RESP_OK_NODATA.

Full cursor update.  Cursor will be loaded from the specified
\field{resource_id} and will be moved to \field{pos}.  The driver must
transfer the cursor into the resource beforehand (using control queue
commands) and make sure the commands to fill the resource are actually
processed (using fencing).

\item[VIRTIO_GPU_CMD_MOVE_CURSOR]
Move cursor.
Request data is \field{struct virtio_gpu_update_cursor}.
Response type is VIRTIO_GPU_RESP_OK_NODATA.

Move cursor to the place specified in \field{pos}.  The other fields
are not used and will be ignored by the device.

\end{description}

\subsection{VGA Compatibility}\label{sec:Device Types / GPU Device / VGA Compatibility}

Applies to Virtio Over PCI only.  The GPU device can come with and
without VGA compatibility.  The PCI class should be DISPLAY_VGA if VGA
compatibility is present and DISPLAY_OTHER otherwise.

VGA compatibility: PCI region 0 has the linear framebuffer, standard
vga registers are present.  Configuring a scanout
(VIRTIO_GPU_CMD_SET_SCANOUT) switches the device from vga
compatibility mode into native virtio mode.  A reset switches it back
into vga compatibility mode.

Note: qemu implementation also provides bochs dispi interface io ports
and mmio bar at pci region 1 and is therefore fully compatible with
the qemu stdvga (see \href{https://git.qemu-project.org/?p=qemu.git;a=blob;f=docs/specs/standard-vga.txt;hb=HEAD}{docs/specs/standard-vga.txt} in the qemu source tree).

\section{GPU Device}\label{sec:Device Types / GPU Device}

virtio-gpu is a virtio based graphics adapter.  It can operate in 2D
mode and in 3D mode.  3D mode will offload rendering ops to
the host gpu and therefore requires a gpu with 3D support on the host
machine.

In 2D mode the virtio-gpu device provides support for ARGB Hardware
cursors and multiple scanouts (aka heads).

\subsection{Device ID}\label{sec:Device Types / GPU Device / Device ID}

16

\subsection{Virtqueues}\label{sec:Device Types / GPU Device / Virtqueues}

\begin{description}
\item[0] controlq - queue for sending control commands
\item[1] cursorq - queue for sending cursor updates
\end{description}

Both queues have the same format.  Each request and each response have
a fixed header, followed by command specific data fields.  The
separate cursor queue is the "fast track" for cursor commands
(VIRTIO_GPU_CMD_UPDATE_CURSOR and VIRTIO_GPU_CMD_MOVE_CURSOR), so they
go through without being delayed by time-consuming commands in the
control queue.

\subsection{Feature bits}\label{sec:Device Types / GPU Device / Feature bits}

\begin{description}
\item[VIRTIO_GPU_F_VIRGL (0)] virgl 3D mode is supported.
\item[VIRTIO_GPU_F_EDID  (1)] EDID is supported.
\item[VIRTIO_GPU_F_RESOURCE_UUID (2)] assigning resources UUIDs for export
  to other virtio devices is supported.
\item[VIRTIO_GPU_F_RESOURCE_BLOB (3)] creating and using size-based blob
  resources is supported.
\item[VIRTIO_GPU_F_CONTEXT_INIT (4)] multiple context types and
  synchronization timelines supported.  Requires VIRTIO_GPU_F_VIRGL.
\end{description}

\subsection{Device configuration layout}\label{sec:Device Types / GPU Device / Device configuration layout}

GPU device configuration uses the following layout structure and
definitions:

\begin{lstlisting}
#define VIRTIO_GPU_EVENT_DISPLAY (1 << 0)

struct virtio_gpu_config {
        le32 events_read;
        le32 events_clear;
        le32 num_scanouts;
        le32 num_capsets;
};
\end{lstlisting}

\subsubsection{Device configuration fields}

\begin{description}
\item[\field{events_read}] signals pending events to the driver.  The
  driver MUST NOT write to this field.
\item[\field{events_clear}] clears pending events in the device.
  Writing a '1' into a bit will clear the corresponding bit in
  \field{events_read}, mimicking write-to-clear behavior.
\item[\field{num_scanouts}] specifies the maximum number of scanouts
  supported by the device.  Minimum value is 1, maximum value is 16.
\item[\field{num_capsets}] specifies the maximum number of capability
  sets supported by the device.  The minimum value is zero.
\end{description}

\subsubsection{Events}

\begin{description}
\item[VIRTIO_GPU_EVENT_DISPLAY] Display configuration has changed.
  The driver SHOULD use the VIRTIO_GPU_CMD_GET_DISPLAY_INFO command to
  fetch the information from the device.  In case EDID support is
  negotiated (VIRTIO_GPU_F_EDID feature flag) the device SHOULD also
  fetch the updated EDID blobs using the VIRTIO_GPU_CMD_GET_EDID
  command.
\end{description}

\devicenormative{\subsection}{Device Initialization}{Device Types / GPU Device / Device Initialization}

The driver SHOULD query the display information from the device using
the VIRTIO_GPU_CMD_GET_DISPLAY_INFO command and use that information
for the initial scanout setup.  In case EDID support is negotiated
(VIRTIO_GPU_F_EDID feature flag) the device SHOULD also fetch the EDID
information using the VIRTIO_GPU_CMD_GET_EDID command.  If no
information is available or all displays are disabled the driver MAY
choose to use a fallback, such as 1024x768 at display 0.

The driver SHOULD query all shared memory regions supported by the device.
If the device supports shared memory, the \field{shmid} of a region MUST
(see \ref{sec:Basic Facilities of a Virtio Device /
Shared Memory Regions}~\nameref{sec:Basic Facilities of a Virtio Device /
Shared Memory Regions}) be one of the following:

\begin{lstlisting}
enum virtio_gpu_shm_id {
        VIRTIO_GPU_SHM_ID_UNDEFINED = 0,
        VIRTIO_GPU_SHM_ID_HOST_VISIBLE = 1,
};
\end{lstlisting}

The shared memory region with VIRTIO_GPU_SHM_ID_HOST_VISIBLE is referred as
the "host visible memory region".  The device MUST support the
VIRTIO_GPU_CMD_RESOURCE_MAP_BLOB and VIRTIO_GPU_CMD_RESOURCE_UNMAP_BLOB
if the host visible memory region is available.

\subsection{Device Operation}\label{sec:Device Types / GPU Device / Device Operation}

The virtio-gpu is based around the concept of resources private to the
host.  The guest must DMA transfer into these resources, unless shared memory
regions are supported. This is a design requirement in order to interface with
future 3D rendering. In the unaccelerated 2D mode there is no support for DMA
transfers from resources, just to them.

Resources are initially simple 2D resources, consisting of a width,
height and format along with an identifier. The guest must then attach
backing store to the resources in order for DMA transfers to
work. This is like a GART in a real GPU.

\subsubsection{Device Operation: Create a framebuffer and configure scanout}

\begin{itemize*}
\item Create a host resource using VIRTIO_GPU_CMD_RESOURCE_CREATE_2D.
\item Allocate a framebuffer from guest ram, and attach it as backing
  storage to the resource just created, using
  VIRTIO_GPU_CMD_RESOURCE_ATTACH_BACKING.  Scatter lists are
  supported, so the framebuffer doesn't need to be contignous in guest
  physical memory.
\item Use VIRTIO_GPU_CMD_SET_SCANOUT to link the framebuffer to
  a display scanout.
\end{itemize*}

\subsubsection{Device Operation: Update a framebuffer and scanout}

\begin{itemize*}
\item Render to your framebuffer memory.
\item Use VIRTIO_GPU_CMD_TRANSFER_TO_HOST_2D to update the host resource
  from guest memory.
\item Use VIRTIO_GPU_CMD_RESOURCE_FLUSH to flush the updated resource
  to the display.
\end{itemize*}

\subsubsection{Device Operation: Using pageflip}

It is possible to create multiple framebuffers, flip between them
using VIRTIO_GPU_CMD_SET_SCANOUT and VIRTIO_GPU_CMD_RESOURCE_FLUSH,
and update the invisible framebuffer using
VIRTIO_GPU_CMD_TRANSFER_TO_HOST_2D.

\subsubsection{Device Operation: Multihead setup}

In case two or more displays are present there are different ways to
configure things:

\begin{itemize*}
\item Create a single framebuffer, link it to all displays
  (mirroring).
\item Create an framebuffer for each display.
\item Create one big framebuffer, configure scanouts to display a
  different rectangle of that framebuffer each.
\end{itemize*}

\devicenormative{\subsubsection}{Device Operation: Command lifecycle and fencing}{Device Types / GPU Device / Device Operation / Device Operation: Command lifecycle and fencing}

The device MAY process controlq commands asyncronously and return them
to the driver before the processing is complete.  If the driver needs
to know when the processing is finished it can set the
VIRTIO_GPU_FLAG_FENCE flag in the request.  The device MUST finish the
processing before returning the command then.

Note: current qemu implementation does asyncrounous processing only in
3d mode, when offloading the processing to the host gpu.

\subsubsection{Device Operation: Configure mouse cursor}

The mouse cursor image is a normal resource, except that it must be
64x64 in size.  The driver MUST create and populate the resource
(using the usual VIRTIO_GPU_CMD_RESOURCE_CREATE_2D,
VIRTIO_GPU_CMD_RESOURCE_ATTACH_BACKING and
VIRTIO_GPU_CMD_TRANSFER_TO_HOST_2D controlq commands) and make sure they
are completed (using VIRTIO_GPU_FLAG_FENCE).

Then VIRTIO_GPU_CMD_UPDATE_CURSOR can be sent to the cursorq to set
the pointer shape and position.  To move the pointer without updating
the shape use VIRTIO_GPU_CMD_MOVE_CURSOR instead.

\subsubsection{Device Operation: Request header}\label{sec:Device Types / GPU Device / Device Operation / Device Operation: Request header}

All requests and responses on the virtqueues have a fixed header
using the following layout structure and definitions:

\begin{lstlisting}
enum virtio_gpu_ctrl_type {

        /* 2d commands */
        VIRTIO_GPU_CMD_GET_DISPLAY_INFO = 0x0100,
        VIRTIO_GPU_CMD_RESOURCE_CREATE_2D,
        VIRTIO_GPU_CMD_RESOURCE_UNREF,
        VIRTIO_GPU_CMD_SET_SCANOUT,
        VIRTIO_GPU_CMD_RESOURCE_FLUSH,
        VIRTIO_GPU_CMD_TRANSFER_TO_HOST_2D,
        VIRTIO_GPU_CMD_RESOURCE_ATTACH_BACKING,
        VIRTIO_GPU_CMD_RESOURCE_DETACH_BACKING,
        VIRTIO_GPU_CMD_GET_CAPSET_INFO,
        VIRTIO_GPU_CMD_GET_CAPSET,
        VIRTIO_GPU_CMD_GET_EDID,
        VIRTIO_GPU_CMD_RESOURCE_ASSIGN_UUID,
        VIRTIO_GPU_CMD_RESOURCE_CREATE_BLOB,
        VIRTIO_GPU_CMD_SET_SCANOUT_BLOB,

        /* 3d commands */
        VIRTIO_GPU_CMD_CTX_CREATE = 0x0200,
        VIRTIO_GPU_CMD_CTX_DESTROY,
        VIRTIO_GPU_CMD_CTX_ATTACH_RESOURCE,
        VIRTIO_GPU_CMD_CTX_DETACH_RESOURCE,
        VIRTIO_GPU_CMD_RESOURCE_CREATE_3D,
        VIRTIO_GPU_CMD_TRANSFER_TO_HOST_3D,
        VIRTIO_GPU_CMD_TRANSFER_FROM_HOST_3D,
        VIRTIO_GPU_CMD_SUBMIT_3D,
        VIRTIO_GPU_CMD_RESOURCE_MAP_BLOB,
        VIRTIO_GPU_CMD_RESOURCE_UNMAP_BLOB,

        /* cursor commands */
        VIRTIO_GPU_CMD_UPDATE_CURSOR = 0x0300,
        VIRTIO_GPU_CMD_MOVE_CURSOR,

        /* success responses */
        VIRTIO_GPU_RESP_OK_NODATA = 0x1100,
        VIRTIO_GPU_RESP_OK_DISPLAY_INFO,
        VIRTIO_GPU_RESP_OK_CAPSET_INFO,
        VIRTIO_GPU_RESP_OK_CAPSET,
        VIRTIO_GPU_RESP_OK_EDID,
        VIRTIO_GPU_RESP_OK_RESOURCE_UUID,
        VIRTIO_GPU_RESP_OK_MAP_INFO,

        /* error responses */
        VIRTIO_GPU_RESP_ERR_UNSPEC = 0x1200,
        VIRTIO_GPU_RESP_ERR_OUT_OF_MEMORY,
        VIRTIO_GPU_RESP_ERR_INVALID_SCANOUT_ID,
        VIRTIO_GPU_RESP_ERR_INVALID_RESOURCE_ID,
        VIRTIO_GPU_RESP_ERR_INVALID_CONTEXT_ID,
        VIRTIO_GPU_RESP_ERR_INVALID_PARAMETER,
};

#define VIRTIO_GPU_FLAG_FENCE (1 << 0)
#define VIRTIO_GPU_FLAG_INFO_RING_IDX (1 << 1)

struct virtio_gpu_ctrl_hdr {
        le32 type;
        le32 flags;
        le64 fence_id;
        le32 ctx_id;
        u8 ring_idx;
        u8 padding[3];
};
\end{lstlisting}

The fixed header \field{struct virtio_gpu_ctrl_hdr} in each
request includes the following fields:

\begin{description}
\item[\field{type}] specifies the type of the driver request
  (VIRTIO_GPU_CMD_*) or device response (VIRTIO_GPU_RESP_*).
\item[\field{flags}] request / response flags.
\item[\field{fence_id}] If the driver sets the VIRTIO_GPU_FLAG_FENCE
  bit in the request \field{flags} field the device MUST:
  \begin{itemize*}
  \item set VIRTIO_GPU_FLAG_FENCE bit in the response,
  \item copy the content of the \field{fence_id} field from the
    request to the response, and
  \item send the response only after command processing is complete.
  \end{itemize*}
\item[\field{ctx_id}] Rendering context (used in 3D mode only).
\item[\field{ring_idx}] If VIRTIO_GPU_F_CONTEXT_INIT is supported, then
  the driver MAY set VIRTIO_GPU_FLAG_INFO_RING_IDX bit in the request
  \field{flags}.  In that case:
  \begin{itemize*}
  \item \field{ring_idx} indicates the value of a context-specific ring
   index.  The minimum value is 0 and maximum value is 63 (inclusive).
  \item If VIRTIO_GPU_FLAG_FENCE is set, \field{fence_id} acts as a
   sequence number on the synchronization timeline defined by
   \field{ctx_idx} and the ring index.
  \item If VIRTIO_GPU_FLAG_FENCE is set and when the command associated
   with \field{fence_id} is complete, the device MUST send a response for
   all outstanding commands with a sequence number less than or equal to
   \field{fence_id} on the same synchronization timeline.
  \end{itemize*}
\end{description}

On success the device will return VIRTIO_GPU_RESP_OK_NODATA in
case there is no payload.  Otherwise the \field{type} field will
indicate the kind of payload.

On error the device will return one of the
VIRTIO_GPU_RESP_ERR_* error codes.

\subsubsection{Device Operation: controlq}\label{sec:Device Types / GPU Device / Device Operation / Device Operation: controlq}

For any coordinates given 0,0 is top left, larger x moves right,
larger y moves down.

\begin{description}

\item[VIRTIO_GPU_CMD_GET_DISPLAY_INFO] Retrieve the current output
  configuration.  No request data (just bare \field{struct
    virtio_gpu_ctrl_hdr}).  Response type is
  VIRTIO_GPU_RESP_OK_DISPLAY_INFO, response data is \field{struct
    virtio_gpu_resp_display_info}.

\begin{lstlisting}
#define VIRTIO_GPU_MAX_SCANOUTS 16

struct virtio_gpu_rect {
        le32 x;
        le32 y;
        le32 width;
        le32 height;
};

struct virtio_gpu_resp_display_info {
        struct virtio_gpu_ctrl_hdr hdr;
        struct virtio_gpu_display_one {
                struct virtio_gpu_rect r;
                le32 enabled;
                le32 flags;
        } pmodes[VIRTIO_GPU_MAX_SCANOUTS];
};
\end{lstlisting}

The response contains a list of per-scanout information.  The info
contains whether the scanout is enabled and what its preferred
position and size is.

The size (fields \field{width} and \field{height}) is similar to the
native panel resolution in EDID display information, except that in
the virtual machine case the size can change when the host window
representing the guest display is gets resized.

The position (fields \field{x} and \field{y}) describe how the
displays are arranged (i.e. which is -- for example -- the left
display).

The \field{enabled} field is set when the user enabled the display.
It is roughly the same as the connected state of a phyiscal display
connector.

\item[VIRTIO_GPU_CMD_GET_EDID] Retrieve the EDID data for a given
  scanout.  Request data is \field{struct virtio_gpu_get_edid}).
  Response type is VIRTIO_GPU_RESP_OK_EDID, response data is
  \field{struct virtio_gpu_resp_edid}.  Support is optional and
  negotiated using the VIRTIO_GPU_F_EDID feature flag.

\begin{lstlisting}
struct virtio_gpu_get_edid {
        struct virtio_gpu_ctrl_hdr hdr;
        le32 scanout;
        le32 padding;
};

struct virtio_gpu_resp_edid {
        struct virtio_gpu_ctrl_hdr hdr;
        le32 size;
        le32 padding;
        u8 edid[1024];
};
\end{lstlisting}

The response contains the EDID display data blob (as specified by
VESA) for the scanout.

\item[VIRTIO_GPU_CMD_RESOURCE_CREATE_2D] Create a 2D resource on the
  host.  Request data is \field{struct virtio_gpu_resource_create_2d}.
  Response type is VIRTIO_GPU_RESP_OK_NODATA.

\begin{lstlisting}
enum virtio_gpu_formats {
        VIRTIO_GPU_FORMAT_B8G8R8A8_UNORM  = 1,
        VIRTIO_GPU_FORMAT_B8G8R8X8_UNORM  = 2,
        VIRTIO_GPU_FORMAT_A8R8G8B8_UNORM  = 3,
        VIRTIO_GPU_FORMAT_X8R8G8B8_UNORM  = 4,

        VIRTIO_GPU_FORMAT_R8G8B8A8_UNORM  = 67,
        VIRTIO_GPU_FORMAT_X8B8G8R8_UNORM  = 68,

        VIRTIO_GPU_FORMAT_A8B8G8R8_UNORM  = 121,
        VIRTIO_GPU_FORMAT_R8G8B8X8_UNORM  = 134,
};

struct virtio_gpu_resource_create_2d {
        struct virtio_gpu_ctrl_hdr hdr;
        le32 resource_id;
        le32 format;
        le32 width;
        le32 height;
};
\end{lstlisting}

This creates a 2D resource on the host with the specified width,
height and format.  The resource ids are generated by the guest.

\item[VIRTIO_GPU_CMD_RESOURCE_UNREF] Destroy a resource.  Request data
  is \field{struct virtio_gpu_resource_unref}.  Response type is
  VIRTIO_GPU_RESP_OK_NODATA.

\begin{lstlisting}
struct virtio_gpu_resource_unref {
        struct virtio_gpu_ctrl_hdr hdr;
        le32 resource_id;
        le32 padding;
};
\end{lstlisting}

This informs the host that a resource is no longer required by the
guest.

\item[VIRTIO_GPU_CMD_SET_SCANOUT] Set the scanout parameters for a
  single output.  Request data is \field{struct
    virtio_gpu_set_scanout}.  Response type is
  VIRTIO_GPU_RESP_OK_NODATA.

\begin{lstlisting}
struct virtio_gpu_set_scanout {
        struct virtio_gpu_ctrl_hdr hdr;
        struct virtio_gpu_rect r;
        le32 scanout_id;
        le32 resource_id;
};
\end{lstlisting}

This sets the scanout parameters for a single scanout. The resource_id
is the resource to be scanned out from, along with a rectangle.

Scanout rectangles must be completely covered by the underlying
resource.  Overlapping (or identical) scanouts are allowed, typical
use case is screen mirroring.

The driver can use resource_id = 0 to disable a scanout.

\item[VIRTIO_GPU_CMD_RESOURCE_FLUSH] Flush a scanout resource Request
  data is \field{struct virtio_gpu_resource_flush}.  Response type is
  VIRTIO_GPU_RESP_OK_NODATA.

\begin{lstlisting}
struct virtio_gpu_resource_flush {
        struct virtio_gpu_ctrl_hdr hdr;
        struct virtio_gpu_rect r;
        le32 resource_id;
        le32 padding;
};
\end{lstlisting}

This flushes a resource to screen.  It takes a rectangle and a
resource id, and flushes any scanouts the resource is being used on.

\item[VIRTIO_GPU_CMD_TRANSFER_TO_HOST_2D] Transfer from guest memory
  to host resource.  Request data is \field{struct
    virtio_gpu_transfer_to_host_2d}.  Response type is
  VIRTIO_GPU_RESP_OK_NODATA.

\begin{lstlisting}
struct virtio_gpu_transfer_to_host_2d {
        struct virtio_gpu_ctrl_hdr hdr;
        struct virtio_gpu_rect r;
        le64 offset;
        le32 resource_id;
        le32 padding;
};
\end{lstlisting}

This takes a resource id along with an destination offset into the
resource, and a box to transfer to the host backing for the resource.

\item[VIRTIO_GPU_CMD_RESOURCE_ATTACH_BACKING] Assign backing pages to
  a resource.  Request data is \field{struct
    virtio_gpu_resource_attach_backing}, followed by \field{struct
    virtio_gpu_mem_entry} entries.  Response type is
  VIRTIO_GPU_RESP_OK_NODATA.

\begin{lstlisting}
struct virtio_gpu_resource_attach_backing {
        struct virtio_gpu_ctrl_hdr hdr;
        le32 resource_id;
        le32 nr_entries;
};

struct virtio_gpu_mem_entry {
        le64 addr;
        le32 length;
        le32 padding;
};
\end{lstlisting}

This assign an array of guest pages as the backing store for a
resource. These pages are then used for the transfer operations for
that resource from that point on.

\item[VIRTIO_GPU_CMD_RESOURCE_DETACH_BACKING] Detach backing pages
  from a resource.  Request data is \field{struct
    virtio_gpu_resource_detach_backing}.  Response type is
  VIRTIO_GPU_RESP_OK_NODATA.

\begin{lstlisting}
struct virtio_gpu_resource_detach_backing {
        struct virtio_gpu_ctrl_hdr hdr;
        le32 resource_id;
        le32 padding;
};
\end{lstlisting}

This detaches any backing pages from a resource, to be used in case of
guest swapping or object destruction.

\item[VIRTIO_GPU_CMD_GET_CAPSET_INFO] Gets the information associated with
  a particular \field{capset_index}, which MUST less than \field{num_capsets}
  defined in the device configuration.  Request data is
  \field{struct virtio_gpu_get_capset_info}.  Response type is
  VIRTIO_GPU_RESP_OK_CAPSET_INFO.

  On success, \field{struct virtio_gpu_resp_capset_info} contains the
  \field{capset_id}, \field{capset_max_version}, \field{capset_max_size}
  associated with capset at the specified {capset_idex}.  field{capset_id} MUST
  be one of the following (see listing for values):

  \begin{itemize*}
  \item \href{https://gitlab.freedesktop.org/virgl/virglrenderer/-/blob/master/src/virgl_hw.h#L526}{VIRTIO_GPU_CAPSET_VIRGL} --
	the first edition of Virgl (Gallium OpenGL) protocol.
  \item \href{https://gitlab.freedesktop.org/virgl/virglrenderer/-/blob/master/src/virgl_hw.h#L550}{VIRTIO_GPU_CAPSET_VIRGL2} --
	the second edition of Virgl (Gallium OpenGL) protocol after the capset fix.
  \item \href{https://android.googlesource.com/device/generic/vulkan-cereal/+/refs/heads/master/protocols/}{VIRTIO_GPU_CAPSET_GFXSTREAM} --
	gfxtream's (mostly) autogenerated GLES and Vulkan streaming protocols.
  \item \href{https://gitlab.freedesktop.org/olv/venus-protocol}{VIRTIO_GPU_CAPSET_VENUS} --
	Mesa's (mostly) autogenerated Vulkan protocol.
  \item \href{https://chromium.googlesource.com/chromiumos/platform/crosvm/+/refs/heads/main/rutabaga_gfx/src/cross_domain/cross_domain_protocol.rs}{VIRTIO_GPU_CAPSET_CROSS_DOMAIN} --
	protocol for display virtualization via Wayland proxying.
  \end{itemize*}

\begin{lstlisting}
struct virtio_gpu_get_capset_info {
        struct virtio_gpu_ctrl_hdr hdr;
        le32 capset_index;
        le32 padding;
};

#define VIRTIO_GPU_CAPSET_VIRGL 1
#define VIRTIO_GPU_CAPSET_VIRGL2 2
#define VIRTIO_GPU_CAPSET_GFXSTREAM 3
#define VIRTIO_GPU_CAPSET_VENUS 4
#define VIRTIO_GPU_CAPSET_CROSS_DOMAIN 5
struct virtio_gpu_resp_capset_info {
        struct virtio_gpu_ctrl_hdr hdr;
        le32 capset_id;
        le32 capset_max_version;
        le32 capset_max_size;
        le32 padding;
};
\end{lstlisting}

\item[VIRTIO_GPU_CMD_GET_CAPSET] Gets the capset associated with a
  particular \field{capset_id} and \field{capset_version}.  Request data is
  \field{struct virtio_gpu_get_capset}.  Response type is
  VIRTIO_GPU_RESP_OK_CAPSET.

\begin{lstlisting}
struct virtio_gpu_get_capset {
        struct virtio_gpu_ctrl_hdr hdr;
        le32 capset_id;
        le32 capset_version;
};

struct virtio_gpu_resp_capset {
        struct virtio_gpu_ctrl_hdr hdr;
        u8 capset_data[];
};
\end{lstlisting}

\item[VIRTIO_GPU_CMD_RESOURCE_ASSIGN_UUID] Creates an exported object from
  a resource. Request data is \field{struct
    virtio_gpu_resource_assign_uuid}.  Response type is
  VIRTIO_GPU_RESP_OK_RESOURCE_UUID, response data is \field{struct
    virtio_gpu_resp_resource_uuid}. Support is optional and negotiated
    using the VIRTIO_GPU_F_RESOURCE_UUID feature flag.

\begin{lstlisting}
struct virtio_gpu_resource_assign_uuid {
        struct virtio_gpu_ctrl_hdr hdr;
        le32 resource_id;
        le32 padding;
};

struct virtio_gpu_resp_resource_uuid {
        struct virtio_gpu_ctrl_hdr hdr;
        u8 uuid[16];
};
\end{lstlisting}

The response contains a UUID which identifies the exported object created from
the host private resource. Note that if the resource has an attached backing,
modifications made to the host private resource through the exported object by
other devices are not visible in the attached backing until they are transferred
into the backing.

\item[VIRTIO_GPU_CMD_RESOURCE_CREATE_BLOB] Creates a virtio-gpu blob
  resource. Request data is \field{struct
  virtio_gpu_resource_create_blob}, followed by \field{struct
  virtio_gpu_mem_entry} entries. Response type is
  VIRTIO_GPU_RESP_OK_NODATA. Support is optional and negotiated
  using the VIRTIO_GPU_F_RESOURCE_BLOB feature flag.

\begin{lstlisting}
#define VIRTIO_GPU_BLOB_MEM_GUEST             0x0001
#define VIRTIO_GPU_BLOB_MEM_HOST3D            0x0002
#define VIRTIO_GPU_BLOB_MEM_HOST3D_GUEST      0x0003

#define VIRTIO_GPU_BLOB_FLAG_USE_MAPPABLE     0x0001
#define VIRTIO_GPU_BLOB_FLAG_USE_SHAREABLE    0x0002
#define VIRTIO_GPU_BLOB_FLAG_USE_CROSS_DEVICE 0x0004

struct virtio_gpu_resource_create_blob {
       struct virtio_gpu_ctrl_hdr hdr;
       le32 resource_id;
       le32 blob_mem;
       le32 blob_flags;
       le32 nr_entries;
       le64 blob_id;
       le64 size;
};

\end{lstlisting}

A blob resource is a container for:

  \begin{itemize*}
  \item a guest memory allocation (referred to as a
  "guest-only blob resource").
  \item a host memory allocation (referred to as a
  "host-only blob resource").
  \item a guest memory and host memory allocation (referred
  to as a "default blob resource").
  \end{itemize*}

The memory properties of the blob resource MUST be described by
\field{blob_mem}, which MUST be non-zero.

For default and guest-only blob resources, \field{nr_entries} guest
memory entries may be assigned to the resource.  For default blob resources
(i.e, when \field{blob_mem} is VIRTIO_GPU_BLOB_MEM_HOST3D_GUEST), these
memory entries are used as a shadow buffer for the host memory. To
facilitate drivers that support swap-in and swap-out, \field{nr_entries} may
be zero and VIRTIO_GPU_CMD_RESOURCE_ATTACH_BACKING may be subsequently used.
VIRTIO_GPU_CMD_RESOURCE_DETACH_BACKING may be used to unassign memory entries.

\field{blob_mem} can only be VIRTIO_GPU_BLOB_MEM_HOST3D and
VIRTIO_GPU_BLOB_MEM_HOST3D_GUEST if VIRTIO_GPU_F_VIRGL is supported.
VIRTIO_GPU_BLOB_MEM_GUEST is valid regardless whether VIRTIO_GPU_F_VIRGL
is supported or not.

For VIRTIO_GPU_BLOB_MEM_HOST3D and VIRTIO_GPU_BLOB_MEM_HOST3D_GUEST, the
virtio-gpu resource MUST be created from the rendering context local object
identified by the \field{blob_id}. The actual allocation is done via
VIRTIO_GPU_CMD_SUBMIT_3D.

The driver MUST inform the device if the blob resource is used for
memory access, sharing between driver instances and/or sharing with
other devices. This is done via the \field{blob_flags} field.

If VIRTIO_GPU_F_VIRGL is set, both VIRTIO_GPU_CMD_TRANSFER_TO_HOST_3D
and VIRTIO_GPU_CMD_TRANSFER_FROM_HOST_3D may be used to update the
resource. There is no restriction on the image/buffer view the driver
has on the blob resource.

\item[VIRTIO_GPU_CMD_SET_SCANOUT_BLOB] sets scanout parameters for a
   blob resource. Request data is
  \field{struct virtio_gpu_set_scanout_blob}. Response type is
  VIRTIO_GPU_RESP_OK_NODATA. Support is optional and negotiated
  using the VIRTIO_GPU_F_RESOURCE_BLOB feature flag.

\begin{lstlisting}
struct virtio_gpu_set_scanout_blob {
       struct virtio_gpu_ctrl_hdr hdr;
       struct virtio_gpu_rect r;
       le32 scanout_id;
       le32 resource_id;
       le32 width;
       le32 height;
       le32 format;
       le32 padding;
       le32 strides[4];
       le32 offsets[4];
};
\end{lstlisting}

The rectangle \field{r} represents the portion of the blob resource being
displayed. The rest is the metadata associated with the blob resource. The
format MUST be one of \field{enum virtio_gpu_formats}.  The format MAY be
compressed with header and data planes.

\end{description}

\subsubsection{Device Operation: controlq (3d)}\label{sec:Device Types / GPU Device / Device Operation / Device Operation: controlq (3d)}

These commands are supported by the device if the VIRTIO_GPU_F_VIRGL
feature flag is set.

\begin{description}

\item[VIRTIO_GPU_CMD_CTX_CREATE] creates a context for submitting an opaque
  command stream.  Request data is \field{struct virtio_gpu_ctx_create}.
  Response type is VIRTIO_GPU_RESP_OK_NODATA.

\begin{lstlisting}
#define VIRTIO_GPU_CONTEXT_INIT_CAPSET_ID_MASK 0x000000ff;
struct virtio_gpu_ctx_create {
       struct virtio_gpu_ctrl_hdr hdr;
       le32 nlen;
       le32 context_init;
       char debug_name[64];
};
\end{lstlisting}

The implementation MUST create a context for the given \field{ctx_id} in
the \field{hdr}.  For debugging purposes, a \field{debug_name} and it's
length \field{nlen} is provided by the driver.  If
VIRTIO_GPU_F_CONTEXT_INIT is supported, then lower 8 bits of
\field{context_init} MAY contain the \field{capset_id} associated with
context.  In that case, then the device MUST create a context that can
handle the specified command stream.

If the lower 8-bits of the \field{context_init} are zero, then the type of
the context is determined by the device.

\item[VIRTIO_GPU_CMD_CTX_DESTROY]
\item[VIRTIO_GPU_CMD_CTX_ATTACH_RESOURCE]
\item[VIRTIO_GPU_CMD_CTX_DETACH_RESOURCE]
  Manage virtio-gpu 3d contexts.

\item[VIRTIO_GPU_CMD_RESOURCE_CREATE_3D]
  Create virtio-gpu 3d resources.

\item[VIRTIO_GPU_CMD_TRANSFER_TO_HOST_3D]
\item[VIRTIO_GPU_CMD_TRANSFER_FROM_HOST_3D]
  Transfer data from and to virtio-gpu 3d resources.

\item[VIRTIO_GPU_CMD_SUBMIT_3D]
  Submit an opaque command stream.  The type of the command stream is
  determined when creating a context.

\item[VIRTIO_GPU_CMD_RESOURCE_MAP_BLOB] maps a host-only
  blob resource into an offset in the host visible memory region. Request
  data is \field{struct virtio_gpu_resource_map_blob}.  The driver MUST
  not map a blob resource that is already mapped.  Response type is
  VIRTIO_GPU_RESP_OK_MAP_INFO. Support is optional and negotiated
  using the VIRTIO_GPU_F_RESOURCE_BLOB feature flag and checking for
  the presence of the host visible memory region.

\begin{lstlisting}
struct virtio_gpu_resource_map_blob {
        struct virtio_gpu_ctrl_hdr hdr;
        le32 resource_id;
        le32 padding;
        le64 offset;
};

#define VIRTIO_GPU_MAP_CACHE_MASK      0x0f
#define VIRTIO_GPU_MAP_CACHE_NONE      0x00
#define VIRTIO_GPU_MAP_CACHE_CACHED    0x01
#define VIRTIO_GPU_MAP_CACHE_UNCACHED  0x02
#define VIRTIO_GPU_MAP_CACHE_WC        0x03
struct virtio_gpu_resp_map_info {
        struct virtio_gpu_ctrl_hdr hdr;
        u32 map_info;
        u32 padding;
};
\end{lstlisting}

\item[VIRTIO_GPU_CMD_RESOURCE_UNMAP_BLOB] unmaps a
  host-only blob resource from the host visible memory region. Request data
  is \field{struct virtio_gpu_resource_unmap_blob}.  Response type is
  VIRTIO_GPU_RESP_OK_NODATA.  Support is optional and negotiated
  using the VIRTIO_GPU_F_RESOURCE_BLOB feature flag and checking for
  the presence of the host visible memory region.

\begin{lstlisting}
struct virtio_gpu_resource_unmap_blob {
        struct virtio_gpu_ctrl_hdr hdr;
        le32 resource_id;
        le32 padding;
};
\end{lstlisting}

\end{description}

\subsubsection{Device Operation: cursorq}\label{sec:Device Types / GPU Device / Device Operation / Device Operation: cursorq}

Both cursorq commands use the same command struct.

\begin{lstlisting}
struct virtio_gpu_cursor_pos {
        le32 scanout_id;
        le32 x;
        le32 y;
        le32 padding;
};

struct virtio_gpu_update_cursor {
        struct virtio_gpu_ctrl_hdr hdr;
        struct virtio_gpu_cursor_pos pos;
        le32 resource_id;
        le32 hot_x;
        le32 hot_y;
        le32 padding;
};
\end{lstlisting}

\begin{description}

\item[VIRTIO_GPU_CMD_UPDATE_CURSOR]
Update cursor.
Request data is \field{struct virtio_gpu_update_cursor}.
Response type is VIRTIO_GPU_RESP_OK_NODATA.

Full cursor update.  Cursor will be loaded from the specified
\field{resource_id} and will be moved to \field{pos}.  The driver must
transfer the cursor into the resource beforehand (using control queue
commands) and make sure the commands to fill the resource are actually
processed (using fencing).

\item[VIRTIO_GPU_CMD_MOVE_CURSOR]
Move cursor.
Request data is \field{struct virtio_gpu_update_cursor}.
Response type is VIRTIO_GPU_RESP_OK_NODATA.

Move cursor to the place specified in \field{pos}.  The other fields
are not used and will be ignored by the device.

\end{description}

\subsection{VGA Compatibility}\label{sec:Device Types / GPU Device / VGA Compatibility}

Applies to Virtio Over PCI only.  The GPU device can come with and
without VGA compatibility.  The PCI class should be DISPLAY_VGA if VGA
compatibility is present and DISPLAY_OTHER otherwise.

VGA compatibility: PCI region 0 has the linear framebuffer, standard
vga registers are present.  Configuring a scanout
(VIRTIO_GPU_CMD_SET_SCANOUT) switches the device from vga
compatibility mode into native virtio mode.  A reset switches it back
into vga compatibility mode.

Note: qemu implementation also provides bochs dispi interface io ports
and mmio bar at pci region 1 and is therefore fully compatible with
the qemu stdvga (see \href{https://git.qemu-project.org/?p=qemu.git;a=blob;f=docs/specs/standard-vga.txt;hb=HEAD}{docs/specs/standard-vga.txt} in the qemu source tree).

\section{GPU Device}\label{sec:Device Types / GPU Device}

virtio-gpu is a virtio based graphics adapter.  It can operate in 2D
mode and in 3D mode.  3D mode will offload rendering ops to
the host gpu and therefore requires a gpu with 3D support on the host
machine.

In 2D mode the virtio-gpu device provides support for ARGB Hardware
cursors and multiple scanouts (aka heads).

\subsection{Device ID}\label{sec:Device Types / GPU Device / Device ID}

16

\subsection{Virtqueues}\label{sec:Device Types / GPU Device / Virtqueues}

\begin{description}
\item[0] controlq - queue for sending control commands
\item[1] cursorq - queue for sending cursor updates
\end{description}

Both queues have the same format.  Each request and each response have
a fixed header, followed by command specific data fields.  The
separate cursor queue is the "fast track" for cursor commands
(VIRTIO_GPU_CMD_UPDATE_CURSOR and VIRTIO_GPU_CMD_MOVE_CURSOR), so they
go through without being delayed by time-consuming commands in the
control queue.

\subsection{Feature bits}\label{sec:Device Types / GPU Device / Feature bits}

\begin{description}
\item[VIRTIO_GPU_F_VIRGL (0)] virgl 3D mode is supported.
\item[VIRTIO_GPU_F_EDID  (1)] EDID is supported.
\item[VIRTIO_GPU_F_RESOURCE_UUID (2)] assigning resources UUIDs for export
  to other virtio devices is supported.
\item[VIRTIO_GPU_F_RESOURCE_BLOB (3)] creating and using size-based blob
  resources is supported.
\item[VIRTIO_GPU_F_CONTEXT_INIT (4)] multiple context types and
  synchronization timelines supported.  Requires VIRTIO_GPU_F_VIRGL.
\end{description}

\subsection{Device configuration layout}\label{sec:Device Types / GPU Device / Device configuration layout}

GPU device configuration uses the following layout structure and
definitions:

\begin{lstlisting}
#define VIRTIO_GPU_EVENT_DISPLAY (1 << 0)

struct virtio_gpu_config {
        le32 events_read;
        le32 events_clear;
        le32 num_scanouts;
        le32 num_capsets;
};
\end{lstlisting}

\subsubsection{Device configuration fields}

\begin{description}
\item[\field{events_read}] signals pending events to the driver.  The
  driver MUST NOT write to this field.
\item[\field{events_clear}] clears pending events in the device.
  Writing a '1' into a bit will clear the corresponding bit in
  \field{events_read}, mimicking write-to-clear behavior.
\item[\field{num_scanouts}] specifies the maximum number of scanouts
  supported by the device.  Minimum value is 1, maximum value is 16.
\item[\field{num_capsets}] specifies the maximum number of capability
  sets supported by the device.  The minimum value is zero.
\end{description}

\subsubsection{Events}

\begin{description}
\item[VIRTIO_GPU_EVENT_DISPLAY] Display configuration has changed.
  The driver SHOULD use the VIRTIO_GPU_CMD_GET_DISPLAY_INFO command to
  fetch the information from the device.  In case EDID support is
  negotiated (VIRTIO_GPU_F_EDID feature flag) the device SHOULD also
  fetch the updated EDID blobs using the VIRTIO_GPU_CMD_GET_EDID
  command.
\end{description}

\devicenormative{\subsection}{Device Initialization}{Device Types / GPU Device / Device Initialization}

The driver SHOULD query the display information from the device using
the VIRTIO_GPU_CMD_GET_DISPLAY_INFO command and use that information
for the initial scanout setup.  In case EDID support is negotiated
(VIRTIO_GPU_F_EDID feature flag) the device SHOULD also fetch the EDID
information using the VIRTIO_GPU_CMD_GET_EDID command.  If no
information is available or all displays are disabled the driver MAY
choose to use a fallback, such as 1024x768 at display 0.

The driver SHOULD query all shared memory regions supported by the device.
If the device supports shared memory, the \field{shmid} of a region MUST
(see \ref{sec:Basic Facilities of a Virtio Device /
Shared Memory Regions}~\nameref{sec:Basic Facilities of a Virtio Device /
Shared Memory Regions}) be one of the following:

\begin{lstlisting}
enum virtio_gpu_shm_id {
        VIRTIO_GPU_SHM_ID_UNDEFINED = 0,
        VIRTIO_GPU_SHM_ID_HOST_VISIBLE = 1,
};
\end{lstlisting}

The shared memory region with VIRTIO_GPU_SHM_ID_HOST_VISIBLE is referred as
the "host visible memory region".  The device MUST support the
VIRTIO_GPU_CMD_RESOURCE_MAP_BLOB and VIRTIO_GPU_CMD_RESOURCE_UNMAP_BLOB
if the host visible memory region is available.

\subsection{Device Operation}\label{sec:Device Types / GPU Device / Device Operation}

The virtio-gpu is based around the concept of resources private to the
host.  The guest must DMA transfer into these resources, unless shared memory
regions are supported. This is a design requirement in order to interface with
future 3D rendering. In the unaccelerated 2D mode there is no support for DMA
transfers from resources, just to them.

Resources are initially simple 2D resources, consisting of a width,
height and format along with an identifier. The guest must then attach
backing store to the resources in order for DMA transfers to
work. This is like a GART in a real GPU.

\subsubsection{Device Operation: Create a framebuffer and configure scanout}

\begin{itemize*}
\item Create a host resource using VIRTIO_GPU_CMD_RESOURCE_CREATE_2D.
\item Allocate a framebuffer from guest ram, and attach it as backing
  storage to the resource just created, using
  VIRTIO_GPU_CMD_RESOURCE_ATTACH_BACKING.  Scatter lists are
  supported, so the framebuffer doesn't need to be contignous in guest
  physical memory.
\item Use VIRTIO_GPU_CMD_SET_SCANOUT to link the framebuffer to
  a display scanout.
\end{itemize*}

\subsubsection{Device Operation: Update a framebuffer and scanout}

\begin{itemize*}
\item Render to your framebuffer memory.
\item Use VIRTIO_GPU_CMD_TRANSFER_TO_HOST_2D to update the host resource
  from guest memory.
\item Use VIRTIO_GPU_CMD_RESOURCE_FLUSH to flush the updated resource
  to the display.
\end{itemize*}

\subsubsection{Device Operation: Using pageflip}

It is possible to create multiple framebuffers, flip between them
using VIRTIO_GPU_CMD_SET_SCANOUT and VIRTIO_GPU_CMD_RESOURCE_FLUSH,
and update the invisible framebuffer using
VIRTIO_GPU_CMD_TRANSFER_TO_HOST_2D.

\subsubsection{Device Operation: Multihead setup}

In case two or more displays are present there are different ways to
configure things:

\begin{itemize*}
\item Create a single framebuffer, link it to all displays
  (mirroring).
\item Create an framebuffer for each display.
\item Create one big framebuffer, configure scanouts to display a
  different rectangle of that framebuffer each.
\end{itemize*}

\devicenormative{\subsubsection}{Device Operation: Command lifecycle and fencing}{Device Types / GPU Device / Device Operation / Device Operation: Command lifecycle and fencing}

The device MAY process controlq commands asyncronously and return them
to the driver before the processing is complete.  If the driver needs
to know when the processing is finished it can set the
VIRTIO_GPU_FLAG_FENCE flag in the request.  The device MUST finish the
processing before returning the command then.

Note: current qemu implementation does asyncrounous processing only in
3d mode, when offloading the processing to the host gpu.

\subsubsection{Device Operation: Configure mouse cursor}

The mouse cursor image is a normal resource, except that it must be
64x64 in size.  The driver MUST create and populate the resource
(using the usual VIRTIO_GPU_CMD_RESOURCE_CREATE_2D,
VIRTIO_GPU_CMD_RESOURCE_ATTACH_BACKING and
VIRTIO_GPU_CMD_TRANSFER_TO_HOST_2D controlq commands) and make sure they
are completed (using VIRTIO_GPU_FLAG_FENCE).

Then VIRTIO_GPU_CMD_UPDATE_CURSOR can be sent to the cursorq to set
the pointer shape and position.  To move the pointer without updating
the shape use VIRTIO_GPU_CMD_MOVE_CURSOR instead.

\subsubsection{Device Operation: Request header}\label{sec:Device Types / GPU Device / Device Operation / Device Operation: Request header}

All requests and responses on the virtqueues have a fixed header
using the following layout structure and definitions:

\begin{lstlisting}
enum virtio_gpu_ctrl_type {

        /* 2d commands */
        VIRTIO_GPU_CMD_GET_DISPLAY_INFO = 0x0100,
        VIRTIO_GPU_CMD_RESOURCE_CREATE_2D,
        VIRTIO_GPU_CMD_RESOURCE_UNREF,
        VIRTIO_GPU_CMD_SET_SCANOUT,
        VIRTIO_GPU_CMD_RESOURCE_FLUSH,
        VIRTIO_GPU_CMD_TRANSFER_TO_HOST_2D,
        VIRTIO_GPU_CMD_RESOURCE_ATTACH_BACKING,
        VIRTIO_GPU_CMD_RESOURCE_DETACH_BACKING,
        VIRTIO_GPU_CMD_GET_CAPSET_INFO,
        VIRTIO_GPU_CMD_GET_CAPSET,
        VIRTIO_GPU_CMD_GET_EDID,
        VIRTIO_GPU_CMD_RESOURCE_ASSIGN_UUID,
        VIRTIO_GPU_CMD_RESOURCE_CREATE_BLOB,
        VIRTIO_GPU_CMD_SET_SCANOUT_BLOB,

        /* 3d commands */
        VIRTIO_GPU_CMD_CTX_CREATE = 0x0200,
        VIRTIO_GPU_CMD_CTX_DESTROY,
        VIRTIO_GPU_CMD_CTX_ATTACH_RESOURCE,
        VIRTIO_GPU_CMD_CTX_DETACH_RESOURCE,
        VIRTIO_GPU_CMD_RESOURCE_CREATE_3D,
        VIRTIO_GPU_CMD_TRANSFER_TO_HOST_3D,
        VIRTIO_GPU_CMD_TRANSFER_FROM_HOST_3D,
        VIRTIO_GPU_CMD_SUBMIT_3D,
        VIRTIO_GPU_CMD_RESOURCE_MAP_BLOB,
        VIRTIO_GPU_CMD_RESOURCE_UNMAP_BLOB,

        /* cursor commands */
        VIRTIO_GPU_CMD_UPDATE_CURSOR = 0x0300,
        VIRTIO_GPU_CMD_MOVE_CURSOR,

        /* success responses */
        VIRTIO_GPU_RESP_OK_NODATA = 0x1100,
        VIRTIO_GPU_RESP_OK_DISPLAY_INFO,
        VIRTIO_GPU_RESP_OK_CAPSET_INFO,
        VIRTIO_GPU_RESP_OK_CAPSET,
        VIRTIO_GPU_RESP_OK_EDID,
        VIRTIO_GPU_RESP_OK_RESOURCE_UUID,
        VIRTIO_GPU_RESP_OK_MAP_INFO,

        /* error responses */
        VIRTIO_GPU_RESP_ERR_UNSPEC = 0x1200,
        VIRTIO_GPU_RESP_ERR_OUT_OF_MEMORY,
        VIRTIO_GPU_RESP_ERR_INVALID_SCANOUT_ID,
        VIRTIO_GPU_RESP_ERR_INVALID_RESOURCE_ID,
        VIRTIO_GPU_RESP_ERR_INVALID_CONTEXT_ID,
        VIRTIO_GPU_RESP_ERR_INVALID_PARAMETER,
};

#define VIRTIO_GPU_FLAG_FENCE (1 << 0)
#define VIRTIO_GPU_FLAG_INFO_RING_IDX (1 << 1)

struct virtio_gpu_ctrl_hdr {
        le32 type;
        le32 flags;
        le64 fence_id;
        le32 ctx_id;
        u8 ring_idx;
        u8 padding[3];
};
\end{lstlisting}

The fixed header \field{struct virtio_gpu_ctrl_hdr} in each
request includes the following fields:

\begin{description}
\item[\field{type}] specifies the type of the driver request
  (VIRTIO_GPU_CMD_*) or device response (VIRTIO_GPU_RESP_*).
\item[\field{flags}] request / response flags.
\item[\field{fence_id}] If the driver sets the VIRTIO_GPU_FLAG_FENCE
  bit in the request \field{flags} field the device MUST:
  \begin{itemize*}
  \item set VIRTIO_GPU_FLAG_FENCE bit in the response,
  \item copy the content of the \field{fence_id} field from the
    request to the response, and
  \item send the response only after command processing is complete.
  \end{itemize*}
\item[\field{ctx_id}] Rendering context (used in 3D mode only).
\item[\field{ring_idx}] If VIRTIO_GPU_F_CONTEXT_INIT is supported, then
  the driver MAY set VIRTIO_GPU_FLAG_INFO_RING_IDX bit in the request
  \field{flags}.  In that case:
  \begin{itemize*}
  \item \field{ring_idx} indicates the value of a context-specific ring
   index.  The minimum value is 0 and maximum value is 63 (inclusive).
  \item If VIRTIO_GPU_FLAG_FENCE is set, \field{fence_id} acts as a
   sequence number on the synchronization timeline defined by
   \field{ctx_idx} and the ring index.
  \item If VIRTIO_GPU_FLAG_FENCE is set and when the command associated
   with \field{fence_id} is complete, the device MUST send a response for
   all outstanding commands with a sequence number less than or equal to
   \field{fence_id} on the same synchronization timeline.
  \end{itemize*}
\end{description}

On success the device will return VIRTIO_GPU_RESP_OK_NODATA in
case there is no payload.  Otherwise the \field{type} field will
indicate the kind of payload.

On error the device will return one of the
VIRTIO_GPU_RESP_ERR_* error codes.

\subsubsection{Device Operation: controlq}\label{sec:Device Types / GPU Device / Device Operation / Device Operation: controlq}

For any coordinates given 0,0 is top left, larger x moves right,
larger y moves down.

\begin{description}

\item[VIRTIO_GPU_CMD_GET_DISPLAY_INFO] Retrieve the current output
  configuration.  No request data (just bare \field{struct
    virtio_gpu_ctrl_hdr}).  Response type is
  VIRTIO_GPU_RESP_OK_DISPLAY_INFO, response data is \field{struct
    virtio_gpu_resp_display_info}.

\begin{lstlisting}
#define VIRTIO_GPU_MAX_SCANOUTS 16

struct virtio_gpu_rect {
        le32 x;
        le32 y;
        le32 width;
        le32 height;
};

struct virtio_gpu_resp_display_info {
        struct virtio_gpu_ctrl_hdr hdr;
        struct virtio_gpu_display_one {
                struct virtio_gpu_rect r;
                le32 enabled;
                le32 flags;
        } pmodes[VIRTIO_GPU_MAX_SCANOUTS];
};
\end{lstlisting}

The response contains a list of per-scanout information.  The info
contains whether the scanout is enabled and what its preferred
position and size is.

The size (fields \field{width} and \field{height}) is similar to the
native panel resolution in EDID display information, except that in
the virtual machine case the size can change when the host window
representing the guest display is gets resized.

The position (fields \field{x} and \field{y}) describe how the
displays are arranged (i.e. which is -- for example -- the left
display).

The \field{enabled} field is set when the user enabled the display.
It is roughly the same as the connected state of a phyiscal display
connector.

\item[VIRTIO_GPU_CMD_GET_EDID] Retrieve the EDID data for a given
  scanout.  Request data is \field{struct virtio_gpu_get_edid}).
  Response type is VIRTIO_GPU_RESP_OK_EDID, response data is
  \field{struct virtio_gpu_resp_edid}.  Support is optional and
  negotiated using the VIRTIO_GPU_F_EDID feature flag.

\begin{lstlisting}
struct virtio_gpu_get_edid {
        struct virtio_gpu_ctrl_hdr hdr;
        le32 scanout;
        le32 padding;
};

struct virtio_gpu_resp_edid {
        struct virtio_gpu_ctrl_hdr hdr;
        le32 size;
        le32 padding;
        u8 edid[1024];
};
\end{lstlisting}

The response contains the EDID display data blob (as specified by
VESA) for the scanout.

\item[VIRTIO_GPU_CMD_RESOURCE_CREATE_2D] Create a 2D resource on the
  host.  Request data is \field{struct virtio_gpu_resource_create_2d}.
  Response type is VIRTIO_GPU_RESP_OK_NODATA.

\begin{lstlisting}
enum virtio_gpu_formats {
        VIRTIO_GPU_FORMAT_B8G8R8A8_UNORM  = 1,
        VIRTIO_GPU_FORMAT_B8G8R8X8_UNORM  = 2,
        VIRTIO_GPU_FORMAT_A8R8G8B8_UNORM  = 3,
        VIRTIO_GPU_FORMAT_X8R8G8B8_UNORM  = 4,

        VIRTIO_GPU_FORMAT_R8G8B8A8_UNORM  = 67,
        VIRTIO_GPU_FORMAT_X8B8G8R8_UNORM  = 68,

        VIRTIO_GPU_FORMAT_A8B8G8R8_UNORM  = 121,
        VIRTIO_GPU_FORMAT_R8G8B8X8_UNORM  = 134,
};

struct virtio_gpu_resource_create_2d {
        struct virtio_gpu_ctrl_hdr hdr;
        le32 resource_id;
        le32 format;
        le32 width;
        le32 height;
};
\end{lstlisting}

This creates a 2D resource on the host with the specified width,
height and format.  The resource ids are generated by the guest.

\item[VIRTIO_GPU_CMD_RESOURCE_UNREF] Destroy a resource.  Request data
  is \field{struct virtio_gpu_resource_unref}.  Response type is
  VIRTIO_GPU_RESP_OK_NODATA.

\begin{lstlisting}
struct virtio_gpu_resource_unref {
        struct virtio_gpu_ctrl_hdr hdr;
        le32 resource_id;
        le32 padding;
};
\end{lstlisting}

This informs the host that a resource is no longer required by the
guest.

\item[VIRTIO_GPU_CMD_SET_SCANOUT] Set the scanout parameters for a
  single output.  Request data is \field{struct
    virtio_gpu_set_scanout}.  Response type is
  VIRTIO_GPU_RESP_OK_NODATA.

\begin{lstlisting}
struct virtio_gpu_set_scanout {
        struct virtio_gpu_ctrl_hdr hdr;
        struct virtio_gpu_rect r;
        le32 scanout_id;
        le32 resource_id;
};
\end{lstlisting}

This sets the scanout parameters for a single scanout. The resource_id
is the resource to be scanned out from, along with a rectangle.

Scanout rectangles must be completely covered by the underlying
resource.  Overlapping (or identical) scanouts are allowed, typical
use case is screen mirroring.

The driver can use resource_id = 0 to disable a scanout.

\item[VIRTIO_GPU_CMD_RESOURCE_FLUSH] Flush a scanout resource Request
  data is \field{struct virtio_gpu_resource_flush}.  Response type is
  VIRTIO_GPU_RESP_OK_NODATA.

\begin{lstlisting}
struct virtio_gpu_resource_flush {
        struct virtio_gpu_ctrl_hdr hdr;
        struct virtio_gpu_rect r;
        le32 resource_id;
        le32 padding;
};
\end{lstlisting}

This flushes a resource to screen.  It takes a rectangle and a
resource id, and flushes any scanouts the resource is being used on.

\item[VIRTIO_GPU_CMD_TRANSFER_TO_HOST_2D] Transfer from guest memory
  to host resource.  Request data is \field{struct
    virtio_gpu_transfer_to_host_2d}.  Response type is
  VIRTIO_GPU_RESP_OK_NODATA.

\begin{lstlisting}
struct virtio_gpu_transfer_to_host_2d {
        struct virtio_gpu_ctrl_hdr hdr;
        struct virtio_gpu_rect r;
        le64 offset;
        le32 resource_id;
        le32 padding;
};
\end{lstlisting}

This takes a resource id along with an destination offset into the
resource, and a box to transfer to the host backing for the resource.

\item[VIRTIO_GPU_CMD_RESOURCE_ATTACH_BACKING] Assign backing pages to
  a resource.  Request data is \field{struct
    virtio_gpu_resource_attach_backing}, followed by \field{struct
    virtio_gpu_mem_entry} entries.  Response type is
  VIRTIO_GPU_RESP_OK_NODATA.

\begin{lstlisting}
struct virtio_gpu_resource_attach_backing {
        struct virtio_gpu_ctrl_hdr hdr;
        le32 resource_id;
        le32 nr_entries;
};

struct virtio_gpu_mem_entry {
        le64 addr;
        le32 length;
        le32 padding;
};
\end{lstlisting}

This assign an array of guest pages as the backing store for a
resource. These pages are then used for the transfer operations for
that resource from that point on.

\item[VIRTIO_GPU_CMD_RESOURCE_DETACH_BACKING] Detach backing pages
  from a resource.  Request data is \field{struct
    virtio_gpu_resource_detach_backing}.  Response type is
  VIRTIO_GPU_RESP_OK_NODATA.

\begin{lstlisting}
struct virtio_gpu_resource_detach_backing {
        struct virtio_gpu_ctrl_hdr hdr;
        le32 resource_id;
        le32 padding;
};
\end{lstlisting}

This detaches any backing pages from a resource, to be used in case of
guest swapping or object destruction.

\item[VIRTIO_GPU_CMD_GET_CAPSET_INFO] Gets the information associated with
  a particular \field{capset_index}, which MUST less than \field{num_capsets}
  defined in the device configuration.  Request data is
  \field{struct virtio_gpu_get_capset_info}.  Response type is
  VIRTIO_GPU_RESP_OK_CAPSET_INFO.

  On success, \field{struct virtio_gpu_resp_capset_info} contains the
  \field{capset_id}, \field{capset_max_version}, \field{capset_max_size}
  associated with capset at the specified {capset_idex}.  field{capset_id} MUST
  be one of the following (see listing for values):

  \begin{itemize*}
  \item \href{https://gitlab.freedesktop.org/virgl/virglrenderer/-/blob/master/src/virgl_hw.h#L526}{VIRTIO_GPU_CAPSET_VIRGL} --
	the first edition of Virgl (Gallium OpenGL) protocol.
  \item \href{https://gitlab.freedesktop.org/virgl/virglrenderer/-/blob/master/src/virgl_hw.h#L550}{VIRTIO_GPU_CAPSET_VIRGL2} --
	the second edition of Virgl (Gallium OpenGL) protocol after the capset fix.
  \item \href{https://android.googlesource.com/device/generic/vulkan-cereal/+/refs/heads/master/protocols/}{VIRTIO_GPU_CAPSET_GFXSTREAM} --
	gfxtream's (mostly) autogenerated GLES and Vulkan streaming protocols.
  \item \href{https://gitlab.freedesktop.org/olv/venus-protocol}{VIRTIO_GPU_CAPSET_VENUS} --
	Mesa's (mostly) autogenerated Vulkan protocol.
  \item \href{https://chromium.googlesource.com/chromiumos/platform/crosvm/+/refs/heads/main/rutabaga_gfx/src/cross_domain/cross_domain_protocol.rs}{VIRTIO_GPU_CAPSET_CROSS_DOMAIN} --
	protocol for display virtualization via Wayland proxying.
  \end{itemize*}

\begin{lstlisting}
struct virtio_gpu_get_capset_info {
        struct virtio_gpu_ctrl_hdr hdr;
        le32 capset_index;
        le32 padding;
};

#define VIRTIO_GPU_CAPSET_VIRGL 1
#define VIRTIO_GPU_CAPSET_VIRGL2 2
#define VIRTIO_GPU_CAPSET_GFXSTREAM 3
#define VIRTIO_GPU_CAPSET_VENUS 4
#define VIRTIO_GPU_CAPSET_CROSS_DOMAIN 5
struct virtio_gpu_resp_capset_info {
        struct virtio_gpu_ctrl_hdr hdr;
        le32 capset_id;
        le32 capset_max_version;
        le32 capset_max_size;
        le32 padding;
};
\end{lstlisting}

\item[VIRTIO_GPU_CMD_GET_CAPSET] Gets the capset associated with a
  particular \field{capset_id} and \field{capset_version}.  Request data is
  \field{struct virtio_gpu_get_capset}.  Response type is
  VIRTIO_GPU_RESP_OK_CAPSET.

\begin{lstlisting}
struct virtio_gpu_get_capset {
        struct virtio_gpu_ctrl_hdr hdr;
        le32 capset_id;
        le32 capset_version;
};

struct virtio_gpu_resp_capset {
        struct virtio_gpu_ctrl_hdr hdr;
        u8 capset_data[];
};
\end{lstlisting}

\item[VIRTIO_GPU_CMD_RESOURCE_ASSIGN_UUID] Creates an exported object from
  a resource. Request data is \field{struct
    virtio_gpu_resource_assign_uuid}.  Response type is
  VIRTIO_GPU_RESP_OK_RESOURCE_UUID, response data is \field{struct
    virtio_gpu_resp_resource_uuid}. Support is optional and negotiated
    using the VIRTIO_GPU_F_RESOURCE_UUID feature flag.

\begin{lstlisting}
struct virtio_gpu_resource_assign_uuid {
        struct virtio_gpu_ctrl_hdr hdr;
        le32 resource_id;
        le32 padding;
};

struct virtio_gpu_resp_resource_uuid {
        struct virtio_gpu_ctrl_hdr hdr;
        u8 uuid[16];
};
\end{lstlisting}

The response contains a UUID which identifies the exported object created from
the host private resource. Note that if the resource has an attached backing,
modifications made to the host private resource through the exported object by
other devices are not visible in the attached backing until they are transferred
into the backing.

\item[VIRTIO_GPU_CMD_RESOURCE_CREATE_BLOB] Creates a virtio-gpu blob
  resource. Request data is \field{struct
  virtio_gpu_resource_create_blob}, followed by \field{struct
  virtio_gpu_mem_entry} entries. Response type is
  VIRTIO_GPU_RESP_OK_NODATA. Support is optional and negotiated
  using the VIRTIO_GPU_F_RESOURCE_BLOB feature flag.

\begin{lstlisting}
#define VIRTIO_GPU_BLOB_MEM_GUEST             0x0001
#define VIRTIO_GPU_BLOB_MEM_HOST3D            0x0002
#define VIRTIO_GPU_BLOB_MEM_HOST3D_GUEST      0x0003

#define VIRTIO_GPU_BLOB_FLAG_USE_MAPPABLE     0x0001
#define VIRTIO_GPU_BLOB_FLAG_USE_SHAREABLE    0x0002
#define VIRTIO_GPU_BLOB_FLAG_USE_CROSS_DEVICE 0x0004

struct virtio_gpu_resource_create_blob {
       struct virtio_gpu_ctrl_hdr hdr;
       le32 resource_id;
       le32 blob_mem;
       le32 blob_flags;
       le32 nr_entries;
       le64 blob_id;
       le64 size;
};

\end{lstlisting}

A blob resource is a container for:

  \begin{itemize*}
  \item a guest memory allocation (referred to as a
  "guest-only blob resource").
  \item a host memory allocation (referred to as a
  "host-only blob resource").
  \item a guest memory and host memory allocation (referred
  to as a "default blob resource").
  \end{itemize*}

The memory properties of the blob resource MUST be described by
\field{blob_mem}, which MUST be non-zero.

For default and guest-only blob resources, \field{nr_entries} guest
memory entries may be assigned to the resource.  For default blob resources
(i.e, when \field{blob_mem} is VIRTIO_GPU_BLOB_MEM_HOST3D_GUEST), these
memory entries are used as a shadow buffer for the host memory. To
facilitate drivers that support swap-in and swap-out, \field{nr_entries} may
be zero and VIRTIO_GPU_CMD_RESOURCE_ATTACH_BACKING may be subsequently used.
VIRTIO_GPU_CMD_RESOURCE_DETACH_BACKING may be used to unassign memory entries.

\field{blob_mem} can only be VIRTIO_GPU_BLOB_MEM_HOST3D and
VIRTIO_GPU_BLOB_MEM_HOST3D_GUEST if VIRTIO_GPU_F_VIRGL is supported.
VIRTIO_GPU_BLOB_MEM_GUEST is valid regardless whether VIRTIO_GPU_F_VIRGL
is supported or not.

For VIRTIO_GPU_BLOB_MEM_HOST3D and VIRTIO_GPU_BLOB_MEM_HOST3D_GUEST, the
virtio-gpu resource MUST be created from the rendering context local object
identified by the \field{blob_id}. The actual allocation is done via
VIRTIO_GPU_CMD_SUBMIT_3D.

The driver MUST inform the device if the blob resource is used for
memory access, sharing between driver instances and/or sharing with
other devices. This is done via the \field{blob_flags} field.

If VIRTIO_GPU_F_VIRGL is set, both VIRTIO_GPU_CMD_TRANSFER_TO_HOST_3D
and VIRTIO_GPU_CMD_TRANSFER_FROM_HOST_3D may be used to update the
resource. There is no restriction on the image/buffer view the driver
has on the blob resource.

\item[VIRTIO_GPU_CMD_SET_SCANOUT_BLOB] sets scanout parameters for a
   blob resource. Request data is
  \field{struct virtio_gpu_set_scanout_blob}. Response type is
  VIRTIO_GPU_RESP_OK_NODATA. Support is optional and negotiated
  using the VIRTIO_GPU_F_RESOURCE_BLOB feature flag.

\begin{lstlisting}
struct virtio_gpu_set_scanout_blob {
       struct virtio_gpu_ctrl_hdr hdr;
       struct virtio_gpu_rect r;
       le32 scanout_id;
       le32 resource_id;
       le32 width;
       le32 height;
       le32 format;
       le32 padding;
       le32 strides[4];
       le32 offsets[4];
};
\end{lstlisting}

The rectangle \field{r} represents the portion of the blob resource being
displayed. The rest is the metadata associated with the blob resource. The
format MUST be one of \field{enum virtio_gpu_formats}.  The format MAY be
compressed with header and data planes.

\end{description}

\subsubsection{Device Operation: controlq (3d)}\label{sec:Device Types / GPU Device / Device Operation / Device Operation: controlq (3d)}

These commands are supported by the device if the VIRTIO_GPU_F_VIRGL
feature flag is set.

\begin{description}

\item[VIRTIO_GPU_CMD_CTX_CREATE] creates a context for submitting an opaque
  command stream.  Request data is \field{struct virtio_gpu_ctx_create}.
  Response type is VIRTIO_GPU_RESP_OK_NODATA.

\begin{lstlisting}
#define VIRTIO_GPU_CONTEXT_INIT_CAPSET_ID_MASK 0x000000ff;
struct virtio_gpu_ctx_create {
       struct virtio_gpu_ctrl_hdr hdr;
       le32 nlen;
       le32 context_init;
       char debug_name[64];
};
\end{lstlisting}

The implementation MUST create a context for the given \field{ctx_id} in
the \field{hdr}.  For debugging purposes, a \field{debug_name} and it's
length \field{nlen} is provided by the driver.  If
VIRTIO_GPU_F_CONTEXT_INIT is supported, then lower 8 bits of
\field{context_init} MAY contain the \field{capset_id} associated with
context.  In that case, then the device MUST create a context that can
handle the specified command stream.

If the lower 8-bits of the \field{context_init} are zero, then the type of
the context is determined by the device.

\item[VIRTIO_GPU_CMD_CTX_DESTROY]
\item[VIRTIO_GPU_CMD_CTX_ATTACH_RESOURCE]
\item[VIRTIO_GPU_CMD_CTX_DETACH_RESOURCE]
  Manage virtio-gpu 3d contexts.

\item[VIRTIO_GPU_CMD_RESOURCE_CREATE_3D]
  Create virtio-gpu 3d resources.

\item[VIRTIO_GPU_CMD_TRANSFER_TO_HOST_3D]
\item[VIRTIO_GPU_CMD_TRANSFER_FROM_HOST_3D]
  Transfer data from and to virtio-gpu 3d resources.

\item[VIRTIO_GPU_CMD_SUBMIT_3D]
  Submit an opaque command stream.  The type of the command stream is
  determined when creating a context.

\item[VIRTIO_GPU_CMD_RESOURCE_MAP_BLOB] maps a host-only
  blob resource into an offset in the host visible memory region. Request
  data is \field{struct virtio_gpu_resource_map_blob}.  The driver MUST
  not map a blob resource that is already mapped.  Response type is
  VIRTIO_GPU_RESP_OK_MAP_INFO. Support is optional and negotiated
  using the VIRTIO_GPU_F_RESOURCE_BLOB feature flag and checking for
  the presence of the host visible memory region.

\begin{lstlisting}
struct virtio_gpu_resource_map_blob {
        struct virtio_gpu_ctrl_hdr hdr;
        le32 resource_id;
        le32 padding;
        le64 offset;
};

#define VIRTIO_GPU_MAP_CACHE_MASK      0x0f
#define VIRTIO_GPU_MAP_CACHE_NONE      0x00
#define VIRTIO_GPU_MAP_CACHE_CACHED    0x01
#define VIRTIO_GPU_MAP_CACHE_UNCACHED  0x02
#define VIRTIO_GPU_MAP_CACHE_WC        0x03
struct virtio_gpu_resp_map_info {
        struct virtio_gpu_ctrl_hdr hdr;
        u32 map_info;
        u32 padding;
};
\end{lstlisting}

\item[VIRTIO_GPU_CMD_RESOURCE_UNMAP_BLOB] unmaps a
  host-only blob resource from the host visible memory region. Request data
  is \field{struct virtio_gpu_resource_unmap_blob}.  Response type is
  VIRTIO_GPU_RESP_OK_NODATA.  Support is optional and negotiated
  using the VIRTIO_GPU_F_RESOURCE_BLOB feature flag and checking for
  the presence of the host visible memory region.

\begin{lstlisting}
struct virtio_gpu_resource_unmap_blob {
        struct virtio_gpu_ctrl_hdr hdr;
        le32 resource_id;
        le32 padding;
};
\end{lstlisting}

\end{description}

\subsubsection{Device Operation: cursorq}\label{sec:Device Types / GPU Device / Device Operation / Device Operation: cursorq}

Both cursorq commands use the same command struct.

\begin{lstlisting}
struct virtio_gpu_cursor_pos {
        le32 scanout_id;
        le32 x;
        le32 y;
        le32 padding;
};

struct virtio_gpu_update_cursor {
        struct virtio_gpu_ctrl_hdr hdr;
        struct virtio_gpu_cursor_pos pos;
        le32 resource_id;
        le32 hot_x;
        le32 hot_y;
        le32 padding;
};
\end{lstlisting}

\begin{description}

\item[VIRTIO_GPU_CMD_UPDATE_CURSOR]
Update cursor.
Request data is \field{struct virtio_gpu_update_cursor}.
Response type is VIRTIO_GPU_RESP_OK_NODATA.

Full cursor update.  Cursor will be loaded from the specified
\field{resource_id} and will be moved to \field{pos}.  The driver must
transfer the cursor into the resource beforehand (using control queue
commands) and make sure the commands to fill the resource are actually
processed (using fencing).

\item[VIRTIO_GPU_CMD_MOVE_CURSOR]
Move cursor.
Request data is \field{struct virtio_gpu_update_cursor}.
Response type is VIRTIO_GPU_RESP_OK_NODATA.

Move cursor to the place specified in \field{pos}.  The other fields
are not used and will be ignored by the device.

\end{description}

\subsection{VGA Compatibility}\label{sec:Device Types / GPU Device / VGA Compatibility}

Applies to Virtio Over PCI only.  The GPU device can come with and
without VGA compatibility.  The PCI class should be DISPLAY_VGA if VGA
compatibility is present and DISPLAY_OTHER otherwise.

VGA compatibility: PCI region 0 has the linear framebuffer, standard
vga registers are present.  Configuring a scanout
(VIRTIO_GPU_CMD_SET_SCANOUT) switches the device from vga
compatibility mode into native virtio mode.  A reset switches it back
into vga compatibility mode.

Note: qemu implementation also provides bochs dispi interface io ports
and mmio bar at pci region 1 and is therefore fully compatible with
the qemu stdvga (see \href{https://git.qemu-project.org/?p=qemu.git;a=blob;f=docs/specs/standard-vga.txt;hb=HEAD}{docs/specs/standard-vga.txt} in the qemu source tree).

\section{GPU Device}\label{sec:Device Types / GPU Device}

virtio-gpu is a virtio based graphics adapter.  It can operate in 2D
mode and in 3D mode.  3D mode will offload rendering ops to
the host gpu and therefore requires a gpu with 3D support on the host
machine.

In 2D mode the virtio-gpu device provides support for ARGB Hardware
cursors and multiple scanouts (aka heads).

\subsection{Device ID}\label{sec:Device Types / GPU Device / Device ID}

16

\subsection{Virtqueues}\label{sec:Device Types / GPU Device / Virtqueues}

\begin{description}
\item[0] controlq - queue for sending control commands
\item[1] cursorq - queue for sending cursor updates
\end{description}

Both queues have the same format.  Each request and each response have
a fixed header, followed by command specific data fields.  The
separate cursor queue is the "fast track" for cursor commands
(VIRTIO_GPU_CMD_UPDATE_CURSOR and VIRTIO_GPU_CMD_MOVE_CURSOR), so they
go through without being delayed by time-consuming commands in the
control queue.

\subsection{Feature bits}\label{sec:Device Types / GPU Device / Feature bits}

\begin{description}
\item[VIRTIO_GPU_F_VIRGL (0)] virgl 3D mode is supported.
\item[VIRTIO_GPU_F_EDID  (1)] EDID is supported.
\item[VIRTIO_GPU_F_RESOURCE_UUID (2)] assigning resources UUIDs for export
  to other virtio devices is supported.
\item[VIRTIO_GPU_F_RESOURCE_BLOB (3)] creating and using size-based blob
  resources is supported.
\item[VIRTIO_GPU_F_CONTEXT_INIT (4)] multiple context types and
  synchronization timelines supported.  Requires VIRTIO_GPU_F_VIRGL.
\end{description}

\subsection{Device configuration layout}\label{sec:Device Types / GPU Device / Device configuration layout}

GPU device configuration uses the following layout structure and
definitions:

\begin{lstlisting}
#define VIRTIO_GPU_EVENT_DISPLAY (1 << 0)

struct virtio_gpu_config {
        le32 events_read;
        le32 events_clear;
        le32 num_scanouts;
        le32 num_capsets;
};
\end{lstlisting}

\subsubsection{Device configuration fields}

\begin{description}
\item[\field{events_read}] signals pending events to the driver.  The
  driver MUST NOT write to this field.
\item[\field{events_clear}] clears pending events in the device.
  Writing a '1' into a bit will clear the corresponding bit in
  \field{events_read}, mimicking write-to-clear behavior.
\item[\field{num_scanouts}] specifies the maximum number of scanouts
  supported by the device.  Minimum value is 1, maximum value is 16.
\item[\field{num_capsets}] specifies the maximum number of capability
  sets supported by the device.  The minimum value is zero.
\end{description}

\subsubsection{Events}

\begin{description}
\item[VIRTIO_GPU_EVENT_DISPLAY] Display configuration has changed.
  The driver SHOULD use the VIRTIO_GPU_CMD_GET_DISPLAY_INFO command to
  fetch the information from the device.  In case EDID support is
  negotiated (VIRTIO_GPU_F_EDID feature flag) the device SHOULD also
  fetch the updated EDID blobs using the VIRTIO_GPU_CMD_GET_EDID
  command.
\end{description}

\devicenormative{\subsection}{Device Initialization}{Device Types / GPU Device / Device Initialization}

The driver SHOULD query the display information from the device using
the VIRTIO_GPU_CMD_GET_DISPLAY_INFO command and use that information
for the initial scanout setup.  In case EDID support is negotiated
(VIRTIO_GPU_F_EDID feature flag) the device SHOULD also fetch the EDID
information using the VIRTIO_GPU_CMD_GET_EDID command.  If no
information is available or all displays are disabled the driver MAY
choose to use a fallback, such as 1024x768 at display 0.

The driver SHOULD query all shared memory regions supported by the device.
If the device supports shared memory, the \field{shmid} of a region MUST
(see \ref{sec:Basic Facilities of a Virtio Device /
Shared Memory Regions}~\nameref{sec:Basic Facilities of a Virtio Device /
Shared Memory Regions}) be one of the following:

\begin{lstlisting}
enum virtio_gpu_shm_id {
        VIRTIO_GPU_SHM_ID_UNDEFINED = 0,
        VIRTIO_GPU_SHM_ID_HOST_VISIBLE = 1,
};
\end{lstlisting}

The shared memory region with VIRTIO_GPU_SHM_ID_HOST_VISIBLE is referred as
the "host visible memory region".  The device MUST support the
VIRTIO_GPU_CMD_RESOURCE_MAP_BLOB and VIRTIO_GPU_CMD_RESOURCE_UNMAP_BLOB
if the host visible memory region is available.

\subsection{Device Operation}\label{sec:Device Types / GPU Device / Device Operation}

The virtio-gpu is based around the concept of resources private to the
host.  The guest must DMA transfer into these resources, unless shared memory
regions are supported. This is a design requirement in order to interface with
future 3D rendering. In the unaccelerated 2D mode there is no support for DMA
transfers from resources, just to them.

Resources are initially simple 2D resources, consisting of a width,
height and format along with an identifier. The guest must then attach
backing store to the resources in order for DMA transfers to
work. This is like a GART in a real GPU.

\subsubsection{Device Operation: Create a framebuffer and configure scanout}

\begin{itemize*}
\item Create a host resource using VIRTIO_GPU_CMD_RESOURCE_CREATE_2D.
\item Allocate a framebuffer from guest ram, and attach it as backing
  storage to the resource just created, using
  VIRTIO_GPU_CMD_RESOURCE_ATTACH_BACKING.  Scatter lists are
  supported, so the framebuffer doesn't need to be contignous in guest
  physical memory.
\item Use VIRTIO_GPU_CMD_SET_SCANOUT to link the framebuffer to
  a display scanout.
\end{itemize*}

\subsubsection{Device Operation: Update a framebuffer and scanout}

\begin{itemize*}
\item Render to your framebuffer memory.
\item Use VIRTIO_GPU_CMD_TRANSFER_TO_HOST_2D to update the host resource
  from guest memory.
\item Use VIRTIO_GPU_CMD_RESOURCE_FLUSH to flush the updated resource
  to the display.
\end{itemize*}

\subsubsection{Device Operation: Using pageflip}

It is possible to create multiple framebuffers, flip between them
using VIRTIO_GPU_CMD_SET_SCANOUT and VIRTIO_GPU_CMD_RESOURCE_FLUSH,
and update the invisible framebuffer using
VIRTIO_GPU_CMD_TRANSFER_TO_HOST_2D.

\subsubsection{Device Operation: Multihead setup}

In case two or more displays are present there are different ways to
configure things:

\begin{itemize*}
\item Create a single framebuffer, link it to all displays
  (mirroring).
\item Create an framebuffer for each display.
\item Create one big framebuffer, configure scanouts to display a
  different rectangle of that framebuffer each.
\end{itemize*}

\devicenormative{\subsubsection}{Device Operation: Command lifecycle and fencing}{Device Types / GPU Device / Device Operation / Device Operation: Command lifecycle and fencing}

The device MAY process controlq commands asyncronously and return them
to the driver before the processing is complete.  If the driver needs
to know when the processing is finished it can set the
VIRTIO_GPU_FLAG_FENCE flag in the request.  The device MUST finish the
processing before returning the command then.

Note: current qemu implementation does asyncrounous processing only in
3d mode, when offloading the processing to the host gpu.

\subsubsection{Device Operation: Configure mouse cursor}

The mouse cursor image is a normal resource, except that it must be
64x64 in size.  The driver MUST create and populate the resource
(using the usual VIRTIO_GPU_CMD_RESOURCE_CREATE_2D,
VIRTIO_GPU_CMD_RESOURCE_ATTACH_BACKING and
VIRTIO_GPU_CMD_TRANSFER_TO_HOST_2D controlq commands) and make sure they
are completed (using VIRTIO_GPU_FLAG_FENCE).

Then VIRTIO_GPU_CMD_UPDATE_CURSOR can be sent to the cursorq to set
the pointer shape and position.  To move the pointer without updating
the shape use VIRTIO_GPU_CMD_MOVE_CURSOR instead.

\subsubsection{Device Operation: Request header}\label{sec:Device Types / GPU Device / Device Operation / Device Operation: Request header}

All requests and responses on the virtqueues have a fixed header
using the following layout structure and definitions:

\begin{lstlisting}
enum virtio_gpu_ctrl_type {

        /* 2d commands */
        VIRTIO_GPU_CMD_GET_DISPLAY_INFO = 0x0100,
        VIRTIO_GPU_CMD_RESOURCE_CREATE_2D,
        VIRTIO_GPU_CMD_RESOURCE_UNREF,
        VIRTIO_GPU_CMD_SET_SCANOUT,
        VIRTIO_GPU_CMD_RESOURCE_FLUSH,
        VIRTIO_GPU_CMD_TRANSFER_TO_HOST_2D,
        VIRTIO_GPU_CMD_RESOURCE_ATTACH_BACKING,
        VIRTIO_GPU_CMD_RESOURCE_DETACH_BACKING,
        VIRTIO_GPU_CMD_GET_CAPSET_INFO,
        VIRTIO_GPU_CMD_GET_CAPSET,
        VIRTIO_GPU_CMD_GET_EDID,
        VIRTIO_GPU_CMD_RESOURCE_ASSIGN_UUID,
        VIRTIO_GPU_CMD_RESOURCE_CREATE_BLOB,
        VIRTIO_GPU_CMD_SET_SCANOUT_BLOB,

        /* 3d commands */
        VIRTIO_GPU_CMD_CTX_CREATE = 0x0200,
        VIRTIO_GPU_CMD_CTX_DESTROY,
        VIRTIO_GPU_CMD_CTX_ATTACH_RESOURCE,
        VIRTIO_GPU_CMD_CTX_DETACH_RESOURCE,
        VIRTIO_GPU_CMD_RESOURCE_CREATE_3D,
        VIRTIO_GPU_CMD_TRANSFER_TO_HOST_3D,
        VIRTIO_GPU_CMD_TRANSFER_FROM_HOST_3D,
        VIRTIO_GPU_CMD_SUBMIT_3D,
        VIRTIO_GPU_CMD_RESOURCE_MAP_BLOB,
        VIRTIO_GPU_CMD_RESOURCE_UNMAP_BLOB,

        /* cursor commands */
        VIRTIO_GPU_CMD_UPDATE_CURSOR = 0x0300,
        VIRTIO_GPU_CMD_MOVE_CURSOR,

        /* success responses */
        VIRTIO_GPU_RESP_OK_NODATA = 0x1100,
        VIRTIO_GPU_RESP_OK_DISPLAY_INFO,
        VIRTIO_GPU_RESP_OK_CAPSET_INFO,
        VIRTIO_GPU_RESP_OK_CAPSET,
        VIRTIO_GPU_RESP_OK_EDID,
        VIRTIO_GPU_RESP_OK_RESOURCE_UUID,
        VIRTIO_GPU_RESP_OK_MAP_INFO,

        /* error responses */
        VIRTIO_GPU_RESP_ERR_UNSPEC = 0x1200,
        VIRTIO_GPU_RESP_ERR_OUT_OF_MEMORY,
        VIRTIO_GPU_RESP_ERR_INVALID_SCANOUT_ID,
        VIRTIO_GPU_RESP_ERR_INVALID_RESOURCE_ID,
        VIRTIO_GPU_RESP_ERR_INVALID_CONTEXT_ID,
        VIRTIO_GPU_RESP_ERR_INVALID_PARAMETER,
};

#define VIRTIO_GPU_FLAG_FENCE (1 << 0)
#define VIRTIO_GPU_FLAG_INFO_RING_IDX (1 << 1)

struct virtio_gpu_ctrl_hdr {
        le32 type;
        le32 flags;
        le64 fence_id;
        le32 ctx_id;
        u8 ring_idx;
        u8 padding[3];
};
\end{lstlisting}

The fixed header \field{struct virtio_gpu_ctrl_hdr} in each
request includes the following fields:

\begin{description}
\item[\field{type}] specifies the type of the driver request
  (VIRTIO_GPU_CMD_*) or device response (VIRTIO_GPU_RESP_*).
\item[\field{flags}] request / response flags.
\item[\field{fence_id}] If the driver sets the VIRTIO_GPU_FLAG_FENCE
  bit in the request \field{flags} field the device MUST:
  \begin{itemize*}
  \item set VIRTIO_GPU_FLAG_FENCE bit in the response,
  \item copy the content of the \field{fence_id} field from the
    request to the response, and
  \item send the response only after command processing is complete.
  \end{itemize*}
\item[\field{ctx_id}] Rendering context (used in 3D mode only).
\item[\field{ring_idx}] If VIRTIO_GPU_F_CONTEXT_INIT is supported, then
  the driver MAY set VIRTIO_GPU_FLAG_INFO_RING_IDX bit in the request
  \field{flags}.  In that case:
  \begin{itemize*}
  \item \field{ring_idx} indicates the value of a context-specific ring
   index.  The minimum value is 0 and maximum value is 63 (inclusive).
  \item If VIRTIO_GPU_FLAG_FENCE is set, \field{fence_id} acts as a
   sequence number on the synchronization timeline defined by
   \field{ctx_idx} and the ring index.
  \item If VIRTIO_GPU_FLAG_FENCE is set and when the command associated
   with \field{fence_id} is complete, the device MUST send a response for
   all outstanding commands with a sequence number less than or equal to
   \field{fence_id} on the same synchronization timeline.
  \end{itemize*}
\end{description}

On success the device will return VIRTIO_GPU_RESP_OK_NODATA in
case there is no payload.  Otherwise the \field{type} field will
indicate the kind of payload.

On error the device will return one of the
VIRTIO_GPU_RESP_ERR_* error codes.

\subsubsection{Device Operation: controlq}\label{sec:Device Types / GPU Device / Device Operation / Device Operation: controlq}

For any coordinates given 0,0 is top left, larger x moves right,
larger y moves down.

\begin{description}

\item[VIRTIO_GPU_CMD_GET_DISPLAY_INFO] Retrieve the current output
  configuration.  No request data (just bare \field{struct
    virtio_gpu_ctrl_hdr}).  Response type is
  VIRTIO_GPU_RESP_OK_DISPLAY_INFO, response data is \field{struct
    virtio_gpu_resp_display_info}.

\begin{lstlisting}
#define VIRTIO_GPU_MAX_SCANOUTS 16

struct virtio_gpu_rect {
        le32 x;
        le32 y;
        le32 width;
        le32 height;
};

struct virtio_gpu_resp_display_info {
        struct virtio_gpu_ctrl_hdr hdr;
        struct virtio_gpu_display_one {
                struct virtio_gpu_rect r;
                le32 enabled;
                le32 flags;
        } pmodes[VIRTIO_GPU_MAX_SCANOUTS];
};
\end{lstlisting}

The response contains a list of per-scanout information.  The info
contains whether the scanout is enabled and what its preferred
position and size is.

The size (fields \field{width} and \field{height}) is similar to the
native panel resolution in EDID display information, except that in
the virtual machine case the size can change when the host window
representing the guest display is gets resized.

The position (fields \field{x} and \field{y}) describe how the
displays are arranged (i.e. which is -- for example -- the left
display).

The \field{enabled} field is set when the user enabled the display.
It is roughly the same as the connected state of a phyiscal display
connector.

\item[VIRTIO_GPU_CMD_GET_EDID] Retrieve the EDID data for a given
  scanout.  Request data is \field{struct virtio_gpu_get_edid}).
  Response type is VIRTIO_GPU_RESP_OK_EDID, response data is
  \field{struct virtio_gpu_resp_edid}.  Support is optional and
  negotiated using the VIRTIO_GPU_F_EDID feature flag.

\begin{lstlisting}
struct virtio_gpu_get_edid {
        struct virtio_gpu_ctrl_hdr hdr;
        le32 scanout;
        le32 padding;
};

struct virtio_gpu_resp_edid {
        struct virtio_gpu_ctrl_hdr hdr;
        le32 size;
        le32 padding;
        u8 edid[1024];
};
\end{lstlisting}

The response contains the EDID display data blob (as specified by
VESA) for the scanout.

\item[VIRTIO_GPU_CMD_RESOURCE_CREATE_2D] Create a 2D resource on the
  host.  Request data is \field{struct virtio_gpu_resource_create_2d}.
  Response type is VIRTIO_GPU_RESP_OK_NODATA.

\begin{lstlisting}
enum virtio_gpu_formats {
        VIRTIO_GPU_FORMAT_B8G8R8A8_UNORM  = 1,
        VIRTIO_GPU_FORMAT_B8G8R8X8_UNORM  = 2,
        VIRTIO_GPU_FORMAT_A8R8G8B8_UNORM  = 3,
        VIRTIO_GPU_FORMAT_X8R8G8B8_UNORM  = 4,

        VIRTIO_GPU_FORMAT_R8G8B8A8_UNORM  = 67,
        VIRTIO_GPU_FORMAT_X8B8G8R8_UNORM  = 68,

        VIRTIO_GPU_FORMAT_A8B8G8R8_UNORM  = 121,
        VIRTIO_GPU_FORMAT_R8G8B8X8_UNORM  = 134,
};

struct virtio_gpu_resource_create_2d {
        struct virtio_gpu_ctrl_hdr hdr;
        le32 resource_id;
        le32 format;
        le32 width;
        le32 height;
};
\end{lstlisting}

This creates a 2D resource on the host with the specified width,
height and format.  The resource ids are generated by the guest.

\item[VIRTIO_GPU_CMD_RESOURCE_UNREF] Destroy a resource.  Request data
  is \field{struct virtio_gpu_resource_unref}.  Response type is
  VIRTIO_GPU_RESP_OK_NODATA.

\begin{lstlisting}
struct virtio_gpu_resource_unref {
        struct virtio_gpu_ctrl_hdr hdr;
        le32 resource_id;
        le32 padding;
};
\end{lstlisting}

This informs the host that a resource is no longer required by the
guest.

\item[VIRTIO_GPU_CMD_SET_SCANOUT] Set the scanout parameters for a
  single output.  Request data is \field{struct
    virtio_gpu_set_scanout}.  Response type is
  VIRTIO_GPU_RESP_OK_NODATA.

\begin{lstlisting}
struct virtio_gpu_set_scanout {
        struct virtio_gpu_ctrl_hdr hdr;
        struct virtio_gpu_rect r;
        le32 scanout_id;
        le32 resource_id;
};
\end{lstlisting}

This sets the scanout parameters for a single scanout. The resource_id
is the resource to be scanned out from, along with a rectangle.

Scanout rectangles must be completely covered by the underlying
resource.  Overlapping (or identical) scanouts are allowed, typical
use case is screen mirroring.

The driver can use resource_id = 0 to disable a scanout.

\item[VIRTIO_GPU_CMD_RESOURCE_FLUSH] Flush a scanout resource Request
  data is \field{struct virtio_gpu_resource_flush}.  Response type is
  VIRTIO_GPU_RESP_OK_NODATA.

\begin{lstlisting}
struct virtio_gpu_resource_flush {
        struct virtio_gpu_ctrl_hdr hdr;
        struct virtio_gpu_rect r;
        le32 resource_id;
        le32 padding;
};
\end{lstlisting}

This flushes a resource to screen.  It takes a rectangle and a
resource id, and flushes any scanouts the resource is being used on.

\item[VIRTIO_GPU_CMD_TRANSFER_TO_HOST_2D] Transfer from guest memory
  to host resource.  Request data is \field{struct
    virtio_gpu_transfer_to_host_2d}.  Response type is
  VIRTIO_GPU_RESP_OK_NODATA.

\begin{lstlisting}
struct virtio_gpu_transfer_to_host_2d {
        struct virtio_gpu_ctrl_hdr hdr;
        struct virtio_gpu_rect r;
        le64 offset;
        le32 resource_id;
        le32 padding;
};
\end{lstlisting}

This takes a resource id along with an destination offset into the
resource, and a box to transfer to the host backing for the resource.

\item[VIRTIO_GPU_CMD_RESOURCE_ATTACH_BACKING] Assign backing pages to
  a resource.  Request data is \field{struct
    virtio_gpu_resource_attach_backing}, followed by \field{struct
    virtio_gpu_mem_entry} entries.  Response type is
  VIRTIO_GPU_RESP_OK_NODATA.

\begin{lstlisting}
struct virtio_gpu_resource_attach_backing {
        struct virtio_gpu_ctrl_hdr hdr;
        le32 resource_id;
        le32 nr_entries;
};

struct virtio_gpu_mem_entry {
        le64 addr;
        le32 length;
        le32 padding;
};
\end{lstlisting}

This assign an array of guest pages as the backing store for a
resource. These pages are then used for the transfer operations for
that resource from that point on.

\item[VIRTIO_GPU_CMD_RESOURCE_DETACH_BACKING] Detach backing pages
  from a resource.  Request data is \field{struct
    virtio_gpu_resource_detach_backing}.  Response type is
  VIRTIO_GPU_RESP_OK_NODATA.

\begin{lstlisting}
struct virtio_gpu_resource_detach_backing {
        struct virtio_gpu_ctrl_hdr hdr;
        le32 resource_id;
        le32 padding;
};
\end{lstlisting}

This detaches any backing pages from a resource, to be used in case of
guest swapping or object destruction.

\item[VIRTIO_GPU_CMD_GET_CAPSET_INFO] Gets the information associated with
  a particular \field{capset_index}, which MUST less than \field{num_capsets}
  defined in the device configuration.  Request data is
  \field{struct virtio_gpu_get_capset_info}.  Response type is
  VIRTIO_GPU_RESP_OK_CAPSET_INFO.

  On success, \field{struct virtio_gpu_resp_capset_info} contains the
  \field{capset_id}, \field{capset_max_version}, \field{capset_max_size}
  associated with capset at the specified {capset_idex}.  field{capset_id} MUST
  be one of the following (see listing for values):

  \begin{itemize*}
  \item \href{https://gitlab.freedesktop.org/virgl/virglrenderer/-/blob/master/src/virgl_hw.h#L526}{VIRTIO_GPU_CAPSET_VIRGL} --
	the first edition of Virgl (Gallium OpenGL) protocol.
  \item \href{https://gitlab.freedesktop.org/virgl/virglrenderer/-/blob/master/src/virgl_hw.h#L550}{VIRTIO_GPU_CAPSET_VIRGL2} --
	the second edition of Virgl (Gallium OpenGL) protocol after the capset fix.
  \item \href{https://android.googlesource.com/device/generic/vulkan-cereal/+/refs/heads/master/protocols/}{VIRTIO_GPU_CAPSET_GFXSTREAM} --
	gfxtream's (mostly) autogenerated GLES and Vulkan streaming protocols.
  \item \href{https://gitlab.freedesktop.org/olv/venus-protocol}{VIRTIO_GPU_CAPSET_VENUS} --
	Mesa's (mostly) autogenerated Vulkan protocol.
  \item \href{https://chromium.googlesource.com/chromiumos/platform/crosvm/+/refs/heads/main/rutabaga_gfx/src/cross_domain/cross_domain_protocol.rs}{VIRTIO_GPU_CAPSET_CROSS_DOMAIN} --
	protocol for display virtualization via Wayland proxying.
  \end{itemize*}

\begin{lstlisting}
struct virtio_gpu_get_capset_info {
        struct virtio_gpu_ctrl_hdr hdr;
        le32 capset_index;
        le32 padding;
};

#define VIRTIO_GPU_CAPSET_VIRGL 1
#define VIRTIO_GPU_CAPSET_VIRGL2 2
#define VIRTIO_GPU_CAPSET_GFXSTREAM 3
#define VIRTIO_GPU_CAPSET_VENUS 4
#define VIRTIO_GPU_CAPSET_CROSS_DOMAIN 5
struct virtio_gpu_resp_capset_info {
        struct virtio_gpu_ctrl_hdr hdr;
        le32 capset_id;
        le32 capset_max_version;
        le32 capset_max_size;
        le32 padding;
};
\end{lstlisting}

\item[VIRTIO_GPU_CMD_GET_CAPSET] Gets the capset associated with a
  particular \field{capset_id} and \field{capset_version}.  Request data is
  \field{struct virtio_gpu_get_capset}.  Response type is
  VIRTIO_GPU_RESP_OK_CAPSET.

\begin{lstlisting}
struct virtio_gpu_get_capset {
        struct virtio_gpu_ctrl_hdr hdr;
        le32 capset_id;
        le32 capset_version;
};

struct virtio_gpu_resp_capset {
        struct virtio_gpu_ctrl_hdr hdr;
        u8 capset_data[];
};
\end{lstlisting}

\item[VIRTIO_GPU_CMD_RESOURCE_ASSIGN_UUID] Creates an exported object from
  a resource. Request data is \field{struct
    virtio_gpu_resource_assign_uuid}.  Response type is
  VIRTIO_GPU_RESP_OK_RESOURCE_UUID, response data is \field{struct
    virtio_gpu_resp_resource_uuid}. Support is optional and negotiated
    using the VIRTIO_GPU_F_RESOURCE_UUID feature flag.

\begin{lstlisting}
struct virtio_gpu_resource_assign_uuid {
        struct virtio_gpu_ctrl_hdr hdr;
        le32 resource_id;
        le32 padding;
};

struct virtio_gpu_resp_resource_uuid {
        struct virtio_gpu_ctrl_hdr hdr;
        u8 uuid[16];
};
\end{lstlisting}

The response contains a UUID which identifies the exported object created from
the host private resource. Note that if the resource has an attached backing,
modifications made to the host private resource through the exported object by
other devices are not visible in the attached backing until they are transferred
into the backing.

\item[VIRTIO_GPU_CMD_RESOURCE_CREATE_BLOB] Creates a virtio-gpu blob
  resource. Request data is \field{struct
  virtio_gpu_resource_create_blob}, followed by \field{struct
  virtio_gpu_mem_entry} entries. Response type is
  VIRTIO_GPU_RESP_OK_NODATA. Support is optional and negotiated
  using the VIRTIO_GPU_F_RESOURCE_BLOB feature flag.

\begin{lstlisting}
#define VIRTIO_GPU_BLOB_MEM_GUEST             0x0001
#define VIRTIO_GPU_BLOB_MEM_HOST3D            0x0002
#define VIRTIO_GPU_BLOB_MEM_HOST3D_GUEST      0x0003

#define VIRTIO_GPU_BLOB_FLAG_USE_MAPPABLE     0x0001
#define VIRTIO_GPU_BLOB_FLAG_USE_SHAREABLE    0x0002
#define VIRTIO_GPU_BLOB_FLAG_USE_CROSS_DEVICE 0x0004

struct virtio_gpu_resource_create_blob {
       struct virtio_gpu_ctrl_hdr hdr;
       le32 resource_id;
       le32 blob_mem;
       le32 blob_flags;
       le32 nr_entries;
       le64 blob_id;
       le64 size;
};

\end{lstlisting}

A blob resource is a container for:

  \begin{itemize*}
  \item a guest memory allocation (referred to as a
  "guest-only blob resource").
  \item a host memory allocation (referred to as a
  "host-only blob resource").
  \item a guest memory and host memory allocation (referred
  to as a "default blob resource").
  \end{itemize*}

The memory properties of the blob resource MUST be described by
\field{blob_mem}, which MUST be non-zero.

For default and guest-only blob resources, \field{nr_entries} guest
memory entries may be assigned to the resource.  For default blob resources
(i.e, when \field{blob_mem} is VIRTIO_GPU_BLOB_MEM_HOST3D_GUEST), these
memory entries are used as a shadow buffer for the host memory. To
facilitate drivers that support swap-in and swap-out, \field{nr_entries} may
be zero and VIRTIO_GPU_CMD_RESOURCE_ATTACH_BACKING may be subsequently used.
VIRTIO_GPU_CMD_RESOURCE_DETACH_BACKING may be used to unassign memory entries.

\field{blob_mem} can only be VIRTIO_GPU_BLOB_MEM_HOST3D and
VIRTIO_GPU_BLOB_MEM_HOST3D_GUEST if VIRTIO_GPU_F_VIRGL is supported.
VIRTIO_GPU_BLOB_MEM_GUEST is valid regardless whether VIRTIO_GPU_F_VIRGL
is supported or not.

For VIRTIO_GPU_BLOB_MEM_HOST3D and VIRTIO_GPU_BLOB_MEM_HOST3D_GUEST, the
virtio-gpu resource MUST be created from the rendering context local object
identified by the \field{blob_id}. The actual allocation is done via
VIRTIO_GPU_CMD_SUBMIT_3D.

The driver MUST inform the device if the blob resource is used for
memory access, sharing between driver instances and/or sharing with
other devices. This is done via the \field{blob_flags} field.

If VIRTIO_GPU_F_VIRGL is set, both VIRTIO_GPU_CMD_TRANSFER_TO_HOST_3D
and VIRTIO_GPU_CMD_TRANSFER_FROM_HOST_3D may be used to update the
resource. There is no restriction on the image/buffer view the driver
has on the blob resource.

\item[VIRTIO_GPU_CMD_SET_SCANOUT_BLOB] sets scanout parameters for a
   blob resource. Request data is
  \field{struct virtio_gpu_set_scanout_blob}. Response type is
  VIRTIO_GPU_RESP_OK_NODATA. Support is optional and negotiated
  using the VIRTIO_GPU_F_RESOURCE_BLOB feature flag.

\begin{lstlisting}
struct virtio_gpu_set_scanout_blob {
       struct virtio_gpu_ctrl_hdr hdr;
       struct virtio_gpu_rect r;
       le32 scanout_id;
       le32 resource_id;
       le32 width;
       le32 height;
       le32 format;
       le32 padding;
       le32 strides[4];
       le32 offsets[4];
};
\end{lstlisting}

The rectangle \field{r} represents the portion of the blob resource being
displayed. The rest is the metadata associated with the blob resource. The
format MUST be one of \field{enum virtio_gpu_formats}.  The format MAY be
compressed with header and data planes.

\end{description}

\subsubsection{Device Operation: controlq (3d)}\label{sec:Device Types / GPU Device / Device Operation / Device Operation: controlq (3d)}

These commands are supported by the device if the VIRTIO_GPU_F_VIRGL
feature flag is set.

\begin{description}

\item[VIRTIO_GPU_CMD_CTX_CREATE] creates a context for submitting an opaque
  command stream.  Request data is \field{struct virtio_gpu_ctx_create}.
  Response type is VIRTIO_GPU_RESP_OK_NODATA.

\begin{lstlisting}
#define VIRTIO_GPU_CONTEXT_INIT_CAPSET_ID_MASK 0x000000ff;
struct virtio_gpu_ctx_create {
       struct virtio_gpu_ctrl_hdr hdr;
       le32 nlen;
       le32 context_init;
       char debug_name[64];
};
\end{lstlisting}

The implementation MUST create a context for the given \field{ctx_id} in
the \field{hdr}.  For debugging purposes, a \field{debug_name} and it's
length \field{nlen} is provided by the driver.  If
VIRTIO_GPU_F_CONTEXT_INIT is supported, then lower 8 bits of
\field{context_init} MAY contain the \field{capset_id} associated with
context.  In that case, then the device MUST create a context that can
handle the specified command stream.

If the lower 8-bits of the \field{context_init} are zero, then the type of
the context is determined by the device.

\item[VIRTIO_GPU_CMD_CTX_DESTROY]
\item[VIRTIO_GPU_CMD_CTX_ATTACH_RESOURCE]
\item[VIRTIO_GPU_CMD_CTX_DETACH_RESOURCE]
  Manage virtio-gpu 3d contexts.

\item[VIRTIO_GPU_CMD_RESOURCE_CREATE_3D]
  Create virtio-gpu 3d resources.

\item[VIRTIO_GPU_CMD_TRANSFER_TO_HOST_3D]
\item[VIRTIO_GPU_CMD_TRANSFER_FROM_HOST_3D]
  Transfer data from and to virtio-gpu 3d resources.

\item[VIRTIO_GPU_CMD_SUBMIT_3D]
  Submit an opaque command stream.  The type of the command stream is
  determined when creating a context.

\item[VIRTIO_GPU_CMD_RESOURCE_MAP_BLOB] maps a host-only
  blob resource into an offset in the host visible memory region. Request
  data is \field{struct virtio_gpu_resource_map_blob}.  The driver MUST
  not map a blob resource that is already mapped.  Response type is
  VIRTIO_GPU_RESP_OK_MAP_INFO. Support is optional and negotiated
  using the VIRTIO_GPU_F_RESOURCE_BLOB feature flag and checking for
  the presence of the host visible memory region.

\begin{lstlisting}
struct virtio_gpu_resource_map_blob {
        struct virtio_gpu_ctrl_hdr hdr;
        le32 resource_id;
        le32 padding;
        le64 offset;
};

#define VIRTIO_GPU_MAP_CACHE_MASK      0x0f
#define VIRTIO_GPU_MAP_CACHE_NONE      0x00
#define VIRTIO_GPU_MAP_CACHE_CACHED    0x01
#define VIRTIO_GPU_MAP_CACHE_UNCACHED  0x02
#define VIRTIO_GPU_MAP_CACHE_WC        0x03
struct virtio_gpu_resp_map_info {
        struct virtio_gpu_ctrl_hdr hdr;
        u32 map_info;
        u32 padding;
};
\end{lstlisting}

\item[VIRTIO_GPU_CMD_RESOURCE_UNMAP_BLOB] unmaps a
  host-only blob resource from the host visible memory region. Request data
  is \field{struct virtio_gpu_resource_unmap_blob}.  Response type is
  VIRTIO_GPU_RESP_OK_NODATA.  Support is optional and negotiated
  using the VIRTIO_GPU_F_RESOURCE_BLOB feature flag and checking for
  the presence of the host visible memory region.

\begin{lstlisting}
struct virtio_gpu_resource_unmap_blob {
        struct virtio_gpu_ctrl_hdr hdr;
        le32 resource_id;
        le32 padding;
};
\end{lstlisting}

\end{description}

\subsubsection{Device Operation: cursorq}\label{sec:Device Types / GPU Device / Device Operation / Device Operation: cursorq}

Both cursorq commands use the same command struct.

\begin{lstlisting}
struct virtio_gpu_cursor_pos {
        le32 scanout_id;
        le32 x;
        le32 y;
        le32 padding;
};

struct virtio_gpu_update_cursor {
        struct virtio_gpu_ctrl_hdr hdr;
        struct virtio_gpu_cursor_pos pos;
        le32 resource_id;
        le32 hot_x;
        le32 hot_y;
        le32 padding;
};
\end{lstlisting}

\begin{description}

\item[VIRTIO_GPU_CMD_UPDATE_CURSOR]
Update cursor.
Request data is \field{struct virtio_gpu_update_cursor}.
Response type is VIRTIO_GPU_RESP_OK_NODATA.

Full cursor update.  Cursor will be loaded from the specified
\field{resource_id} and will be moved to \field{pos}.  The driver must
transfer the cursor into the resource beforehand (using control queue
commands) and make sure the commands to fill the resource are actually
processed (using fencing).

\item[VIRTIO_GPU_CMD_MOVE_CURSOR]
Move cursor.
Request data is \field{struct virtio_gpu_update_cursor}.
Response type is VIRTIO_GPU_RESP_OK_NODATA.

Move cursor to the place specified in \field{pos}.  The other fields
are not used and will be ignored by the device.

\end{description}

\subsection{VGA Compatibility}\label{sec:Device Types / GPU Device / VGA Compatibility}

Applies to Virtio Over PCI only.  The GPU device can come with and
without VGA compatibility.  The PCI class should be DISPLAY_VGA if VGA
compatibility is present and DISPLAY_OTHER otherwise.

VGA compatibility: PCI region 0 has the linear framebuffer, standard
vga registers are present.  Configuring a scanout
(VIRTIO_GPU_CMD_SET_SCANOUT) switches the device from vga
compatibility mode into native virtio mode.  A reset switches it back
into vga compatibility mode.

Note: qemu implementation also provides bochs dispi interface io ports
and mmio bar at pci region 1 and is therefore fully compatible with
the qemu stdvga (see \href{https://git.qemu-project.org/?p=qemu.git;a=blob;f=docs/specs/standard-vga.txt;hb=HEAD}{docs/specs/standard-vga.txt} in the qemu source tree).

\section{GPU Device}\label{sec:Device Types / GPU Device}

virtio-gpu is a virtio based graphics adapter.  It can operate in 2D
mode and in 3D mode.  3D mode will offload rendering ops to
the host gpu and therefore requires a gpu with 3D support on the host
machine.

In 2D mode the virtio-gpu device provides support for ARGB Hardware
cursors and multiple scanouts (aka heads).

\subsection{Device ID}\label{sec:Device Types / GPU Device / Device ID}

16

\subsection{Virtqueues}\label{sec:Device Types / GPU Device / Virtqueues}

\begin{description}
\item[0] controlq - queue for sending control commands
\item[1] cursorq - queue for sending cursor updates
\end{description}

Both queues have the same format.  Each request and each response have
a fixed header, followed by command specific data fields.  The
separate cursor queue is the "fast track" for cursor commands
(VIRTIO_GPU_CMD_UPDATE_CURSOR and VIRTIO_GPU_CMD_MOVE_CURSOR), so they
go through without being delayed by time-consuming commands in the
control queue.

\subsection{Feature bits}\label{sec:Device Types / GPU Device / Feature bits}

\begin{description}
\item[VIRTIO_GPU_F_VIRGL (0)] virgl 3D mode is supported.
\item[VIRTIO_GPU_F_EDID  (1)] EDID is supported.
\item[VIRTIO_GPU_F_RESOURCE_UUID (2)] assigning resources UUIDs for export
  to other virtio devices is supported.
\item[VIRTIO_GPU_F_RESOURCE_BLOB (3)] creating and using size-based blob
  resources is supported.
\item[VIRTIO_GPU_F_CONTEXT_INIT (4)] multiple context types and
  synchronization timelines supported.  Requires VIRTIO_GPU_F_VIRGL.
\end{description}

\subsection{Device configuration layout}\label{sec:Device Types / GPU Device / Device configuration layout}

GPU device configuration uses the following layout structure and
definitions:

\begin{lstlisting}
#define VIRTIO_GPU_EVENT_DISPLAY (1 << 0)

struct virtio_gpu_config {
        le32 events_read;
        le32 events_clear;
        le32 num_scanouts;
        le32 num_capsets;
};
\end{lstlisting}

\subsubsection{Device configuration fields}

\begin{description}
\item[\field{events_read}] signals pending events to the driver.  The
  driver MUST NOT write to this field.
\item[\field{events_clear}] clears pending events in the device.
  Writing a '1' into a bit will clear the corresponding bit in
  \field{events_read}, mimicking write-to-clear behavior.
\item[\field{num_scanouts}] specifies the maximum number of scanouts
  supported by the device.  Minimum value is 1, maximum value is 16.
\item[\field{num_capsets}] specifies the maximum number of capability
  sets supported by the device.  The minimum value is zero.
\end{description}

\subsubsection{Events}

\begin{description}
\item[VIRTIO_GPU_EVENT_DISPLAY] Display configuration has changed.
  The driver SHOULD use the VIRTIO_GPU_CMD_GET_DISPLAY_INFO command to
  fetch the information from the device.  In case EDID support is
  negotiated (VIRTIO_GPU_F_EDID feature flag) the device SHOULD also
  fetch the updated EDID blobs using the VIRTIO_GPU_CMD_GET_EDID
  command.
\end{description}

\devicenormative{\subsection}{Device Initialization}{Device Types / GPU Device / Device Initialization}

The driver SHOULD query the display information from the device using
the VIRTIO_GPU_CMD_GET_DISPLAY_INFO command and use that information
for the initial scanout setup.  In case EDID support is negotiated
(VIRTIO_GPU_F_EDID feature flag) the device SHOULD also fetch the EDID
information using the VIRTIO_GPU_CMD_GET_EDID command.  If no
information is available or all displays are disabled the driver MAY
choose to use a fallback, such as 1024x768 at display 0.

The driver SHOULD query all shared memory regions supported by the device.
If the device supports shared memory, the \field{shmid} of a region MUST
(see \ref{sec:Basic Facilities of a Virtio Device /
Shared Memory Regions}~\nameref{sec:Basic Facilities of a Virtio Device /
Shared Memory Regions}) be one of the following:

\begin{lstlisting}
enum virtio_gpu_shm_id {
        VIRTIO_GPU_SHM_ID_UNDEFINED = 0,
        VIRTIO_GPU_SHM_ID_HOST_VISIBLE = 1,
};
\end{lstlisting}

The shared memory region with VIRTIO_GPU_SHM_ID_HOST_VISIBLE is referred as
the "host visible memory region".  The device MUST support the
VIRTIO_GPU_CMD_RESOURCE_MAP_BLOB and VIRTIO_GPU_CMD_RESOURCE_UNMAP_BLOB
if the host visible memory region is available.

\subsection{Device Operation}\label{sec:Device Types / GPU Device / Device Operation}

The virtio-gpu is based around the concept of resources private to the
host.  The guest must DMA transfer into these resources, unless shared memory
regions are supported. This is a design requirement in order to interface with
future 3D rendering. In the unaccelerated 2D mode there is no support for DMA
transfers from resources, just to them.

Resources are initially simple 2D resources, consisting of a width,
height and format along with an identifier. The guest must then attach
backing store to the resources in order for DMA transfers to
work. This is like a GART in a real GPU.

\subsubsection{Device Operation: Create a framebuffer and configure scanout}

\begin{itemize*}
\item Create a host resource using VIRTIO_GPU_CMD_RESOURCE_CREATE_2D.
\item Allocate a framebuffer from guest ram, and attach it as backing
  storage to the resource just created, using
  VIRTIO_GPU_CMD_RESOURCE_ATTACH_BACKING.  Scatter lists are
  supported, so the framebuffer doesn't need to be contignous in guest
  physical memory.
\item Use VIRTIO_GPU_CMD_SET_SCANOUT to link the framebuffer to
  a display scanout.
\end{itemize*}

\subsubsection{Device Operation: Update a framebuffer and scanout}

\begin{itemize*}
\item Render to your framebuffer memory.
\item Use VIRTIO_GPU_CMD_TRANSFER_TO_HOST_2D to update the host resource
  from guest memory.
\item Use VIRTIO_GPU_CMD_RESOURCE_FLUSH to flush the updated resource
  to the display.
\end{itemize*}

\subsubsection{Device Operation: Using pageflip}

It is possible to create multiple framebuffers, flip between them
using VIRTIO_GPU_CMD_SET_SCANOUT and VIRTIO_GPU_CMD_RESOURCE_FLUSH,
and update the invisible framebuffer using
VIRTIO_GPU_CMD_TRANSFER_TO_HOST_2D.

\subsubsection{Device Operation: Multihead setup}

In case two or more displays are present there are different ways to
configure things:

\begin{itemize*}
\item Create a single framebuffer, link it to all displays
  (mirroring).
\item Create an framebuffer for each display.
\item Create one big framebuffer, configure scanouts to display a
  different rectangle of that framebuffer each.
\end{itemize*}

\devicenormative{\subsubsection}{Device Operation: Command lifecycle and fencing}{Device Types / GPU Device / Device Operation / Device Operation: Command lifecycle and fencing}

The device MAY process controlq commands asyncronously and return them
to the driver before the processing is complete.  If the driver needs
to know when the processing is finished it can set the
VIRTIO_GPU_FLAG_FENCE flag in the request.  The device MUST finish the
processing before returning the command then.

Note: current qemu implementation does asyncrounous processing only in
3d mode, when offloading the processing to the host gpu.

\subsubsection{Device Operation: Configure mouse cursor}

The mouse cursor image is a normal resource, except that it must be
64x64 in size.  The driver MUST create and populate the resource
(using the usual VIRTIO_GPU_CMD_RESOURCE_CREATE_2D,
VIRTIO_GPU_CMD_RESOURCE_ATTACH_BACKING and
VIRTIO_GPU_CMD_TRANSFER_TO_HOST_2D controlq commands) and make sure they
are completed (using VIRTIO_GPU_FLAG_FENCE).

Then VIRTIO_GPU_CMD_UPDATE_CURSOR can be sent to the cursorq to set
the pointer shape and position.  To move the pointer without updating
the shape use VIRTIO_GPU_CMD_MOVE_CURSOR instead.

\subsubsection{Device Operation: Request header}\label{sec:Device Types / GPU Device / Device Operation / Device Operation: Request header}

All requests and responses on the virtqueues have a fixed header
using the following layout structure and definitions:

\begin{lstlisting}
enum virtio_gpu_ctrl_type {

        /* 2d commands */
        VIRTIO_GPU_CMD_GET_DISPLAY_INFO = 0x0100,
        VIRTIO_GPU_CMD_RESOURCE_CREATE_2D,
        VIRTIO_GPU_CMD_RESOURCE_UNREF,
        VIRTIO_GPU_CMD_SET_SCANOUT,
        VIRTIO_GPU_CMD_RESOURCE_FLUSH,
        VIRTIO_GPU_CMD_TRANSFER_TO_HOST_2D,
        VIRTIO_GPU_CMD_RESOURCE_ATTACH_BACKING,
        VIRTIO_GPU_CMD_RESOURCE_DETACH_BACKING,
        VIRTIO_GPU_CMD_GET_CAPSET_INFO,
        VIRTIO_GPU_CMD_GET_CAPSET,
        VIRTIO_GPU_CMD_GET_EDID,
        VIRTIO_GPU_CMD_RESOURCE_ASSIGN_UUID,
        VIRTIO_GPU_CMD_RESOURCE_CREATE_BLOB,
        VIRTIO_GPU_CMD_SET_SCANOUT_BLOB,

        /* 3d commands */
        VIRTIO_GPU_CMD_CTX_CREATE = 0x0200,
        VIRTIO_GPU_CMD_CTX_DESTROY,
        VIRTIO_GPU_CMD_CTX_ATTACH_RESOURCE,
        VIRTIO_GPU_CMD_CTX_DETACH_RESOURCE,
        VIRTIO_GPU_CMD_RESOURCE_CREATE_3D,
        VIRTIO_GPU_CMD_TRANSFER_TO_HOST_3D,
        VIRTIO_GPU_CMD_TRANSFER_FROM_HOST_3D,
        VIRTIO_GPU_CMD_SUBMIT_3D,
        VIRTIO_GPU_CMD_RESOURCE_MAP_BLOB,
        VIRTIO_GPU_CMD_RESOURCE_UNMAP_BLOB,

        /* cursor commands */
        VIRTIO_GPU_CMD_UPDATE_CURSOR = 0x0300,
        VIRTIO_GPU_CMD_MOVE_CURSOR,

        /* success responses */
        VIRTIO_GPU_RESP_OK_NODATA = 0x1100,
        VIRTIO_GPU_RESP_OK_DISPLAY_INFO,
        VIRTIO_GPU_RESP_OK_CAPSET_INFO,
        VIRTIO_GPU_RESP_OK_CAPSET,
        VIRTIO_GPU_RESP_OK_EDID,
        VIRTIO_GPU_RESP_OK_RESOURCE_UUID,
        VIRTIO_GPU_RESP_OK_MAP_INFO,

        /* error responses */
        VIRTIO_GPU_RESP_ERR_UNSPEC = 0x1200,
        VIRTIO_GPU_RESP_ERR_OUT_OF_MEMORY,
        VIRTIO_GPU_RESP_ERR_INVALID_SCANOUT_ID,
        VIRTIO_GPU_RESP_ERR_INVALID_RESOURCE_ID,
        VIRTIO_GPU_RESP_ERR_INVALID_CONTEXT_ID,
        VIRTIO_GPU_RESP_ERR_INVALID_PARAMETER,
};

#define VIRTIO_GPU_FLAG_FENCE (1 << 0)
#define VIRTIO_GPU_FLAG_INFO_RING_IDX (1 << 1)

struct virtio_gpu_ctrl_hdr {
        le32 type;
        le32 flags;
        le64 fence_id;
        le32 ctx_id;
        u8 ring_idx;
        u8 padding[3];
};
\end{lstlisting}

The fixed header \field{struct virtio_gpu_ctrl_hdr} in each
request includes the following fields:

\begin{description}
\item[\field{type}] specifies the type of the driver request
  (VIRTIO_GPU_CMD_*) or device response (VIRTIO_GPU_RESP_*).
\item[\field{flags}] request / response flags.
\item[\field{fence_id}] If the driver sets the VIRTIO_GPU_FLAG_FENCE
  bit in the request \field{flags} field the device MUST:
  \begin{itemize*}
  \item set VIRTIO_GPU_FLAG_FENCE bit in the response,
  \item copy the content of the \field{fence_id} field from the
    request to the response, and
  \item send the response only after command processing is complete.
  \end{itemize*}
\item[\field{ctx_id}] Rendering context (used in 3D mode only).
\item[\field{ring_idx}] If VIRTIO_GPU_F_CONTEXT_INIT is supported, then
  the driver MAY set VIRTIO_GPU_FLAG_INFO_RING_IDX bit in the request
  \field{flags}.  In that case:
  \begin{itemize*}
  \item \field{ring_idx} indicates the value of a context-specific ring
   index.  The minimum value is 0 and maximum value is 63 (inclusive).
  \item If VIRTIO_GPU_FLAG_FENCE is set, \field{fence_id} acts as a
   sequence number on the synchronization timeline defined by
   \field{ctx_idx} and the ring index.
  \item If VIRTIO_GPU_FLAG_FENCE is set and when the command associated
   with \field{fence_id} is complete, the device MUST send a response for
   all outstanding commands with a sequence number less than or equal to
   \field{fence_id} on the same synchronization timeline.
  \end{itemize*}
\end{description}

On success the device will return VIRTIO_GPU_RESP_OK_NODATA in
case there is no payload.  Otherwise the \field{type} field will
indicate the kind of payload.

On error the device will return one of the
VIRTIO_GPU_RESP_ERR_* error codes.

\subsubsection{Device Operation: controlq}\label{sec:Device Types / GPU Device / Device Operation / Device Operation: controlq}

For any coordinates given 0,0 is top left, larger x moves right,
larger y moves down.

\begin{description}

\item[VIRTIO_GPU_CMD_GET_DISPLAY_INFO] Retrieve the current output
  configuration.  No request data (just bare \field{struct
    virtio_gpu_ctrl_hdr}).  Response type is
  VIRTIO_GPU_RESP_OK_DISPLAY_INFO, response data is \field{struct
    virtio_gpu_resp_display_info}.

\begin{lstlisting}
#define VIRTIO_GPU_MAX_SCANOUTS 16

struct virtio_gpu_rect {
        le32 x;
        le32 y;
        le32 width;
        le32 height;
};

struct virtio_gpu_resp_display_info {
        struct virtio_gpu_ctrl_hdr hdr;
        struct virtio_gpu_display_one {
                struct virtio_gpu_rect r;
                le32 enabled;
                le32 flags;
        } pmodes[VIRTIO_GPU_MAX_SCANOUTS];
};
\end{lstlisting}

The response contains a list of per-scanout information.  The info
contains whether the scanout is enabled and what its preferred
position and size is.

The size (fields \field{width} and \field{height}) is similar to the
native panel resolution in EDID display information, except that in
the virtual machine case the size can change when the host window
representing the guest display is gets resized.

The position (fields \field{x} and \field{y}) describe how the
displays are arranged (i.e. which is -- for example -- the left
display).

The \field{enabled} field is set when the user enabled the display.
It is roughly the same as the connected state of a phyiscal display
connector.

\item[VIRTIO_GPU_CMD_GET_EDID] Retrieve the EDID data for a given
  scanout.  Request data is \field{struct virtio_gpu_get_edid}).
  Response type is VIRTIO_GPU_RESP_OK_EDID, response data is
  \field{struct virtio_gpu_resp_edid}.  Support is optional and
  negotiated using the VIRTIO_GPU_F_EDID feature flag.

\begin{lstlisting}
struct virtio_gpu_get_edid {
        struct virtio_gpu_ctrl_hdr hdr;
        le32 scanout;
        le32 padding;
};

struct virtio_gpu_resp_edid {
        struct virtio_gpu_ctrl_hdr hdr;
        le32 size;
        le32 padding;
        u8 edid[1024];
};
\end{lstlisting}

The response contains the EDID display data blob (as specified by
VESA) for the scanout.

\item[VIRTIO_GPU_CMD_RESOURCE_CREATE_2D] Create a 2D resource on the
  host.  Request data is \field{struct virtio_gpu_resource_create_2d}.
  Response type is VIRTIO_GPU_RESP_OK_NODATA.

\begin{lstlisting}
enum virtio_gpu_formats {
        VIRTIO_GPU_FORMAT_B8G8R8A8_UNORM  = 1,
        VIRTIO_GPU_FORMAT_B8G8R8X8_UNORM  = 2,
        VIRTIO_GPU_FORMAT_A8R8G8B8_UNORM  = 3,
        VIRTIO_GPU_FORMAT_X8R8G8B8_UNORM  = 4,

        VIRTIO_GPU_FORMAT_R8G8B8A8_UNORM  = 67,
        VIRTIO_GPU_FORMAT_X8B8G8R8_UNORM  = 68,

        VIRTIO_GPU_FORMAT_A8B8G8R8_UNORM  = 121,
        VIRTIO_GPU_FORMAT_R8G8B8X8_UNORM  = 134,
};

struct virtio_gpu_resource_create_2d {
        struct virtio_gpu_ctrl_hdr hdr;
        le32 resource_id;
        le32 format;
        le32 width;
        le32 height;
};
\end{lstlisting}

This creates a 2D resource on the host with the specified width,
height and format.  The resource ids are generated by the guest.

\item[VIRTIO_GPU_CMD_RESOURCE_UNREF] Destroy a resource.  Request data
  is \field{struct virtio_gpu_resource_unref}.  Response type is
  VIRTIO_GPU_RESP_OK_NODATA.

\begin{lstlisting}
struct virtio_gpu_resource_unref {
        struct virtio_gpu_ctrl_hdr hdr;
        le32 resource_id;
        le32 padding;
};
\end{lstlisting}

This informs the host that a resource is no longer required by the
guest.

\item[VIRTIO_GPU_CMD_SET_SCANOUT] Set the scanout parameters for a
  single output.  Request data is \field{struct
    virtio_gpu_set_scanout}.  Response type is
  VIRTIO_GPU_RESP_OK_NODATA.

\begin{lstlisting}
struct virtio_gpu_set_scanout {
        struct virtio_gpu_ctrl_hdr hdr;
        struct virtio_gpu_rect r;
        le32 scanout_id;
        le32 resource_id;
};
\end{lstlisting}

This sets the scanout parameters for a single scanout. The resource_id
is the resource to be scanned out from, along with a rectangle.

Scanout rectangles must be completely covered by the underlying
resource.  Overlapping (or identical) scanouts are allowed, typical
use case is screen mirroring.

The driver can use resource_id = 0 to disable a scanout.

\item[VIRTIO_GPU_CMD_RESOURCE_FLUSH] Flush a scanout resource Request
  data is \field{struct virtio_gpu_resource_flush}.  Response type is
  VIRTIO_GPU_RESP_OK_NODATA.

\begin{lstlisting}
struct virtio_gpu_resource_flush {
        struct virtio_gpu_ctrl_hdr hdr;
        struct virtio_gpu_rect r;
        le32 resource_id;
        le32 padding;
};
\end{lstlisting}

This flushes a resource to screen.  It takes a rectangle and a
resource id, and flushes any scanouts the resource is being used on.

\item[VIRTIO_GPU_CMD_TRANSFER_TO_HOST_2D] Transfer from guest memory
  to host resource.  Request data is \field{struct
    virtio_gpu_transfer_to_host_2d}.  Response type is
  VIRTIO_GPU_RESP_OK_NODATA.

\begin{lstlisting}
struct virtio_gpu_transfer_to_host_2d {
        struct virtio_gpu_ctrl_hdr hdr;
        struct virtio_gpu_rect r;
        le64 offset;
        le32 resource_id;
        le32 padding;
};
\end{lstlisting}

This takes a resource id along with an destination offset into the
resource, and a box to transfer to the host backing for the resource.

\item[VIRTIO_GPU_CMD_RESOURCE_ATTACH_BACKING] Assign backing pages to
  a resource.  Request data is \field{struct
    virtio_gpu_resource_attach_backing}, followed by \field{struct
    virtio_gpu_mem_entry} entries.  Response type is
  VIRTIO_GPU_RESP_OK_NODATA.

\begin{lstlisting}
struct virtio_gpu_resource_attach_backing {
        struct virtio_gpu_ctrl_hdr hdr;
        le32 resource_id;
        le32 nr_entries;
};

struct virtio_gpu_mem_entry {
        le64 addr;
        le32 length;
        le32 padding;
};
\end{lstlisting}

This assign an array of guest pages as the backing store for a
resource. These pages are then used for the transfer operations for
that resource from that point on.

\item[VIRTIO_GPU_CMD_RESOURCE_DETACH_BACKING] Detach backing pages
  from a resource.  Request data is \field{struct
    virtio_gpu_resource_detach_backing}.  Response type is
  VIRTIO_GPU_RESP_OK_NODATA.

\begin{lstlisting}
struct virtio_gpu_resource_detach_backing {
        struct virtio_gpu_ctrl_hdr hdr;
        le32 resource_id;
        le32 padding;
};
\end{lstlisting}

This detaches any backing pages from a resource, to be used in case of
guest swapping or object destruction.

\item[VIRTIO_GPU_CMD_GET_CAPSET_INFO] Gets the information associated with
  a particular \field{capset_index}, which MUST less than \field{num_capsets}
  defined in the device configuration.  Request data is
  \field{struct virtio_gpu_get_capset_info}.  Response type is
  VIRTIO_GPU_RESP_OK_CAPSET_INFO.

  On success, \field{struct virtio_gpu_resp_capset_info} contains the
  \field{capset_id}, \field{capset_max_version}, \field{capset_max_size}
  associated with capset at the specified {capset_idex}.  field{capset_id} MUST
  be one of the following (see listing for values):

  \begin{itemize*}
  \item \href{https://gitlab.freedesktop.org/virgl/virglrenderer/-/blob/master/src/virgl_hw.h#L526}{VIRTIO_GPU_CAPSET_VIRGL} --
	the first edition of Virgl (Gallium OpenGL) protocol.
  \item \href{https://gitlab.freedesktop.org/virgl/virglrenderer/-/blob/master/src/virgl_hw.h#L550}{VIRTIO_GPU_CAPSET_VIRGL2} --
	the second edition of Virgl (Gallium OpenGL) protocol after the capset fix.
  \item \href{https://android.googlesource.com/device/generic/vulkan-cereal/+/refs/heads/master/protocols/}{VIRTIO_GPU_CAPSET_GFXSTREAM} --
	gfxtream's (mostly) autogenerated GLES and Vulkan streaming protocols.
  \item \href{https://gitlab.freedesktop.org/olv/venus-protocol}{VIRTIO_GPU_CAPSET_VENUS} --
	Mesa's (mostly) autogenerated Vulkan protocol.
  \item \href{https://chromium.googlesource.com/chromiumos/platform/crosvm/+/refs/heads/main/rutabaga_gfx/src/cross_domain/cross_domain_protocol.rs}{VIRTIO_GPU_CAPSET_CROSS_DOMAIN} --
	protocol for display virtualization via Wayland proxying.
  \end{itemize*}

\begin{lstlisting}
struct virtio_gpu_get_capset_info {
        struct virtio_gpu_ctrl_hdr hdr;
        le32 capset_index;
        le32 padding;
};

#define VIRTIO_GPU_CAPSET_VIRGL 1
#define VIRTIO_GPU_CAPSET_VIRGL2 2
#define VIRTIO_GPU_CAPSET_GFXSTREAM 3
#define VIRTIO_GPU_CAPSET_VENUS 4
#define VIRTIO_GPU_CAPSET_CROSS_DOMAIN 5
struct virtio_gpu_resp_capset_info {
        struct virtio_gpu_ctrl_hdr hdr;
        le32 capset_id;
        le32 capset_max_version;
        le32 capset_max_size;
        le32 padding;
};
\end{lstlisting}

\item[VIRTIO_GPU_CMD_GET_CAPSET] Gets the capset associated with a
  particular \field{capset_id} and \field{capset_version}.  Request data is
  \field{struct virtio_gpu_get_capset}.  Response type is
  VIRTIO_GPU_RESP_OK_CAPSET.

\begin{lstlisting}
struct virtio_gpu_get_capset {
        struct virtio_gpu_ctrl_hdr hdr;
        le32 capset_id;
        le32 capset_version;
};

struct virtio_gpu_resp_capset {
        struct virtio_gpu_ctrl_hdr hdr;
        u8 capset_data[];
};
\end{lstlisting}

\item[VIRTIO_GPU_CMD_RESOURCE_ASSIGN_UUID] Creates an exported object from
  a resource. Request data is \field{struct
    virtio_gpu_resource_assign_uuid}.  Response type is
  VIRTIO_GPU_RESP_OK_RESOURCE_UUID, response data is \field{struct
    virtio_gpu_resp_resource_uuid}. Support is optional and negotiated
    using the VIRTIO_GPU_F_RESOURCE_UUID feature flag.

\begin{lstlisting}
struct virtio_gpu_resource_assign_uuid {
        struct virtio_gpu_ctrl_hdr hdr;
        le32 resource_id;
        le32 padding;
};

struct virtio_gpu_resp_resource_uuid {
        struct virtio_gpu_ctrl_hdr hdr;
        u8 uuid[16];
};
\end{lstlisting}

The response contains a UUID which identifies the exported object created from
the host private resource. Note that if the resource has an attached backing,
modifications made to the host private resource through the exported object by
other devices are not visible in the attached backing until they are transferred
into the backing.

\item[VIRTIO_GPU_CMD_RESOURCE_CREATE_BLOB] Creates a virtio-gpu blob
  resource. Request data is \field{struct
  virtio_gpu_resource_create_blob}, followed by \field{struct
  virtio_gpu_mem_entry} entries. Response type is
  VIRTIO_GPU_RESP_OK_NODATA. Support is optional and negotiated
  using the VIRTIO_GPU_F_RESOURCE_BLOB feature flag.

\begin{lstlisting}
#define VIRTIO_GPU_BLOB_MEM_GUEST             0x0001
#define VIRTIO_GPU_BLOB_MEM_HOST3D            0x0002
#define VIRTIO_GPU_BLOB_MEM_HOST3D_GUEST      0x0003

#define VIRTIO_GPU_BLOB_FLAG_USE_MAPPABLE     0x0001
#define VIRTIO_GPU_BLOB_FLAG_USE_SHAREABLE    0x0002
#define VIRTIO_GPU_BLOB_FLAG_USE_CROSS_DEVICE 0x0004

struct virtio_gpu_resource_create_blob {
       struct virtio_gpu_ctrl_hdr hdr;
       le32 resource_id;
       le32 blob_mem;
       le32 blob_flags;
       le32 nr_entries;
       le64 blob_id;
       le64 size;
};

\end{lstlisting}

A blob resource is a container for:

  \begin{itemize*}
  \item a guest memory allocation (referred to as a
  "guest-only blob resource").
  \item a host memory allocation (referred to as a
  "host-only blob resource").
  \item a guest memory and host memory allocation (referred
  to as a "default blob resource").
  \end{itemize*}

The memory properties of the blob resource MUST be described by
\field{blob_mem}, which MUST be non-zero.

For default and guest-only blob resources, \field{nr_entries} guest
memory entries may be assigned to the resource.  For default blob resources
(i.e, when \field{blob_mem} is VIRTIO_GPU_BLOB_MEM_HOST3D_GUEST), these
memory entries are used as a shadow buffer for the host memory. To
facilitate drivers that support swap-in and swap-out, \field{nr_entries} may
be zero and VIRTIO_GPU_CMD_RESOURCE_ATTACH_BACKING may be subsequently used.
VIRTIO_GPU_CMD_RESOURCE_DETACH_BACKING may be used to unassign memory entries.

\field{blob_mem} can only be VIRTIO_GPU_BLOB_MEM_HOST3D and
VIRTIO_GPU_BLOB_MEM_HOST3D_GUEST if VIRTIO_GPU_F_VIRGL is supported.
VIRTIO_GPU_BLOB_MEM_GUEST is valid regardless whether VIRTIO_GPU_F_VIRGL
is supported or not.

For VIRTIO_GPU_BLOB_MEM_HOST3D and VIRTIO_GPU_BLOB_MEM_HOST3D_GUEST, the
virtio-gpu resource MUST be created from the rendering context local object
identified by the \field{blob_id}. The actual allocation is done via
VIRTIO_GPU_CMD_SUBMIT_3D.

The driver MUST inform the device if the blob resource is used for
memory access, sharing between driver instances and/or sharing with
other devices. This is done via the \field{blob_flags} field.

If VIRTIO_GPU_F_VIRGL is set, both VIRTIO_GPU_CMD_TRANSFER_TO_HOST_3D
and VIRTIO_GPU_CMD_TRANSFER_FROM_HOST_3D may be used to update the
resource. There is no restriction on the image/buffer view the driver
has on the blob resource.

\item[VIRTIO_GPU_CMD_SET_SCANOUT_BLOB] sets scanout parameters for a
   blob resource. Request data is
  \field{struct virtio_gpu_set_scanout_blob}. Response type is
  VIRTIO_GPU_RESP_OK_NODATA. Support is optional and negotiated
  using the VIRTIO_GPU_F_RESOURCE_BLOB feature flag.

\begin{lstlisting}
struct virtio_gpu_set_scanout_blob {
       struct virtio_gpu_ctrl_hdr hdr;
       struct virtio_gpu_rect r;
       le32 scanout_id;
       le32 resource_id;
       le32 width;
       le32 height;
       le32 format;
       le32 padding;
       le32 strides[4];
       le32 offsets[4];
};
\end{lstlisting}

The rectangle \field{r} represents the portion of the blob resource being
displayed. The rest is the metadata associated with the blob resource. The
format MUST be one of \field{enum virtio_gpu_formats}.  The format MAY be
compressed with header and data planes.

\end{description}

\subsubsection{Device Operation: controlq (3d)}\label{sec:Device Types / GPU Device / Device Operation / Device Operation: controlq (3d)}

These commands are supported by the device if the VIRTIO_GPU_F_VIRGL
feature flag is set.

\begin{description}

\item[VIRTIO_GPU_CMD_CTX_CREATE] creates a context for submitting an opaque
  command stream.  Request data is \field{struct virtio_gpu_ctx_create}.
  Response type is VIRTIO_GPU_RESP_OK_NODATA.

\begin{lstlisting}
#define VIRTIO_GPU_CONTEXT_INIT_CAPSET_ID_MASK 0x000000ff;
struct virtio_gpu_ctx_create {
       struct virtio_gpu_ctrl_hdr hdr;
       le32 nlen;
       le32 context_init;
       char debug_name[64];
};
\end{lstlisting}

The implementation MUST create a context for the given \field{ctx_id} in
the \field{hdr}.  For debugging purposes, a \field{debug_name} and it's
length \field{nlen} is provided by the driver.  If
VIRTIO_GPU_F_CONTEXT_INIT is supported, then lower 8 bits of
\field{context_init} MAY contain the \field{capset_id} associated with
context.  In that case, then the device MUST create a context that can
handle the specified command stream.

If the lower 8-bits of the \field{context_init} are zero, then the type of
the context is determined by the device.

\item[VIRTIO_GPU_CMD_CTX_DESTROY]
\item[VIRTIO_GPU_CMD_CTX_ATTACH_RESOURCE]
\item[VIRTIO_GPU_CMD_CTX_DETACH_RESOURCE]
  Manage virtio-gpu 3d contexts.

\item[VIRTIO_GPU_CMD_RESOURCE_CREATE_3D]
  Create virtio-gpu 3d resources.

\item[VIRTIO_GPU_CMD_TRANSFER_TO_HOST_3D]
\item[VIRTIO_GPU_CMD_TRANSFER_FROM_HOST_3D]
  Transfer data from and to virtio-gpu 3d resources.

\item[VIRTIO_GPU_CMD_SUBMIT_3D]
  Submit an opaque command stream.  The type of the command stream is
  determined when creating a context.

\item[VIRTIO_GPU_CMD_RESOURCE_MAP_BLOB] maps a host-only
  blob resource into an offset in the host visible memory region. Request
  data is \field{struct virtio_gpu_resource_map_blob}.  The driver MUST
  not map a blob resource that is already mapped.  Response type is
  VIRTIO_GPU_RESP_OK_MAP_INFO. Support is optional and negotiated
  using the VIRTIO_GPU_F_RESOURCE_BLOB feature flag and checking for
  the presence of the host visible memory region.

\begin{lstlisting}
struct virtio_gpu_resource_map_blob {
        struct virtio_gpu_ctrl_hdr hdr;
        le32 resource_id;
        le32 padding;
        le64 offset;
};

#define VIRTIO_GPU_MAP_CACHE_MASK      0x0f
#define VIRTIO_GPU_MAP_CACHE_NONE      0x00
#define VIRTIO_GPU_MAP_CACHE_CACHED    0x01
#define VIRTIO_GPU_MAP_CACHE_UNCACHED  0x02
#define VIRTIO_GPU_MAP_CACHE_WC        0x03
struct virtio_gpu_resp_map_info {
        struct virtio_gpu_ctrl_hdr hdr;
        u32 map_info;
        u32 padding;
};
\end{lstlisting}

\item[VIRTIO_GPU_CMD_RESOURCE_UNMAP_BLOB] unmaps a
  host-only blob resource from the host visible memory region. Request data
  is \field{struct virtio_gpu_resource_unmap_blob}.  Response type is
  VIRTIO_GPU_RESP_OK_NODATA.  Support is optional and negotiated
  using the VIRTIO_GPU_F_RESOURCE_BLOB feature flag and checking for
  the presence of the host visible memory region.

\begin{lstlisting}
struct virtio_gpu_resource_unmap_blob {
        struct virtio_gpu_ctrl_hdr hdr;
        le32 resource_id;
        le32 padding;
};
\end{lstlisting}

\end{description}

\subsubsection{Device Operation: cursorq}\label{sec:Device Types / GPU Device / Device Operation / Device Operation: cursorq}

Both cursorq commands use the same command struct.

\begin{lstlisting}
struct virtio_gpu_cursor_pos {
        le32 scanout_id;
        le32 x;
        le32 y;
        le32 padding;
};

struct virtio_gpu_update_cursor {
        struct virtio_gpu_ctrl_hdr hdr;
        struct virtio_gpu_cursor_pos pos;
        le32 resource_id;
        le32 hot_x;
        le32 hot_y;
        le32 padding;
};
\end{lstlisting}

\begin{description}

\item[VIRTIO_GPU_CMD_UPDATE_CURSOR]
Update cursor.
Request data is \field{struct virtio_gpu_update_cursor}.
Response type is VIRTIO_GPU_RESP_OK_NODATA.

Full cursor update.  Cursor will be loaded from the specified
\field{resource_id} and will be moved to \field{pos}.  The driver must
transfer the cursor into the resource beforehand (using control queue
commands) and make sure the commands to fill the resource are actually
processed (using fencing).

\item[VIRTIO_GPU_CMD_MOVE_CURSOR]
Move cursor.
Request data is \field{struct virtio_gpu_update_cursor}.
Response type is VIRTIO_GPU_RESP_OK_NODATA.

Move cursor to the place specified in \field{pos}.  The other fields
are not used and will be ignored by the device.

\end{description}

\subsection{VGA Compatibility}\label{sec:Device Types / GPU Device / VGA Compatibility}

Applies to Virtio Over PCI only.  The GPU device can come with and
without VGA compatibility.  The PCI class should be DISPLAY_VGA if VGA
compatibility is present and DISPLAY_OTHER otherwise.

VGA compatibility: PCI region 0 has the linear framebuffer, standard
vga registers are present.  Configuring a scanout
(VIRTIO_GPU_CMD_SET_SCANOUT) switches the device from vga
compatibility mode into native virtio mode.  A reset switches it back
into vga compatibility mode.

Note: qemu implementation also provides bochs dispi interface io ports
and mmio bar at pci region 1 and is therefore fully compatible with
the qemu stdvga (see \href{https://git.qemu-project.org/?p=qemu.git;a=blob;f=docs/specs/standard-vga.txt;hb=HEAD}{docs/specs/standard-vga.txt} in the qemu source tree).

\section{GPU Device}\label{sec:Device Types / GPU Device}

virtio-gpu is a virtio based graphics adapter.  It can operate in 2D
mode and in 3D mode.  3D mode will offload rendering ops to
the host gpu and therefore requires a gpu with 3D support on the host
machine.

In 2D mode the virtio-gpu device provides support for ARGB Hardware
cursors and multiple scanouts (aka heads).

\subsection{Device ID}\label{sec:Device Types / GPU Device / Device ID}

16

\subsection{Virtqueues}\label{sec:Device Types / GPU Device / Virtqueues}

\begin{description}
\item[0] controlq - queue for sending control commands
\item[1] cursorq - queue for sending cursor updates
\end{description}

Both queues have the same format.  Each request and each response have
a fixed header, followed by command specific data fields.  The
separate cursor queue is the "fast track" for cursor commands
(VIRTIO_GPU_CMD_UPDATE_CURSOR and VIRTIO_GPU_CMD_MOVE_CURSOR), so they
go through without being delayed by time-consuming commands in the
control queue.

\subsection{Feature bits}\label{sec:Device Types / GPU Device / Feature bits}

\begin{description}
\item[VIRTIO_GPU_F_VIRGL (0)] virgl 3D mode is supported.
\item[VIRTIO_GPU_F_EDID  (1)] EDID is supported.
\item[VIRTIO_GPU_F_RESOURCE_UUID (2)] assigning resources UUIDs for export
  to other virtio devices is supported.
\item[VIRTIO_GPU_F_RESOURCE_BLOB (3)] creating and using size-based blob
  resources is supported.
\item[VIRTIO_GPU_F_CONTEXT_INIT (4)] multiple context types and
  synchronization timelines supported.  Requires VIRTIO_GPU_F_VIRGL.
\end{description}

\subsection{Device configuration layout}\label{sec:Device Types / GPU Device / Device configuration layout}

GPU device configuration uses the following layout structure and
definitions:

\begin{lstlisting}
#define VIRTIO_GPU_EVENT_DISPLAY (1 << 0)

struct virtio_gpu_config {
        le32 events_read;
        le32 events_clear;
        le32 num_scanouts;
        le32 num_capsets;
};
\end{lstlisting}

\subsubsection{Device configuration fields}

\begin{description}
\item[\field{events_read}] signals pending events to the driver.  The
  driver MUST NOT write to this field.
\item[\field{events_clear}] clears pending events in the device.
  Writing a '1' into a bit will clear the corresponding bit in
  \field{events_read}, mimicking write-to-clear behavior.
\item[\field{num_scanouts}] specifies the maximum number of scanouts
  supported by the device.  Minimum value is 1, maximum value is 16.
\item[\field{num_capsets}] specifies the maximum number of capability
  sets supported by the device.  The minimum value is zero.
\end{description}

\subsubsection{Events}

\begin{description}
\item[VIRTIO_GPU_EVENT_DISPLAY] Display configuration has changed.
  The driver SHOULD use the VIRTIO_GPU_CMD_GET_DISPLAY_INFO command to
  fetch the information from the device.  In case EDID support is
  negotiated (VIRTIO_GPU_F_EDID feature flag) the device SHOULD also
  fetch the updated EDID blobs using the VIRTIO_GPU_CMD_GET_EDID
  command.
\end{description}

\devicenormative{\subsection}{Device Initialization}{Device Types / GPU Device / Device Initialization}

The driver SHOULD query the display information from the device using
the VIRTIO_GPU_CMD_GET_DISPLAY_INFO command and use that information
for the initial scanout setup.  In case EDID support is negotiated
(VIRTIO_GPU_F_EDID feature flag) the device SHOULD also fetch the EDID
information using the VIRTIO_GPU_CMD_GET_EDID command.  If no
information is available or all displays are disabled the driver MAY
choose to use a fallback, such as 1024x768 at display 0.

The driver SHOULD query all shared memory regions supported by the device.
If the device supports shared memory, the \field{shmid} of a region MUST
(see \ref{sec:Basic Facilities of a Virtio Device /
Shared Memory Regions}~\nameref{sec:Basic Facilities of a Virtio Device /
Shared Memory Regions}) be one of the following:

\begin{lstlisting}
enum virtio_gpu_shm_id {
        VIRTIO_GPU_SHM_ID_UNDEFINED = 0,
        VIRTIO_GPU_SHM_ID_HOST_VISIBLE = 1,
};
\end{lstlisting}

The shared memory region with VIRTIO_GPU_SHM_ID_HOST_VISIBLE is referred as
the "host visible memory region".  The device MUST support the
VIRTIO_GPU_CMD_RESOURCE_MAP_BLOB and VIRTIO_GPU_CMD_RESOURCE_UNMAP_BLOB
if the host visible memory region is available.

\subsection{Device Operation}\label{sec:Device Types / GPU Device / Device Operation}

The virtio-gpu is based around the concept of resources private to the
host.  The guest must DMA transfer into these resources, unless shared memory
regions are supported. This is a design requirement in order to interface with
future 3D rendering. In the unaccelerated 2D mode there is no support for DMA
transfers from resources, just to them.

Resources are initially simple 2D resources, consisting of a width,
height and format along with an identifier. The guest must then attach
backing store to the resources in order for DMA transfers to
work. This is like a GART in a real GPU.

\subsubsection{Device Operation: Create a framebuffer and configure scanout}

\begin{itemize*}
\item Create a host resource using VIRTIO_GPU_CMD_RESOURCE_CREATE_2D.
\item Allocate a framebuffer from guest ram, and attach it as backing
  storage to the resource just created, using
  VIRTIO_GPU_CMD_RESOURCE_ATTACH_BACKING.  Scatter lists are
  supported, so the framebuffer doesn't need to be contignous in guest
  physical memory.
\item Use VIRTIO_GPU_CMD_SET_SCANOUT to link the framebuffer to
  a display scanout.
\end{itemize*}

\subsubsection{Device Operation: Update a framebuffer and scanout}

\begin{itemize*}
\item Render to your framebuffer memory.
\item Use VIRTIO_GPU_CMD_TRANSFER_TO_HOST_2D to update the host resource
  from guest memory.
\item Use VIRTIO_GPU_CMD_RESOURCE_FLUSH to flush the updated resource
  to the display.
\end{itemize*}

\subsubsection{Device Operation: Using pageflip}

It is possible to create multiple framebuffers, flip between them
using VIRTIO_GPU_CMD_SET_SCANOUT and VIRTIO_GPU_CMD_RESOURCE_FLUSH,
and update the invisible framebuffer using
VIRTIO_GPU_CMD_TRANSFER_TO_HOST_2D.

\subsubsection{Device Operation: Multihead setup}

In case two or more displays are present there are different ways to
configure things:

\begin{itemize*}
\item Create a single framebuffer, link it to all displays
  (mirroring).
\item Create an framebuffer for each display.
\item Create one big framebuffer, configure scanouts to display a
  different rectangle of that framebuffer each.
\end{itemize*}

\devicenormative{\subsubsection}{Device Operation: Command lifecycle and fencing}{Device Types / GPU Device / Device Operation / Device Operation: Command lifecycle and fencing}

The device MAY process controlq commands asyncronously and return them
to the driver before the processing is complete.  If the driver needs
to know when the processing is finished it can set the
VIRTIO_GPU_FLAG_FENCE flag in the request.  The device MUST finish the
processing before returning the command then.

Note: current qemu implementation does asyncrounous processing only in
3d mode, when offloading the processing to the host gpu.

\subsubsection{Device Operation: Configure mouse cursor}

The mouse cursor image is a normal resource, except that it must be
64x64 in size.  The driver MUST create and populate the resource
(using the usual VIRTIO_GPU_CMD_RESOURCE_CREATE_2D,
VIRTIO_GPU_CMD_RESOURCE_ATTACH_BACKING and
VIRTIO_GPU_CMD_TRANSFER_TO_HOST_2D controlq commands) and make sure they
are completed (using VIRTIO_GPU_FLAG_FENCE).

Then VIRTIO_GPU_CMD_UPDATE_CURSOR can be sent to the cursorq to set
the pointer shape and position.  To move the pointer without updating
the shape use VIRTIO_GPU_CMD_MOVE_CURSOR instead.

\subsubsection{Device Operation: Request header}\label{sec:Device Types / GPU Device / Device Operation / Device Operation: Request header}

All requests and responses on the virtqueues have a fixed header
using the following layout structure and definitions:

\begin{lstlisting}
enum virtio_gpu_ctrl_type {

        /* 2d commands */
        VIRTIO_GPU_CMD_GET_DISPLAY_INFO = 0x0100,
        VIRTIO_GPU_CMD_RESOURCE_CREATE_2D,
        VIRTIO_GPU_CMD_RESOURCE_UNREF,
        VIRTIO_GPU_CMD_SET_SCANOUT,
        VIRTIO_GPU_CMD_RESOURCE_FLUSH,
        VIRTIO_GPU_CMD_TRANSFER_TO_HOST_2D,
        VIRTIO_GPU_CMD_RESOURCE_ATTACH_BACKING,
        VIRTIO_GPU_CMD_RESOURCE_DETACH_BACKING,
        VIRTIO_GPU_CMD_GET_CAPSET_INFO,
        VIRTIO_GPU_CMD_GET_CAPSET,
        VIRTIO_GPU_CMD_GET_EDID,
        VIRTIO_GPU_CMD_RESOURCE_ASSIGN_UUID,
        VIRTIO_GPU_CMD_RESOURCE_CREATE_BLOB,
        VIRTIO_GPU_CMD_SET_SCANOUT_BLOB,

        /* 3d commands */
        VIRTIO_GPU_CMD_CTX_CREATE = 0x0200,
        VIRTIO_GPU_CMD_CTX_DESTROY,
        VIRTIO_GPU_CMD_CTX_ATTACH_RESOURCE,
        VIRTIO_GPU_CMD_CTX_DETACH_RESOURCE,
        VIRTIO_GPU_CMD_RESOURCE_CREATE_3D,
        VIRTIO_GPU_CMD_TRANSFER_TO_HOST_3D,
        VIRTIO_GPU_CMD_TRANSFER_FROM_HOST_3D,
        VIRTIO_GPU_CMD_SUBMIT_3D,
        VIRTIO_GPU_CMD_RESOURCE_MAP_BLOB,
        VIRTIO_GPU_CMD_RESOURCE_UNMAP_BLOB,

        /* cursor commands */
        VIRTIO_GPU_CMD_UPDATE_CURSOR = 0x0300,
        VIRTIO_GPU_CMD_MOVE_CURSOR,

        /* success responses */
        VIRTIO_GPU_RESP_OK_NODATA = 0x1100,
        VIRTIO_GPU_RESP_OK_DISPLAY_INFO,
        VIRTIO_GPU_RESP_OK_CAPSET_INFO,
        VIRTIO_GPU_RESP_OK_CAPSET,
        VIRTIO_GPU_RESP_OK_EDID,
        VIRTIO_GPU_RESP_OK_RESOURCE_UUID,
        VIRTIO_GPU_RESP_OK_MAP_INFO,

        /* error responses */
        VIRTIO_GPU_RESP_ERR_UNSPEC = 0x1200,
        VIRTIO_GPU_RESP_ERR_OUT_OF_MEMORY,
        VIRTIO_GPU_RESP_ERR_INVALID_SCANOUT_ID,
        VIRTIO_GPU_RESP_ERR_INVALID_RESOURCE_ID,
        VIRTIO_GPU_RESP_ERR_INVALID_CONTEXT_ID,
        VIRTIO_GPU_RESP_ERR_INVALID_PARAMETER,
};

#define VIRTIO_GPU_FLAG_FENCE (1 << 0)
#define VIRTIO_GPU_FLAG_INFO_RING_IDX (1 << 1)

struct virtio_gpu_ctrl_hdr {
        le32 type;
        le32 flags;
        le64 fence_id;
        le32 ctx_id;
        u8 ring_idx;
        u8 padding[3];
};
\end{lstlisting}

The fixed header \field{struct virtio_gpu_ctrl_hdr} in each
request includes the following fields:

\begin{description}
\item[\field{type}] specifies the type of the driver request
  (VIRTIO_GPU_CMD_*) or device response (VIRTIO_GPU_RESP_*).
\item[\field{flags}] request / response flags.
\item[\field{fence_id}] If the driver sets the VIRTIO_GPU_FLAG_FENCE
  bit in the request \field{flags} field the device MUST:
  \begin{itemize*}
  \item set VIRTIO_GPU_FLAG_FENCE bit in the response,
  \item copy the content of the \field{fence_id} field from the
    request to the response, and
  \item send the response only after command processing is complete.
  \end{itemize*}
\item[\field{ctx_id}] Rendering context (used in 3D mode only).
\item[\field{ring_idx}] If VIRTIO_GPU_F_CONTEXT_INIT is supported, then
  the driver MAY set VIRTIO_GPU_FLAG_INFO_RING_IDX bit in the request
  \field{flags}.  In that case:
  \begin{itemize*}
  \item \field{ring_idx} indicates the value of a context-specific ring
   index.  The minimum value is 0 and maximum value is 63 (inclusive).
  \item If VIRTIO_GPU_FLAG_FENCE is set, \field{fence_id} acts as a
   sequence number on the synchronization timeline defined by
   \field{ctx_idx} and the ring index.
  \item If VIRTIO_GPU_FLAG_FENCE is set and when the command associated
   with \field{fence_id} is complete, the device MUST send a response for
   all outstanding commands with a sequence number less than or equal to
   \field{fence_id} on the same synchronization timeline.
  \end{itemize*}
\end{description}

On success the device will return VIRTIO_GPU_RESP_OK_NODATA in
case there is no payload.  Otherwise the \field{type} field will
indicate the kind of payload.

On error the device will return one of the
VIRTIO_GPU_RESP_ERR_* error codes.

\subsubsection{Device Operation: controlq}\label{sec:Device Types / GPU Device / Device Operation / Device Operation: controlq}

For any coordinates given 0,0 is top left, larger x moves right,
larger y moves down.

\begin{description}

\item[VIRTIO_GPU_CMD_GET_DISPLAY_INFO] Retrieve the current output
  configuration.  No request data (just bare \field{struct
    virtio_gpu_ctrl_hdr}).  Response type is
  VIRTIO_GPU_RESP_OK_DISPLAY_INFO, response data is \field{struct
    virtio_gpu_resp_display_info}.

\begin{lstlisting}
#define VIRTIO_GPU_MAX_SCANOUTS 16

struct virtio_gpu_rect {
        le32 x;
        le32 y;
        le32 width;
        le32 height;
};

struct virtio_gpu_resp_display_info {
        struct virtio_gpu_ctrl_hdr hdr;
        struct virtio_gpu_display_one {
                struct virtio_gpu_rect r;
                le32 enabled;
                le32 flags;
        } pmodes[VIRTIO_GPU_MAX_SCANOUTS];
};
\end{lstlisting}

The response contains a list of per-scanout information.  The info
contains whether the scanout is enabled and what its preferred
position and size is.

The size (fields \field{width} and \field{height}) is similar to the
native panel resolution in EDID display information, except that in
the virtual machine case the size can change when the host window
representing the guest display is gets resized.

The position (fields \field{x} and \field{y}) describe how the
displays are arranged (i.e. which is -- for example -- the left
display).

The \field{enabled} field is set when the user enabled the display.
It is roughly the same as the connected state of a phyiscal display
connector.

\item[VIRTIO_GPU_CMD_GET_EDID] Retrieve the EDID data for a given
  scanout.  Request data is \field{struct virtio_gpu_get_edid}).
  Response type is VIRTIO_GPU_RESP_OK_EDID, response data is
  \field{struct virtio_gpu_resp_edid}.  Support is optional and
  negotiated using the VIRTIO_GPU_F_EDID feature flag.

\begin{lstlisting}
struct virtio_gpu_get_edid {
        struct virtio_gpu_ctrl_hdr hdr;
        le32 scanout;
        le32 padding;
};

struct virtio_gpu_resp_edid {
        struct virtio_gpu_ctrl_hdr hdr;
        le32 size;
        le32 padding;
        u8 edid[1024];
};
\end{lstlisting}

The response contains the EDID display data blob (as specified by
VESA) for the scanout.

\item[VIRTIO_GPU_CMD_RESOURCE_CREATE_2D] Create a 2D resource on the
  host.  Request data is \field{struct virtio_gpu_resource_create_2d}.
  Response type is VIRTIO_GPU_RESP_OK_NODATA.

\begin{lstlisting}
enum virtio_gpu_formats {
        VIRTIO_GPU_FORMAT_B8G8R8A8_UNORM  = 1,
        VIRTIO_GPU_FORMAT_B8G8R8X8_UNORM  = 2,
        VIRTIO_GPU_FORMAT_A8R8G8B8_UNORM  = 3,
        VIRTIO_GPU_FORMAT_X8R8G8B8_UNORM  = 4,

        VIRTIO_GPU_FORMAT_R8G8B8A8_UNORM  = 67,
        VIRTIO_GPU_FORMAT_X8B8G8R8_UNORM  = 68,

        VIRTIO_GPU_FORMAT_A8B8G8R8_UNORM  = 121,
        VIRTIO_GPU_FORMAT_R8G8B8X8_UNORM  = 134,
};

struct virtio_gpu_resource_create_2d {
        struct virtio_gpu_ctrl_hdr hdr;
        le32 resource_id;
        le32 format;
        le32 width;
        le32 height;
};
\end{lstlisting}

This creates a 2D resource on the host with the specified width,
height and format.  The resource ids are generated by the guest.

\item[VIRTIO_GPU_CMD_RESOURCE_UNREF] Destroy a resource.  Request data
  is \field{struct virtio_gpu_resource_unref}.  Response type is
  VIRTIO_GPU_RESP_OK_NODATA.

\begin{lstlisting}
struct virtio_gpu_resource_unref {
        struct virtio_gpu_ctrl_hdr hdr;
        le32 resource_id;
        le32 padding;
};
\end{lstlisting}

This informs the host that a resource is no longer required by the
guest.

\item[VIRTIO_GPU_CMD_SET_SCANOUT] Set the scanout parameters for a
  single output.  Request data is \field{struct
    virtio_gpu_set_scanout}.  Response type is
  VIRTIO_GPU_RESP_OK_NODATA.

\begin{lstlisting}
struct virtio_gpu_set_scanout {
        struct virtio_gpu_ctrl_hdr hdr;
        struct virtio_gpu_rect r;
        le32 scanout_id;
        le32 resource_id;
};
\end{lstlisting}

This sets the scanout parameters for a single scanout. The resource_id
is the resource to be scanned out from, along with a rectangle.

Scanout rectangles must be completely covered by the underlying
resource.  Overlapping (or identical) scanouts are allowed, typical
use case is screen mirroring.

The driver can use resource_id = 0 to disable a scanout.

\item[VIRTIO_GPU_CMD_RESOURCE_FLUSH] Flush a scanout resource Request
  data is \field{struct virtio_gpu_resource_flush}.  Response type is
  VIRTIO_GPU_RESP_OK_NODATA.

\begin{lstlisting}
struct virtio_gpu_resource_flush {
        struct virtio_gpu_ctrl_hdr hdr;
        struct virtio_gpu_rect r;
        le32 resource_id;
        le32 padding;
};
\end{lstlisting}

This flushes a resource to screen.  It takes a rectangle and a
resource id, and flushes any scanouts the resource is being used on.

\item[VIRTIO_GPU_CMD_TRANSFER_TO_HOST_2D] Transfer from guest memory
  to host resource.  Request data is \field{struct
    virtio_gpu_transfer_to_host_2d}.  Response type is
  VIRTIO_GPU_RESP_OK_NODATA.

\begin{lstlisting}
struct virtio_gpu_transfer_to_host_2d {
        struct virtio_gpu_ctrl_hdr hdr;
        struct virtio_gpu_rect r;
        le64 offset;
        le32 resource_id;
        le32 padding;
};
\end{lstlisting}

This takes a resource id along with an destination offset into the
resource, and a box to transfer to the host backing for the resource.

\item[VIRTIO_GPU_CMD_RESOURCE_ATTACH_BACKING] Assign backing pages to
  a resource.  Request data is \field{struct
    virtio_gpu_resource_attach_backing}, followed by \field{struct
    virtio_gpu_mem_entry} entries.  Response type is
  VIRTIO_GPU_RESP_OK_NODATA.

\begin{lstlisting}
struct virtio_gpu_resource_attach_backing {
        struct virtio_gpu_ctrl_hdr hdr;
        le32 resource_id;
        le32 nr_entries;
};

struct virtio_gpu_mem_entry {
        le64 addr;
        le32 length;
        le32 padding;
};
\end{lstlisting}

This assign an array of guest pages as the backing store for a
resource. These pages are then used for the transfer operations for
that resource from that point on.

\item[VIRTIO_GPU_CMD_RESOURCE_DETACH_BACKING] Detach backing pages
  from a resource.  Request data is \field{struct
    virtio_gpu_resource_detach_backing}.  Response type is
  VIRTIO_GPU_RESP_OK_NODATA.

\begin{lstlisting}
struct virtio_gpu_resource_detach_backing {
        struct virtio_gpu_ctrl_hdr hdr;
        le32 resource_id;
        le32 padding;
};
\end{lstlisting}

This detaches any backing pages from a resource, to be used in case of
guest swapping or object destruction.

\item[VIRTIO_GPU_CMD_GET_CAPSET_INFO] Gets the information associated with
  a particular \field{capset_index}, which MUST less than \field{num_capsets}
  defined in the device configuration.  Request data is
  \field{struct virtio_gpu_get_capset_info}.  Response type is
  VIRTIO_GPU_RESP_OK_CAPSET_INFO.

  On success, \field{struct virtio_gpu_resp_capset_info} contains the
  \field{capset_id}, \field{capset_max_version}, \field{capset_max_size}
  associated with capset at the specified {capset_idex}.  field{capset_id} MUST
  be one of the following (see listing for values):

  \begin{itemize*}
  \item \href{https://gitlab.freedesktop.org/virgl/virglrenderer/-/blob/master/src/virgl_hw.h#L526}{VIRTIO_GPU_CAPSET_VIRGL} --
	the first edition of Virgl (Gallium OpenGL) protocol.
  \item \href{https://gitlab.freedesktop.org/virgl/virglrenderer/-/blob/master/src/virgl_hw.h#L550}{VIRTIO_GPU_CAPSET_VIRGL2} --
	the second edition of Virgl (Gallium OpenGL) protocol after the capset fix.
  \item \href{https://android.googlesource.com/device/generic/vulkan-cereal/+/refs/heads/master/protocols/}{VIRTIO_GPU_CAPSET_GFXSTREAM} --
	gfxtream's (mostly) autogenerated GLES and Vulkan streaming protocols.
  \item \href{https://gitlab.freedesktop.org/olv/venus-protocol}{VIRTIO_GPU_CAPSET_VENUS} --
	Mesa's (mostly) autogenerated Vulkan protocol.
  \item \href{https://chromium.googlesource.com/chromiumos/platform/crosvm/+/refs/heads/main/rutabaga_gfx/src/cross_domain/cross_domain_protocol.rs}{VIRTIO_GPU_CAPSET_CROSS_DOMAIN} --
	protocol for display virtualization via Wayland proxying.
  \end{itemize*}

\begin{lstlisting}
struct virtio_gpu_get_capset_info {
        struct virtio_gpu_ctrl_hdr hdr;
        le32 capset_index;
        le32 padding;
};

#define VIRTIO_GPU_CAPSET_VIRGL 1
#define VIRTIO_GPU_CAPSET_VIRGL2 2
#define VIRTIO_GPU_CAPSET_GFXSTREAM 3
#define VIRTIO_GPU_CAPSET_VENUS 4
#define VIRTIO_GPU_CAPSET_CROSS_DOMAIN 5
struct virtio_gpu_resp_capset_info {
        struct virtio_gpu_ctrl_hdr hdr;
        le32 capset_id;
        le32 capset_max_version;
        le32 capset_max_size;
        le32 padding;
};
\end{lstlisting}

\item[VIRTIO_GPU_CMD_GET_CAPSET] Gets the capset associated with a
  particular \field{capset_id} and \field{capset_version}.  Request data is
  \field{struct virtio_gpu_get_capset}.  Response type is
  VIRTIO_GPU_RESP_OK_CAPSET.

\begin{lstlisting}
struct virtio_gpu_get_capset {
        struct virtio_gpu_ctrl_hdr hdr;
        le32 capset_id;
        le32 capset_version;
};

struct virtio_gpu_resp_capset {
        struct virtio_gpu_ctrl_hdr hdr;
        u8 capset_data[];
};
\end{lstlisting}

\item[VIRTIO_GPU_CMD_RESOURCE_ASSIGN_UUID] Creates an exported object from
  a resource. Request data is \field{struct
    virtio_gpu_resource_assign_uuid}.  Response type is
  VIRTIO_GPU_RESP_OK_RESOURCE_UUID, response data is \field{struct
    virtio_gpu_resp_resource_uuid}. Support is optional and negotiated
    using the VIRTIO_GPU_F_RESOURCE_UUID feature flag.

\begin{lstlisting}
struct virtio_gpu_resource_assign_uuid {
        struct virtio_gpu_ctrl_hdr hdr;
        le32 resource_id;
        le32 padding;
};

struct virtio_gpu_resp_resource_uuid {
        struct virtio_gpu_ctrl_hdr hdr;
        u8 uuid[16];
};
\end{lstlisting}

The response contains a UUID which identifies the exported object created from
the host private resource. Note that if the resource has an attached backing,
modifications made to the host private resource through the exported object by
other devices are not visible in the attached backing until they are transferred
into the backing.

\item[VIRTIO_GPU_CMD_RESOURCE_CREATE_BLOB] Creates a virtio-gpu blob
  resource. Request data is \field{struct
  virtio_gpu_resource_create_blob}, followed by \field{struct
  virtio_gpu_mem_entry} entries. Response type is
  VIRTIO_GPU_RESP_OK_NODATA. Support is optional and negotiated
  using the VIRTIO_GPU_F_RESOURCE_BLOB feature flag.

\begin{lstlisting}
#define VIRTIO_GPU_BLOB_MEM_GUEST             0x0001
#define VIRTIO_GPU_BLOB_MEM_HOST3D            0x0002
#define VIRTIO_GPU_BLOB_MEM_HOST3D_GUEST      0x0003

#define VIRTIO_GPU_BLOB_FLAG_USE_MAPPABLE     0x0001
#define VIRTIO_GPU_BLOB_FLAG_USE_SHAREABLE    0x0002
#define VIRTIO_GPU_BLOB_FLAG_USE_CROSS_DEVICE 0x0004

struct virtio_gpu_resource_create_blob {
       struct virtio_gpu_ctrl_hdr hdr;
       le32 resource_id;
       le32 blob_mem;
       le32 blob_flags;
       le32 nr_entries;
       le64 blob_id;
       le64 size;
};

\end{lstlisting}

A blob resource is a container for:

  \begin{itemize*}
  \item a guest memory allocation (referred to as a
  "guest-only blob resource").
  \item a host memory allocation (referred to as a
  "host-only blob resource").
  \item a guest memory and host memory allocation (referred
  to as a "default blob resource").
  \end{itemize*}

The memory properties of the blob resource MUST be described by
\field{blob_mem}, which MUST be non-zero.

For default and guest-only blob resources, \field{nr_entries} guest
memory entries may be assigned to the resource.  For default blob resources
(i.e, when \field{blob_mem} is VIRTIO_GPU_BLOB_MEM_HOST3D_GUEST), these
memory entries are used as a shadow buffer for the host memory. To
facilitate drivers that support swap-in and swap-out, \field{nr_entries} may
be zero and VIRTIO_GPU_CMD_RESOURCE_ATTACH_BACKING may be subsequently used.
VIRTIO_GPU_CMD_RESOURCE_DETACH_BACKING may be used to unassign memory entries.

\field{blob_mem} can only be VIRTIO_GPU_BLOB_MEM_HOST3D and
VIRTIO_GPU_BLOB_MEM_HOST3D_GUEST if VIRTIO_GPU_F_VIRGL is supported.
VIRTIO_GPU_BLOB_MEM_GUEST is valid regardless whether VIRTIO_GPU_F_VIRGL
is supported or not.

For VIRTIO_GPU_BLOB_MEM_HOST3D and VIRTIO_GPU_BLOB_MEM_HOST3D_GUEST, the
virtio-gpu resource MUST be created from the rendering context local object
identified by the \field{blob_id}. The actual allocation is done via
VIRTIO_GPU_CMD_SUBMIT_3D.

The driver MUST inform the device if the blob resource is used for
memory access, sharing between driver instances and/or sharing with
other devices. This is done via the \field{blob_flags} field.

If VIRTIO_GPU_F_VIRGL is set, both VIRTIO_GPU_CMD_TRANSFER_TO_HOST_3D
and VIRTIO_GPU_CMD_TRANSFER_FROM_HOST_3D may be used to update the
resource. There is no restriction on the image/buffer view the driver
has on the blob resource.

\item[VIRTIO_GPU_CMD_SET_SCANOUT_BLOB] sets scanout parameters for a
   blob resource. Request data is
  \field{struct virtio_gpu_set_scanout_blob}. Response type is
  VIRTIO_GPU_RESP_OK_NODATA. Support is optional and negotiated
  using the VIRTIO_GPU_F_RESOURCE_BLOB feature flag.

\begin{lstlisting}
struct virtio_gpu_set_scanout_blob {
       struct virtio_gpu_ctrl_hdr hdr;
       struct virtio_gpu_rect r;
       le32 scanout_id;
       le32 resource_id;
       le32 width;
       le32 height;
       le32 format;
       le32 padding;
       le32 strides[4];
       le32 offsets[4];
};
\end{lstlisting}

The rectangle \field{r} represents the portion of the blob resource being
displayed. The rest is the metadata associated with the blob resource. The
format MUST be one of \field{enum virtio_gpu_formats}.  The format MAY be
compressed with header and data planes.

\end{description}

\subsubsection{Device Operation: controlq (3d)}\label{sec:Device Types / GPU Device / Device Operation / Device Operation: controlq (3d)}

These commands are supported by the device if the VIRTIO_GPU_F_VIRGL
feature flag is set.

\begin{description}

\item[VIRTIO_GPU_CMD_CTX_CREATE] creates a context for submitting an opaque
  command stream.  Request data is \field{struct virtio_gpu_ctx_create}.
  Response type is VIRTIO_GPU_RESP_OK_NODATA.

\begin{lstlisting}
#define VIRTIO_GPU_CONTEXT_INIT_CAPSET_ID_MASK 0x000000ff;
struct virtio_gpu_ctx_create {
       struct virtio_gpu_ctrl_hdr hdr;
       le32 nlen;
       le32 context_init;
       char debug_name[64];
};
\end{lstlisting}

The implementation MUST create a context for the given \field{ctx_id} in
the \field{hdr}.  For debugging purposes, a \field{debug_name} and it's
length \field{nlen} is provided by the driver.  If
VIRTIO_GPU_F_CONTEXT_INIT is supported, then lower 8 bits of
\field{context_init} MAY contain the \field{capset_id} associated with
context.  In that case, then the device MUST create a context that can
handle the specified command stream.

If the lower 8-bits of the \field{context_init} are zero, then the type of
the context is determined by the device.

\item[VIRTIO_GPU_CMD_CTX_DESTROY]
\item[VIRTIO_GPU_CMD_CTX_ATTACH_RESOURCE]
\item[VIRTIO_GPU_CMD_CTX_DETACH_RESOURCE]
  Manage virtio-gpu 3d contexts.

\item[VIRTIO_GPU_CMD_RESOURCE_CREATE_3D]
  Create virtio-gpu 3d resources.

\item[VIRTIO_GPU_CMD_TRANSFER_TO_HOST_3D]
\item[VIRTIO_GPU_CMD_TRANSFER_FROM_HOST_3D]
  Transfer data from and to virtio-gpu 3d resources.

\item[VIRTIO_GPU_CMD_SUBMIT_3D]
  Submit an opaque command stream.  The type of the command stream is
  determined when creating a context.

\item[VIRTIO_GPU_CMD_RESOURCE_MAP_BLOB] maps a host-only
  blob resource into an offset in the host visible memory region. Request
  data is \field{struct virtio_gpu_resource_map_blob}.  The driver MUST
  not map a blob resource that is already mapped.  Response type is
  VIRTIO_GPU_RESP_OK_MAP_INFO. Support is optional and negotiated
  using the VIRTIO_GPU_F_RESOURCE_BLOB feature flag and checking for
  the presence of the host visible memory region.

\begin{lstlisting}
struct virtio_gpu_resource_map_blob {
        struct virtio_gpu_ctrl_hdr hdr;
        le32 resource_id;
        le32 padding;
        le64 offset;
};

#define VIRTIO_GPU_MAP_CACHE_MASK      0x0f
#define VIRTIO_GPU_MAP_CACHE_NONE      0x00
#define VIRTIO_GPU_MAP_CACHE_CACHED    0x01
#define VIRTIO_GPU_MAP_CACHE_UNCACHED  0x02
#define VIRTIO_GPU_MAP_CACHE_WC        0x03
struct virtio_gpu_resp_map_info {
        struct virtio_gpu_ctrl_hdr hdr;
        u32 map_info;
        u32 padding;
};
\end{lstlisting}

\item[VIRTIO_GPU_CMD_RESOURCE_UNMAP_BLOB] unmaps a
  host-only blob resource from the host visible memory region. Request data
  is \field{struct virtio_gpu_resource_unmap_blob}.  Response type is
  VIRTIO_GPU_RESP_OK_NODATA.  Support is optional and negotiated
  using the VIRTIO_GPU_F_RESOURCE_BLOB feature flag and checking for
  the presence of the host visible memory region.

\begin{lstlisting}
struct virtio_gpu_resource_unmap_blob {
        struct virtio_gpu_ctrl_hdr hdr;
        le32 resource_id;
        le32 padding;
};
\end{lstlisting}

\end{description}

\subsubsection{Device Operation: cursorq}\label{sec:Device Types / GPU Device / Device Operation / Device Operation: cursorq}

Both cursorq commands use the same command struct.

\begin{lstlisting}
struct virtio_gpu_cursor_pos {
        le32 scanout_id;
        le32 x;
        le32 y;
        le32 padding;
};

struct virtio_gpu_update_cursor {
        struct virtio_gpu_ctrl_hdr hdr;
        struct virtio_gpu_cursor_pos pos;
        le32 resource_id;
        le32 hot_x;
        le32 hot_y;
        le32 padding;
};
\end{lstlisting}

\begin{description}

\item[VIRTIO_GPU_CMD_UPDATE_CURSOR]
Update cursor.
Request data is \field{struct virtio_gpu_update_cursor}.
Response type is VIRTIO_GPU_RESP_OK_NODATA.

Full cursor update.  Cursor will be loaded from the specified
\field{resource_id} and will be moved to \field{pos}.  The driver must
transfer the cursor into the resource beforehand (using control queue
commands) and make sure the commands to fill the resource are actually
processed (using fencing).

\item[VIRTIO_GPU_CMD_MOVE_CURSOR]
Move cursor.
Request data is \field{struct virtio_gpu_update_cursor}.
Response type is VIRTIO_GPU_RESP_OK_NODATA.

Move cursor to the place specified in \field{pos}.  The other fields
are not used and will be ignored by the device.

\end{description}

\subsection{VGA Compatibility}\label{sec:Device Types / GPU Device / VGA Compatibility}

Applies to Virtio Over PCI only.  The GPU device can come with and
without VGA compatibility.  The PCI class should be DISPLAY_VGA if VGA
compatibility is present and DISPLAY_OTHER otherwise.

VGA compatibility: PCI region 0 has the linear framebuffer, standard
vga registers are present.  Configuring a scanout
(VIRTIO_GPU_CMD_SET_SCANOUT) switches the device from vga
compatibility mode into native virtio mode.  A reset switches it back
into vga compatibility mode.

Note: qemu implementation also provides bochs dispi interface io ports
and mmio bar at pci region 1 and is therefore fully compatible with
the qemu stdvga (see \href{https://git.qemu-project.org/?p=qemu.git;a=blob;f=docs/specs/standard-vga.txt;hb=HEAD}{docs/specs/standard-vga.txt} in the qemu source tree).

\section{GPU Device}\label{sec:Device Types / GPU Device}

virtio-gpu is a virtio based graphics adapter.  It can operate in 2D
mode and in 3D mode.  3D mode will offload rendering ops to
the host gpu and therefore requires a gpu with 3D support on the host
machine.

In 2D mode the virtio-gpu device provides support for ARGB Hardware
cursors and multiple scanouts (aka heads).

\subsection{Device ID}\label{sec:Device Types / GPU Device / Device ID}

16

\subsection{Virtqueues}\label{sec:Device Types / GPU Device / Virtqueues}

\begin{description}
\item[0] controlq - queue for sending control commands
\item[1] cursorq - queue for sending cursor updates
\end{description}

Both queues have the same format.  Each request and each response have
a fixed header, followed by command specific data fields.  The
separate cursor queue is the "fast track" for cursor commands
(VIRTIO_GPU_CMD_UPDATE_CURSOR and VIRTIO_GPU_CMD_MOVE_CURSOR), so they
go through without being delayed by time-consuming commands in the
control queue.

\subsection{Feature bits}\label{sec:Device Types / GPU Device / Feature bits}

\begin{description}
\item[VIRTIO_GPU_F_VIRGL (0)] virgl 3D mode is supported.
\item[VIRTIO_GPU_F_EDID  (1)] EDID is supported.
\item[VIRTIO_GPU_F_RESOURCE_UUID (2)] assigning resources UUIDs for export
  to other virtio devices is supported.
\item[VIRTIO_GPU_F_RESOURCE_BLOB (3)] creating and using size-based blob
  resources is supported.
\item[VIRTIO_GPU_F_CONTEXT_INIT (4)] multiple context types and
  synchronization timelines supported.  Requires VIRTIO_GPU_F_VIRGL.
\end{description}

\subsection{Device configuration layout}\label{sec:Device Types / GPU Device / Device configuration layout}

GPU device configuration uses the following layout structure and
definitions:

\begin{lstlisting}
#define VIRTIO_GPU_EVENT_DISPLAY (1 << 0)

struct virtio_gpu_config {
        le32 events_read;
        le32 events_clear;
        le32 num_scanouts;
        le32 num_capsets;
};
\end{lstlisting}

\subsubsection{Device configuration fields}

\begin{description}
\item[\field{events_read}] signals pending events to the driver.  The
  driver MUST NOT write to this field.
\item[\field{events_clear}] clears pending events in the device.
  Writing a '1' into a bit will clear the corresponding bit in
  \field{events_read}, mimicking write-to-clear behavior.
\item[\field{num_scanouts}] specifies the maximum number of scanouts
  supported by the device.  Minimum value is 1, maximum value is 16.
\item[\field{num_capsets}] specifies the maximum number of capability
  sets supported by the device.  The minimum value is zero.
\end{description}

\subsubsection{Events}

\begin{description}
\item[VIRTIO_GPU_EVENT_DISPLAY] Display configuration has changed.
  The driver SHOULD use the VIRTIO_GPU_CMD_GET_DISPLAY_INFO command to
  fetch the information from the device.  In case EDID support is
  negotiated (VIRTIO_GPU_F_EDID feature flag) the device SHOULD also
  fetch the updated EDID blobs using the VIRTIO_GPU_CMD_GET_EDID
  command.
\end{description}

\devicenormative{\subsection}{Device Initialization}{Device Types / GPU Device / Device Initialization}

The driver SHOULD query the display information from the device using
the VIRTIO_GPU_CMD_GET_DISPLAY_INFO command and use that information
for the initial scanout setup.  In case EDID support is negotiated
(VIRTIO_GPU_F_EDID feature flag) the device SHOULD also fetch the EDID
information using the VIRTIO_GPU_CMD_GET_EDID command.  If no
information is available or all displays are disabled the driver MAY
choose to use a fallback, such as 1024x768 at display 0.

The driver SHOULD query all shared memory regions supported by the device.
If the device supports shared memory, the \field{shmid} of a region MUST
(see \ref{sec:Basic Facilities of a Virtio Device /
Shared Memory Regions}~\nameref{sec:Basic Facilities of a Virtio Device /
Shared Memory Regions}) be one of the following:

\begin{lstlisting}
enum virtio_gpu_shm_id {
        VIRTIO_GPU_SHM_ID_UNDEFINED = 0,
        VIRTIO_GPU_SHM_ID_HOST_VISIBLE = 1,
};
\end{lstlisting}

The shared memory region with VIRTIO_GPU_SHM_ID_HOST_VISIBLE is referred as
the "host visible memory region".  The device MUST support the
VIRTIO_GPU_CMD_RESOURCE_MAP_BLOB and VIRTIO_GPU_CMD_RESOURCE_UNMAP_BLOB
if the host visible memory region is available.

\subsection{Device Operation}\label{sec:Device Types / GPU Device / Device Operation}

The virtio-gpu is based around the concept of resources private to the
host.  The guest must DMA transfer into these resources, unless shared memory
regions are supported. This is a design requirement in order to interface with
future 3D rendering. In the unaccelerated 2D mode there is no support for DMA
transfers from resources, just to them.

Resources are initially simple 2D resources, consisting of a width,
height and format along with an identifier. The guest must then attach
backing store to the resources in order for DMA transfers to
work. This is like a GART in a real GPU.

\subsubsection{Device Operation: Create a framebuffer and configure scanout}

\begin{itemize*}
\item Create a host resource using VIRTIO_GPU_CMD_RESOURCE_CREATE_2D.
\item Allocate a framebuffer from guest ram, and attach it as backing
  storage to the resource just created, using
  VIRTIO_GPU_CMD_RESOURCE_ATTACH_BACKING.  Scatter lists are
  supported, so the framebuffer doesn't need to be contignous in guest
  physical memory.
\item Use VIRTIO_GPU_CMD_SET_SCANOUT to link the framebuffer to
  a display scanout.
\end{itemize*}

\subsubsection{Device Operation: Update a framebuffer and scanout}

\begin{itemize*}
\item Render to your framebuffer memory.
\item Use VIRTIO_GPU_CMD_TRANSFER_TO_HOST_2D to update the host resource
  from guest memory.
\item Use VIRTIO_GPU_CMD_RESOURCE_FLUSH to flush the updated resource
  to the display.
\end{itemize*}

\subsubsection{Device Operation: Using pageflip}

It is possible to create multiple framebuffers, flip between them
using VIRTIO_GPU_CMD_SET_SCANOUT and VIRTIO_GPU_CMD_RESOURCE_FLUSH,
and update the invisible framebuffer using
VIRTIO_GPU_CMD_TRANSFER_TO_HOST_2D.

\subsubsection{Device Operation: Multihead setup}

In case two or more displays are present there are different ways to
configure things:

\begin{itemize*}
\item Create a single framebuffer, link it to all displays
  (mirroring).
\item Create an framebuffer for each display.
\item Create one big framebuffer, configure scanouts to display a
  different rectangle of that framebuffer each.
\end{itemize*}

\devicenormative{\subsubsection}{Device Operation: Command lifecycle and fencing}{Device Types / GPU Device / Device Operation / Device Operation: Command lifecycle and fencing}

The device MAY process controlq commands asyncronously and return them
to the driver before the processing is complete.  If the driver needs
to know when the processing is finished it can set the
VIRTIO_GPU_FLAG_FENCE flag in the request.  The device MUST finish the
processing before returning the command then.

Note: current qemu implementation does asyncrounous processing only in
3d mode, when offloading the processing to the host gpu.

\subsubsection{Device Operation: Configure mouse cursor}

The mouse cursor image is a normal resource, except that it must be
64x64 in size.  The driver MUST create and populate the resource
(using the usual VIRTIO_GPU_CMD_RESOURCE_CREATE_2D,
VIRTIO_GPU_CMD_RESOURCE_ATTACH_BACKING and
VIRTIO_GPU_CMD_TRANSFER_TO_HOST_2D controlq commands) and make sure they
are completed (using VIRTIO_GPU_FLAG_FENCE).

Then VIRTIO_GPU_CMD_UPDATE_CURSOR can be sent to the cursorq to set
the pointer shape and position.  To move the pointer without updating
the shape use VIRTIO_GPU_CMD_MOVE_CURSOR instead.

\subsubsection{Device Operation: Request header}\label{sec:Device Types / GPU Device / Device Operation / Device Operation: Request header}

All requests and responses on the virtqueues have a fixed header
using the following layout structure and definitions:

\begin{lstlisting}
enum virtio_gpu_ctrl_type {

        /* 2d commands */
        VIRTIO_GPU_CMD_GET_DISPLAY_INFO = 0x0100,
        VIRTIO_GPU_CMD_RESOURCE_CREATE_2D,
        VIRTIO_GPU_CMD_RESOURCE_UNREF,
        VIRTIO_GPU_CMD_SET_SCANOUT,
        VIRTIO_GPU_CMD_RESOURCE_FLUSH,
        VIRTIO_GPU_CMD_TRANSFER_TO_HOST_2D,
        VIRTIO_GPU_CMD_RESOURCE_ATTACH_BACKING,
        VIRTIO_GPU_CMD_RESOURCE_DETACH_BACKING,
        VIRTIO_GPU_CMD_GET_CAPSET_INFO,
        VIRTIO_GPU_CMD_GET_CAPSET,
        VIRTIO_GPU_CMD_GET_EDID,
        VIRTIO_GPU_CMD_RESOURCE_ASSIGN_UUID,
        VIRTIO_GPU_CMD_RESOURCE_CREATE_BLOB,
        VIRTIO_GPU_CMD_SET_SCANOUT_BLOB,

        /* 3d commands */
        VIRTIO_GPU_CMD_CTX_CREATE = 0x0200,
        VIRTIO_GPU_CMD_CTX_DESTROY,
        VIRTIO_GPU_CMD_CTX_ATTACH_RESOURCE,
        VIRTIO_GPU_CMD_CTX_DETACH_RESOURCE,
        VIRTIO_GPU_CMD_RESOURCE_CREATE_3D,
        VIRTIO_GPU_CMD_TRANSFER_TO_HOST_3D,
        VIRTIO_GPU_CMD_TRANSFER_FROM_HOST_3D,
        VIRTIO_GPU_CMD_SUBMIT_3D,
        VIRTIO_GPU_CMD_RESOURCE_MAP_BLOB,
        VIRTIO_GPU_CMD_RESOURCE_UNMAP_BLOB,

        /* cursor commands */
        VIRTIO_GPU_CMD_UPDATE_CURSOR = 0x0300,
        VIRTIO_GPU_CMD_MOVE_CURSOR,

        /* success responses */
        VIRTIO_GPU_RESP_OK_NODATA = 0x1100,
        VIRTIO_GPU_RESP_OK_DISPLAY_INFO,
        VIRTIO_GPU_RESP_OK_CAPSET_INFO,
        VIRTIO_GPU_RESP_OK_CAPSET,
        VIRTIO_GPU_RESP_OK_EDID,
        VIRTIO_GPU_RESP_OK_RESOURCE_UUID,
        VIRTIO_GPU_RESP_OK_MAP_INFO,

        /* error responses */
        VIRTIO_GPU_RESP_ERR_UNSPEC = 0x1200,
        VIRTIO_GPU_RESP_ERR_OUT_OF_MEMORY,
        VIRTIO_GPU_RESP_ERR_INVALID_SCANOUT_ID,
        VIRTIO_GPU_RESP_ERR_INVALID_RESOURCE_ID,
        VIRTIO_GPU_RESP_ERR_INVALID_CONTEXT_ID,
        VIRTIO_GPU_RESP_ERR_INVALID_PARAMETER,
};

#define VIRTIO_GPU_FLAG_FENCE (1 << 0)
#define VIRTIO_GPU_FLAG_INFO_RING_IDX (1 << 1)

struct virtio_gpu_ctrl_hdr {
        le32 type;
        le32 flags;
        le64 fence_id;
        le32 ctx_id;
        u8 ring_idx;
        u8 padding[3];
};
\end{lstlisting}

The fixed header \field{struct virtio_gpu_ctrl_hdr} in each
request includes the following fields:

\begin{description}
\item[\field{type}] specifies the type of the driver request
  (VIRTIO_GPU_CMD_*) or device response (VIRTIO_GPU_RESP_*).
\item[\field{flags}] request / response flags.
\item[\field{fence_id}] If the driver sets the VIRTIO_GPU_FLAG_FENCE
  bit in the request \field{flags} field the device MUST:
  \begin{itemize*}
  \item set VIRTIO_GPU_FLAG_FENCE bit in the response,
  \item copy the content of the \field{fence_id} field from the
    request to the response, and
  \item send the response only after command processing is complete.
  \end{itemize*}
\item[\field{ctx_id}] Rendering context (used in 3D mode only).
\item[\field{ring_idx}] If VIRTIO_GPU_F_CONTEXT_INIT is supported, then
  the driver MAY set VIRTIO_GPU_FLAG_INFO_RING_IDX bit in the request
  \field{flags}.  In that case:
  \begin{itemize*}
  \item \field{ring_idx} indicates the value of a context-specific ring
   index.  The minimum value is 0 and maximum value is 63 (inclusive).
  \item If VIRTIO_GPU_FLAG_FENCE is set, \field{fence_id} acts as a
   sequence number on the synchronization timeline defined by
   \field{ctx_idx} and the ring index.
  \item If VIRTIO_GPU_FLAG_FENCE is set and when the command associated
   with \field{fence_id} is complete, the device MUST send a response for
   all outstanding commands with a sequence number less than or equal to
   \field{fence_id} on the same synchronization timeline.
  \end{itemize*}
\end{description}

On success the device will return VIRTIO_GPU_RESP_OK_NODATA in
case there is no payload.  Otherwise the \field{type} field will
indicate the kind of payload.

On error the device will return one of the
VIRTIO_GPU_RESP_ERR_* error codes.

\subsubsection{Device Operation: controlq}\label{sec:Device Types / GPU Device / Device Operation / Device Operation: controlq}

For any coordinates given 0,0 is top left, larger x moves right,
larger y moves down.

\begin{description}

\item[VIRTIO_GPU_CMD_GET_DISPLAY_INFO] Retrieve the current output
  configuration.  No request data (just bare \field{struct
    virtio_gpu_ctrl_hdr}).  Response type is
  VIRTIO_GPU_RESP_OK_DISPLAY_INFO, response data is \field{struct
    virtio_gpu_resp_display_info}.

\begin{lstlisting}
#define VIRTIO_GPU_MAX_SCANOUTS 16

struct virtio_gpu_rect {
        le32 x;
        le32 y;
        le32 width;
        le32 height;
};

struct virtio_gpu_resp_display_info {
        struct virtio_gpu_ctrl_hdr hdr;
        struct virtio_gpu_display_one {
                struct virtio_gpu_rect r;
                le32 enabled;
                le32 flags;
        } pmodes[VIRTIO_GPU_MAX_SCANOUTS];
};
\end{lstlisting}

The response contains a list of per-scanout information.  The info
contains whether the scanout is enabled and what its preferred
position and size is.

The size (fields \field{width} and \field{height}) is similar to the
native panel resolution in EDID display information, except that in
the virtual machine case the size can change when the host window
representing the guest display is gets resized.

The position (fields \field{x} and \field{y}) describe how the
displays are arranged (i.e. which is -- for example -- the left
display).

The \field{enabled} field is set when the user enabled the display.
It is roughly the same as the connected state of a phyiscal display
connector.

\item[VIRTIO_GPU_CMD_GET_EDID] Retrieve the EDID data for a given
  scanout.  Request data is \field{struct virtio_gpu_get_edid}).
  Response type is VIRTIO_GPU_RESP_OK_EDID, response data is
  \field{struct virtio_gpu_resp_edid}.  Support is optional and
  negotiated using the VIRTIO_GPU_F_EDID feature flag.

\begin{lstlisting}
struct virtio_gpu_get_edid {
        struct virtio_gpu_ctrl_hdr hdr;
        le32 scanout;
        le32 padding;
};

struct virtio_gpu_resp_edid {
        struct virtio_gpu_ctrl_hdr hdr;
        le32 size;
        le32 padding;
        u8 edid[1024];
};
\end{lstlisting}

The response contains the EDID display data blob (as specified by
VESA) for the scanout.

\item[VIRTIO_GPU_CMD_RESOURCE_CREATE_2D] Create a 2D resource on the
  host.  Request data is \field{struct virtio_gpu_resource_create_2d}.
  Response type is VIRTIO_GPU_RESP_OK_NODATA.

\begin{lstlisting}
enum virtio_gpu_formats {
        VIRTIO_GPU_FORMAT_B8G8R8A8_UNORM  = 1,
        VIRTIO_GPU_FORMAT_B8G8R8X8_UNORM  = 2,
        VIRTIO_GPU_FORMAT_A8R8G8B8_UNORM  = 3,
        VIRTIO_GPU_FORMAT_X8R8G8B8_UNORM  = 4,

        VIRTIO_GPU_FORMAT_R8G8B8A8_UNORM  = 67,
        VIRTIO_GPU_FORMAT_X8B8G8R8_UNORM  = 68,

        VIRTIO_GPU_FORMAT_A8B8G8R8_UNORM  = 121,
        VIRTIO_GPU_FORMAT_R8G8B8X8_UNORM  = 134,
};

struct virtio_gpu_resource_create_2d {
        struct virtio_gpu_ctrl_hdr hdr;
        le32 resource_id;
        le32 format;
        le32 width;
        le32 height;
};
\end{lstlisting}

This creates a 2D resource on the host with the specified width,
height and format.  The resource ids are generated by the guest.

\item[VIRTIO_GPU_CMD_RESOURCE_UNREF] Destroy a resource.  Request data
  is \field{struct virtio_gpu_resource_unref}.  Response type is
  VIRTIO_GPU_RESP_OK_NODATA.

\begin{lstlisting}
struct virtio_gpu_resource_unref {
        struct virtio_gpu_ctrl_hdr hdr;
        le32 resource_id;
        le32 padding;
};
\end{lstlisting}

This informs the host that a resource is no longer required by the
guest.

\item[VIRTIO_GPU_CMD_SET_SCANOUT] Set the scanout parameters for a
  single output.  Request data is \field{struct
    virtio_gpu_set_scanout}.  Response type is
  VIRTIO_GPU_RESP_OK_NODATA.

\begin{lstlisting}
struct virtio_gpu_set_scanout {
        struct virtio_gpu_ctrl_hdr hdr;
        struct virtio_gpu_rect r;
        le32 scanout_id;
        le32 resource_id;
};
\end{lstlisting}

This sets the scanout parameters for a single scanout. The resource_id
is the resource to be scanned out from, along with a rectangle.

Scanout rectangles must be completely covered by the underlying
resource.  Overlapping (or identical) scanouts are allowed, typical
use case is screen mirroring.

The driver can use resource_id = 0 to disable a scanout.

\item[VIRTIO_GPU_CMD_RESOURCE_FLUSH] Flush a scanout resource Request
  data is \field{struct virtio_gpu_resource_flush}.  Response type is
  VIRTIO_GPU_RESP_OK_NODATA.

\begin{lstlisting}
struct virtio_gpu_resource_flush {
        struct virtio_gpu_ctrl_hdr hdr;
        struct virtio_gpu_rect r;
        le32 resource_id;
        le32 padding;
};
\end{lstlisting}

This flushes a resource to screen.  It takes a rectangle and a
resource id, and flushes any scanouts the resource is being used on.

\item[VIRTIO_GPU_CMD_TRANSFER_TO_HOST_2D] Transfer from guest memory
  to host resource.  Request data is \field{struct
    virtio_gpu_transfer_to_host_2d}.  Response type is
  VIRTIO_GPU_RESP_OK_NODATA.

\begin{lstlisting}
struct virtio_gpu_transfer_to_host_2d {
        struct virtio_gpu_ctrl_hdr hdr;
        struct virtio_gpu_rect r;
        le64 offset;
        le32 resource_id;
        le32 padding;
};
\end{lstlisting}

This takes a resource id along with an destination offset into the
resource, and a box to transfer to the host backing for the resource.

\item[VIRTIO_GPU_CMD_RESOURCE_ATTACH_BACKING] Assign backing pages to
  a resource.  Request data is \field{struct
    virtio_gpu_resource_attach_backing}, followed by \field{struct
    virtio_gpu_mem_entry} entries.  Response type is
  VIRTIO_GPU_RESP_OK_NODATA.

\begin{lstlisting}
struct virtio_gpu_resource_attach_backing {
        struct virtio_gpu_ctrl_hdr hdr;
        le32 resource_id;
        le32 nr_entries;
};

struct virtio_gpu_mem_entry {
        le64 addr;
        le32 length;
        le32 padding;
};
\end{lstlisting}

This assign an array of guest pages as the backing store for a
resource. These pages are then used for the transfer operations for
that resource from that point on.

\item[VIRTIO_GPU_CMD_RESOURCE_DETACH_BACKING] Detach backing pages
  from a resource.  Request data is \field{struct
    virtio_gpu_resource_detach_backing}.  Response type is
  VIRTIO_GPU_RESP_OK_NODATA.

\begin{lstlisting}
struct virtio_gpu_resource_detach_backing {
        struct virtio_gpu_ctrl_hdr hdr;
        le32 resource_id;
        le32 padding;
};
\end{lstlisting}

This detaches any backing pages from a resource, to be used in case of
guest swapping or object destruction.

\item[VIRTIO_GPU_CMD_GET_CAPSET_INFO] Gets the information associated with
  a particular \field{capset_index}, which MUST less than \field{num_capsets}
  defined in the device configuration.  Request data is
  \field{struct virtio_gpu_get_capset_info}.  Response type is
  VIRTIO_GPU_RESP_OK_CAPSET_INFO.

  On success, \field{struct virtio_gpu_resp_capset_info} contains the
  \field{capset_id}, \field{capset_max_version}, \field{capset_max_size}
  associated with capset at the specified {capset_idex}.  field{capset_id} MUST
  be one of the following (see listing for values):

  \begin{itemize*}
  \item \href{https://gitlab.freedesktop.org/virgl/virglrenderer/-/blob/master/src/virgl_hw.h#L526}{VIRTIO_GPU_CAPSET_VIRGL} --
	the first edition of Virgl (Gallium OpenGL) protocol.
  \item \href{https://gitlab.freedesktop.org/virgl/virglrenderer/-/blob/master/src/virgl_hw.h#L550}{VIRTIO_GPU_CAPSET_VIRGL2} --
	the second edition of Virgl (Gallium OpenGL) protocol after the capset fix.
  \item \href{https://android.googlesource.com/device/generic/vulkan-cereal/+/refs/heads/master/protocols/}{VIRTIO_GPU_CAPSET_GFXSTREAM} --
	gfxtream's (mostly) autogenerated GLES and Vulkan streaming protocols.
  \item \href{https://gitlab.freedesktop.org/olv/venus-protocol}{VIRTIO_GPU_CAPSET_VENUS} --
	Mesa's (mostly) autogenerated Vulkan protocol.
  \item \href{https://chromium.googlesource.com/chromiumos/platform/crosvm/+/refs/heads/main/rutabaga_gfx/src/cross_domain/cross_domain_protocol.rs}{VIRTIO_GPU_CAPSET_CROSS_DOMAIN} --
	protocol for display virtualization via Wayland proxying.
  \end{itemize*}

\begin{lstlisting}
struct virtio_gpu_get_capset_info {
        struct virtio_gpu_ctrl_hdr hdr;
        le32 capset_index;
        le32 padding;
};

#define VIRTIO_GPU_CAPSET_VIRGL 1
#define VIRTIO_GPU_CAPSET_VIRGL2 2
#define VIRTIO_GPU_CAPSET_GFXSTREAM 3
#define VIRTIO_GPU_CAPSET_VENUS 4
#define VIRTIO_GPU_CAPSET_CROSS_DOMAIN 5
struct virtio_gpu_resp_capset_info {
        struct virtio_gpu_ctrl_hdr hdr;
        le32 capset_id;
        le32 capset_max_version;
        le32 capset_max_size;
        le32 padding;
};
\end{lstlisting}

\item[VIRTIO_GPU_CMD_GET_CAPSET] Gets the capset associated with a
  particular \field{capset_id} and \field{capset_version}.  Request data is
  \field{struct virtio_gpu_get_capset}.  Response type is
  VIRTIO_GPU_RESP_OK_CAPSET.

\begin{lstlisting}
struct virtio_gpu_get_capset {
        struct virtio_gpu_ctrl_hdr hdr;
        le32 capset_id;
        le32 capset_version;
};

struct virtio_gpu_resp_capset {
        struct virtio_gpu_ctrl_hdr hdr;
        u8 capset_data[];
};
\end{lstlisting}

\item[VIRTIO_GPU_CMD_RESOURCE_ASSIGN_UUID] Creates an exported object from
  a resource. Request data is \field{struct
    virtio_gpu_resource_assign_uuid}.  Response type is
  VIRTIO_GPU_RESP_OK_RESOURCE_UUID, response data is \field{struct
    virtio_gpu_resp_resource_uuid}. Support is optional and negotiated
    using the VIRTIO_GPU_F_RESOURCE_UUID feature flag.

\begin{lstlisting}
struct virtio_gpu_resource_assign_uuid {
        struct virtio_gpu_ctrl_hdr hdr;
        le32 resource_id;
        le32 padding;
};

struct virtio_gpu_resp_resource_uuid {
        struct virtio_gpu_ctrl_hdr hdr;
        u8 uuid[16];
};
\end{lstlisting}

The response contains a UUID which identifies the exported object created from
the host private resource. Note that if the resource has an attached backing,
modifications made to the host private resource through the exported object by
other devices are not visible in the attached backing until they are transferred
into the backing.

\item[VIRTIO_GPU_CMD_RESOURCE_CREATE_BLOB] Creates a virtio-gpu blob
  resource. Request data is \field{struct
  virtio_gpu_resource_create_blob}, followed by \field{struct
  virtio_gpu_mem_entry} entries. Response type is
  VIRTIO_GPU_RESP_OK_NODATA. Support is optional and negotiated
  using the VIRTIO_GPU_F_RESOURCE_BLOB feature flag.

\begin{lstlisting}
#define VIRTIO_GPU_BLOB_MEM_GUEST             0x0001
#define VIRTIO_GPU_BLOB_MEM_HOST3D            0x0002
#define VIRTIO_GPU_BLOB_MEM_HOST3D_GUEST      0x0003

#define VIRTIO_GPU_BLOB_FLAG_USE_MAPPABLE     0x0001
#define VIRTIO_GPU_BLOB_FLAG_USE_SHAREABLE    0x0002
#define VIRTIO_GPU_BLOB_FLAG_USE_CROSS_DEVICE 0x0004

struct virtio_gpu_resource_create_blob {
       struct virtio_gpu_ctrl_hdr hdr;
       le32 resource_id;
       le32 blob_mem;
       le32 blob_flags;
       le32 nr_entries;
       le64 blob_id;
       le64 size;
};

\end{lstlisting}

A blob resource is a container for:

  \begin{itemize*}
  \item a guest memory allocation (referred to as a
  "guest-only blob resource").
  \item a host memory allocation (referred to as a
  "host-only blob resource").
  \item a guest memory and host memory allocation (referred
  to as a "default blob resource").
  \end{itemize*}

The memory properties of the blob resource MUST be described by
\field{blob_mem}, which MUST be non-zero.

For default and guest-only blob resources, \field{nr_entries} guest
memory entries may be assigned to the resource.  For default blob resources
(i.e, when \field{blob_mem} is VIRTIO_GPU_BLOB_MEM_HOST3D_GUEST), these
memory entries are used as a shadow buffer for the host memory. To
facilitate drivers that support swap-in and swap-out, \field{nr_entries} may
be zero and VIRTIO_GPU_CMD_RESOURCE_ATTACH_BACKING may be subsequently used.
VIRTIO_GPU_CMD_RESOURCE_DETACH_BACKING may be used to unassign memory entries.

\field{blob_mem} can only be VIRTIO_GPU_BLOB_MEM_HOST3D and
VIRTIO_GPU_BLOB_MEM_HOST3D_GUEST if VIRTIO_GPU_F_VIRGL is supported.
VIRTIO_GPU_BLOB_MEM_GUEST is valid regardless whether VIRTIO_GPU_F_VIRGL
is supported or not.

For VIRTIO_GPU_BLOB_MEM_HOST3D and VIRTIO_GPU_BLOB_MEM_HOST3D_GUEST, the
virtio-gpu resource MUST be created from the rendering context local object
identified by the \field{blob_id}. The actual allocation is done via
VIRTIO_GPU_CMD_SUBMIT_3D.

The driver MUST inform the device if the blob resource is used for
memory access, sharing between driver instances and/or sharing with
other devices. This is done via the \field{blob_flags} field.

If VIRTIO_GPU_F_VIRGL is set, both VIRTIO_GPU_CMD_TRANSFER_TO_HOST_3D
and VIRTIO_GPU_CMD_TRANSFER_FROM_HOST_3D may be used to update the
resource. There is no restriction on the image/buffer view the driver
has on the blob resource.

\item[VIRTIO_GPU_CMD_SET_SCANOUT_BLOB] sets scanout parameters for a
   blob resource. Request data is
  \field{struct virtio_gpu_set_scanout_blob}. Response type is
  VIRTIO_GPU_RESP_OK_NODATA. Support is optional and negotiated
  using the VIRTIO_GPU_F_RESOURCE_BLOB feature flag.

\begin{lstlisting}
struct virtio_gpu_set_scanout_blob {
       struct virtio_gpu_ctrl_hdr hdr;
       struct virtio_gpu_rect r;
       le32 scanout_id;
       le32 resource_id;
       le32 width;
       le32 height;
       le32 format;
       le32 padding;
       le32 strides[4];
       le32 offsets[4];
};
\end{lstlisting}

The rectangle \field{r} represents the portion of the blob resource being
displayed. The rest is the metadata associated with the blob resource. The
format MUST be one of \field{enum virtio_gpu_formats}.  The format MAY be
compressed with header and data planes.

\end{description}

\subsubsection{Device Operation: controlq (3d)}\label{sec:Device Types / GPU Device / Device Operation / Device Operation: controlq (3d)}

These commands are supported by the device if the VIRTIO_GPU_F_VIRGL
feature flag is set.

\begin{description}

\item[VIRTIO_GPU_CMD_CTX_CREATE] creates a context for submitting an opaque
  command stream.  Request data is \field{struct virtio_gpu_ctx_create}.
  Response type is VIRTIO_GPU_RESP_OK_NODATA.

\begin{lstlisting}
#define VIRTIO_GPU_CONTEXT_INIT_CAPSET_ID_MASK 0x000000ff;
struct virtio_gpu_ctx_create {
       struct virtio_gpu_ctrl_hdr hdr;
       le32 nlen;
       le32 context_init;
       char debug_name[64];
};
\end{lstlisting}

The implementation MUST create a context for the given \field{ctx_id} in
the \field{hdr}.  For debugging purposes, a \field{debug_name} and it's
length \field{nlen} is provided by the driver.  If
VIRTIO_GPU_F_CONTEXT_INIT is supported, then lower 8 bits of
\field{context_init} MAY contain the \field{capset_id} associated with
context.  In that case, then the device MUST create a context that can
handle the specified command stream.

If the lower 8-bits of the \field{context_init} are zero, then the type of
the context is determined by the device.

\item[VIRTIO_GPU_CMD_CTX_DESTROY]
\item[VIRTIO_GPU_CMD_CTX_ATTACH_RESOURCE]
\item[VIRTIO_GPU_CMD_CTX_DETACH_RESOURCE]
  Manage virtio-gpu 3d contexts.

\item[VIRTIO_GPU_CMD_RESOURCE_CREATE_3D]
  Create virtio-gpu 3d resources.

\item[VIRTIO_GPU_CMD_TRANSFER_TO_HOST_3D]
\item[VIRTIO_GPU_CMD_TRANSFER_FROM_HOST_3D]
  Transfer data from and to virtio-gpu 3d resources.

\item[VIRTIO_GPU_CMD_SUBMIT_3D]
  Submit an opaque command stream.  The type of the command stream is
  determined when creating a context.

\item[VIRTIO_GPU_CMD_RESOURCE_MAP_BLOB] maps a host-only
  blob resource into an offset in the host visible memory region. Request
  data is \field{struct virtio_gpu_resource_map_blob}.  The driver MUST
  not map a blob resource that is already mapped.  Response type is
  VIRTIO_GPU_RESP_OK_MAP_INFO. Support is optional and negotiated
  using the VIRTIO_GPU_F_RESOURCE_BLOB feature flag and checking for
  the presence of the host visible memory region.

\begin{lstlisting}
struct virtio_gpu_resource_map_blob {
        struct virtio_gpu_ctrl_hdr hdr;
        le32 resource_id;
        le32 padding;
        le64 offset;
};

#define VIRTIO_GPU_MAP_CACHE_MASK      0x0f
#define VIRTIO_GPU_MAP_CACHE_NONE      0x00
#define VIRTIO_GPU_MAP_CACHE_CACHED    0x01
#define VIRTIO_GPU_MAP_CACHE_UNCACHED  0x02
#define VIRTIO_GPU_MAP_CACHE_WC        0x03
struct virtio_gpu_resp_map_info {
        struct virtio_gpu_ctrl_hdr hdr;
        u32 map_info;
        u32 padding;
};
\end{lstlisting}

\item[VIRTIO_GPU_CMD_RESOURCE_UNMAP_BLOB] unmaps a
  host-only blob resource from the host visible memory region. Request data
  is \field{struct virtio_gpu_resource_unmap_blob}.  Response type is
  VIRTIO_GPU_RESP_OK_NODATA.  Support is optional and negotiated
  using the VIRTIO_GPU_F_RESOURCE_BLOB feature flag and checking for
  the presence of the host visible memory region.

\begin{lstlisting}
struct virtio_gpu_resource_unmap_blob {
        struct virtio_gpu_ctrl_hdr hdr;
        le32 resource_id;
        le32 padding;
};
\end{lstlisting}

\end{description}

\subsubsection{Device Operation: cursorq}\label{sec:Device Types / GPU Device / Device Operation / Device Operation: cursorq}

Both cursorq commands use the same command struct.

\begin{lstlisting}
struct virtio_gpu_cursor_pos {
        le32 scanout_id;
        le32 x;
        le32 y;
        le32 padding;
};

struct virtio_gpu_update_cursor {
        struct virtio_gpu_ctrl_hdr hdr;
        struct virtio_gpu_cursor_pos pos;
        le32 resource_id;
        le32 hot_x;
        le32 hot_y;
        le32 padding;
};
\end{lstlisting}

\begin{description}

\item[VIRTIO_GPU_CMD_UPDATE_CURSOR]
Update cursor.
Request data is \field{struct virtio_gpu_update_cursor}.
Response type is VIRTIO_GPU_RESP_OK_NODATA.

Full cursor update.  Cursor will be loaded from the specified
\field{resource_id} and will be moved to \field{pos}.  The driver must
transfer the cursor into the resource beforehand (using control queue
commands) and make sure the commands to fill the resource are actually
processed (using fencing).

\item[VIRTIO_GPU_CMD_MOVE_CURSOR]
Move cursor.
Request data is \field{struct virtio_gpu_update_cursor}.
Response type is VIRTIO_GPU_RESP_OK_NODATA.

Move cursor to the place specified in \field{pos}.  The other fields
are not used and will be ignored by the device.

\end{description}

\subsection{VGA Compatibility}\label{sec:Device Types / GPU Device / VGA Compatibility}

Applies to Virtio Over PCI only.  The GPU device can come with and
without VGA compatibility.  The PCI class should be DISPLAY_VGA if VGA
compatibility is present and DISPLAY_OTHER otherwise.

VGA compatibility: PCI region 0 has the linear framebuffer, standard
vga registers are present.  Configuring a scanout
(VIRTIO_GPU_CMD_SET_SCANOUT) switches the device from vga
compatibility mode into native virtio mode.  A reset switches it back
into vga compatibility mode.

Note: qemu implementation also provides bochs dispi interface io ports
and mmio bar at pci region 1 and is therefore fully compatible with
the qemu stdvga (see \href{https://git.qemu-project.org/?p=qemu.git;a=blob;f=docs/specs/standard-vga.txt;hb=HEAD}{docs/specs/standard-vga.txt} in the qemu source tree).

\section{GPU Device}\label{sec:Device Types / GPU Device}

virtio-gpu is a virtio based graphics adapter.  It can operate in 2D
mode and in 3D mode.  3D mode will offload rendering ops to
the host gpu and therefore requires a gpu with 3D support on the host
machine.

In 2D mode the virtio-gpu device provides support for ARGB Hardware
cursors and multiple scanouts (aka heads).

\subsection{Device ID}\label{sec:Device Types / GPU Device / Device ID}

16

\subsection{Virtqueues}\label{sec:Device Types / GPU Device / Virtqueues}

\begin{description}
\item[0] controlq - queue for sending control commands
\item[1] cursorq - queue for sending cursor updates
\end{description}

Both queues have the same format.  Each request and each response have
a fixed header, followed by command specific data fields.  The
separate cursor queue is the "fast track" for cursor commands
(VIRTIO_GPU_CMD_UPDATE_CURSOR and VIRTIO_GPU_CMD_MOVE_CURSOR), so they
go through without being delayed by time-consuming commands in the
control queue.

\subsection{Feature bits}\label{sec:Device Types / GPU Device / Feature bits}

\begin{description}
\item[VIRTIO_GPU_F_VIRGL (0)] virgl 3D mode is supported.
\item[VIRTIO_GPU_F_EDID  (1)] EDID is supported.
\item[VIRTIO_GPU_F_RESOURCE_UUID (2)] assigning resources UUIDs for export
  to other virtio devices is supported.
\item[VIRTIO_GPU_F_RESOURCE_BLOB (3)] creating and using size-based blob
  resources is supported.
\item[VIRTIO_GPU_F_CONTEXT_INIT (4)] multiple context types and
  synchronization timelines supported.  Requires VIRTIO_GPU_F_VIRGL.
\end{description}

\subsection{Device configuration layout}\label{sec:Device Types / GPU Device / Device configuration layout}

GPU device configuration uses the following layout structure and
definitions:

\begin{lstlisting}
#define VIRTIO_GPU_EVENT_DISPLAY (1 << 0)

struct virtio_gpu_config {
        le32 events_read;
        le32 events_clear;
        le32 num_scanouts;
        le32 num_capsets;
};
\end{lstlisting}

\subsubsection{Device configuration fields}

\begin{description}
\item[\field{events_read}] signals pending events to the driver.  The
  driver MUST NOT write to this field.
\item[\field{events_clear}] clears pending events in the device.
  Writing a '1' into a bit will clear the corresponding bit in
  \field{events_read}, mimicking write-to-clear behavior.
\item[\field{num_scanouts}] specifies the maximum number of scanouts
  supported by the device.  Minimum value is 1, maximum value is 16.
\item[\field{num_capsets}] specifies the maximum number of capability
  sets supported by the device.  The minimum value is zero.
\end{description}

\subsubsection{Events}

\begin{description}
\item[VIRTIO_GPU_EVENT_DISPLAY] Display configuration has changed.
  The driver SHOULD use the VIRTIO_GPU_CMD_GET_DISPLAY_INFO command to
  fetch the information from the device.  In case EDID support is
  negotiated (VIRTIO_GPU_F_EDID feature flag) the device SHOULD also
  fetch the updated EDID blobs using the VIRTIO_GPU_CMD_GET_EDID
  command.
\end{description}

\devicenormative{\subsection}{Device Initialization}{Device Types / GPU Device / Device Initialization}

The driver SHOULD query the display information from the device using
the VIRTIO_GPU_CMD_GET_DISPLAY_INFO command and use that information
for the initial scanout setup.  In case EDID support is negotiated
(VIRTIO_GPU_F_EDID feature flag) the device SHOULD also fetch the EDID
information using the VIRTIO_GPU_CMD_GET_EDID command.  If no
information is available or all displays are disabled the driver MAY
choose to use a fallback, such as 1024x768 at display 0.

The driver SHOULD query all shared memory regions supported by the device.
If the device supports shared memory, the \field{shmid} of a region MUST
(see \ref{sec:Basic Facilities of a Virtio Device /
Shared Memory Regions}~\nameref{sec:Basic Facilities of a Virtio Device /
Shared Memory Regions}) be one of the following:

\begin{lstlisting}
enum virtio_gpu_shm_id {
        VIRTIO_GPU_SHM_ID_UNDEFINED = 0,
        VIRTIO_GPU_SHM_ID_HOST_VISIBLE = 1,
};
\end{lstlisting}

The shared memory region with VIRTIO_GPU_SHM_ID_HOST_VISIBLE is referred as
the "host visible memory region".  The device MUST support the
VIRTIO_GPU_CMD_RESOURCE_MAP_BLOB and VIRTIO_GPU_CMD_RESOURCE_UNMAP_BLOB
if the host visible memory region is available.

\subsection{Device Operation}\label{sec:Device Types / GPU Device / Device Operation}

The virtio-gpu is based around the concept of resources private to the
host.  The guest must DMA transfer into these resources, unless shared memory
regions are supported. This is a design requirement in order to interface with
future 3D rendering. In the unaccelerated 2D mode there is no support for DMA
transfers from resources, just to them.

Resources are initially simple 2D resources, consisting of a width,
height and format along with an identifier. The guest must then attach
backing store to the resources in order for DMA transfers to
work. This is like a GART in a real GPU.

\subsubsection{Device Operation: Create a framebuffer and configure scanout}

\begin{itemize*}
\item Create a host resource using VIRTIO_GPU_CMD_RESOURCE_CREATE_2D.
\item Allocate a framebuffer from guest ram, and attach it as backing
  storage to the resource just created, using
  VIRTIO_GPU_CMD_RESOURCE_ATTACH_BACKING.  Scatter lists are
  supported, so the framebuffer doesn't need to be contignous in guest
  physical memory.
\item Use VIRTIO_GPU_CMD_SET_SCANOUT to link the framebuffer to
  a display scanout.
\end{itemize*}

\subsubsection{Device Operation: Update a framebuffer and scanout}

\begin{itemize*}
\item Render to your framebuffer memory.
\item Use VIRTIO_GPU_CMD_TRANSFER_TO_HOST_2D to update the host resource
  from guest memory.
\item Use VIRTIO_GPU_CMD_RESOURCE_FLUSH to flush the updated resource
  to the display.
\end{itemize*}

\subsubsection{Device Operation: Using pageflip}

It is possible to create multiple framebuffers, flip between them
using VIRTIO_GPU_CMD_SET_SCANOUT and VIRTIO_GPU_CMD_RESOURCE_FLUSH,
and update the invisible framebuffer using
VIRTIO_GPU_CMD_TRANSFER_TO_HOST_2D.

\subsubsection{Device Operation: Multihead setup}

In case two or more displays are present there are different ways to
configure things:

\begin{itemize*}
\item Create a single framebuffer, link it to all displays
  (mirroring).
\item Create an framebuffer for each display.
\item Create one big framebuffer, configure scanouts to display a
  different rectangle of that framebuffer each.
\end{itemize*}

\devicenormative{\subsubsection}{Device Operation: Command lifecycle and fencing}{Device Types / GPU Device / Device Operation / Device Operation: Command lifecycle and fencing}

The device MAY process controlq commands asyncronously and return them
to the driver before the processing is complete.  If the driver needs
to know when the processing is finished it can set the
VIRTIO_GPU_FLAG_FENCE flag in the request.  The device MUST finish the
processing before returning the command then.

Note: current qemu implementation does asyncrounous processing only in
3d mode, when offloading the processing to the host gpu.

\subsubsection{Device Operation: Configure mouse cursor}

The mouse cursor image is a normal resource, except that it must be
64x64 in size.  The driver MUST create and populate the resource
(using the usual VIRTIO_GPU_CMD_RESOURCE_CREATE_2D,
VIRTIO_GPU_CMD_RESOURCE_ATTACH_BACKING and
VIRTIO_GPU_CMD_TRANSFER_TO_HOST_2D controlq commands) and make sure they
are completed (using VIRTIO_GPU_FLAG_FENCE).

Then VIRTIO_GPU_CMD_UPDATE_CURSOR can be sent to the cursorq to set
the pointer shape and position.  To move the pointer without updating
the shape use VIRTIO_GPU_CMD_MOVE_CURSOR instead.

\subsubsection{Device Operation: Request header}\label{sec:Device Types / GPU Device / Device Operation / Device Operation: Request header}

All requests and responses on the virtqueues have a fixed header
using the following layout structure and definitions:

\begin{lstlisting}
enum virtio_gpu_ctrl_type {

        /* 2d commands */
        VIRTIO_GPU_CMD_GET_DISPLAY_INFO = 0x0100,
        VIRTIO_GPU_CMD_RESOURCE_CREATE_2D,
        VIRTIO_GPU_CMD_RESOURCE_UNREF,
        VIRTIO_GPU_CMD_SET_SCANOUT,
        VIRTIO_GPU_CMD_RESOURCE_FLUSH,
        VIRTIO_GPU_CMD_TRANSFER_TO_HOST_2D,
        VIRTIO_GPU_CMD_RESOURCE_ATTACH_BACKING,
        VIRTIO_GPU_CMD_RESOURCE_DETACH_BACKING,
        VIRTIO_GPU_CMD_GET_CAPSET_INFO,
        VIRTIO_GPU_CMD_GET_CAPSET,
        VIRTIO_GPU_CMD_GET_EDID,
        VIRTIO_GPU_CMD_RESOURCE_ASSIGN_UUID,
        VIRTIO_GPU_CMD_RESOURCE_CREATE_BLOB,
        VIRTIO_GPU_CMD_SET_SCANOUT_BLOB,

        /* 3d commands */
        VIRTIO_GPU_CMD_CTX_CREATE = 0x0200,
        VIRTIO_GPU_CMD_CTX_DESTROY,
        VIRTIO_GPU_CMD_CTX_ATTACH_RESOURCE,
        VIRTIO_GPU_CMD_CTX_DETACH_RESOURCE,
        VIRTIO_GPU_CMD_RESOURCE_CREATE_3D,
        VIRTIO_GPU_CMD_TRANSFER_TO_HOST_3D,
        VIRTIO_GPU_CMD_TRANSFER_FROM_HOST_3D,
        VIRTIO_GPU_CMD_SUBMIT_3D,
        VIRTIO_GPU_CMD_RESOURCE_MAP_BLOB,
        VIRTIO_GPU_CMD_RESOURCE_UNMAP_BLOB,

        /* cursor commands */
        VIRTIO_GPU_CMD_UPDATE_CURSOR = 0x0300,
        VIRTIO_GPU_CMD_MOVE_CURSOR,

        /* success responses */
        VIRTIO_GPU_RESP_OK_NODATA = 0x1100,
        VIRTIO_GPU_RESP_OK_DISPLAY_INFO,
        VIRTIO_GPU_RESP_OK_CAPSET_INFO,
        VIRTIO_GPU_RESP_OK_CAPSET,
        VIRTIO_GPU_RESP_OK_EDID,
        VIRTIO_GPU_RESP_OK_RESOURCE_UUID,
        VIRTIO_GPU_RESP_OK_MAP_INFO,

        /* error responses */
        VIRTIO_GPU_RESP_ERR_UNSPEC = 0x1200,
        VIRTIO_GPU_RESP_ERR_OUT_OF_MEMORY,
        VIRTIO_GPU_RESP_ERR_INVALID_SCANOUT_ID,
        VIRTIO_GPU_RESP_ERR_INVALID_RESOURCE_ID,
        VIRTIO_GPU_RESP_ERR_INVALID_CONTEXT_ID,
        VIRTIO_GPU_RESP_ERR_INVALID_PARAMETER,
};

#define VIRTIO_GPU_FLAG_FENCE (1 << 0)
#define VIRTIO_GPU_FLAG_INFO_RING_IDX (1 << 1)

struct virtio_gpu_ctrl_hdr {
        le32 type;
        le32 flags;
        le64 fence_id;
        le32 ctx_id;
        u8 ring_idx;
        u8 padding[3];
};
\end{lstlisting}

The fixed header \field{struct virtio_gpu_ctrl_hdr} in each
request includes the following fields:

\begin{description}
\item[\field{type}] specifies the type of the driver request
  (VIRTIO_GPU_CMD_*) or device response (VIRTIO_GPU_RESP_*).
\item[\field{flags}] request / response flags.
\item[\field{fence_id}] If the driver sets the VIRTIO_GPU_FLAG_FENCE
  bit in the request \field{flags} field the device MUST:
  \begin{itemize*}
  \item set VIRTIO_GPU_FLAG_FENCE bit in the response,
  \item copy the content of the \field{fence_id} field from the
    request to the response, and
  \item send the response only after command processing is complete.
  \end{itemize*}
\item[\field{ctx_id}] Rendering context (used in 3D mode only).
\item[\field{ring_idx}] If VIRTIO_GPU_F_CONTEXT_INIT is supported, then
  the driver MAY set VIRTIO_GPU_FLAG_INFO_RING_IDX bit in the request
  \field{flags}.  In that case:
  \begin{itemize*}
  \item \field{ring_idx} indicates the value of a context-specific ring
   index.  The minimum value is 0 and maximum value is 63 (inclusive).
  \item If VIRTIO_GPU_FLAG_FENCE is set, \field{fence_id} acts as a
   sequence number on the synchronization timeline defined by
   \field{ctx_idx} and the ring index.
  \item If VIRTIO_GPU_FLAG_FENCE is set and when the command associated
   with \field{fence_id} is complete, the device MUST send a response for
   all outstanding commands with a sequence number less than or equal to
   \field{fence_id} on the same synchronization timeline.
  \end{itemize*}
\end{description}

On success the device will return VIRTIO_GPU_RESP_OK_NODATA in
case there is no payload.  Otherwise the \field{type} field will
indicate the kind of payload.

On error the device will return one of the
VIRTIO_GPU_RESP_ERR_* error codes.

\subsubsection{Device Operation: controlq}\label{sec:Device Types / GPU Device / Device Operation / Device Operation: controlq}

For any coordinates given 0,0 is top left, larger x moves right,
larger y moves down.

\begin{description}

\item[VIRTIO_GPU_CMD_GET_DISPLAY_INFO] Retrieve the current output
  configuration.  No request data (just bare \field{struct
    virtio_gpu_ctrl_hdr}).  Response type is
  VIRTIO_GPU_RESP_OK_DISPLAY_INFO, response data is \field{struct
    virtio_gpu_resp_display_info}.

\begin{lstlisting}
#define VIRTIO_GPU_MAX_SCANOUTS 16

struct virtio_gpu_rect {
        le32 x;
        le32 y;
        le32 width;
        le32 height;
};

struct virtio_gpu_resp_display_info {
        struct virtio_gpu_ctrl_hdr hdr;
        struct virtio_gpu_display_one {
                struct virtio_gpu_rect r;
                le32 enabled;
                le32 flags;
        } pmodes[VIRTIO_GPU_MAX_SCANOUTS];
};
\end{lstlisting}

The response contains a list of per-scanout information.  The info
contains whether the scanout is enabled and what its preferred
position and size is.

The size (fields \field{width} and \field{height}) is similar to the
native panel resolution in EDID display information, except that in
the virtual machine case the size can change when the host window
representing the guest display is gets resized.

The position (fields \field{x} and \field{y}) describe how the
displays are arranged (i.e. which is -- for example -- the left
display).

The \field{enabled} field is set when the user enabled the display.
It is roughly the same as the connected state of a phyiscal display
connector.

\item[VIRTIO_GPU_CMD_GET_EDID] Retrieve the EDID data for a given
  scanout.  Request data is \field{struct virtio_gpu_get_edid}).
  Response type is VIRTIO_GPU_RESP_OK_EDID, response data is
  \field{struct virtio_gpu_resp_edid}.  Support is optional and
  negotiated using the VIRTIO_GPU_F_EDID feature flag.

\begin{lstlisting}
struct virtio_gpu_get_edid {
        struct virtio_gpu_ctrl_hdr hdr;
        le32 scanout;
        le32 padding;
};

struct virtio_gpu_resp_edid {
        struct virtio_gpu_ctrl_hdr hdr;
        le32 size;
        le32 padding;
        u8 edid[1024];
};
\end{lstlisting}

The response contains the EDID display data blob (as specified by
VESA) for the scanout.

\item[VIRTIO_GPU_CMD_RESOURCE_CREATE_2D] Create a 2D resource on the
  host.  Request data is \field{struct virtio_gpu_resource_create_2d}.
  Response type is VIRTIO_GPU_RESP_OK_NODATA.

\begin{lstlisting}
enum virtio_gpu_formats {
        VIRTIO_GPU_FORMAT_B8G8R8A8_UNORM  = 1,
        VIRTIO_GPU_FORMAT_B8G8R8X8_UNORM  = 2,
        VIRTIO_GPU_FORMAT_A8R8G8B8_UNORM  = 3,
        VIRTIO_GPU_FORMAT_X8R8G8B8_UNORM  = 4,

        VIRTIO_GPU_FORMAT_R8G8B8A8_UNORM  = 67,
        VIRTIO_GPU_FORMAT_X8B8G8R8_UNORM  = 68,

        VIRTIO_GPU_FORMAT_A8B8G8R8_UNORM  = 121,
        VIRTIO_GPU_FORMAT_R8G8B8X8_UNORM  = 134,
};

struct virtio_gpu_resource_create_2d {
        struct virtio_gpu_ctrl_hdr hdr;
        le32 resource_id;
        le32 format;
        le32 width;
        le32 height;
};
\end{lstlisting}

This creates a 2D resource on the host with the specified width,
height and format.  The resource ids are generated by the guest.

\item[VIRTIO_GPU_CMD_RESOURCE_UNREF] Destroy a resource.  Request data
  is \field{struct virtio_gpu_resource_unref}.  Response type is
  VIRTIO_GPU_RESP_OK_NODATA.

\begin{lstlisting}
struct virtio_gpu_resource_unref {
        struct virtio_gpu_ctrl_hdr hdr;
        le32 resource_id;
        le32 padding;
};
\end{lstlisting}

This informs the host that a resource is no longer required by the
guest.

\item[VIRTIO_GPU_CMD_SET_SCANOUT] Set the scanout parameters for a
  single output.  Request data is \field{struct
    virtio_gpu_set_scanout}.  Response type is
  VIRTIO_GPU_RESP_OK_NODATA.

\begin{lstlisting}
struct virtio_gpu_set_scanout {
        struct virtio_gpu_ctrl_hdr hdr;
        struct virtio_gpu_rect r;
        le32 scanout_id;
        le32 resource_id;
};
\end{lstlisting}

This sets the scanout parameters for a single scanout. The resource_id
is the resource to be scanned out from, along with a rectangle.

Scanout rectangles must be completely covered by the underlying
resource.  Overlapping (or identical) scanouts are allowed, typical
use case is screen mirroring.

The driver can use resource_id = 0 to disable a scanout.

\item[VIRTIO_GPU_CMD_RESOURCE_FLUSH] Flush a scanout resource Request
  data is \field{struct virtio_gpu_resource_flush}.  Response type is
  VIRTIO_GPU_RESP_OK_NODATA.

\begin{lstlisting}
struct virtio_gpu_resource_flush {
        struct virtio_gpu_ctrl_hdr hdr;
        struct virtio_gpu_rect r;
        le32 resource_id;
        le32 padding;
};
\end{lstlisting}

This flushes a resource to screen.  It takes a rectangle and a
resource id, and flushes any scanouts the resource is being used on.

\item[VIRTIO_GPU_CMD_TRANSFER_TO_HOST_2D] Transfer from guest memory
  to host resource.  Request data is \field{struct
    virtio_gpu_transfer_to_host_2d}.  Response type is
  VIRTIO_GPU_RESP_OK_NODATA.

\begin{lstlisting}
struct virtio_gpu_transfer_to_host_2d {
        struct virtio_gpu_ctrl_hdr hdr;
        struct virtio_gpu_rect r;
        le64 offset;
        le32 resource_id;
        le32 padding;
};
\end{lstlisting}

This takes a resource id along with an destination offset into the
resource, and a box to transfer to the host backing for the resource.

\item[VIRTIO_GPU_CMD_RESOURCE_ATTACH_BACKING] Assign backing pages to
  a resource.  Request data is \field{struct
    virtio_gpu_resource_attach_backing}, followed by \field{struct
    virtio_gpu_mem_entry} entries.  Response type is
  VIRTIO_GPU_RESP_OK_NODATA.

\begin{lstlisting}
struct virtio_gpu_resource_attach_backing {
        struct virtio_gpu_ctrl_hdr hdr;
        le32 resource_id;
        le32 nr_entries;
};

struct virtio_gpu_mem_entry {
        le64 addr;
        le32 length;
        le32 padding;
};
\end{lstlisting}

This assign an array of guest pages as the backing store for a
resource. These pages are then used for the transfer operations for
that resource from that point on.

\item[VIRTIO_GPU_CMD_RESOURCE_DETACH_BACKING] Detach backing pages
  from a resource.  Request data is \field{struct
    virtio_gpu_resource_detach_backing}.  Response type is
  VIRTIO_GPU_RESP_OK_NODATA.

\begin{lstlisting}
struct virtio_gpu_resource_detach_backing {
        struct virtio_gpu_ctrl_hdr hdr;
        le32 resource_id;
        le32 padding;
};
\end{lstlisting}

This detaches any backing pages from a resource, to be used in case of
guest swapping or object destruction.

\item[VIRTIO_GPU_CMD_GET_CAPSET_INFO] Gets the information associated with
  a particular \field{capset_index}, which MUST less than \field{num_capsets}
  defined in the device configuration.  Request data is
  \field{struct virtio_gpu_get_capset_info}.  Response type is
  VIRTIO_GPU_RESP_OK_CAPSET_INFO.

  On success, \field{struct virtio_gpu_resp_capset_info} contains the
  \field{capset_id}, \field{capset_max_version}, \field{capset_max_size}
  associated with capset at the specified {capset_idex}.  field{capset_id} MUST
  be one of the following (see listing for values):

  \begin{itemize*}
  \item \href{https://gitlab.freedesktop.org/virgl/virglrenderer/-/blob/master/src/virgl_hw.h#L526}{VIRTIO_GPU_CAPSET_VIRGL} --
	the first edition of Virgl (Gallium OpenGL) protocol.
  \item \href{https://gitlab.freedesktop.org/virgl/virglrenderer/-/blob/master/src/virgl_hw.h#L550}{VIRTIO_GPU_CAPSET_VIRGL2} --
	the second edition of Virgl (Gallium OpenGL) protocol after the capset fix.
  \item \href{https://android.googlesource.com/device/generic/vulkan-cereal/+/refs/heads/master/protocols/}{VIRTIO_GPU_CAPSET_GFXSTREAM} --
	gfxtream's (mostly) autogenerated GLES and Vulkan streaming protocols.
  \item \href{https://gitlab.freedesktop.org/olv/venus-protocol}{VIRTIO_GPU_CAPSET_VENUS} --
	Mesa's (mostly) autogenerated Vulkan protocol.
  \item \href{https://chromium.googlesource.com/chromiumos/platform/crosvm/+/refs/heads/main/rutabaga_gfx/src/cross_domain/cross_domain_protocol.rs}{VIRTIO_GPU_CAPSET_CROSS_DOMAIN} --
	protocol for display virtualization via Wayland proxying.
  \end{itemize*}

\begin{lstlisting}
struct virtio_gpu_get_capset_info {
        struct virtio_gpu_ctrl_hdr hdr;
        le32 capset_index;
        le32 padding;
};

#define VIRTIO_GPU_CAPSET_VIRGL 1
#define VIRTIO_GPU_CAPSET_VIRGL2 2
#define VIRTIO_GPU_CAPSET_GFXSTREAM 3
#define VIRTIO_GPU_CAPSET_VENUS 4
#define VIRTIO_GPU_CAPSET_CROSS_DOMAIN 5
struct virtio_gpu_resp_capset_info {
        struct virtio_gpu_ctrl_hdr hdr;
        le32 capset_id;
        le32 capset_max_version;
        le32 capset_max_size;
        le32 padding;
};
\end{lstlisting}

\item[VIRTIO_GPU_CMD_GET_CAPSET] Gets the capset associated with a
  particular \field{capset_id} and \field{capset_version}.  Request data is
  \field{struct virtio_gpu_get_capset}.  Response type is
  VIRTIO_GPU_RESP_OK_CAPSET.

\begin{lstlisting}
struct virtio_gpu_get_capset {
        struct virtio_gpu_ctrl_hdr hdr;
        le32 capset_id;
        le32 capset_version;
};

struct virtio_gpu_resp_capset {
        struct virtio_gpu_ctrl_hdr hdr;
        u8 capset_data[];
};
\end{lstlisting}

\item[VIRTIO_GPU_CMD_RESOURCE_ASSIGN_UUID] Creates an exported object from
  a resource. Request data is \field{struct
    virtio_gpu_resource_assign_uuid}.  Response type is
  VIRTIO_GPU_RESP_OK_RESOURCE_UUID, response data is \field{struct
    virtio_gpu_resp_resource_uuid}. Support is optional and negotiated
    using the VIRTIO_GPU_F_RESOURCE_UUID feature flag.

\begin{lstlisting}
struct virtio_gpu_resource_assign_uuid {
        struct virtio_gpu_ctrl_hdr hdr;
        le32 resource_id;
        le32 padding;
};

struct virtio_gpu_resp_resource_uuid {
        struct virtio_gpu_ctrl_hdr hdr;
        u8 uuid[16];
};
\end{lstlisting}

The response contains a UUID which identifies the exported object created from
the host private resource. Note that if the resource has an attached backing,
modifications made to the host private resource through the exported object by
other devices are not visible in the attached backing until they are transferred
into the backing.

\item[VIRTIO_GPU_CMD_RESOURCE_CREATE_BLOB] Creates a virtio-gpu blob
  resource. Request data is \field{struct
  virtio_gpu_resource_create_blob}, followed by \field{struct
  virtio_gpu_mem_entry} entries. Response type is
  VIRTIO_GPU_RESP_OK_NODATA. Support is optional and negotiated
  using the VIRTIO_GPU_F_RESOURCE_BLOB feature flag.

\begin{lstlisting}
#define VIRTIO_GPU_BLOB_MEM_GUEST             0x0001
#define VIRTIO_GPU_BLOB_MEM_HOST3D            0x0002
#define VIRTIO_GPU_BLOB_MEM_HOST3D_GUEST      0x0003

#define VIRTIO_GPU_BLOB_FLAG_USE_MAPPABLE     0x0001
#define VIRTIO_GPU_BLOB_FLAG_USE_SHAREABLE    0x0002
#define VIRTIO_GPU_BLOB_FLAG_USE_CROSS_DEVICE 0x0004

struct virtio_gpu_resource_create_blob {
       struct virtio_gpu_ctrl_hdr hdr;
       le32 resource_id;
       le32 blob_mem;
       le32 blob_flags;
       le32 nr_entries;
       le64 blob_id;
       le64 size;
};

\end{lstlisting}

A blob resource is a container for:

  \begin{itemize*}
  \item a guest memory allocation (referred to as a
  "guest-only blob resource").
  \item a host memory allocation (referred to as a
  "host-only blob resource").
  \item a guest memory and host memory allocation (referred
  to as a "default blob resource").
  \end{itemize*}

The memory properties of the blob resource MUST be described by
\field{blob_mem}, which MUST be non-zero.

For default and guest-only blob resources, \field{nr_entries} guest
memory entries may be assigned to the resource.  For default blob resources
(i.e, when \field{blob_mem} is VIRTIO_GPU_BLOB_MEM_HOST3D_GUEST), these
memory entries are used as a shadow buffer for the host memory. To
facilitate drivers that support swap-in and swap-out, \field{nr_entries} may
be zero and VIRTIO_GPU_CMD_RESOURCE_ATTACH_BACKING may be subsequently used.
VIRTIO_GPU_CMD_RESOURCE_DETACH_BACKING may be used to unassign memory entries.

\field{blob_mem} can only be VIRTIO_GPU_BLOB_MEM_HOST3D and
VIRTIO_GPU_BLOB_MEM_HOST3D_GUEST if VIRTIO_GPU_F_VIRGL is supported.
VIRTIO_GPU_BLOB_MEM_GUEST is valid regardless whether VIRTIO_GPU_F_VIRGL
is supported or not.

For VIRTIO_GPU_BLOB_MEM_HOST3D and VIRTIO_GPU_BLOB_MEM_HOST3D_GUEST, the
virtio-gpu resource MUST be created from the rendering context local object
identified by the \field{blob_id}. The actual allocation is done via
VIRTIO_GPU_CMD_SUBMIT_3D.

The driver MUST inform the device if the blob resource is used for
memory access, sharing between driver instances and/or sharing with
other devices. This is done via the \field{blob_flags} field.

If VIRTIO_GPU_F_VIRGL is set, both VIRTIO_GPU_CMD_TRANSFER_TO_HOST_3D
and VIRTIO_GPU_CMD_TRANSFER_FROM_HOST_3D may be used to update the
resource. There is no restriction on the image/buffer view the driver
has on the blob resource.

\item[VIRTIO_GPU_CMD_SET_SCANOUT_BLOB] sets scanout parameters for a
   blob resource. Request data is
  \field{struct virtio_gpu_set_scanout_blob}. Response type is
  VIRTIO_GPU_RESP_OK_NODATA. Support is optional and negotiated
  using the VIRTIO_GPU_F_RESOURCE_BLOB feature flag.

\begin{lstlisting}
struct virtio_gpu_set_scanout_blob {
       struct virtio_gpu_ctrl_hdr hdr;
       struct virtio_gpu_rect r;
       le32 scanout_id;
       le32 resource_id;
       le32 width;
       le32 height;
       le32 format;
       le32 padding;
       le32 strides[4];
       le32 offsets[4];
};
\end{lstlisting}

The rectangle \field{r} represents the portion of the blob resource being
displayed. The rest is the metadata associated with the blob resource. The
format MUST be one of \field{enum virtio_gpu_formats}.  The format MAY be
compressed with header and data planes.

\end{description}

\subsubsection{Device Operation: controlq (3d)}\label{sec:Device Types / GPU Device / Device Operation / Device Operation: controlq (3d)}

These commands are supported by the device if the VIRTIO_GPU_F_VIRGL
feature flag is set.

\begin{description}

\item[VIRTIO_GPU_CMD_CTX_CREATE] creates a context for submitting an opaque
  command stream.  Request data is \field{struct virtio_gpu_ctx_create}.
  Response type is VIRTIO_GPU_RESP_OK_NODATA.

\begin{lstlisting}
#define VIRTIO_GPU_CONTEXT_INIT_CAPSET_ID_MASK 0x000000ff;
struct virtio_gpu_ctx_create {
       struct virtio_gpu_ctrl_hdr hdr;
       le32 nlen;
       le32 context_init;
       char debug_name[64];
};
\end{lstlisting}

The implementation MUST create a context for the given \field{ctx_id} in
the \field{hdr}.  For debugging purposes, a \field{debug_name} and it's
length \field{nlen} is provided by the driver.  If
VIRTIO_GPU_F_CONTEXT_INIT is supported, then lower 8 bits of
\field{context_init} MAY contain the \field{capset_id} associated with
context.  In that case, then the device MUST create a context that can
handle the specified command stream.

If the lower 8-bits of the \field{context_init} are zero, then the type of
the context is determined by the device.

\item[VIRTIO_GPU_CMD_CTX_DESTROY]
\item[VIRTIO_GPU_CMD_CTX_ATTACH_RESOURCE]
\item[VIRTIO_GPU_CMD_CTX_DETACH_RESOURCE]
  Manage virtio-gpu 3d contexts.

\item[VIRTIO_GPU_CMD_RESOURCE_CREATE_3D]
  Create virtio-gpu 3d resources.

\item[VIRTIO_GPU_CMD_TRANSFER_TO_HOST_3D]
\item[VIRTIO_GPU_CMD_TRANSFER_FROM_HOST_3D]
  Transfer data from and to virtio-gpu 3d resources.

\item[VIRTIO_GPU_CMD_SUBMIT_3D]
  Submit an opaque command stream.  The type of the command stream is
  determined when creating a context.

\item[VIRTIO_GPU_CMD_RESOURCE_MAP_BLOB] maps a host-only
  blob resource into an offset in the host visible memory region. Request
  data is \field{struct virtio_gpu_resource_map_blob}.  The driver MUST
  not map a blob resource that is already mapped.  Response type is
  VIRTIO_GPU_RESP_OK_MAP_INFO. Support is optional and negotiated
  using the VIRTIO_GPU_F_RESOURCE_BLOB feature flag and checking for
  the presence of the host visible memory region.

\begin{lstlisting}
struct virtio_gpu_resource_map_blob {
        struct virtio_gpu_ctrl_hdr hdr;
        le32 resource_id;
        le32 padding;
        le64 offset;
};

#define VIRTIO_GPU_MAP_CACHE_MASK      0x0f
#define VIRTIO_GPU_MAP_CACHE_NONE      0x00
#define VIRTIO_GPU_MAP_CACHE_CACHED    0x01
#define VIRTIO_GPU_MAP_CACHE_UNCACHED  0x02
#define VIRTIO_GPU_MAP_CACHE_WC        0x03
struct virtio_gpu_resp_map_info {
        struct virtio_gpu_ctrl_hdr hdr;
        u32 map_info;
        u32 padding;
};
\end{lstlisting}

\item[VIRTIO_GPU_CMD_RESOURCE_UNMAP_BLOB] unmaps a
  host-only blob resource from the host visible memory region. Request data
  is \field{struct virtio_gpu_resource_unmap_blob}.  Response type is
  VIRTIO_GPU_RESP_OK_NODATA.  Support is optional and negotiated
  using the VIRTIO_GPU_F_RESOURCE_BLOB feature flag and checking for
  the presence of the host visible memory region.

\begin{lstlisting}
struct virtio_gpu_resource_unmap_blob {
        struct virtio_gpu_ctrl_hdr hdr;
        le32 resource_id;
        le32 padding;
};
\end{lstlisting}

\end{description}

\subsubsection{Device Operation: cursorq}\label{sec:Device Types / GPU Device / Device Operation / Device Operation: cursorq}

Both cursorq commands use the same command struct.

\begin{lstlisting}
struct virtio_gpu_cursor_pos {
        le32 scanout_id;
        le32 x;
        le32 y;
        le32 padding;
};

struct virtio_gpu_update_cursor {
        struct virtio_gpu_ctrl_hdr hdr;
        struct virtio_gpu_cursor_pos pos;
        le32 resource_id;
        le32 hot_x;
        le32 hot_y;
        le32 padding;
};
\end{lstlisting}

\begin{description}

\item[VIRTIO_GPU_CMD_UPDATE_CURSOR]
Update cursor.
Request data is \field{struct virtio_gpu_update_cursor}.
Response type is VIRTIO_GPU_RESP_OK_NODATA.

Full cursor update.  Cursor will be loaded from the specified
\field{resource_id} and will be moved to \field{pos}.  The driver must
transfer the cursor into the resource beforehand (using control queue
commands) and make sure the commands to fill the resource are actually
processed (using fencing).

\item[VIRTIO_GPU_CMD_MOVE_CURSOR]
Move cursor.
Request data is \field{struct virtio_gpu_update_cursor}.
Response type is VIRTIO_GPU_RESP_OK_NODATA.

Move cursor to the place specified in \field{pos}.  The other fields
are not used and will be ignored by the device.

\end{description}

\subsection{VGA Compatibility}\label{sec:Device Types / GPU Device / VGA Compatibility}

Applies to Virtio Over PCI only.  The GPU device can come with and
without VGA compatibility.  The PCI class should be DISPLAY_VGA if VGA
compatibility is present and DISPLAY_OTHER otherwise.

VGA compatibility: PCI region 0 has the linear framebuffer, standard
vga registers are present.  Configuring a scanout
(VIRTIO_GPU_CMD_SET_SCANOUT) switches the device from vga
compatibility mode into native virtio mode.  A reset switches it back
into vga compatibility mode.

Note: qemu implementation also provides bochs dispi interface io ports
and mmio bar at pci region 1 and is therefore fully compatible with
the qemu stdvga (see \href{https://git.qemu-project.org/?p=qemu.git;a=blob;f=docs/specs/standard-vga.txt;hb=HEAD}{docs/specs/standard-vga.txt} in the qemu source tree).

\section{GPU Device}\label{sec:Device Types / GPU Device}

virtio-gpu is a virtio based graphics adapter.  It can operate in 2D
mode and in 3D mode.  3D mode will offload rendering ops to
the host gpu and therefore requires a gpu with 3D support on the host
machine.

In 2D mode the virtio-gpu device provides support for ARGB Hardware
cursors and multiple scanouts (aka heads).

\subsection{Device ID}\label{sec:Device Types / GPU Device / Device ID}

16

\subsection{Virtqueues}\label{sec:Device Types / GPU Device / Virtqueues}

\begin{description}
\item[0] controlq - queue for sending control commands
\item[1] cursorq - queue for sending cursor updates
\end{description}

Both queues have the same format.  Each request and each response have
a fixed header, followed by command specific data fields.  The
separate cursor queue is the "fast track" for cursor commands
(VIRTIO_GPU_CMD_UPDATE_CURSOR and VIRTIO_GPU_CMD_MOVE_CURSOR), so they
go through without being delayed by time-consuming commands in the
control queue.

\subsection{Feature bits}\label{sec:Device Types / GPU Device / Feature bits}

\begin{description}
\item[VIRTIO_GPU_F_VIRGL (0)] virgl 3D mode is supported.
\item[VIRTIO_GPU_F_EDID  (1)] EDID is supported.
\item[VIRTIO_GPU_F_RESOURCE_UUID (2)] assigning resources UUIDs for export
  to other virtio devices is supported.
\item[VIRTIO_GPU_F_RESOURCE_BLOB (3)] creating and using size-based blob
  resources is supported.
\item[VIRTIO_GPU_F_CONTEXT_INIT (4)] multiple context types and
  synchronization timelines supported.  Requires VIRTIO_GPU_F_VIRGL.
\end{description}

\subsection{Device configuration layout}\label{sec:Device Types / GPU Device / Device configuration layout}

GPU device configuration uses the following layout structure and
definitions:

\begin{lstlisting}
#define VIRTIO_GPU_EVENT_DISPLAY (1 << 0)

struct virtio_gpu_config {
        le32 events_read;
        le32 events_clear;
        le32 num_scanouts;
        le32 num_capsets;
};
\end{lstlisting}

\subsubsection{Device configuration fields}

\begin{description}
\item[\field{events_read}] signals pending events to the driver.  The
  driver MUST NOT write to this field.
\item[\field{events_clear}] clears pending events in the device.
  Writing a '1' into a bit will clear the corresponding bit in
  \field{events_read}, mimicking write-to-clear behavior.
\item[\field{num_scanouts}] specifies the maximum number of scanouts
  supported by the device.  Minimum value is 1, maximum value is 16.
\item[\field{num_capsets}] specifies the maximum number of capability
  sets supported by the device.  The minimum value is zero.
\end{description}

\subsubsection{Events}

\begin{description}
\item[VIRTIO_GPU_EVENT_DISPLAY] Display configuration has changed.
  The driver SHOULD use the VIRTIO_GPU_CMD_GET_DISPLAY_INFO command to
  fetch the information from the device.  In case EDID support is
  negotiated (VIRTIO_GPU_F_EDID feature flag) the device SHOULD also
  fetch the updated EDID blobs using the VIRTIO_GPU_CMD_GET_EDID
  command.
\end{description}

\devicenormative{\subsection}{Device Initialization}{Device Types / GPU Device / Device Initialization}

The driver SHOULD query the display information from the device using
the VIRTIO_GPU_CMD_GET_DISPLAY_INFO command and use that information
for the initial scanout setup.  In case EDID support is negotiated
(VIRTIO_GPU_F_EDID feature flag) the device SHOULD also fetch the EDID
information using the VIRTIO_GPU_CMD_GET_EDID command.  If no
information is available or all displays are disabled the driver MAY
choose to use a fallback, such as 1024x768 at display 0.

The driver SHOULD query all shared memory regions supported by the device.
If the device supports shared memory, the \field{shmid} of a region MUST
(see \ref{sec:Basic Facilities of a Virtio Device /
Shared Memory Regions}~\nameref{sec:Basic Facilities of a Virtio Device /
Shared Memory Regions}) be one of the following:

\begin{lstlisting}
enum virtio_gpu_shm_id {
        VIRTIO_GPU_SHM_ID_UNDEFINED = 0,
        VIRTIO_GPU_SHM_ID_HOST_VISIBLE = 1,
};
\end{lstlisting}

The shared memory region with VIRTIO_GPU_SHM_ID_HOST_VISIBLE is referred as
the "host visible memory region".  The device MUST support the
VIRTIO_GPU_CMD_RESOURCE_MAP_BLOB and VIRTIO_GPU_CMD_RESOURCE_UNMAP_BLOB
if the host visible memory region is available.

\subsection{Device Operation}\label{sec:Device Types / GPU Device / Device Operation}

The virtio-gpu is based around the concept of resources private to the
host.  The guest must DMA transfer into these resources, unless shared memory
regions are supported. This is a design requirement in order to interface with
future 3D rendering. In the unaccelerated 2D mode there is no support for DMA
transfers from resources, just to them.

Resources are initially simple 2D resources, consisting of a width,
height and format along with an identifier. The guest must then attach
backing store to the resources in order for DMA transfers to
work. This is like a GART in a real GPU.

\subsubsection{Device Operation: Create a framebuffer and configure scanout}

\begin{itemize*}
\item Create a host resource using VIRTIO_GPU_CMD_RESOURCE_CREATE_2D.
\item Allocate a framebuffer from guest ram, and attach it as backing
  storage to the resource just created, using
  VIRTIO_GPU_CMD_RESOURCE_ATTACH_BACKING.  Scatter lists are
  supported, so the framebuffer doesn't need to be contignous in guest
  physical memory.
\item Use VIRTIO_GPU_CMD_SET_SCANOUT to link the framebuffer to
  a display scanout.
\end{itemize*}

\subsubsection{Device Operation: Update a framebuffer and scanout}

\begin{itemize*}
\item Render to your framebuffer memory.
\item Use VIRTIO_GPU_CMD_TRANSFER_TO_HOST_2D to update the host resource
  from guest memory.
\item Use VIRTIO_GPU_CMD_RESOURCE_FLUSH to flush the updated resource
  to the display.
\end{itemize*}

\subsubsection{Device Operation: Using pageflip}

It is possible to create multiple framebuffers, flip between them
using VIRTIO_GPU_CMD_SET_SCANOUT and VIRTIO_GPU_CMD_RESOURCE_FLUSH,
and update the invisible framebuffer using
VIRTIO_GPU_CMD_TRANSFER_TO_HOST_2D.

\subsubsection{Device Operation: Multihead setup}

In case two or more displays are present there are different ways to
configure things:

\begin{itemize*}
\item Create a single framebuffer, link it to all displays
  (mirroring).
\item Create an framebuffer for each display.
\item Create one big framebuffer, configure scanouts to display a
  different rectangle of that framebuffer each.
\end{itemize*}

\devicenormative{\subsubsection}{Device Operation: Command lifecycle and fencing}{Device Types / GPU Device / Device Operation / Device Operation: Command lifecycle and fencing}

The device MAY process controlq commands asyncronously and return them
to the driver before the processing is complete.  If the driver needs
to know when the processing is finished it can set the
VIRTIO_GPU_FLAG_FENCE flag in the request.  The device MUST finish the
processing before returning the command then.

Note: current qemu implementation does asyncrounous processing only in
3d mode, when offloading the processing to the host gpu.

\subsubsection{Device Operation: Configure mouse cursor}

The mouse cursor image is a normal resource, except that it must be
64x64 in size.  The driver MUST create and populate the resource
(using the usual VIRTIO_GPU_CMD_RESOURCE_CREATE_2D,
VIRTIO_GPU_CMD_RESOURCE_ATTACH_BACKING and
VIRTIO_GPU_CMD_TRANSFER_TO_HOST_2D controlq commands) and make sure they
are completed (using VIRTIO_GPU_FLAG_FENCE).

Then VIRTIO_GPU_CMD_UPDATE_CURSOR can be sent to the cursorq to set
the pointer shape and position.  To move the pointer without updating
the shape use VIRTIO_GPU_CMD_MOVE_CURSOR instead.

\subsubsection{Device Operation: Request header}\label{sec:Device Types / GPU Device / Device Operation / Device Operation: Request header}

All requests and responses on the virtqueues have a fixed header
using the following layout structure and definitions:

\begin{lstlisting}
enum virtio_gpu_ctrl_type {

        /* 2d commands */
        VIRTIO_GPU_CMD_GET_DISPLAY_INFO = 0x0100,
        VIRTIO_GPU_CMD_RESOURCE_CREATE_2D,
        VIRTIO_GPU_CMD_RESOURCE_UNREF,
        VIRTIO_GPU_CMD_SET_SCANOUT,
        VIRTIO_GPU_CMD_RESOURCE_FLUSH,
        VIRTIO_GPU_CMD_TRANSFER_TO_HOST_2D,
        VIRTIO_GPU_CMD_RESOURCE_ATTACH_BACKING,
        VIRTIO_GPU_CMD_RESOURCE_DETACH_BACKING,
        VIRTIO_GPU_CMD_GET_CAPSET_INFO,
        VIRTIO_GPU_CMD_GET_CAPSET,
        VIRTIO_GPU_CMD_GET_EDID,
        VIRTIO_GPU_CMD_RESOURCE_ASSIGN_UUID,
        VIRTIO_GPU_CMD_RESOURCE_CREATE_BLOB,
        VIRTIO_GPU_CMD_SET_SCANOUT_BLOB,

        /* 3d commands */
        VIRTIO_GPU_CMD_CTX_CREATE = 0x0200,
        VIRTIO_GPU_CMD_CTX_DESTROY,
        VIRTIO_GPU_CMD_CTX_ATTACH_RESOURCE,
        VIRTIO_GPU_CMD_CTX_DETACH_RESOURCE,
        VIRTIO_GPU_CMD_RESOURCE_CREATE_3D,
        VIRTIO_GPU_CMD_TRANSFER_TO_HOST_3D,
        VIRTIO_GPU_CMD_TRANSFER_FROM_HOST_3D,
        VIRTIO_GPU_CMD_SUBMIT_3D,
        VIRTIO_GPU_CMD_RESOURCE_MAP_BLOB,
        VIRTIO_GPU_CMD_RESOURCE_UNMAP_BLOB,

        /* cursor commands */
        VIRTIO_GPU_CMD_UPDATE_CURSOR = 0x0300,
        VIRTIO_GPU_CMD_MOVE_CURSOR,

        /* success responses */
        VIRTIO_GPU_RESP_OK_NODATA = 0x1100,
        VIRTIO_GPU_RESP_OK_DISPLAY_INFO,
        VIRTIO_GPU_RESP_OK_CAPSET_INFO,
        VIRTIO_GPU_RESP_OK_CAPSET,
        VIRTIO_GPU_RESP_OK_EDID,
        VIRTIO_GPU_RESP_OK_RESOURCE_UUID,
        VIRTIO_GPU_RESP_OK_MAP_INFO,

        /* error responses */
        VIRTIO_GPU_RESP_ERR_UNSPEC = 0x1200,
        VIRTIO_GPU_RESP_ERR_OUT_OF_MEMORY,
        VIRTIO_GPU_RESP_ERR_INVALID_SCANOUT_ID,
        VIRTIO_GPU_RESP_ERR_INVALID_RESOURCE_ID,
        VIRTIO_GPU_RESP_ERR_INVALID_CONTEXT_ID,
        VIRTIO_GPU_RESP_ERR_INVALID_PARAMETER,
};

#define VIRTIO_GPU_FLAG_FENCE (1 << 0)
#define VIRTIO_GPU_FLAG_INFO_RING_IDX (1 << 1)

struct virtio_gpu_ctrl_hdr {
        le32 type;
        le32 flags;
        le64 fence_id;
        le32 ctx_id;
        u8 ring_idx;
        u8 padding[3];
};
\end{lstlisting}

The fixed header \field{struct virtio_gpu_ctrl_hdr} in each
request includes the following fields:

\begin{description}
\item[\field{type}] specifies the type of the driver request
  (VIRTIO_GPU_CMD_*) or device response (VIRTIO_GPU_RESP_*).
\item[\field{flags}] request / response flags.
\item[\field{fence_id}] If the driver sets the VIRTIO_GPU_FLAG_FENCE
  bit in the request \field{flags} field the device MUST:
  \begin{itemize*}
  \item set VIRTIO_GPU_FLAG_FENCE bit in the response,
  \item copy the content of the \field{fence_id} field from the
    request to the response, and
  \item send the response only after command processing is complete.
  \end{itemize*}
\item[\field{ctx_id}] Rendering context (used in 3D mode only).
\item[\field{ring_idx}] If VIRTIO_GPU_F_CONTEXT_INIT is supported, then
  the driver MAY set VIRTIO_GPU_FLAG_INFO_RING_IDX bit in the request
  \field{flags}.  In that case:
  \begin{itemize*}
  \item \field{ring_idx} indicates the value of a context-specific ring
   index.  The minimum value is 0 and maximum value is 63 (inclusive).
  \item If VIRTIO_GPU_FLAG_FENCE is set, \field{fence_id} acts as a
   sequence number on the synchronization timeline defined by
   \field{ctx_idx} and the ring index.
  \item If VIRTIO_GPU_FLAG_FENCE is set and when the command associated
   with \field{fence_id} is complete, the device MUST send a response for
   all outstanding commands with a sequence number less than or equal to
   \field{fence_id} on the same synchronization timeline.
  \end{itemize*}
\end{description}

On success the device will return VIRTIO_GPU_RESP_OK_NODATA in
case there is no payload.  Otherwise the \field{type} field will
indicate the kind of payload.

On error the device will return one of the
VIRTIO_GPU_RESP_ERR_* error codes.

\subsubsection{Device Operation: controlq}\label{sec:Device Types / GPU Device / Device Operation / Device Operation: controlq}

For any coordinates given 0,0 is top left, larger x moves right,
larger y moves down.

\begin{description}

\item[VIRTIO_GPU_CMD_GET_DISPLAY_INFO] Retrieve the current output
  configuration.  No request data (just bare \field{struct
    virtio_gpu_ctrl_hdr}).  Response type is
  VIRTIO_GPU_RESP_OK_DISPLAY_INFO, response data is \field{struct
    virtio_gpu_resp_display_info}.

\begin{lstlisting}
#define VIRTIO_GPU_MAX_SCANOUTS 16

struct virtio_gpu_rect {
        le32 x;
        le32 y;
        le32 width;
        le32 height;
};

struct virtio_gpu_resp_display_info {
        struct virtio_gpu_ctrl_hdr hdr;
        struct virtio_gpu_display_one {
                struct virtio_gpu_rect r;
                le32 enabled;
                le32 flags;
        } pmodes[VIRTIO_GPU_MAX_SCANOUTS];
};
\end{lstlisting}

The response contains a list of per-scanout information.  The info
contains whether the scanout is enabled and what its preferred
position and size is.

The size (fields \field{width} and \field{height}) is similar to the
native panel resolution in EDID display information, except that in
the virtual machine case the size can change when the host window
representing the guest display is gets resized.

The position (fields \field{x} and \field{y}) describe how the
displays are arranged (i.e. which is -- for example -- the left
display).

The \field{enabled} field is set when the user enabled the display.
It is roughly the same as the connected state of a phyiscal display
connector.

\item[VIRTIO_GPU_CMD_GET_EDID] Retrieve the EDID data for a given
  scanout.  Request data is \field{struct virtio_gpu_get_edid}).
  Response type is VIRTIO_GPU_RESP_OK_EDID, response data is
  \field{struct virtio_gpu_resp_edid}.  Support is optional and
  negotiated using the VIRTIO_GPU_F_EDID feature flag.

\begin{lstlisting}
struct virtio_gpu_get_edid {
        struct virtio_gpu_ctrl_hdr hdr;
        le32 scanout;
        le32 padding;
};

struct virtio_gpu_resp_edid {
        struct virtio_gpu_ctrl_hdr hdr;
        le32 size;
        le32 padding;
        u8 edid[1024];
};
\end{lstlisting}

The response contains the EDID display data blob (as specified by
VESA) for the scanout.

\item[VIRTIO_GPU_CMD_RESOURCE_CREATE_2D] Create a 2D resource on the
  host.  Request data is \field{struct virtio_gpu_resource_create_2d}.
  Response type is VIRTIO_GPU_RESP_OK_NODATA.

\begin{lstlisting}
enum virtio_gpu_formats {
        VIRTIO_GPU_FORMAT_B8G8R8A8_UNORM  = 1,
        VIRTIO_GPU_FORMAT_B8G8R8X8_UNORM  = 2,
        VIRTIO_GPU_FORMAT_A8R8G8B8_UNORM  = 3,
        VIRTIO_GPU_FORMAT_X8R8G8B8_UNORM  = 4,

        VIRTIO_GPU_FORMAT_R8G8B8A8_UNORM  = 67,
        VIRTIO_GPU_FORMAT_X8B8G8R8_UNORM  = 68,

        VIRTIO_GPU_FORMAT_A8B8G8R8_UNORM  = 121,
        VIRTIO_GPU_FORMAT_R8G8B8X8_UNORM  = 134,
};

struct virtio_gpu_resource_create_2d {
        struct virtio_gpu_ctrl_hdr hdr;
        le32 resource_id;
        le32 format;
        le32 width;
        le32 height;
};
\end{lstlisting}

This creates a 2D resource on the host with the specified width,
height and format.  The resource ids are generated by the guest.

\item[VIRTIO_GPU_CMD_RESOURCE_UNREF] Destroy a resource.  Request data
  is \field{struct virtio_gpu_resource_unref}.  Response type is
  VIRTIO_GPU_RESP_OK_NODATA.

\begin{lstlisting}
struct virtio_gpu_resource_unref {
        struct virtio_gpu_ctrl_hdr hdr;
        le32 resource_id;
        le32 padding;
};
\end{lstlisting}

This informs the host that a resource is no longer required by the
guest.

\item[VIRTIO_GPU_CMD_SET_SCANOUT] Set the scanout parameters for a
  single output.  Request data is \field{struct
    virtio_gpu_set_scanout}.  Response type is
  VIRTIO_GPU_RESP_OK_NODATA.

\begin{lstlisting}
struct virtio_gpu_set_scanout {
        struct virtio_gpu_ctrl_hdr hdr;
        struct virtio_gpu_rect r;
        le32 scanout_id;
        le32 resource_id;
};
\end{lstlisting}

This sets the scanout parameters for a single scanout. The resource_id
is the resource to be scanned out from, along with a rectangle.

Scanout rectangles must be completely covered by the underlying
resource.  Overlapping (or identical) scanouts are allowed, typical
use case is screen mirroring.

The driver can use resource_id = 0 to disable a scanout.

\item[VIRTIO_GPU_CMD_RESOURCE_FLUSH] Flush a scanout resource Request
  data is \field{struct virtio_gpu_resource_flush}.  Response type is
  VIRTIO_GPU_RESP_OK_NODATA.

\begin{lstlisting}
struct virtio_gpu_resource_flush {
        struct virtio_gpu_ctrl_hdr hdr;
        struct virtio_gpu_rect r;
        le32 resource_id;
        le32 padding;
};
\end{lstlisting}

This flushes a resource to screen.  It takes a rectangle and a
resource id, and flushes any scanouts the resource is being used on.

\item[VIRTIO_GPU_CMD_TRANSFER_TO_HOST_2D] Transfer from guest memory
  to host resource.  Request data is \field{struct
    virtio_gpu_transfer_to_host_2d}.  Response type is
  VIRTIO_GPU_RESP_OK_NODATA.

\begin{lstlisting}
struct virtio_gpu_transfer_to_host_2d {
        struct virtio_gpu_ctrl_hdr hdr;
        struct virtio_gpu_rect r;
        le64 offset;
        le32 resource_id;
        le32 padding;
};
\end{lstlisting}

This takes a resource id along with an destination offset into the
resource, and a box to transfer to the host backing for the resource.

\item[VIRTIO_GPU_CMD_RESOURCE_ATTACH_BACKING] Assign backing pages to
  a resource.  Request data is \field{struct
    virtio_gpu_resource_attach_backing}, followed by \field{struct
    virtio_gpu_mem_entry} entries.  Response type is
  VIRTIO_GPU_RESP_OK_NODATA.

\begin{lstlisting}
struct virtio_gpu_resource_attach_backing {
        struct virtio_gpu_ctrl_hdr hdr;
        le32 resource_id;
        le32 nr_entries;
};

struct virtio_gpu_mem_entry {
        le64 addr;
        le32 length;
        le32 padding;
};
\end{lstlisting}

This assign an array of guest pages as the backing store for a
resource. These pages are then used for the transfer operations for
that resource from that point on.

\item[VIRTIO_GPU_CMD_RESOURCE_DETACH_BACKING] Detach backing pages
  from a resource.  Request data is \field{struct
    virtio_gpu_resource_detach_backing}.  Response type is
  VIRTIO_GPU_RESP_OK_NODATA.

\begin{lstlisting}
struct virtio_gpu_resource_detach_backing {
        struct virtio_gpu_ctrl_hdr hdr;
        le32 resource_id;
        le32 padding;
};
\end{lstlisting}

This detaches any backing pages from a resource, to be used in case of
guest swapping or object destruction.

\item[VIRTIO_GPU_CMD_GET_CAPSET_INFO] Gets the information associated with
  a particular \field{capset_index}, which MUST less than \field{num_capsets}
  defined in the device configuration.  Request data is
  \field{struct virtio_gpu_get_capset_info}.  Response type is
  VIRTIO_GPU_RESP_OK_CAPSET_INFO.

  On success, \field{struct virtio_gpu_resp_capset_info} contains the
  \field{capset_id}, \field{capset_max_version}, \field{capset_max_size}
  associated with capset at the specified {capset_idex}.  field{capset_id} MUST
  be one of the following (see listing for values):

  \begin{itemize*}
  \item \href{https://gitlab.freedesktop.org/virgl/virglrenderer/-/blob/master/src/virgl_hw.h#L526}{VIRTIO_GPU_CAPSET_VIRGL} --
	the first edition of Virgl (Gallium OpenGL) protocol.
  \item \href{https://gitlab.freedesktop.org/virgl/virglrenderer/-/blob/master/src/virgl_hw.h#L550}{VIRTIO_GPU_CAPSET_VIRGL2} --
	the second edition of Virgl (Gallium OpenGL) protocol after the capset fix.
  \item \href{https://android.googlesource.com/device/generic/vulkan-cereal/+/refs/heads/master/protocols/}{VIRTIO_GPU_CAPSET_GFXSTREAM} --
	gfxtream's (mostly) autogenerated GLES and Vulkan streaming protocols.
  \item \href{https://gitlab.freedesktop.org/olv/venus-protocol}{VIRTIO_GPU_CAPSET_VENUS} --
	Mesa's (mostly) autogenerated Vulkan protocol.
  \item \href{https://chromium.googlesource.com/chromiumos/platform/crosvm/+/refs/heads/main/rutabaga_gfx/src/cross_domain/cross_domain_protocol.rs}{VIRTIO_GPU_CAPSET_CROSS_DOMAIN} --
	protocol for display virtualization via Wayland proxying.
  \end{itemize*}

\begin{lstlisting}
struct virtio_gpu_get_capset_info {
        struct virtio_gpu_ctrl_hdr hdr;
        le32 capset_index;
        le32 padding;
};

#define VIRTIO_GPU_CAPSET_VIRGL 1
#define VIRTIO_GPU_CAPSET_VIRGL2 2
#define VIRTIO_GPU_CAPSET_GFXSTREAM 3
#define VIRTIO_GPU_CAPSET_VENUS 4
#define VIRTIO_GPU_CAPSET_CROSS_DOMAIN 5
struct virtio_gpu_resp_capset_info {
        struct virtio_gpu_ctrl_hdr hdr;
        le32 capset_id;
        le32 capset_max_version;
        le32 capset_max_size;
        le32 padding;
};
\end{lstlisting}

\item[VIRTIO_GPU_CMD_GET_CAPSET] Gets the capset associated with a
  particular \field{capset_id} and \field{capset_version}.  Request data is
  \field{struct virtio_gpu_get_capset}.  Response type is
  VIRTIO_GPU_RESP_OK_CAPSET.

\begin{lstlisting}
struct virtio_gpu_get_capset {
        struct virtio_gpu_ctrl_hdr hdr;
        le32 capset_id;
        le32 capset_version;
};

struct virtio_gpu_resp_capset {
        struct virtio_gpu_ctrl_hdr hdr;
        u8 capset_data[];
};
\end{lstlisting}

\item[VIRTIO_GPU_CMD_RESOURCE_ASSIGN_UUID] Creates an exported object from
  a resource. Request data is \field{struct
    virtio_gpu_resource_assign_uuid}.  Response type is
  VIRTIO_GPU_RESP_OK_RESOURCE_UUID, response data is \field{struct
    virtio_gpu_resp_resource_uuid}. Support is optional and negotiated
    using the VIRTIO_GPU_F_RESOURCE_UUID feature flag.

\begin{lstlisting}
struct virtio_gpu_resource_assign_uuid {
        struct virtio_gpu_ctrl_hdr hdr;
        le32 resource_id;
        le32 padding;
};

struct virtio_gpu_resp_resource_uuid {
        struct virtio_gpu_ctrl_hdr hdr;
        u8 uuid[16];
};
\end{lstlisting}

The response contains a UUID which identifies the exported object created from
the host private resource. Note that if the resource has an attached backing,
modifications made to the host private resource through the exported object by
other devices are not visible in the attached backing until they are transferred
into the backing.

\item[VIRTIO_GPU_CMD_RESOURCE_CREATE_BLOB] Creates a virtio-gpu blob
  resource. Request data is \field{struct
  virtio_gpu_resource_create_blob}, followed by \field{struct
  virtio_gpu_mem_entry} entries. Response type is
  VIRTIO_GPU_RESP_OK_NODATA. Support is optional and negotiated
  using the VIRTIO_GPU_F_RESOURCE_BLOB feature flag.

\begin{lstlisting}
#define VIRTIO_GPU_BLOB_MEM_GUEST             0x0001
#define VIRTIO_GPU_BLOB_MEM_HOST3D            0x0002
#define VIRTIO_GPU_BLOB_MEM_HOST3D_GUEST      0x0003

#define VIRTIO_GPU_BLOB_FLAG_USE_MAPPABLE     0x0001
#define VIRTIO_GPU_BLOB_FLAG_USE_SHAREABLE    0x0002
#define VIRTIO_GPU_BLOB_FLAG_USE_CROSS_DEVICE 0x0004

struct virtio_gpu_resource_create_blob {
       struct virtio_gpu_ctrl_hdr hdr;
       le32 resource_id;
       le32 blob_mem;
       le32 blob_flags;
       le32 nr_entries;
       le64 blob_id;
       le64 size;
};

\end{lstlisting}

A blob resource is a container for:

  \begin{itemize*}
  \item a guest memory allocation (referred to as a
  "guest-only blob resource").
  \item a host memory allocation (referred to as a
  "host-only blob resource").
  \item a guest memory and host memory allocation (referred
  to as a "default blob resource").
  \end{itemize*}

The memory properties of the blob resource MUST be described by
\field{blob_mem}, which MUST be non-zero.

For default and guest-only blob resources, \field{nr_entries} guest
memory entries may be assigned to the resource.  For default blob resources
(i.e, when \field{blob_mem} is VIRTIO_GPU_BLOB_MEM_HOST3D_GUEST), these
memory entries are used as a shadow buffer for the host memory. To
facilitate drivers that support swap-in and swap-out, \field{nr_entries} may
be zero and VIRTIO_GPU_CMD_RESOURCE_ATTACH_BACKING may be subsequently used.
VIRTIO_GPU_CMD_RESOURCE_DETACH_BACKING may be used to unassign memory entries.

\field{blob_mem} can only be VIRTIO_GPU_BLOB_MEM_HOST3D and
VIRTIO_GPU_BLOB_MEM_HOST3D_GUEST if VIRTIO_GPU_F_VIRGL is supported.
VIRTIO_GPU_BLOB_MEM_GUEST is valid regardless whether VIRTIO_GPU_F_VIRGL
is supported or not.

For VIRTIO_GPU_BLOB_MEM_HOST3D and VIRTIO_GPU_BLOB_MEM_HOST3D_GUEST, the
virtio-gpu resource MUST be created from the rendering context local object
identified by the \field{blob_id}. The actual allocation is done via
VIRTIO_GPU_CMD_SUBMIT_3D.

The driver MUST inform the device if the blob resource is used for
memory access, sharing between driver instances and/or sharing with
other devices. This is done via the \field{blob_flags} field.

If VIRTIO_GPU_F_VIRGL is set, both VIRTIO_GPU_CMD_TRANSFER_TO_HOST_3D
and VIRTIO_GPU_CMD_TRANSFER_FROM_HOST_3D may be used to update the
resource. There is no restriction on the image/buffer view the driver
has on the blob resource.

\item[VIRTIO_GPU_CMD_SET_SCANOUT_BLOB] sets scanout parameters for a
   blob resource. Request data is
  \field{struct virtio_gpu_set_scanout_blob}. Response type is
  VIRTIO_GPU_RESP_OK_NODATA. Support is optional and negotiated
  using the VIRTIO_GPU_F_RESOURCE_BLOB feature flag.

\begin{lstlisting}
struct virtio_gpu_set_scanout_blob {
       struct virtio_gpu_ctrl_hdr hdr;
       struct virtio_gpu_rect r;
       le32 scanout_id;
       le32 resource_id;
       le32 width;
       le32 height;
       le32 format;
       le32 padding;
       le32 strides[4];
       le32 offsets[4];
};
\end{lstlisting}

The rectangle \field{r} represents the portion of the blob resource being
displayed. The rest is the metadata associated with the blob resource. The
format MUST be one of \field{enum virtio_gpu_formats}.  The format MAY be
compressed with header and data planes.

\end{description}

\subsubsection{Device Operation: controlq (3d)}\label{sec:Device Types / GPU Device / Device Operation / Device Operation: controlq (3d)}

These commands are supported by the device if the VIRTIO_GPU_F_VIRGL
feature flag is set.

\begin{description}

\item[VIRTIO_GPU_CMD_CTX_CREATE] creates a context for submitting an opaque
  command stream.  Request data is \field{struct virtio_gpu_ctx_create}.
  Response type is VIRTIO_GPU_RESP_OK_NODATA.

\begin{lstlisting}
#define VIRTIO_GPU_CONTEXT_INIT_CAPSET_ID_MASK 0x000000ff;
struct virtio_gpu_ctx_create {
       struct virtio_gpu_ctrl_hdr hdr;
       le32 nlen;
       le32 context_init;
       char debug_name[64];
};
\end{lstlisting}

The implementation MUST create a context for the given \field{ctx_id} in
the \field{hdr}.  For debugging purposes, a \field{debug_name} and it's
length \field{nlen} is provided by the driver.  If
VIRTIO_GPU_F_CONTEXT_INIT is supported, then lower 8 bits of
\field{context_init} MAY contain the \field{capset_id} associated with
context.  In that case, then the device MUST create a context that can
handle the specified command stream.

If the lower 8-bits of the \field{context_init} are zero, then the type of
the context is determined by the device.

\item[VIRTIO_GPU_CMD_CTX_DESTROY]
\item[VIRTIO_GPU_CMD_CTX_ATTACH_RESOURCE]
\item[VIRTIO_GPU_CMD_CTX_DETACH_RESOURCE]
  Manage virtio-gpu 3d contexts.

\item[VIRTIO_GPU_CMD_RESOURCE_CREATE_3D]
  Create virtio-gpu 3d resources.

\item[VIRTIO_GPU_CMD_TRANSFER_TO_HOST_3D]
\item[VIRTIO_GPU_CMD_TRANSFER_FROM_HOST_3D]
  Transfer data from and to virtio-gpu 3d resources.

\item[VIRTIO_GPU_CMD_SUBMIT_3D]
  Submit an opaque command stream.  The type of the command stream is
  determined when creating a context.

\item[VIRTIO_GPU_CMD_RESOURCE_MAP_BLOB] maps a host-only
  blob resource into an offset in the host visible memory region. Request
  data is \field{struct virtio_gpu_resource_map_blob}.  The driver MUST
  not map a blob resource that is already mapped.  Response type is
  VIRTIO_GPU_RESP_OK_MAP_INFO. Support is optional and negotiated
  using the VIRTIO_GPU_F_RESOURCE_BLOB feature flag and checking for
  the presence of the host visible memory region.

\begin{lstlisting}
struct virtio_gpu_resource_map_blob {
        struct virtio_gpu_ctrl_hdr hdr;
        le32 resource_id;
        le32 padding;
        le64 offset;
};

#define VIRTIO_GPU_MAP_CACHE_MASK      0x0f
#define VIRTIO_GPU_MAP_CACHE_NONE      0x00
#define VIRTIO_GPU_MAP_CACHE_CACHED    0x01
#define VIRTIO_GPU_MAP_CACHE_UNCACHED  0x02
#define VIRTIO_GPU_MAP_CACHE_WC        0x03
struct virtio_gpu_resp_map_info {
        struct virtio_gpu_ctrl_hdr hdr;
        u32 map_info;
        u32 padding;
};
\end{lstlisting}

\item[VIRTIO_GPU_CMD_RESOURCE_UNMAP_BLOB] unmaps a
  host-only blob resource from the host visible memory region. Request data
  is \field{struct virtio_gpu_resource_unmap_blob}.  Response type is
  VIRTIO_GPU_RESP_OK_NODATA.  Support is optional and negotiated
  using the VIRTIO_GPU_F_RESOURCE_BLOB feature flag and checking for
  the presence of the host visible memory region.

\begin{lstlisting}
struct virtio_gpu_resource_unmap_blob {
        struct virtio_gpu_ctrl_hdr hdr;
        le32 resource_id;
        le32 padding;
};
\end{lstlisting}

\end{description}

\subsubsection{Device Operation: cursorq}\label{sec:Device Types / GPU Device / Device Operation / Device Operation: cursorq}

Both cursorq commands use the same command struct.

\begin{lstlisting}
struct virtio_gpu_cursor_pos {
        le32 scanout_id;
        le32 x;
        le32 y;
        le32 padding;
};

struct virtio_gpu_update_cursor {
        struct virtio_gpu_ctrl_hdr hdr;
        struct virtio_gpu_cursor_pos pos;
        le32 resource_id;
        le32 hot_x;
        le32 hot_y;
        le32 padding;
};
\end{lstlisting}

\begin{description}

\item[VIRTIO_GPU_CMD_UPDATE_CURSOR]
Update cursor.
Request data is \field{struct virtio_gpu_update_cursor}.
Response type is VIRTIO_GPU_RESP_OK_NODATA.

Full cursor update.  Cursor will be loaded from the specified
\field{resource_id} and will be moved to \field{pos}.  The driver must
transfer the cursor into the resource beforehand (using control queue
commands) and make sure the commands to fill the resource are actually
processed (using fencing).

\item[VIRTIO_GPU_CMD_MOVE_CURSOR]
Move cursor.
Request data is \field{struct virtio_gpu_update_cursor}.
Response type is VIRTIO_GPU_RESP_OK_NODATA.

Move cursor to the place specified in \field{pos}.  The other fields
are not used and will be ignored by the device.

\end{description}

\subsection{VGA Compatibility}\label{sec:Device Types / GPU Device / VGA Compatibility}

Applies to Virtio Over PCI only.  The GPU device can come with and
without VGA compatibility.  The PCI class should be DISPLAY_VGA if VGA
compatibility is present and DISPLAY_OTHER otherwise.

VGA compatibility: PCI region 0 has the linear framebuffer, standard
vga registers are present.  Configuring a scanout
(VIRTIO_GPU_CMD_SET_SCANOUT) switches the device from vga
compatibility mode into native virtio mode.  A reset switches it back
into vga compatibility mode.

Note: qemu implementation also provides bochs dispi interface io ports
and mmio bar at pci region 1 and is therefore fully compatible with
the qemu stdvga (see \href{https://git.qemu-project.org/?p=qemu.git;a=blob;f=docs/specs/standard-vga.txt;hb=HEAD}{docs/specs/standard-vga.txt} in the qemu source tree).

\section{GPU Device}\label{sec:Device Types / GPU Device}

virtio-gpu is a virtio based graphics adapter.  It can operate in 2D
mode and in 3D mode.  3D mode will offload rendering ops to
the host gpu and therefore requires a gpu with 3D support on the host
machine.

In 2D mode the virtio-gpu device provides support for ARGB Hardware
cursors and multiple scanouts (aka heads).

\subsection{Device ID}\label{sec:Device Types / GPU Device / Device ID}

16

\subsection{Virtqueues}\label{sec:Device Types / GPU Device / Virtqueues}

\begin{description}
\item[0] controlq - queue for sending control commands
\item[1] cursorq - queue for sending cursor updates
\end{description}

Both queues have the same format.  Each request and each response have
a fixed header, followed by command specific data fields.  The
separate cursor queue is the "fast track" for cursor commands
(VIRTIO_GPU_CMD_UPDATE_CURSOR and VIRTIO_GPU_CMD_MOVE_CURSOR), so they
go through without being delayed by time-consuming commands in the
control queue.

\subsection{Feature bits}\label{sec:Device Types / GPU Device / Feature bits}

\begin{description}
\item[VIRTIO_GPU_F_VIRGL (0)] virgl 3D mode is supported.
\item[VIRTIO_GPU_F_EDID  (1)] EDID is supported.
\item[VIRTIO_GPU_F_RESOURCE_UUID (2)] assigning resources UUIDs for export
  to other virtio devices is supported.
\item[VIRTIO_GPU_F_RESOURCE_BLOB (3)] creating and using size-based blob
  resources is supported.
\item[VIRTIO_GPU_F_CONTEXT_INIT (4)] multiple context types and
  synchronization timelines supported.  Requires VIRTIO_GPU_F_VIRGL.
\end{description}

\subsection{Device configuration layout}\label{sec:Device Types / GPU Device / Device configuration layout}

GPU device configuration uses the following layout structure and
definitions:

\begin{lstlisting}
#define VIRTIO_GPU_EVENT_DISPLAY (1 << 0)

struct virtio_gpu_config {
        le32 events_read;
        le32 events_clear;
        le32 num_scanouts;
        le32 num_capsets;
};
\end{lstlisting}

\subsubsection{Device configuration fields}

\begin{description}
\item[\field{events_read}] signals pending events to the driver.  The
  driver MUST NOT write to this field.
\item[\field{events_clear}] clears pending events in the device.
  Writing a '1' into a bit will clear the corresponding bit in
  \field{events_read}, mimicking write-to-clear behavior.
\item[\field{num_scanouts}] specifies the maximum number of scanouts
  supported by the device.  Minimum value is 1, maximum value is 16.
\item[\field{num_capsets}] specifies the maximum number of capability
  sets supported by the device.  The minimum value is zero.
\end{description}

\subsubsection{Events}

\begin{description}
\item[VIRTIO_GPU_EVENT_DISPLAY] Display configuration has changed.
  The driver SHOULD use the VIRTIO_GPU_CMD_GET_DISPLAY_INFO command to
  fetch the information from the device.  In case EDID support is
  negotiated (VIRTIO_GPU_F_EDID feature flag) the device SHOULD also
  fetch the updated EDID blobs using the VIRTIO_GPU_CMD_GET_EDID
  command.
\end{description}

\devicenormative{\subsection}{Device Initialization}{Device Types / GPU Device / Device Initialization}

The driver SHOULD query the display information from the device using
the VIRTIO_GPU_CMD_GET_DISPLAY_INFO command and use that information
for the initial scanout setup.  In case EDID support is negotiated
(VIRTIO_GPU_F_EDID feature flag) the device SHOULD also fetch the EDID
information using the VIRTIO_GPU_CMD_GET_EDID command.  If no
information is available or all displays are disabled the driver MAY
choose to use a fallback, such as 1024x768 at display 0.

The driver SHOULD query all shared memory regions supported by the device.
If the device supports shared memory, the \field{shmid} of a region MUST
(see \ref{sec:Basic Facilities of a Virtio Device /
Shared Memory Regions}~\nameref{sec:Basic Facilities of a Virtio Device /
Shared Memory Regions}) be one of the following:

\begin{lstlisting}
enum virtio_gpu_shm_id {
        VIRTIO_GPU_SHM_ID_UNDEFINED = 0,
        VIRTIO_GPU_SHM_ID_HOST_VISIBLE = 1,
};
\end{lstlisting}

The shared memory region with VIRTIO_GPU_SHM_ID_HOST_VISIBLE is referred as
the "host visible memory region".  The device MUST support the
VIRTIO_GPU_CMD_RESOURCE_MAP_BLOB and VIRTIO_GPU_CMD_RESOURCE_UNMAP_BLOB
if the host visible memory region is available.

\subsection{Device Operation}\label{sec:Device Types / GPU Device / Device Operation}

The virtio-gpu is based around the concept of resources private to the
host.  The guest must DMA transfer into these resources, unless shared memory
regions are supported. This is a design requirement in order to interface with
future 3D rendering. In the unaccelerated 2D mode there is no support for DMA
transfers from resources, just to them.

Resources are initially simple 2D resources, consisting of a width,
height and format along with an identifier. The guest must then attach
backing store to the resources in order for DMA transfers to
work. This is like a GART in a real GPU.

\subsubsection{Device Operation: Create a framebuffer and configure scanout}

\begin{itemize*}
\item Create a host resource using VIRTIO_GPU_CMD_RESOURCE_CREATE_2D.
\item Allocate a framebuffer from guest ram, and attach it as backing
  storage to the resource just created, using
  VIRTIO_GPU_CMD_RESOURCE_ATTACH_BACKING.  Scatter lists are
  supported, so the framebuffer doesn't need to be contignous in guest
  physical memory.
\item Use VIRTIO_GPU_CMD_SET_SCANOUT to link the framebuffer to
  a display scanout.
\end{itemize*}

\subsubsection{Device Operation: Update a framebuffer and scanout}

\begin{itemize*}
\item Render to your framebuffer memory.
\item Use VIRTIO_GPU_CMD_TRANSFER_TO_HOST_2D to update the host resource
  from guest memory.
\item Use VIRTIO_GPU_CMD_RESOURCE_FLUSH to flush the updated resource
  to the display.
\end{itemize*}

\subsubsection{Device Operation: Using pageflip}

It is possible to create multiple framebuffers, flip between them
using VIRTIO_GPU_CMD_SET_SCANOUT and VIRTIO_GPU_CMD_RESOURCE_FLUSH,
and update the invisible framebuffer using
VIRTIO_GPU_CMD_TRANSFER_TO_HOST_2D.

\subsubsection{Device Operation: Multihead setup}

In case two or more displays are present there are different ways to
configure things:

\begin{itemize*}
\item Create a single framebuffer, link it to all displays
  (mirroring).
\item Create an framebuffer for each display.
\item Create one big framebuffer, configure scanouts to display a
  different rectangle of that framebuffer each.
\end{itemize*}

\devicenormative{\subsubsection}{Device Operation: Command lifecycle and fencing}{Device Types / GPU Device / Device Operation / Device Operation: Command lifecycle and fencing}

The device MAY process controlq commands asyncronously and return them
to the driver before the processing is complete.  If the driver needs
to know when the processing is finished it can set the
VIRTIO_GPU_FLAG_FENCE flag in the request.  The device MUST finish the
processing before returning the command then.

Note: current qemu implementation does asyncrounous processing only in
3d mode, when offloading the processing to the host gpu.

\subsubsection{Device Operation: Configure mouse cursor}

The mouse cursor image is a normal resource, except that it must be
64x64 in size.  The driver MUST create and populate the resource
(using the usual VIRTIO_GPU_CMD_RESOURCE_CREATE_2D,
VIRTIO_GPU_CMD_RESOURCE_ATTACH_BACKING and
VIRTIO_GPU_CMD_TRANSFER_TO_HOST_2D controlq commands) and make sure they
are completed (using VIRTIO_GPU_FLAG_FENCE).

Then VIRTIO_GPU_CMD_UPDATE_CURSOR can be sent to the cursorq to set
the pointer shape and position.  To move the pointer without updating
the shape use VIRTIO_GPU_CMD_MOVE_CURSOR instead.

\subsubsection{Device Operation: Request header}\label{sec:Device Types / GPU Device / Device Operation / Device Operation: Request header}

All requests and responses on the virtqueues have a fixed header
using the following layout structure and definitions:

\begin{lstlisting}
enum virtio_gpu_ctrl_type {

        /* 2d commands */
        VIRTIO_GPU_CMD_GET_DISPLAY_INFO = 0x0100,
        VIRTIO_GPU_CMD_RESOURCE_CREATE_2D,
        VIRTIO_GPU_CMD_RESOURCE_UNREF,
        VIRTIO_GPU_CMD_SET_SCANOUT,
        VIRTIO_GPU_CMD_RESOURCE_FLUSH,
        VIRTIO_GPU_CMD_TRANSFER_TO_HOST_2D,
        VIRTIO_GPU_CMD_RESOURCE_ATTACH_BACKING,
        VIRTIO_GPU_CMD_RESOURCE_DETACH_BACKING,
        VIRTIO_GPU_CMD_GET_CAPSET_INFO,
        VIRTIO_GPU_CMD_GET_CAPSET,
        VIRTIO_GPU_CMD_GET_EDID,
        VIRTIO_GPU_CMD_RESOURCE_ASSIGN_UUID,
        VIRTIO_GPU_CMD_RESOURCE_CREATE_BLOB,
        VIRTIO_GPU_CMD_SET_SCANOUT_BLOB,

        /* 3d commands */
        VIRTIO_GPU_CMD_CTX_CREATE = 0x0200,
        VIRTIO_GPU_CMD_CTX_DESTROY,
        VIRTIO_GPU_CMD_CTX_ATTACH_RESOURCE,
        VIRTIO_GPU_CMD_CTX_DETACH_RESOURCE,
        VIRTIO_GPU_CMD_RESOURCE_CREATE_3D,
        VIRTIO_GPU_CMD_TRANSFER_TO_HOST_3D,
        VIRTIO_GPU_CMD_TRANSFER_FROM_HOST_3D,
        VIRTIO_GPU_CMD_SUBMIT_3D,
        VIRTIO_GPU_CMD_RESOURCE_MAP_BLOB,
        VIRTIO_GPU_CMD_RESOURCE_UNMAP_BLOB,

        /* cursor commands */
        VIRTIO_GPU_CMD_UPDATE_CURSOR = 0x0300,
        VIRTIO_GPU_CMD_MOVE_CURSOR,

        /* success responses */
        VIRTIO_GPU_RESP_OK_NODATA = 0x1100,
        VIRTIO_GPU_RESP_OK_DISPLAY_INFO,
        VIRTIO_GPU_RESP_OK_CAPSET_INFO,
        VIRTIO_GPU_RESP_OK_CAPSET,
        VIRTIO_GPU_RESP_OK_EDID,
        VIRTIO_GPU_RESP_OK_RESOURCE_UUID,
        VIRTIO_GPU_RESP_OK_MAP_INFO,

        /* error responses */
        VIRTIO_GPU_RESP_ERR_UNSPEC = 0x1200,
        VIRTIO_GPU_RESP_ERR_OUT_OF_MEMORY,
        VIRTIO_GPU_RESP_ERR_INVALID_SCANOUT_ID,
        VIRTIO_GPU_RESP_ERR_INVALID_RESOURCE_ID,
        VIRTIO_GPU_RESP_ERR_INVALID_CONTEXT_ID,
        VIRTIO_GPU_RESP_ERR_INVALID_PARAMETER,
};

#define VIRTIO_GPU_FLAG_FENCE (1 << 0)
#define VIRTIO_GPU_FLAG_INFO_RING_IDX (1 << 1)

struct virtio_gpu_ctrl_hdr {
        le32 type;
        le32 flags;
        le64 fence_id;
        le32 ctx_id;
        u8 ring_idx;
        u8 padding[3];
};
\end{lstlisting}

The fixed header \field{struct virtio_gpu_ctrl_hdr} in each
request includes the following fields:

\begin{description}
\item[\field{type}] specifies the type of the driver request
  (VIRTIO_GPU_CMD_*) or device response (VIRTIO_GPU_RESP_*).
\item[\field{flags}] request / response flags.
\item[\field{fence_id}] If the driver sets the VIRTIO_GPU_FLAG_FENCE
  bit in the request \field{flags} field the device MUST:
  \begin{itemize*}
  \item set VIRTIO_GPU_FLAG_FENCE bit in the response,
  \item copy the content of the \field{fence_id} field from the
    request to the response, and
  \item send the response only after command processing is complete.
  \end{itemize*}
\item[\field{ctx_id}] Rendering context (used in 3D mode only).
\item[\field{ring_idx}] If VIRTIO_GPU_F_CONTEXT_INIT is supported, then
  the driver MAY set VIRTIO_GPU_FLAG_INFO_RING_IDX bit in the request
  \field{flags}.  In that case:
  \begin{itemize*}
  \item \field{ring_idx} indicates the value of a context-specific ring
   index.  The minimum value is 0 and maximum value is 63 (inclusive).
  \item If VIRTIO_GPU_FLAG_FENCE is set, \field{fence_id} acts as a
   sequence number on the synchronization timeline defined by
   \field{ctx_idx} and the ring index.
  \item If VIRTIO_GPU_FLAG_FENCE is set and when the command associated
   with \field{fence_id} is complete, the device MUST send a response for
   all outstanding commands with a sequence number less than or equal to
   \field{fence_id} on the same synchronization timeline.
  \end{itemize*}
\end{description}

On success the device will return VIRTIO_GPU_RESP_OK_NODATA in
case there is no payload.  Otherwise the \field{type} field will
indicate the kind of payload.

On error the device will return one of the
VIRTIO_GPU_RESP_ERR_* error codes.

\subsubsection{Device Operation: controlq}\label{sec:Device Types / GPU Device / Device Operation / Device Operation: controlq}

For any coordinates given 0,0 is top left, larger x moves right,
larger y moves down.

\begin{description}

\item[VIRTIO_GPU_CMD_GET_DISPLAY_INFO] Retrieve the current output
  configuration.  No request data (just bare \field{struct
    virtio_gpu_ctrl_hdr}).  Response type is
  VIRTIO_GPU_RESP_OK_DISPLAY_INFO, response data is \field{struct
    virtio_gpu_resp_display_info}.

\begin{lstlisting}
#define VIRTIO_GPU_MAX_SCANOUTS 16

struct virtio_gpu_rect {
        le32 x;
        le32 y;
        le32 width;
        le32 height;
};

struct virtio_gpu_resp_display_info {
        struct virtio_gpu_ctrl_hdr hdr;
        struct virtio_gpu_display_one {
                struct virtio_gpu_rect r;
                le32 enabled;
                le32 flags;
        } pmodes[VIRTIO_GPU_MAX_SCANOUTS];
};
\end{lstlisting}

The response contains a list of per-scanout information.  The info
contains whether the scanout is enabled and what its preferred
position and size is.

The size (fields \field{width} and \field{height}) is similar to the
native panel resolution in EDID display information, except that in
the virtual machine case the size can change when the host window
representing the guest display is gets resized.

The position (fields \field{x} and \field{y}) describe how the
displays are arranged (i.e. which is -- for example -- the left
display).

The \field{enabled} field is set when the user enabled the display.
It is roughly the same as the connected state of a phyiscal display
connector.

\item[VIRTIO_GPU_CMD_GET_EDID] Retrieve the EDID data for a given
  scanout.  Request data is \field{struct virtio_gpu_get_edid}).
  Response type is VIRTIO_GPU_RESP_OK_EDID, response data is
  \field{struct virtio_gpu_resp_edid}.  Support is optional and
  negotiated using the VIRTIO_GPU_F_EDID feature flag.

\begin{lstlisting}
struct virtio_gpu_get_edid {
        struct virtio_gpu_ctrl_hdr hdr;
        le32 scanout;
        le32 padding;
};

struct virtio_gpu_resp_edid {
        struct virtio_gpu_ctrl_hdr hdr;
        le32 size;
        le32 padding;
        u8 edid[1024];
};
\end{lstlisting}

The response contains the EDID display data blob (as specified by
VESA) for the scanout.

\item[VIRTIO_GPU_CMD_RESOURCE_CREATE_2D] Create a 2D resource on the
  host.  Request data is \field{struct virtio_gpu_resource_create_2d}.
  Response type is VIRTIO_GPU_RESP_OK_NODATA.

\begin{lstlisting}
enum virtio_gpu_formats {
        VIRTIO_GPU_FORMAT_B8G8R8A8_UNORM  = 1,
        VIRTIO_GPU_FORMAT_B8G8R8X8_UNORM  = 2,
        VIRTIO_GPU_FORMAT_A8R8G8B8_UNORM  = 3,
        VIRTIO_GPU_FORMAT_X8R8G8B8_UNORM  = 4,

        VIRTIO_GPU_FORMAT_R8G8B8A8_UNORM  = 67,
        VIRTIO_GPU_FORMAT_X8B8G8R8_UNORM  = 68,

        VIRTIO_GPU_FORMAT_A8B8G8R8_UNORM  = 121,
        VIRTIO_GPU_FORMAT_R8G8B8X8_UNORM  = 134,
};

struct virtio_gpu_resource_create_2d {
        struct virtio_gpu_ctrl_hdr hdr;
        le32 resource_id;
        le32 format;
        le32 width;
        le32 height;
};
\end{lstlisting}

This creates a 2D resource on the host with the specified width,
height and format.  The resource ids are generated by the guest.

\item[VIRTIO_GPU_CMD_RESOURCE_UNREF] Destroy a resource.  Request data
  is \field{struct virtio_gpu_resource_unref}.  Response type is
  VIRTIO_GPU_RESP_OK_NODATA.

\begin{lstlisting}
struct virtio_gpu_resource_unref {
        struct virtio_gpu_ctrl_hdr hdr;
        le32 resource_id;
        le32 padding;
};
\end{lstlisting}

This informs the host that a resource is no longer required by the
guest.

\item[VIRTIO_GPU_CMD_SET_SCANOUT] Set the scanout parameters for a
  single output.  Request data is \field{struct
    virtio_gpu_set_scanout}.  Response type is
  VIRTIO_GPU_RESP_OK_NODATA.

\begin{lstlisting}
struct virtio_gpu_set_scanout {
        struct virtio_gpu_ctrl_hdr hdr;
        struct virtio_gpu_rect r;
        le32 scanout_id;
        le32 resource_id;
};
\end{lstlisting}

This sets the scanout parameters for a single scanout. The resource_id
is the resource to be scanned out from, along with a rectangle.

Scanout rectangles must be completely covered by the underlying
resource.  Overlapping (or identical) scanouts are allowed, typical
use case is screen mirroring.

The driver can use resource_id = 0 to disable a scanout.

\item[VIRTIO_GPU_CMD_RESOURCE_FLUSH] Flush a scanout resource Request
  data is \field{struct virtio_gpu_resource_flush}.  Response type is
  VIRTIO_GPU_RESP_OK_NODATA.

\begin{lstlisting}
struct virtio_gpu_resource_flush {
        struct virtio_gpu_ctrl_hdr hdr;
        struct virtio_gpu_rect r;
        le32 resource_id;
        le32 padding;
};
\end{lstlisting}

This flushes a resource to screen.  It takes a rectangle and a
resource id, and flushes any scanouts the resource is being used on.

\item[VIRTIO_GPU_CMD_TRANSFER_TO_HOST_2D] Transfer from guest memory
  to host resource.  Request data is \field{struct
    virtio_gpu_transfer_to_host_2d}.  Response type is
  VIRTIO_GPU_RESP_OK_NODATA.

\begin{lstlisting}
struct virtio_gpu_transfer_to_host_2d {
        struct virtio_gpu_ctrl_hdr hdr;
        struct virtio_gpu_rect r;
        le64 offset;
        le32 resource_id;
        le32 padding;
};
\end{lstlisting}

This takes a resource id along with an destination offset into the
resource, and a box to transfer to the host backing for the resource.

\item[VIRTIO_GPU_CMD_RESOURCE_ATTACH_BACKING] Assign backing pages to
  a resource.  Request data is \field{struct
    virtio_gpu_resource_attach_backing}, followed by \field{struct
    virtio_gpu_mem_entry} entries.  Response type is
  VIRTIO_GPU_RESP_OK_NODATA.

\begin{lstlisting}
struct virtio_gpu_resource_attach_backing {
        struct virtio_gpu_ctrl_hdr hdr;
        le32 resource_id;
        le32 nr_entries;
};

struct virtio_gpu_mem_entry {
        le64 addr;
        le32 length;
        le32 padding;
};
\end{lstlisting}

This assign an array of guest pages as the backing store for a
resource. These pages are then used for the transfer operations for
that resource from that point on.

\item[VIRTIO_GPU_CMD_RESOURCE_DETACH_BACKING] Detach backing pages
  from a resource.  Request data is \field{struct
    virtio_gpu_resource_detach_backing}.  Response type is
  VIRTIO_GPU_RESP_OK_NODATA.

\begin{lstlisting}
struct virtio_gpu_resource_detach_backing {
        struct virtio_gpu_ctrl_hdr hdr;
        le32 resource_id;
        le32 padding;
};
\end{lstlisting}

This detaches any backing pages from a resource, to be used in case of
guest swapping or object destruction.

\item[VIRTIO_GPU_CMD_GET_CAPSET_INFO] Gets the information associated with
  a particular \field{capset_index}, which MUST less than \field{num_capsets}
  defined in the device configuration.  Request data is
  \field{struct virtio_gpu_get_capset_info}.  Response type is
  VIRTIO_GPU_RESP_OK_CAPSET_INFO.

  On success, \field{struct virtio_gpu_resp_capset_info} contains the
  \field{capset_id}, \field{capset_max_version}, \field{capset_max_size}
  associated with capset at the specified {capset_idex}.  field{capset_id} MUST
  be one of the following (see listing for values):

  \begin{itemize*}
  \item \href{https://gitlab.freedesktop.org/virgl/virglrenderer/-/blob/master/src/virgl_hw.h#L526}{VIRTIO_GPU_CAPSET_VIRGL} --
	the first edition of Virgl (Gallium OpenGL) protocol.
  \item \href{https://gitlab.freedesktop.org/virgl/virglrenderer/-/blob/master/src/virgl_hw.h#L550}{VIRTIO_GPU_CAPSET_VIRGL2} --
	the second edition of Virgl (Gallium OpenGL) protocol after the capset fix.
  \item \href{https://android.googlesource.com/device/generic/vulkan-cereal/+/refs/heads/master/protocols/}{VIRTIO_GPU_CAPSET_GFXSTREAM} --
	gfxtream's (mostly) autogenerated GLES and Vulkan streaming protocols.
  \item \href{https://gitlab.freedesktop.org/olv/venus-protocol}{VIRTIO_GPU_CAPSET_VENUS} --
	Mesa's (mostly) autogenerated Vulkan protocol.
  \item \href{https://chromium.googlesource.com/chromiumos/platform/crosvm/+/refs/heads/main/rutabaga_gfx/src/cross_domain/cross_domain_protocol.rs}{VIRTIO_GPU_CAPSET_CROSS_DOMAIN} --
	protocol for display virtualization via Wayland proxying.
  \end{itemize*}

\begin{lstlisting}
struct virtio_gpu_get_capset_info {
        struct virtio_gpu_ctrl_hdr hdr;
        le32 capset_index;
        le32 padding;
};

#define VIRTIO_GPU_CAPSET_VIRGL 1
#define VIRTIO_GPU_CAPSET_VIRGL2 2
#define VIRTIO_GPU_CAPSET_GFXSTREAM 3
#define VIRTIO_GPU_CAPSET_VENUS 4
#define VIRTIO_GPU_CAPSET_CROSS_DOMAIN 5
struct virtio_gpu_resp_capset_info {
        struct virtio_gpu_ctrl_hdr hdr;
        le32 capset_id;
        le32 capset_max_version;
        le32 capset_max_size;
        le32 padding;
};
\end{lstlisting}

\item[VIRTIO_GPU_CMD_GET_CAPSET] Gets the capset associated with a
  particular \field{capset_id} and \field{capset_version}.  Request data is
  \field{struct virtio_gpu_get_capset}.  Response type is
  VIRTIO_GPU_RESP_OK_CAPSET.

\begin{lstlisting}
struct virtio_gpu_get_capset {
        struct virtio_gpu_ctrl_hdr hdr;
        le32 capset_id;
        le32 capset_version;
};

struct virtio_gpu_resp_capset {
        struct virtio_gpu_ctrl_hdr hdr;
        u8 capset_data[];
};
\end{lstlisting}

\item[VIRTIO_GPU_CMD_RESOURCE_ASSIGN_UUID] Creates an exported object from
  a resource. Request data is \field{struct
    virtio_gpu_resource_assign_uuid}.  Response type is
  VIRTIO_GPU_RESP_OK_RESOURCE_UUID, response data is \field{struct
    virtio_gpu_resp_resource_uuid}. Support is optional and negotiated
    using the VIRTIO_GPU_F_RESOURCE_UUID feature flag.

\begin{lstlisting}
struct virtio_gpu_resource_assign_uuid {
        struct virtio_gpu_ctrl_hdr hdr;
        le32 resource_id;
        le32 padding;
};

struct virtio_gpu_resp_resource_uuid {
        struct virtio_gpu_ctrl_hdr hdr;
        u8 uuid[16];
};
\end{lstlisting}

The response contains a UUID which identifies the exported object created from
the host private resource. Note that if the resource has an attached backing,
modifications made to the host private resource through the exported object by
other devices are not visible in the attached backing until they are transferred
into the backing.

\item[VIRTIO_GPU_CMD_RESOURCE_CREATE_BLOB] Creates a virtio-gpu blob
  resource. Request data is \field{struct
  virtio_gpu_resource_create_blob}, followed by \field{struct
  virtio_gpu_mem_entry} entries. Response type is
  VIRTIO_GPU_RESP_OK_NODATA. Support is optional and negotiated
  using the VIRTIO_GPU_F_RESOURCE_BLOB feature flag.

\begin{lstlisting}
#define VIRTIO_GPU_BLOB_MEM_GUEST             0x0001
#define VIRTIO_GPU_BLOB_MEM_HOST3D            0x0002
#define VIRTIO_GPU_BLOB_MEM_HOST3D_GUEST      0x0003

#define VIRTIO_GPU_BLOB_FLAG_USE_MAPPABLE     0x0001
#define VIRTIO_GPU_BLOB_FLAG_USE_SHAREABLE    0x0002
#define VIRTIO_GPU_BLOB_FLAG_USE_CROSS_DEVICE 0x0004

struct virtio_gpu_resource_create_blob {
       struct virtio_gpu_ctrl_hdr hdr;
       le32 resource_id;
       le32 blob_mem;
       le32 blob_flags;
       le32 nr_entries;
       le64 blob_id;
       le64 size;
};

\end{lstlisting}

A blob resource is a container for:

  \begin{itemize*}
  \item a guest memory allocation (referred to as a
  "guest-only blob resource").
  \item a host memory allocation (referred to as a
  "host-only blob resource").
  \item a guest memory and host memory allocation (referred
  to as a "default blob resource").
  \end{itemize*}

The memory properties of the blob resource MUST be described by
\field{blob_mem}, which MUST be non-zero.

For default and guest-only blob resources, \field{nr_entries} guest
memory entries may be assigned to the resource.  For default blob resources
(i.e, when \field{blob_mem} is VIRTIO_GPU_BLOB_MEM_HOST3D_GUEST), these
memory entries are used as a shadow buffer for the host memory. To
facilitate drivers that support swap-in and swap-out, \field{nr_entries} may
be zero and VIRTIO_GPU_CMD_RESOURCE_ATTACH_BACKING may be subsequently used.
VIRTIO_GPU_CMD_RESOURCE_DETACH_BACKING may be used to unassign memory entries.

\field{blob_mem} can only be VIRTIO_GPU_BLOB_MEM_HOST3D and
VIRTIO_GPU_BLOB_MEM_HOST3D_GUEST if VIRTIO_GPU_F_VIRGL is supported.
VIRTIO_GPU_BLOB_MEM_GUEST is valid regardless whether VIRTIO_GPU_F_VIRGL
is supported or not.

For VIRTIO_GPU_BLOB_MEM_HOST3D and VIRTIO_GPU_BLOB_MEM_HOST3D_GUEST, the
virtio-gpu resource MUST be created from the rendering context local object
identified by the \field{blob_id}. The actual allocation is done via
VIRTIO_GPU_CMD_SUBMIT_3D.

The driver MUST inform the device if the blob resource is used for
memory access, sharing between driver instances and/or sharing with
other devices. This is done via the \field{blob_flags} field.

If VIRTIO_GPU_F_VIRGL is set, both VIRTIO_GPU_CMD_TRANSFER_TO_HOST_3D
and VIRTIO_GPU_CMD_TRANSFER_FROM_HOST_3D may be used to update the
resource. There is no restriction on the image/buffer view the driver
has on the blob resource.

\item[VIRTIO_GPU_CMD_SET_SCANOUT_BLOB] sets scanout parameters for a
   blob resource. Request data is
  \field{struct virtio_gpu_set_scanout_blob}. Response type is
  VIRTIO_GPU_RESP_OK_NODATA. Support is optional and negotiated
  using the VIRTIO_GPU_F_RESOURCE_BLOB feature flag.

\begin{lstlisting}
struct virtio_gpu_set_scanout_blob {
       struct virtio_gpu_ctrl_hdr hdr;
       struct virtio_gpu_rect r;
       le32 scanout_id;
       le32 resource_id;
       le32 width;
       le32 height;
       le32 format;
       le32 padding;
       le32 strides[4];
       le32 offsets[4];
};
\end{lstlisting}

The rectangle \field{r} represents the portion of the blob resource being
displayed. The rest is the metadata associated with the blob resource. The
format MUST be one of \field{enum virtio_gpu_formats}.  The format MAY be
compressed with header and data planes.

\end{description}

\subsubsection{Device Operation: controlq (3d)}\label{sec:Device Types / GPU Device / Device Operation / Device Operation: controlq (3d)}

These commands are supported by the device if the VIRTIO_GPU_F_VIRGL
feature flag is set.

\begin{description}

\item[VIRTIO_GPU_CMD_CTX_CREATE] creates a context for submitting an opaque
  command stream.  Request data is \field{struct virtio_gpu_ctx_create}.
  Response type is VIRTIO_GPU_RESP_OK_NODATA.

\begin{lstlisting}
#define VIRTIO_GPU_CONTEXT_INIT_CAPSET_ID_MASK 0x000000ff;
struct virtio_gpu_ctx_create {
       struct virtio_gpu_ctrl_hdr hdr;
       le32 nlen;
       le32 context_init;
       char debug_name[64];
};
\end{lstlisting}

The implementation MUST create a context for the given \field{ctx_id} in
the \field{hdr}.  For debugging purposes, a \field{debug_name} and it's
length \field{nlen} is provided by the driver.  If
VIRTIO_GPU_F_CONTEXT_INIT is supported, then lower 8 bits of
\field{context_init} MAY contain the \field{capset_id} associated with
context.  In that case, then the device MUST create a context that can
handle the specified command stream.

If the lower 8-bits of the \field{context_init} are zero, then the type of
the context is determined by the device.

\item[VIRTIO_GPU_CMD_CTX_DESTROY]
\item[VIRTIO_GPU_CMD_CTX_ATTACH_RESOURCE]
\item[VIRTIO_GPU_CMD_CTX_DETACH_RESOURCE]
  Manage virtio-gpu 3d contexts.

\item[VIRTIO_GPU_CMD_RESOURCE_CREATE_3D]
  Create virtio-gpu 3d resources.

\item[VIRTIO_GPU_CMD_TRANSFER_TO_HOST_3D]
\item[VIRTIO_GPU_CMD_TRANSFER_FROM_HOST_3D]
  Transfer data from and to virtio-gpu 3d resources.

\item[VIRTIO_GPU_CMD_SUBMIT_3D]
  Submit an opaque command stream.  The type of the command stream is
  determined when creating a context.

\item[VIRTIO_GPU_CMD_RESOURCE_MAP_BLOB] maps a host-only
  blob resource into an offset in the host visible memory region. Request
  data is \field{struct virtio_gpu_resource_map_blob}.  The driver MUST
  not map a blob resource that is already mapped.  Response type is
  VIRTIO_GPU_RESP_OK_MAP_INFO. Support is optional and negotiated
  using the VIRTIO_GPU_F_RESOURCE_BLOB feature flag and checking for
  the presence of the host visible memory region.

\begin{lstlisting}
struct virtio_gpu_resource_map_blob {
        struct virtio_gpu_ctrl_hdr hdr;
        le32 resource_id;
        le32 padding;
        le64 offset;
};

#define VIRTIO_GPU_MAP_CACHE_MASK      0x0f
#define VIRTIO_GPU_MAP_CACHE_NONE      0x00
#define VIRTIO_GPU_MAP_CACHE_CACHED    0x01
#define VIRTIO_GPU_MAP_CACHE_UNCACHED  0x02
#define VIRTIO_GPU_MAP_CACHE_WC        0x03
struct virtio_gpu_resp_map_info {
        struct virtio_gpu_ctrl_hdr hdr;
        u32 map_info;
        u32 padding;
};
\end{lstlisting}

\item[VIRTIO_GPU_CMD_RESOURCE_UNMAP_BLOB] unmaps a
  host-only blob resource from the host visible memory region. Request data
  is \field{struct virtio_gpu_resource_unmap_blob}.  Response type is
  VIRTIO_GPU_RESP_OK_NODATA.  Support is optional and negotiated
  using the VIRTIO_GPU_F_RESOURCE_BLOB feature flag and checking for
  the presence of the host visible memory region.

\begin{lstlisting}
struct virtio_gpu_resource_unmap_blob {
        struct virtio_gpu_ctrl_hdr hdr;
        le32 resource_id;
        le32 padding;
};
\end{lstlisting}

\end{description}

\subsubsection{Device Operation: cursorq}\label{sec:Device Types / GPU Device / Device Operation / Device Operation: cursorq}

Both cursorq commands use the same command struct.

\begin{lstlisting}
struct virtio_gpu_cursor_pos {
        le32 scanout_id;
        le32 x;
        le32 y;
        le32 padding;
};

struct virtio_gpu_update_cursor {
        struct virtio_gpu_ctrl_hdr hdr;
        struct virtio_gpu_cursor_pos pos;
        le32 resource_id;
        le32 hot_x;
        le32 hot_y;
        le32 padding;
};
\end{lstlisting}

\begin{description}

\item[VIRTIO_GPU_CMD_UPDATE_CURSOR]
Update cursor.
Request data is \field{struct virtio_gpu_update_cursor}.
Response type is VIRTIO_GPU_RESP_OK_NODATA.

Full cursor update.  Cursor will be loaded from the specified
\field{resource_id} and will be moved to \field{pos}.  The driver must
transfer the cursor into the resource beforehand (using control queue
commands) and make sure the commands to fill the resource are actually
processed (using fencing).

\item[VIRTIO_GPU_CMD_MOVE_CURSOR]
Move cursor.
Request data is \field{struct virtio_gpu_update_cursor}.
Response type is VIRTIO_GPU_RESP_OK_NODATA.

Move cursor to the place specified in \field{pos}.  The other fields
are not used and will be ignored by the device.

\end{description}

\subsection{VGA Compatibility}\label{sec:Device Types / GPU Device / VGA Compatibility}

Applies to Virtio Over PCI only.  The GPU device can come with and
without VGA compatibility.  The PCI class should be DISPLAY_VGA if VGA
compatibility is present and DISPLAY_OTHER otherwise.

VGA compatibility: PCI region 0 has the linear framebuffer, standard
vga registers are present.  Configuring a scanout
(VIRTIO_GPU_CMD_SET_SCANOUT) switches the device from vga
compatibility mode into native virtio mode.  A reset switches it back
into vga compatibility mode.

Note: qemu implementation also provides bochs dispi interface io ports
and mmio bar at pci region 1 and is therefore fully compatible with
the qemu stdvga (see \href{https://git.qemu-project.org/?p=qemu.git;a=blob;f=docs/specs/standard-vga.txt;hb=HEAD}{docs/specs/standard-vga.txt} in the qemu source tree).

\section{GPU Device}\label{sec:Device Types / GPU Device}

virtio-gpu is a virtio based graphics adapter.  It can operate in 2D
mode and in 3D mode.  3D mode will offload rendering ops to
the host gpu and therefore requires a gpu with 3D support on the host
machine.

In 2D mode the virtio-gpu device provides support for ARGB Hardware
cursors and multiple scanouts (aka heads).

\subsection{Device ID}\label{sec:Device Types / GPU Device / Device ID}

16

\subsection{Virtqueues}\label{sec:Device Types / GPU Device / Virtqueues}

\begin{description}
\item[0] controlq - queue for sending control commands
\item[1] cursorq - queue for sending cursor updates
\end{description}

Both queues have the same format.  Each request and each response have
a fixed header, followed by command specific data fields.  The
separate cursor queue is the "fast track" for cursor commands
(VIRTIO_GPU_CMD_UPDATE_CURSOR and VIRTIO_GPU_CMD_MOVE_CURSOR), so they
go through without being delayed by time-consuming commands in the
control queue.

\subsection{Feature bits}\label{sec:Device Types / GPU Device / Feature bits}

\begin{description}
\item[VIRTIO_GPU_F_VIRGL (0)] virgl 3D mode is supported.
\item[VIRTIO_GPU_F_EDID  (1)] EDID is supported.
\item[VIRTIO_GPU_F_RESOURCE_UUID (2)] assigning resources UUIDs for export
  to other virtio devices is supported.
\item[VIRTIO_GPU_F_RESOURCE_BLOB (3)] creating and using size-based blob
  resources is supported.
\item[VIRTIO_GPU_F_CONTEXT_INIT (4)] multiple context types and
  synchronization timelines supported.  Requires VIRTIO_GPU_F_VIRGL.
\end{description}

\subsection{Device configuration layout}\label{sec:Device Types / GPU Device / Device configuration layout}

GPU device configuration uses the following layout structure and
definitions:

\begin{lstlisting}
#define VIRTIO_GPU_EVENT_DISPLAY (1 << 0)

struct virtio_gpu_config {
        le32 events_read;
        le32 events_clear;
        le32 num_scanouts;
        le32 num_capsets;
};
\end{lstlisting}

\subsubsection{Device configuration fields}

\begin{description}
\item[\field{events_read}] signals pending events to the driver.  The
  driver MUST NOT write to this field.
\item[\field{events_clear}] clears pending events in the device.
  Writing a '1' into a bit will clear the corresponding bit in
  \field{events_read}, mimicking write-to-clear behavior.
\item[\field{num_scanouts}] specifies the maximum number of scanouts
  supported by the device.  Minimum value is 1, maximum value is 16.
\item[\field{num_capsets}] specifies the maximum number of capability
  sets supported by the device.  The minimum value is zero.
\end{description}

\subsubsection{Events}

\begin{description}
\item[VIRTIO_GPU_EVENT_DISPLAY] Display configuration has changed.
  The driver SHOULD use the VIRTIO_GPU_CMD_GET_DISPLAY_INFO command to
  fetch the information from the device.  In case EDID support is
  negotiated (VIRTIO_GPU_F_EDID feature flag) the device SHOULD also
  fetch the updated EDID blobs using the VIRTIO_GPU_CMD_GET_EDID
  command.
\end{description}

\devicenormative{\subsection}{Device Initialization}{Device Types / GPU Device / Device Initialization}

The driver SHOULD query the display information from the device using
the VIRTIO_GPU_CMD_GET_DISPLAY_INFO command and use that information
for the initial scanout setup.  In case EDID support is negotiated
(VIRTIO_GPU_F_EDID feature flag) the device SHOULD also fetch the EDID
information using the VIRTIO_GPU_CMD_GET_EDID command.  If no
information is available or all displays are disabled the driver MAY
choose to use a fallback, such as 1024x768 at display 0.

The driver SHOULD query all shared memory regions supported by the device.
If the device supports shared memory, the \field{shmid} of a region MUST
(see \ref{sec:Basic Facilities of a Virtio Device /
Shared Memory Regions}~\nameref{sec:Basic Facilities of a Virtio Device /
Shared Memory Regions}) be one of the following:

\begin{lstlisting}
enum virtio_gpu_shm_id {
        VIRTIO_GPU_SHM_ID_UNDEFINED = 0,
        VIRTIO_GPU_SHM_ID_HOST_VISIBLE = 1,
};
\end{lstlisting}

The shared memory region with VIRTIO_GPU_SHM_ID_HOST_VISIBLE is referred as
the "host visible memory region".  The device MUST support the
VIRTIO_GPU_CMD_RESOURCE_MAP_BLOB and VIRTIO_GPU_CMD_RESOURCE_UNMAP_BLOB
if the host visible memory region is available.

\subsection{Device Operation}\label{sec:Device Types / GPU Device / Device Operation}

The virtio-gpu is based around the concept of resources private to the
host.  The guest must DMA transfer into these resources, unless shared memory
regions are supported. This is a design requirement in order to interface with
future 3D rendering. In the unaccelerated 2D mode there is no support for DMA
transfers from resources, just to them.

Resources are initially simple 2D resources, consisting of a width,
height and format along with an identifier. The guest must then attach
backing store to the resources in order for DMA transfers to
work. This is like a GART in a real GPU.

\subsubsection{Device Operation: Create a framebuffer and configure scanout}

\begin{itemize*}
\item Create a host resource using VIRTIO_GPU_CMD_RESOURCE_CREATE_2D.
\item Allocate a framebuffer from guest ram, and attach it as backing
  storage to the resource just created, using
  VIRTIO_GPU_CMD_RESOURCE_ATTACH_BACKING.  Scatter lists are
  supported, so the framebuffer doesn't need to be contignous in guest
  physical memory.
\item Use VIRTIO_GPU_CMD_SET_SCANOUT to link the framebuffer to
  a display scanout.
\end{itemize*}

\subsubsection{Device Operation: Update a framebuffer and scanout}

\begin{itemize*}
\item Render to your framebuffer memory.
\item Use VIRTIO_GPU_CMD_TRANSFER_TO_HOST_2D to update the host resource
  from guest memory.
\item Use VIRTIO_GPU_CMD_RESOURCE_FLUSH to flush the updated resource
  to the display.
\end{itemize*}

\subsubsection{Device Operation: Using pageflip}

It is possible to create multiple framebuffers, flip between them
using VIRTIO_GPU_CMD_SET_SCANOUT and VIRTIO_GPU_CMD_RESOURCE_FLUSH,
and update the invisible framebuffer using
VIRTIO_GPU_CMD_TRANSFER_TO_HOST_2D.

\subsubsection{Device Operation: Multihead setup}

In case two or more displays are present there are different ways to
configure things:

\begin{itemize*}
\item Create a single framebuffer, link it to all displays
  (mirroring).
\item Create an framebuffer for each display.
\item Create one big framebuffer, configure scanouts to display a
  different rectangle of that framebuffer each.
\end{itemize*}

\devicenormative{\subsubsection}{Device Operation: Command lifecycle and fencing}{Device Types / GPU Device / Device Operation / Device Operation: Command lifecycle and fencing}

The device MAY process controlq commands asyncronously and return them
to the driver before the processing is complete.  If the driver needs
to know when the processing is finished it can set the
VIRTIO_GPU_FLAG_FENCE flag in the request.  The device MUST finish the
processing before returning the command then.

Note: current qemu implementation does asyncrounous processing only in
3d mode, when offloading the processing to the host gpu.

\subsubsection{Device Operation: Configure mouse cursor}

The mouse cursor image is a normal resource, except that it must be
64x64 in size.  The driver MUST create and populate the resource
(using the usual VIRTIO_GPU_CMD_RESOURCE_CREATE_2D,
VIRTIO_GPU_CMD_RESOURCE_ATTACH_BACKING and
VIRTIO_GPU_CMD_TRANSFER_TO_HOST_2D controlq commands) and make sure they
are completed (using VIRTIO_GPU_FLAG_FENCE).

Then VIRTIO_GPU_CMD_UPDATE_CURSOR can be sent to the cursorq to set
the pointer shape and position.  To move the pointer without updating
the shape use VIRTIO_GPU_CMD_MOVE_CURSOR instead.

\subsubsection{Device Operation: Request header}\label{sec:Device Types / GPU Device / Device Operation / Device Operation: Request header}

All requests and responses on the virtqueues have a fixed header
using the following layout structure and definitions:

\begin{lstlisting}
enum virtio_gpu_ctrl_type {

        /* 2d commands */
        VIRTIO_GPU_CMD_GET_DISPLAY_INFO = 0x0100,
        VIRTIO_GPU_CMD_RESOURCE_CREATE_2D,
        VIRTIO_GPU_CMD_RESOURCE_UNREF,
        VIRTIO_GPU_CMD_SET_SCANOUT,
        VIRTIO_GPU_CMD_RESOURCE_FLUSH,
        VIRTIO_GPU_CMD_TRANSFER_TO_HOST_2D,
        VIRTIO_GPU_CMD_RESOURCE_ATTACH_BACKING,
        VIRTIO_GPU_CMD_RESOURCE_DETACH_BACKING,
        VIRTIO_GPU_CMD_GET_CAPSET_INFO,
        VIRTIO_GPU_CMD_GET_CAPSET,
        VIRTIO_GPU_CMD_GET_EDID,
        VIRTIO_GPU_CMD_RESOURCE_ASSIGN_UUID,
        VIRTIO_GPU_CMD_RESOURCE_CREATE_BLOB,
        VIRTIO_GPU_CMD_SET_SCANOUT_BLOB,

        /* 3d commands */
        VIRTIO_GPU_CMD_CTX_CREATE = 0x0200,
        VIRTIO_GPU_CMD_CTX_DESTROY,
        VIRTIO_GPU_CMD_CTX_ATTACH_RESOURCE,
        VIRTIO_GPU_CMD_CTX_DETACH_RESOURCE,
        VIRTIO_GPU_CMD_RESOURCE_CREATE_3D,
        VIRTIO_GPU_CMD_TRANSFER_TO_HOST_3D,
        VIRTIO_GPU_CMD_TRANSFER_FROM_HOST_3D,
        VIRTIO_GPU_CMD_SUBMIT_3D,
        VIRTIO_GPU_CMD_RESOURCE_MAP_BLOB,
        VIRTIO_GPU_CMD_RESOURCE_UNMAP_BLOB,

        /* cursor commands */
        VIRTIO_GPU_CMD_UPDATE_CURSOR = 0x0300,
        VIRTIO_GPU_CMD_MOVE_CURSOR,

        /* success responses */
        VIRTIO_GPU_RESP_OK_NODATA = 0x1100,
        VIRTIO_GPU_RESP_OK_DISPLAY_INFO,
        VIRTIO_GPU_RESP_OK_CAPSET_INFO,
        VIRTIO_GPU_RESP_OK_CAPSET,
        VIRTIO_GPU_RESP_OK_EDID,
        VIRTIO_GPU_RESP_OK_RESOURCE_UUID,
        VIRTIO_GPU_RESP_OK_MAP_INFO,

        /* error responses */
        VIRTIO_GPU_RESP_ERR_UNSPEC = 0x1200,
        VIRTIO_GPU_RESP_ERR_OUT_OF_MEMORY,
        VIRTIO_GPU_RESP_ERR_INVALID_SCANOUT_ID,
        VIRTIO_GPU_RESP_ERR_INVALID_RESOURCE_ID,
        VIRTIO_GPU_RESP_ERR_INVALID_CONTEXT_ID,
        VIRTIO_GPU_RESP_ERR_INVALID_PARAMETER,
};

#define VIRTIO_GPU_FLAG_FENCE (1 << 0)
#define VIRTIO_GPU_FLAG_INFO_RING_IDX (1 << 1)

struct virtio_gpu_ctrl_hdr {
        le32 type;
        le32 flags;
        le64 fence_id;
        le32 ctx_id;
        u8 ring_idx;
        u8 padding[3];
};
\end{lstlisting}

The fixed header \field{struct virtio_gpu_ctrl_hdr} in each
request includes the following fields:

\begin{description}
\item[\field{type}] specifies the type of the driver request
  (VIRTIO_GPU_CMD_*) or device response (VIRTIO_GPU_RESP_*).
\item[\field{flags}] request / response flags.
\item[\field{fence_id}] If the driver sets the VIRTIO_GPU_FLAG_FENCE
  bit in the request \field{flags} field the device MUST:
  \begin{itemize*}
  \item set VIRTIO_GPU_FLAG_FENCE bit in the response,
  \item copy the content of the \field{fence_id} field from the
    request to the response, and
  \item send the response only after command processing is complete.
  \end{itemize*}
\item[\field{ctx_id}] Rendering context (used in 3D mode only).
\item[\field{ring_idx}] If VIRTIO_GPU_F_CONTEXT_INIT is supported, then
  the driver MAY set VIRTIO_GPU_FLAG_INFO_RING_IDX bit in the request
  \field{flags}.  In that case:
  \begin{itemize*}
  \item \field{ring_idx} indicates the value of a context-specific ring
   index.  The minimum value is 0 and maximum value is 63 (inclusive).
  \item If VIRTIO_GPU_FLAG_FENCE is set, \field{fence_id} acts as a
   sequence number on the synchronization timeline defined by
   \field{ctx_idx} and the ring index.
  \item If VIRTIO_GPU_FLAG_FENCE is set and when the command associated
   with \field{fence_id} is complete, the device MUST send a response for
   all outstanding commands with a sequence number less than or equal to
   \field{fence_id} on the same synchronization timeline.
  \end{itemize*}
\end{description}

On success the device will return VIRTIO_GPU_RESP_OK_NODATA in
case there is no payload.  Otherwise the \field{type} field will
indicate the kind of payload.

On error the device will return one of the
VIRTIO_GPU_RESP_ERR_* error codes.

\subsubsection{Device Operation: controlq}\label{sec:Device Types / GPU Device / Device Operation / Device Operation: controlq}

For any coordinates given 0,0 is top left, larger x moves right,
larger y moves down.

\begin{description}

\item[VIRTIO_GPU_CMD_GET_DISPLAY_INFO] Retrieve the current output
  configuration.  No request data (just bare \field{struct
    virtio_gpu_ctrl_hdr}).  Response type is
  VIRTIO_GPU_RESP_OK_DISPLAY_INFO, response data is \field{struct
    virtio_gpu_resp_display_info}.

\begin{lstlisting}
#define VIRTIO_GPU_MAX_SCANOUTS 16

struct virtio_gpu_rect {
        le32 x;
        le32 y;
        le32 width;
        le32 height;
};

struct virtio_gpu_resp_display_info {
        struct virtio_gpu_ctrl_hdr hdr;
        struct virtio_gpu_display_one {
                struct virtio_gpu_rect r;
                le32 enabled;
                le32 flags;
        } pmodes[VIRTIO_GPU_MAX_SCANOUTS];
};
\end{lstlisting}

The response contains a list of per-scanout information.  The info
contains whether the scanout is enabled and what its preferred
position and size is.

The size (fields \field{width} and \field{height}) is similar to the
native panel resolution in EDID display information, except that in
the virtual machine case the size can change when the host window
representing the guest display is gets resized.

The position (fields \field{x} and \field{y}) describe how the
displays are arranged (i.e. which is -- for example -- the left
display).

The \field{enabled} field is set when the user enabled the display.
It is roughly the same as the connected state of a phyiscal display
connector.

\item[VIRTIO_GPU_CMD_GET_EDID] Retrieve the EDID data for a given
  scanout.  Request data is \field{struct virtio_gpu_get_edid}).
  Response type is VIRTIO_GPU_RESP_OK_EDID, response data is
  \field{struct virtio_gpu_resp_edid}.  Support is optional and
  negotiated using the VIRTIO_GPU_F_EDID feature flag.

\begin{lstlisting}
struct virtio_gpu_get_edid {
        struct virtio_gpu_ctrl_hdr hdr;
        le32 scanout;
        le32 padding;
};

struct virtio_gpu_resp_edid {
        struct virtio_gpu_ctrl_hdr hdr;
        le32 size;
        le32 padding;
        u8 edid[1024];
};
\end{lstlisting}

The response contains the EDID display data blob (as specified by
VESA) for the scanout.

\item[VIRTIO_GPU_CMD_RESOURCE_CREATE_2D] Create a 2D resource on the
  host.  Request data is \field{struct virtio_gpu_resource_create_2d}.
  Response type is VIRTIO_GPU_RESP_OK_NODATA.

\begin{lstlisting}
enum virtio_gpu_formats {
        VIRTIO_GPU_FORMAT_B8G8R8A8_UNORM  = 1,
        VIRTIO_GPU_FORMAT_B8G8R8X8_UNORM  = 2,
        VIRTIO_GPU_FORMAT_A8R8G8B8_UNORM  = 3,
        VIRTIO_GPU_FORMAT_X8R8G8B8_UNORM  = 4,

        VIRTIO_GPU_FORMAT_R8G8B8A8_UNORM  = 67,
        VIRTIO_GPU_FORMAT_X8B8G8R8_UNORM  = 68,

        VIRTIO_GPU_FORMAT_A8B8G8R8_UNORM  = 121,
        VIRTIO_GPU_FORMAT_R8G8B8X8_UNORM  = 134,
};

struct virtio_gpu_resource_create_2d {
        struct virtio_gpu_ctrl_hdr hdr;
        le32 resource_id;
        le32 format;
        le32 width;
        le32 height;
};
\end{lstlisting}

This creates a 2D resource on the host with the specified width,
height and format.  The resource ids are generated by the guest.

\item[VIRTIO_GPU_CMD_RESOURCE_UNREF] Destroy a resource.  Request data
  is \field{struct virtio_gpu_resource_unref}.  Response type is
  VIRTIO_GPU_RESP_OK_NODATA.

\begin{lstlisting}
struct virtio_gpu_resource_unref {
        struct virtio_gpu_ctrl_hdr hdr;
        le32 resource_id;
        le32 padding;
};
\end{lstlisting}

This informs the host that a resource is no longer required by the
guest.

\item[VIRTIO_GPU_CMD_SET_SCANOUT] Set the scanout parameters for a
  single output.  Request data is \field{struct
    virtio_gpu_set_scanout}.  Response type is
  VIRTIO_GPU_RESP_OK_NODATA.

\begin{lstlisting}
struct virtio_gpu_set_scanout {
        struct virtio_gpu_ctrl_hdr hdr;
        struct virtio_gpu_rect r;
        le32 scanout_id;
        le32 resource_id;
};
\end{lstlisting}

This sets the scanout parameters for a single scanout. The resource_id
is the resource to be scanned out from, along with a rectangle.

Scanout rectangles must be completely covered by the underlying
resource.  Overlapping (or identical) scanouts are allowed, typical
use case is screen mirroring.

The driver can use resource_id = 0 to disable a scanout.

\item[VIRTIO_GPU_CMD_RESOURCE_FLUSH] Flush a scanout resource Request
  data is \field{struct virtio_gpu_resource_flush}.  Response type is
  VIRTIO_GPU_RESP_OK_NODATA.

\begin{lstlisting}
struct virtio_gpu_resource_flush {
        struct virtio_gpu_ctrl_hdr hdr;
        struct virtio_gpu_rect r;
        le32 resource_id;
        le32 padding;
};
\end{lstlisting}

This flushes a resource to screen.  It takes a rectangle and a
resource id, and flushes any scanouts the resource is being used on.

\item[VIRTIO_GPU_CMD_TRANSFER_TO_HOST_2D] Transfer from guest memory
  to host resource.  Request data is \field{struct
    virtio_gpu_transfer_to_host_2d}.  Response type is
  VIRTIO_GPU_RESP_OK_NODATA.

\begin{lstlisting}
struct virtio_gpu_transfer_to_host_2d {
        struct virtio_gpu_ctrl_hdr hdr;
        struct virtio_gpu_rect r;
        le64 offset;
        le32 resource_id;
        le32 padding;
};
\end{lstlisting}

This takes a resource id along with an destination offset into the
resource, and a box to transfer to the host backing for the resource.

\item[VIRTIO_GPU_CMD_RESOURCE_ATTACH_BACKING] Assign backing pages to
  a resource.  Request data is \field{struct
    virtio_gpu_resource_attach_backing}, followed by \field{struct
    virtio_gpu_mem_entry} entries.  Response type is
  VIRTIO_GPU_RESP_OK_NODATA.

\begin{lstlisting}
struct virtio_gpu_resource_attach_backing {
        struct virtio_gpu_ctrl_hdr hdr;
        le32 resource_id;
        le32 nr_entries;
};

struct virtio_gpu_mem_entry {
        le64 addr;
        le32 length;
        le32 padding;
};
\end{lstlisting}

This assign an array of guest pages as the backing store for a
resource. These pages are then used for the transfer operations for
that resource from that point on.

\item[VIRTIO_GPU_CMD_RESOURCE_DETACH_BACKING] Detach backing pages
  from a resource.  Request data is \field{struct
    virtio_gpu_resource_detach_backing}.  Response type is
  VIRTIO_GPU_RESP_OK_NODATA.

\begin{lstlisting}
struct virtio_gpu_resource_detach_backing {
        struct virtio_gpu_ctrl_hdr hdr;
        le32 resource_id;
        le32 padding;
};
\end{lstlisting}

This detaches any backing pages from a resource, to be used in case of
guest swapping or object destruction.

\item[VIRTIO_GPU_CMD_GET_CAPSET_INFO] Gets the information associated with
  a particular \field{capset_index}, which MUST less than \field{num_capsets}
  defined in the device configuration.  Request data is
  \field{struct virtio_gpu_get_capset_info}.  Response type is
  VIRTIO_GPU_RESP_OK_CAPSET_INFO.

  On success, \field{struct virtio_gpu_resp_capset_info} contains the
  \field{capset_id}, \field{capset_max_version}, \field{capset_max_size}
  associated with capset at the specified {capset_idex}.  field{capset_id} MUST
  be one of the following (see listing for values):

  \begin{itemize*}
  \item \href{https://gitlab.freedesktop.org/virgl/virglrenderer/-/blob/master/src/virgl_hw.h#L526}{VIRTIO_GPU_CAPSET_VIRGL} --
	the first edition of Virgl (Gallium OpenGL) protocol.
  \item \href{https://gitlab.freedesktop.org/virgl/virglrenderer/-/blob/master/src/virgl_hw.h#L550}{VIRTIO_GPU_CAPSET_VIRGL2} --
	the second edition of Virgl (Gallium OpenGL) protocol after the capset fix.
  \item \href{https://android.googlesource.com/device/generic/vulkan-cereal/+/refs/heads/master/protocols/}{VIRTIO_GPU_CAPSET_GFXSTREAM} --
	gfxtream's (mostly) autogenerated GLES and Vulkan streaming protocols.
  \item \href{https://gitlab.freedesktop.org/olv/venus-protocol}{VIRTIO_GPU_CAPSET_VENUS} --
	Mesa's (mostly) autogenerated Vulkan protocol.
  \item \href{https://chromium.googlesource.com/chromiumos/platform/crosvm/+/refs/heads/main/rutabaga_gfx/src/cross_domain/cross_domain_protocol.rs}{VIRTIO_GPU_CAPSET_CROSS_DOMAIN} --
	protocol for display virtualization via Wayland proxying.
  \end{itemize*}

\begin{lstlisting}
struct virtio_gpu_get_capset_info {
        struct virtio_gpu_ctrl_hdr hdr;
        le32 capset_index;
        le32 padding;
};

#define VIRTIO_GPU_CAPSET_VIRGL 1
#define VIRTIO_GPU_CAPSET_VIRGL2 2
#define VIRTIO_GPU_CAPSET_GFXSTREAM 3
#define VIRTIO_GPU_CAPSET_VENUS 4
#define VIRTIO_GPU_CAPSET_CROSS_DOMAIN 5
struct virtio_gpu_resp_capset_info {
        struct virtio_gpu_ctrl_hdr hdr;
        le32 capset_id;
        le32 capset_max_version;
        le32 capset_max_size;
        le32 padding;
};
\end{lstlisting}

\item[VIRTIO_GPU_CMD_GET_CAPSET] Gets the capset associated with a
  particular \field{capset_id} and \field{capset_version}.  Request data is
  \field{struct virtio_gpu_get_capset}.  Response type is
  VIRTIO_GPU_RESP_OK_CAPSET.

\begin{lstlisting}
struct virtio_gpu_get_capset {
        struct virtio_gpu_ctrl_hdr hdr;
        le32 capset_id;
        le32 capset_version;
};

struct virtio_gpu_resp_capset {
        struct virtio_gpu_ctrl_hdr hdr;
        u8 capset_data[];
};
\end{lstlisting}

\item[VIRTIO_GPU_CMD_RESOURCE_ASSIGN_UUID] Creates an exported object from
  a resource. Request data is \field{struct
    virtio_gpu_resource_assign_uuid}.  Response type is
  VIRTIO_GPU_RESP_OK_RESOURCE_UUID, response data is \field{struct
    virtio_gpu_resp_resource_uuid}. Support is optional and negotiated
    using the VIRTIO_GPU_F_RESOURCE_UUID feature flag.

\begin{lstlisting}
struct virtio_gpu_resource_assign_uuid {
        struct virtio_gpu_ctrl_hdr hdr;
        le32 resource_id;
        le32 padding;
};

struct virtio_gpu_resp_resource_uuid {
        struct virtio_gpu_ctrl_hdr hdr;
        u8 uuid[16];
};
\end{lstlisting}

The response contains a UUID which identifies the exported object created from
the host private resource. Note that if the resource has an attached backing,
modifications made to the host private resource through the exported object by
other devices are not visible in the attached backing until they are transferred
into the backing.

\item[VIRTIO_GPU_CMD_RESOURCE_CREATE_BLOB] Creates a virtio-gpu blob
  resource. Request data is \field{struct
  virtio_gpu_resource_create_blob}, followed by \field{struct
  virtio_gpu_mem_entry} entries. Response type is
  VIRTIO_GPU_RESP_OK_NODATA. Support is optional and negotiated
  using the VIRTIO_GPU_F_RESOURCE_BLOB feature flag.

\begin{lstlisting}
#define VIRTIO_GPU_BLOB_MEM_GUEST             0x0001
#define VIRTIO_GPU_BLOB_MEM_HOST3D            0x0002
#define VIRTIO_GPU_BLOB_MEM_HOST3D_GUEST      0x0003

#define VIRTIO_GPU_BLOB_FLAG_USE_MAPPABLE     0x0001
#define VIRTIO_GPU_BLOB_FLAG_USE_SHAREABLE    0x0002
#define VIRTIO_GPU_BLOB_FLAG_USE_CROSS_DEVICE 0x0004

struct virtio_gpu_resource_create_blob {
       struct virtio_gpu_ctrl_hdr hdr;
       le32 resource_id;
       le32 blob_mem;
       le32 blob_flags;
       le32 nr_entries;
       le64 blob_id;
       le64 size;
};

\end{lstlisting}

A blob resource is a container for:

  \begin{itemize*}
  \item a guest memory allocation (referred to as a
  "guest-only blob resource").
  \item a host memory allocation (referred to as a
  "host-only blob resource").
  \item a guest memory and host memory allocation (referred
  to as a "default blob resource").
  \end{itemize*}

The memory properties of the blob resource MUST be described by
\field{blob_mem}, which MUST be non-zero.

For default and guest-only blob resources, \field{nr_entries} guest
memory entries may be assigned to the resource.  For default blob resources
(i.e, when \field{blob_mem} is VIRTIO_GPU_BLOB_MEM_HOST3D_GUEST), these
memory entries are used as a shadow buffer for the host memory. To
facilitate drivers that support swap-in and swap-out, \field{nr_entries} may
be zero and VIRTIO_GPU_CMD_RESOURCE_ATTACH_BACKING may be subsequently used.
VIRTIO_GPU_CMD_RESOURCE_DETACH_BACKING may be used to unassign memory entries.

\field{blob_mem} can only be VIRTIO_GPU_BLOB_MEM_HOST3D and
VIRTIO_GPU_BLOB_MEM_HOST3D_GUEST if VIRTIO_GPU_F_VIRGL is supported.
VIRTIO_GPU_BLOB_MEM_GUEST is valid regardless whether VIRTIO_GPU_F_VIRGL
is supported or not.

For VIRTIO_GPU_BLOB_MEM_HOST3D and VIRTIO_GPU_BLOB_MEM_HOST3D_GUEST, the
virtio-gpu resource MUST be created from the rendering context local object
identified by the \field{blob_id}. The actual allocation is done via
VIRTIO_GPU_CMD_SUBMIT_3D.

The driver MUST inform the device if the blob resource is used for
memory access, sharing between driver instances and/or sharing with
other devices. This is done via the \field{blob_flags} field.

If VIRTIO_GPU_F_VIRGL is set, both VIRTIO_GPU_CMD_TRANSFER_TO_HOST_3D
and VIRTIO_GPU_CMD_TRANSFER_FROM_HOST_3D may be used to update the
resource. There is no restriction on the image/buffer view the driver
has on the blob resource.

\item[VIRTIO_GPU_CMD_SET_SCANOUT_BLOB] sets scanout parameters for a
   blob resource. Request data is
  \field{struct virtio_gpu_set_scanout_blob}. Response type is
  VIRTIO_GPU_RESP_OK_NODATA. Support is optional and negotiated
  using the VIRTIO_GPU_F_RESOURCE_BLOB feature flag.

\begin{lstlisting}
struct virtio_gpu_set_scanout_blob {
       struct virtio_gpu_ctrl_hdr hdr;
       struct virtio_gpu_rect r;
       le32 scanout_id;
       le32 resource_id;
       le32 width;
       le32 height;
       le32 format;
       le32 padding;
       le32 strides[4];
       le32 offsets[4];
};
\end{lstlisting}

The rectangle \field{r} represents the portion of the blob resource being
displayed. The rest is the metadata associated with the blob resource. The
format MUST be one of \field{enum virtio_gpu_formats}.  The format MAY be
compressed with header and data planes.

\end{description}

\subsubsection{Device Operation: controlq (3d)}\label{sec:Device Types / GPU Device / Device Operation / Device Operation: controlq (3d)}

These commands are supported by the device if the VIRTIO_GPU_F_VIRGL
feature flag is set.

\begin{description}

\item[VIRTIO_GPU_CMD_CTX_CREATE] creates a context for submitting an opaque
  command stream.  Request data is \field{struct virtio_gpu_ctx_create}.
  Response type is VIRTIO_GPU_RESP_OK_NODATA.

\begin{lstlisting}
#define VIRTIO_GPU_CONTEXT_INIT_CAPSET_ID_MASK 0x000000ff;
struct virtio_gpu_ctx_create {
       struct virtio_gpu_ctrl_hdr hdr;
       le32 nlen;
       le32 context_init;
       char debug_name[64];
};
\end{lstlisting}

The implementation MUST create a context for the given \field{ctx_id} in
the \field{hdr}.  For debugging purposes, a \field{debug_name} and it's
length \field{nlen} is provided by the driver.  If
VIRTIO_GPU_F_CONTEXT_INIT is supported, then lower 8 bits of
\field{context_init} MAY contain the \field{capset_id} associated with
context.  In that case, then the device MUST create a context that can
handle the specified command stream.

If the lower 8-bits of the \field{context_init} are zero, then the type of
the context is determined by the device.

\item[VIRTIO_GPU_CMD_CTX_DESTROY]
\item[VIRTIO_GPU_CMD_CTX_ATTACH_RESOURCE]
\item[VIRTIO_GPU_CMD_CTX_DETACH_RESOURCE]
  Manage virtio-gpu 3d contexts.

\item[VIRTIO_GPU_CMD_RESOURCE_CREATE_3D]
  Create virtio-gpu 3d resources.

\item[VIRTIO_GPU_CMD_TRANSFER_TO_HOST_3D]
\item[VIRTIO_GPU_CMD_TRANSFER_FROM_HOST_3D]
  Transfer data from and to virtio-gpu 3d resources.

\item[VIRTIO_GPU_CMD_SUBMIT_3D]
  Submit an opaque command stream.  The type of the command stream is
  determined when creating a context.

\item[VIRTIO_GPU_CMD_RESOURCE_MAP_BLOB] maps a host-only
  blob resource into an offset in the host visible memory region. Request
  data is \field{struct virtio_gpu_resource_map_blob}.  The driver MUST
  not map a blob resource that is already mapped.  Response type is
  VIRTIO_GPU_RESP_OK_MAP_INFO. Support is optional and negotiated
  using the VIRTIO_GPU_F_RESOURCE_BLOB feature flag and checking for
  the presence of the host visible memory region.

\begin{lstlisting}
struct virtio_gpu_resource_map_blob {
        struct virtio_gpu_ctrl_hdr hdr;
        le32 resource_id;
        le32 padding;
        le64 offset;
};

#define VIRTIO_GPU_MAP_CACHE_MASK      0x0f
#define VIRTIO_GPU_MAP_CACHE_NONE      0x00
#define VIRTIO_GPU_MAP_CACHE_CACHED    0x01
#define VIRTIO_GPU_MAP_CACHE_UNCACHED  0x02
#define VIRTIO_GPU_MAP_CACHE_WC        0x03
struct virtio_gpu_resp_map_info {
        struct virtio_gpu_ctrl_hdr hdr;
        u32 map_info;
        u32 padding;
};
\end{lstlisting}

\item[VIRTIO_GPU_CMD_RESOURCE_UNMAP_BLOB] unmaps a
  host-only blob resource from the host visible memory region. Request data
  is \field{struct virtio_gpu_resource_unmap_blob}.  Response type is
  VIRTIO_GPU_RESP_OK_NODATA.  Support is optional and negotiated
  using the VIRTIO_GPU_F_RESOURCE_BLOB feature flag and checking for
  the presence of the host visible memory region.

\begin{lstlisting}
struct virtio_gpu_resource_unmap_blob {
        struct virtio_gpu_ctrl_hdr hdr;
        le32 resource_id;
        le32 padding;
};
\end{lstlisting}

\end{description}

\subsubsection{Device Operation: cursorq}\label{sec:Device Types / GPU Device / Device Operation / Device Operation: cursorq}

Both cursorq commands use the same command struct.

\begin{lstlisting}
struct virtio_gpu_cursor_pos {
        le32 scanout_id;
        le32 x;
        le32 y;
        le32 padding;
};

struct virtio_gpu_update_cursor {
        struct virtio_gpu_ctrl_hdr hdr;
        struct virtio_gpu_cursor_pos pos;
        le32 resource_id;
        le32 hot_x;
        le32 hot_y;
        le32 padding;
};
\end{lstlisting}

\begin{description}

\item[VIRTIO_GPU_CMD_UPDATE_CURSOR]
Update cursor.
Request data is \field{struct virtio_gpu_update_cursor}.
Response type is VIRTIO_GPU_RESP_OK_NODATA.

Full cursor update.  Cursor will be loaded from the specified
\field{resource_id} and will be moved to \field{pos}.  The driver must
transfer the cursor into the resource beforehand (using control queue
commands) and make sure the commands to fill the resource are actually
processed (using fencing).

\item[VIRTIO_GPU_CMD_MOVE_CURSOR]
Move cursor.
Request data is \field{struct virtio_gpu_update_cursor}.
Response type is VIRTIO_GPU_RESP_OK_NODATA.

Move cursor to the place specified in \field{pos}.  The other fields
are not used and will be ignored by the device.

\end{description}

\subsection{VGA Compatibility}\label{sec:Device Types / GPU Device / VGA Compatibility}

Applies to Virtio Over PCI only.  The GPU device can come with and
without VGA compatibility.  The PCI class should be DISPLAY_VGA if VGA
compatibility is present and DISPLAY_OTHER otherwise.

VGA compatibility: PCI region 0 has the linear framebuffer, standard
vga registers are present.  Configuring a scanout
(VIRTIO_GPU_CMD_SET_SCANOUT) switches the device from vga
compatibility mode into native virtio mode.  A reset switches it back
into vga compatibility mode.

Note: qemu implementation also provides bochs dispi interface io ports
and mmio bar at pci region 1 and is therefore fully compatible with
the qemu stdvga (see \href{https://git.qemu-project.org/?p=qemu.git;a=blob;f=docs/specs/standard-vga.txt;hb=HEAD}{docs/specs/standard-vga.txt} in the qemu source tree).

\section{GPU Device}\label{sec:Device Types / GPU Device}

virtio-gpu is a virtio based graphics adapter.  It can operate in 2D
mode and in 3D mode.  3D mode will offload rendering ops to
the host gpu and therefore requires a gpu with 3D support on the host
machine.

In 2D mode the virtio-gpu device provides support for ARGB Hardware
cursors and multiple scanouts (aka heads).

\subsection{Device ID}\label{sec:Device Types / GPU Device / Device ID}

16

\subsection{Virtqueues}\label{sec:Device Types / GPU Device / Virtqueues}

\begin{description}
\item[0] controlq - queue for sending control commands
\item[1] cursorq - queue for sending cursor updates
\end{description}

Both queues have the same format.  Each request and each response have
a fixed header, followed by command specific data fields.  The
separate cursor queue is the "fast track" for cursor commands
(VIRTIO_GPU_CMD_UPDATE_CURSOR and VIRTIO_GPU_CMD_MOVE_CURSOR), so they
go through without being delayed by time-consuming commands in the
control queue.

\subsection{Feature bits}\label{sec:Device Types / GPU Device / Feature bits}

\begin{description}
\item[VIRTIO_GPU_F_VIRGL (0)] virgl 3D mode is supported.
\item[VIRTIO_GPU_F_EDID  (1)] EDID is supported.
\item[VIRTIO_GPU_F_RESOURCE_UUID (2)] assigning resources UUIDs for export
  to other virtio devices is supported.
\item[VIRTIO_GPU_F_RESOURCE_BLOB (3)] creating and using size-based blob
  resources is supported.
\item[VIRTIO_GPU_F_CONTEXT_INIT (4)] multiple context types and
  synchronization timelines supported.  Requires VIRTIO_GPU_F_VIRGL.
\end{description}

\subsection{Device configuration layout}\label{sec:Device Types / GPU Device / Device configuration layout}

GPU device configuration uses the following layout structure and
definitions:

\begin{lstlisting}
#define VIRTIO_GPU_EVENT_DISPLAY (1 << 0)

struct virtio_gpu_config {
        le32 events_read;
        le32 events_clear;
        le32 num_scanouts;
        le32 num_capsets;
};
\end{lstlisting}

\subsubsection{Device configuration fields}

\begin{description}
\item[\field{events_read}] signals pending events to the driver.  The
  driver MUST NOT write to this field.
\item[\field{events_clear}] clears pending events in the device.
  Writing a '1' into a bit will clear the corresponding bit in
  \field{events_read}, mimicking write-to-clear behavior.
\item[\field{num_scanouts}] specifies the maximum number of scanouts
  supported by the device.  Minimum value is 1, maximum value is 16.
\item[\field{num_capsets}] specifies the maximum number of capability
  sets supported by the device.  The minimum value is zero.
\end{description}

\subsubsection{Events}

\begin{description}
\item[VIRTIO_GPU_EVENT_DISPLAY] Display configuration has changed.
  The driver SHOULD use the VIRTIO_GPU_CMD_GET_DISPLAY_INFO command to
  fetch the information from the device.  In case EDID support is
  negotiated (VIRTIO_GPU_F_EDID feature flag) the device SHOULD also
  fetch the updated EDID blobs using the VIRTIO_GPU_CMD_GET_EDID
  command.
\end{description}

\devicenormative{\subsection}{Device Initialization}{Device Types / GPU Device / Device Initialization}

The driver SHOULD query the display information from the device using
the VIRTIO_GPU_CMD_GET_DISPLAY_INFO command and use that information
for the initial scanout setup.  In case EDID support is negotiated
(VIRTIO_GPU_F_EDID feature flag) the device SHOULD also fetch the EDID
information using the VIRTIO_GPU_CMD_GET_EDID command.  If no
information is available or all displays are disabled the driver MAY
choose to use a fallback, such as 1024x768 at display 0.

The driver SHOULD query all shared memory regions supported by the device.
If the device supports shared memory, the \field{shmid} of a region MUST
(see \ref{sec:Basic Facilities of a Virtio Device /
Shared Memory Regions}~\nameref{sec:Basic Facilities of a Virtio Device /
Shared Memory Regions}) be one of the following:

\begin{lstlisting}
enum virtio_gpu_shm_id {
        VIRTIO_GPU_SHM_ID_UNDEFINED = 0,
        VIRTIO_GPU_SHM_ID_HOST_VISIBLE = 1,
};
\end{lstlisting}

The shared memory region with VIRTIO_GPU_SHM_ID_HOST_VISIBLE is referred as
the "host visible memory region".  The device MUST support the
VIRTIO_GPU_CMD_RESOURCE_MAP_BLOB and VIRTIO_GPU_CMD_RESOURCE_UNMAP_BLOB
if the host visible memory region is available.

\subsection{Device Operation}\label{sec:Device Types / GPU Device / Device Operation}

The virtio-gpu is based around the concept of resources private to the
host.  The guest must DMA transfer into these resources, unless shared memory
regions are supported. This is a design requirement in order to interface with
future 3D rendering. In the unaccelerated 2D mode there is no support for DMA
transfers from resources, just to them.

Resources are initially simple 2D resources, consisting of a width,
height and format along with an identifier. The guest must then attach
backing store to the resources in order for DMA transfers to
work. This is like a GART in a real GPU.

\subsubsection{Device Operation: Create a framebuffer and configure scanout}

\begin{itemize*}
\item Create a host resource using VIRTIO_GPU_CMD_RESOURCE_CREATE_2D.
\item Allocate a framebuffer from guest ram, and attach it as backing
  storage to the resource just created, using
  VIRTIO_GPU_CMD_RESOURCE_ATTACH_BACKING.  Scatter lists are
  supported, so the framebuffer doesn't need to be contignous in guest
  physical memory.
\item Use VIRTIO_GPU_CMD_SET_SCANOUT to link the framebuffer to
  a display scanout.
\end{itemize*}

\subsubsection{Device Operation: Update a framebuffer and scanout}

\begin{itemize*}
\item Render to your framebuffer memory.
\item Use VIRTIO_GPU_CMD_TRANSFER_TO_HOST_2D to update the host resource
  from guest memory.
\item Use VIRTIO_GPU_CMD_RESOURCE_FLUSH to flush the updated resource
  to the display.
\end{itemize*}

\subsubsection{Device Operation: Using pageflip}

It is possible to create multiple framebuffers, flip between them
using VIRTIO_GPU_CMD_SET_SCANOUT and VIRTIO_GPU_CMD_RESOURCE_FLUSH,
and update the invisible framebuffer using
VIRTIO_GPU_CMD_TRANSFER_TO_HOST_2D.

\subsubsection{Device Operation: Multihead setup}

In case two or more displays are present there are different ways to
configure things:

\begin{itemize*}
\item Create a single framebuffer, link it to all displays
  (mirroring).
\item Create an framebuffer for each display.
\item Create one big framebuffer, configure scanouts to display a
  different rectangle of that framebuffer each.
\end{itemize*}

\devicenormative{\subsubsection}{Device Operation: Command lifecycle and fencing}{Device Types / GPU Device / Device Operation / Device Operation: Command lifecycle and fencing}

The device MAY process controlq commands asyncronously and return them
to the driver before the processing is complete.  If the driver needs
to know when the processing is finished it can set the
VIRTIO_GPU_FLAG_FENCE flag in the request.  The device MUST finish the
processing before returning the command then.

Note: current qemu implementation does asyncrounous processing only in
3d mode, when offloading the processing to the host gpu.

\subsubsection{Device Operation: Configure mouse cursor}

The mouse cursor image is a normal resource, except that it must be
64x64 in size.  The driver MUST create and populate the resource
(using the usual VIRTIO_GPU_CMD_RESOURCE_CREATE_2D,
VIRTIO_GPU_CMD_RESOURCE_ATTACH_BACKING and
VIRTIO_GPU_CMD_TRANSFER_TO_HOST_2D controlq commands) and make sure they
are completed (using VIRTIO_GPU_FLAG_FENCE).

Then VIRTIO_GPU_CMD_UPDATE_CURSOR can be sent to the cursorq to set
the pointer shape and position.  To move the pointer without updating
the shape use VIRTIO_GPU_CMD_MOVE_CURSOR instead.

\subsubsection{Device Operation: Request header}\label{sec:Device Types / GPU Device / Device Operation / Device Operation: Request header}

All requests and responses on the virtqueues have a fixed header
using the following layout structure and definitions:

\begin{lstlisting}
enum virtio_gpu_ctrl_type {

        /* 2d commands */
        VIRTIO_GPU_CMD_GET_DISPLAY_INFO = 0x0100,
        VIRTIO_GPU_CMD_RESOURCE_CREATE_2D,
        VIRTIO_GPU_CMD_RESOURCE_UNREF,
        VIRTIO_GPU_CMD_SET_SCANOUT,
        VIRTIO_GPU_CMD_RESOURCE_FLUSH,
        VIRTIO_GPU_CMD_TRANSFER_TO_HOST_2D,
        VIRTIO_GPU_CMD_RESOURCE_ATTACH_BACKING,
        VIRTIO_GPU_CMD_RESOURCE_DETACH_BACKING,
        VIRTIO_GPU_CMD_GET_CAPSET_INFO,
        VIRTIO_GPU_CMD_GET_CAPSET,
        VIRTIO_GPU_CMD_GET_EDID,
        VIRTIO_GPU_CMD_RESOURCE_ASSIGN_UUID,
        VIRTIO_GPU_CMD_RESOURCE_CREATE_BLOB,
        VIRTIO_GPU_CMD_SET_SCANOUT_BLOB,

        /* 3d commands */
        VIRTIO_GPU_CMD_CTX_CREATE = 0x0200,
        VIRTIO_GPU_CMD_CTX_DESTROY,
        VIRTIO_GPU_CMD_CTX_ATTACH_RESOURCE,
        VIRTIO_GPU_CMD_CTX_DETACH_RESOURCE,
        VIRTIO_GPU_CMD_RESOURCE_CREATE_3D,
        VIRTIO_GPU_CMD_TRANSFER_TO_HOST_3D,
        VIRTIO_GPU_CMD_TRANSFER_FROM_HOST_3D,
        VIRTIO_GPU_CMD_SUBMIT_3D,
        VIRTIO_GPU_CMD_RESOURCE_MAP_BLOB,
        VIRTIO_GPU_CMD_RESOURCE_UNMAP_BLOB,

        /* cursor commands */
        VIRTIO_GPU_CMD_UPDATE_CURSOR = 0x0300,
        VIRTIO_GPU_CMD_MOVE_CURSOR,

        /* success responses */
        VIRTIO_GPU_RESP_OK_NODATA = 0x1100,
        VIRTIO_GPU_RESP_OK_DISPLAY_INFO,
        VIRTIO_GPU_RESP_OK_CAPSET_INFO,
        VIRTIO_GPU_RESP_OK_CAPSET,
        VIRTIO_GPU_RESP_OK_EDID,
        VIRTIO_GPU_RESP_OK_RESOURCE_UUID,
        VIRTIO_GPU_RESP_OK_MAP_INFO,

        /* error responses */
        VIRTIO_GPU_RESP_ERR_UNSPEC = 0x1200,
        VIRTIO_GPU_RESP_ERR_OUT_OF_MEMORY,
        VIRTIO_GPU_RESP_ERR_INVALID_SCANOUT_ID,
        VIRTIO_GPU_RESP_ERR_INVALID_RESOURCE_ID,
        VIRTIO_GPU_RESP_ERR_INVALID_CONTEXT_ID,
        VIRTIO_GPU_RESP_ERR_INVALID_PARAMETER,
};

#define VIRTIO_GPU_FLAG_FENCE (1 << 0)
#define VIRTIO_GPU_FLAG_INFO_RING_IDX (1 << 1)

struct virtio_gpu_ctrl_hdr {
        le32 type;
        le32 flags;
        le64 fence_id;
        le32 ctx_id;
        u8 ring_idx;
        u8 padding[3];
};
\end{lstlisting}

The fixed header \field{struct virtio_gpu_ctrl_hdr} in each
request includes the following fields:

\begin{description}
\item[\field{type}] specifies the type of the driver request
  (VIRTIO_GPU_CMD_*) or device response (VIRTIO_GPU_RESP_*).
\item[\field{flags}] request / response flags.
\item[\field{fence_id}] If the driver sets the VIRTIO_GPU_FLAG_FENCE
  bit in the request \field{flags} field the device MUST:
  \begin{itemize*}
  \item set VIRTIO_GPU_FLAG_FENCE bit in the response,
  \item copy the content of the \field{fence_id} field from the
    request to the response, and
  \item send the response only after command processing is complete.
  \end{itemize*}
\item[\field{ctx_id}] Rendering context (used in 3D mode only).
\item[\field{ring_idx}] If VIRTIO_GPU_F_CONTEXT_INIT is supported, then
  the driver MAY set VIRTIO_GPU_FLAG_INFO_RING_IDX bit in the request
  \field{flags}.  In that case:
  \begin{itemize*}
  \item \field{ring_idx} indicates the value of a context-specific ring
   index.  The minimum value is 0 and maximum value is 63 (inclusive).
  \item If VIRTIO_GPU_FLAG_FENCE is set, \field{fence_id} acts as a
   sequence number on the synchronization timeline defined by
   \field{ctx_idx} and the ring index.
  \item If VIRTIO_GPU_FLAG_FENCE is set and when the command associated
   with \field{fence_id} is complete, the device MUST send a response for
   all outstanding commands with a sequence number less than or equal to
   \field{fence_id} on the same synchronization timeline.
  \end{itemize*}
\end{description}

On success the device will return VIRTIO_GPU_RESP_OK_NODATA in
case there is no payload.  Otherwise the \field{type} field will
indicate the kind of payload.

On error the device will return one of the
VIRTIO_GPU_RESP_ERR_* error codes.

\subsubsection{Device Operation: controlq}\label{sec:Device Types / GPU Device / Device Operation / Device Operation: controlq}

For any coordinates given 0,0 is top left, larger x moves right,
larger y moves down.

\begin{description}

\item[VIRTIO_GPU_CMD_GET_DISPLAY_INFO] Retrieve the current output
  configuration.  No request data (just bare \field{struct
    virtio_gpu_ctrl_hdr}).  Response type is
  VIRTIO_GPU_RESP_OK_DISPLAY_INFO, response data is \field{struct
    virtio_gpu_resp_display_info}.

\begin{lstlisting}
#define VIRTIO_GPU_MAX_SCANOUTS 16

struct virtio_gpu_rect {
        le32 x;
        le32 y;
        le32 width;
        le32 height;
};

struct virtio_gpu_resp_display_info {
        struct virtio_gpu_ctrl_hdr hdr;
        struct virtio_gpu_display_one {
                struct virtio_gpu_rect r;
                le32 enabled;
                le32 flags;
        } pmodes[VIRTIO_GPU_MAX_SCANOUTS];
};
\end{lstlisting}

The response contains a list of per-scanout information.  The info
contains whether the scanout is enabled and what its preferred
position and size is.

The size (fields \field{width} and \field{height}) is similar to the
native panel resolution in EDID display information, except that in
the virtual machine case the size can change when the host window
representing the guest display is gets resized.

The position (fields \field{x} and \field{y}) describe how the
displays are arranged (i.e. which is -- for example -- the left
display).

The \field{enabled} field is set when the user enabled the display.
It is roughly the same as the connected state of a phyiscal display
connector.

\item[VIRTIO_GPU_CMD_GET_EDID] Retrieve the EDID data for a given
  scanout.  Request data is \field{struct virtio_gpu_get_edid}).
  Response type is VIRTIO_GPU_RESP_OK_EDID, response data is
  \field{struct virtio_gpu_resp_edid}.  Support is optional and
  negotiated using the VIRTIO_GPU_F_EDID feature flag.

\begin{lstlisting}
struct virtio_gpu_get_edid {
        struct virtio_gpu_ctrl_hdr hdr;
        le32 scanout;
        le32 padding;
};

struct virtio_gpu_resp_edid {
        struct virtio_gpu_ctrl_hdr hdr;
        le32 size;
        le32 padding;
        u8 edid[1024];
};
\end{lstlisting}

The response contains the EDID display data blob (as specified by
VESA) for the scanout.

\item[VIRTIO_GPU_CMD_RESOURCE_CREATE_2D] Create a 2D resource on the
  host.  Request data is \field{struct virtio_gpu_resource_create_2d}.
  Response type is VIRTIO_GPU_RESP_OK_NODATA.

\begin{lstlisting}
enum virtio_gpu_formats {
        VIRTIO_GPU_FORMAT_B8G8R8A8_UNORM  = 1,
        VIRTIO_GPU_FORMAT_B8G8R8X8_UNORM  = 2,
        VIRTIO_GPU_FORMAT_A8R8G8B8_UNORM  = 3,
        VIRTIO_GPU_FORMAT_X8R8G8B8_UNORM  = 4,

        VIRTIO_GPU_FORMAT_R8G8B8A8_UNORM  = 67,
        VIRTIO_GPU_FORMAT_X8B8G8R8_UNORM  = 68,

        VIRTIO_GPU_FORMAT_A8B8G8R8_UNORM  = 121,
        VIRTIO_GPU_FORMAT_R8G8B8X8_UNORM  = 134,
};

struct virtio_gpu_resource_create_2d {
        struct virtio_gpu_ctrl_hdr hdr;
        le32 resource_id;
        le32 format;
        le32 width;
        le32 height;
};
\end{lstlisting}

This creates a 2D resource on the host with the specified width,
height and format.  The resource ids are generated by the guest.

\item[VIRTIO_GPU_CMD_RESOURCE_UNREF] Destroy a resource.  Request data
  is \field{struct virtio_gpu_resource_unref}.  Response type is
  VIRTIO_GPU_RESP_OK_NODATA.

\begin{lstlisting}
struct virtio_gpu_resource_unref {
        struct virtio_gpu_ctrl_hdr hdr;
        le32 resource_id;
        le32 padding;
};
\end{lstlisting}

This informs the host that a resource is no longer required by the
guest.

\item[VIRTIO_GPU_CMD_SET_SCANOUT] Set the scanout parameters for a
  single output.  Request data is \field{struct
    virtio_gpu_set_scanout}.  Response type is
  VIRTIO_GPU_RESP_OK_NODATA.

\begin{lstlisting}
struct virtio_gpu_set_scanout {
        struct virtio_gpu_ctrl_hdr hdr;
        struct virtio_gpu_rect r;
        le32 scanout_id;
        le32 resource_id;
};
\end{lstlisting}

This sets the scanout parameters for a single scanout. The resource_id
is the resource to be scanned out from, along with a rectangle.

Scanout rectangles must be completely covered by the underlying
resource.  Overlapping (or identical) scanouts are allowed, typical
use case is screen mirroring.

The driver can use resource_id = 0 to disable a scanout.

\item[VIRTIO_GPU_CMD_RESOURCE_FLUSH] Flush a scanout resource Request
  data is \field{struct virtio_gpu_resource_flush}.  Response type is
  VIRTIO_GPU_RESP_OK_NODATA.

\begin{lstlisting}
struct virtio_gpu_resource_flush {
        struct virtio_gpu_ctrl_hdr hdr;
        struct virtio_gpu_rect r;
        le32 resource_id;
        le32 padding;
};
\end{lstlisting}

This flushes a resource to screen.  It takes a rectangle and a
resource id, and flushes any scanouts the resource is being used on.

\item[VIRTIO_GPU_CMD_TRANSFER_TO_HOST_2D] Transfer from guest memory
  to host resource.  Request data is \field{struct
    virtio_gpu_transfer_to_host_2d}.  Response type is
  VIRTIO_GPU_RESP_OK_NODATA.

\begin{lstlisting}
struct virtio_gpu_transfer_to_host_2d {
        struct virtio_gpu_ctrl_hdr hdr;
        struct virtio_gpu_rect r;
        le64 offset;
        le32 resource_id;
        le32 padding;
};
\end{lstlisting}

This takes a resource id along with an destination offset into the
resource, and a box to transfer to the host backing for the resource.

\item[VIRTIO_GPU_CMD_RESOURCE_ATTACH_BACKING] Assign backing pages to
  a resource.  Request data is \field{struct
    virtio_gpu_resource_attach_backing}, followed by \field{struct
    virtio_gpu_mem_entry} entries.  Response type is
  VIRTIO_GPU_RESP_OK_NODATA.

\begin{lstlisting}
struct virtio_gpu_resource_attach_backing {
        struct virtio_gpu_ctrl_hdr hdr;
        le32 resource_id;
        le32 nr_entries;
};

struct virtio_gpu_mem_entry {
        le64 addr;
        le32 length;
        le32 padding;
};
\end{lstlisting}

This assign an array of guest pages as the backing store for a
resource. These pages are then used for the transfer operations for
that resource from that point on.

\item[VIRTIO_GPU_CMD_RESOURCE_DETACH_BACKING] Detach backing pages
  from a resource.  Request data is \field{struct
    virtio_gpu_resource_detach_backing}.  Response type is
  VIRTIO_GPU_RESP_OK_NODATA.

\begin{lstlisting}
struct virtio_gpu_resource_detach_backing {
        struct virtio_gpu_ctrl_hdr hdr;
        le32 resource_id;
        le32 padding;
};
\end{lstlisting}

This detaches any backing pages from a resource, to be used in case of
guest swapping or object destruction.

\item[VIRTIO_GPU_CMD_GET_CAPSET_INFO] Gets the information associated with
  a particular \field{capset_index}, which MUST less than \field{num_capsets}
  defined in the device configuration.  Request data is
  \field{struct virtio_gpu_get_capset_info}.  Response type is
  VIRTIO_GPU_RESP_OK_CAPSET_INFO.

  On success, \field{struct virtio_gpu_resp_capset_info} contains the
  \field{capset_id}, \field{capset_max_version}, \field{capset_max_size}
  associated with capset at the specified {capset_idex}.  field{capset_id} MUST
  be one of the following (see listing for values):

  \begin{itemize*}
  \item \href{https://gitlab.freedesktop.org/virgl/virglrenderer/-/blob/master/src/virgl_hw.h#L526}{VIRTIO_GPU_CAPSET_VIRGL} --
	the first edition of Virgl (Gallium OpenGL) protocol.
  \item \href{https://gitlab.freedesktop.org/virgl/virglrenderer/-/blob/master/src/virgl_hw.h#L550}{VIRTIO_GPU_CAPSET_VIRGL2} --
	the second edition of Virgl (Gallium OpenGL) protocol after the capset fix.
  \item \href{https://android.googlesource.com/device/generic/vulkan-cereal/+/refs/heads/master/protocols/}{VIRTIO_GPU_CAPSET_GFXSTREAM} --
	gfxtream's (mostly) autogenerated GLES and Vulkan streaming protocols.
  \item \href{https://gitlab.freedesktop.org/olv/venus-protocol}{VIRTIO_GPU_CAPSET_VENUS} --
	Mesa's (mostly) autogenerated Vulkan protocol.
  \item \href{https://chromium.googlesource.com/chromiumos/platform/crosvm/+/refs/heads/main/rutabaga_gfx/src/cross_domain/cross_domain_protocol.rs}{VIRTIO_GPU_CAPSET_CROSS_DOMAIN} --
	protocol for display virtualization via Wayland proxying.
  \end{itemize*}

\begin{lstlisting}
struct virtio_gpu_get_capset_info {
        struct virtio_gpu_ctrl_hdr hdr;
        le32 capset_index;
        le32 padding;
};

#define VIRTIO_GPU_CAPSET_VIRGL 1
#define VIRTIO_GPU_CAPSET_VIRGL2 2
#define VIRTIO_GPU_CAPSET_GFXSTREAM 3
#define VIRTIO_GPU_CAPSET_VENUS 4
#define VIRTIO_GPU_CAPSET_CROSS_DOMAIN 5
struct virtio_gpu_resp_capset_info {
        struct virtio_gpu_ctrl_hdr hdr;
        le32 capset_id;
        le32 capset_max_version;
        le32 capset_max_size;
        le32 padding;
};
\end{lstlisting}

\item[VIRTIO_GPU_CMD_GET_CAPSET] Gets the capset associated with a
  particular \field{capset_id} and \field{capset_version}.  Request data is
  \field{struct virtio_gpu_get_capset}.  Response type is
  VIRTIO_GPU_RESP_OK_CAPSET.

\begin{lstlisting}
struct virtio_gpu_get_capset {
        struct virtio_gpu_ctrl_hdr hdr;
        le32 capset_id;
        le32 capset_version;
};

struct virtio_gpu_resp_capset {
        struct virtio_gpu_ctrl_hdr hdr;
        u8 capset_data[];
};
\end{lstlisting}

\item[VIRTIO_GPU_CMD_RESOURCE_ASSIGN_UUID] Creates an exported object from
  a resource. Request data is \field{struct
    virtio_gpu_resource_assign_uuid}.  Response type is
  VIRTIO_GPU_RESP_OK_RESOURCE_UUID, response data is \field{struct
    virtio_gpu_resp_resource_uuid}. Support is optional and negotiated
    using the VIRTIO_GPU_F_RESOURCE_UUID feature flag.

\begin{lstlisting}
struct virtio_gpu_resource_assign_uuid {
        struct virtio_gpu_ctrl_hdr hdr;
        le32 resource_id;
        le32 padding;
};

struct virtio_gpu_resp_resource_uuid {
        struct virtio_gpu_ctrl_hdr hdr;
        u8 uuid[16];
};
\end{lstlisting}

The response contains a UUID which identifies the exported object created from
the host private resource. Note that if the resource has an attached backing,
modifications made to the host private resource through the exported object by
other devices are not visible in the attached backing until they are transferred
into the backing.

\item[VIRTIO_GPU_CMD_RESOURCE_CREATE_BLOB] Creates a virtio-gpu blob
  resource. Request data is \field{struct
  virtio_gpu_resource_create_blob}, followed by \field{struct
  virtio_gpu_mem_entry} entries. Response type is
  VIRTIO_GPU_RESP_OK_NODATA. Support is optional and negotiated
  using the VIRTIO_GPU_F_RESOURCE_BLOB feature flag.

\begin{lstlisting}
#define VIRTIO_GPU_BLOB_MEM_GUEST             0x0001
#define VIRTIO_GPU_BLOB_MEM_HOST3D            0x0002
#define VIRTIO_GPU_BLOB_MEM_HOST3D_GUEST      0x0003

#define VIRTIO_GPU_BLOB_FLAG_USE_MAPPABLE     0x0001
#define VIRTIO_GPU_BLOB_FLAG_USE_SHAREABLE    0x0002
#define VIRTIO_GPU_BLOB_FLAG_USE_CROSS_DEVICE 0x0004

struct virtio_gpu_resource_create_blob {
       struct virtio_gpu_ctrl_hdr hdr;
       le32 resource_id;
       le32 blob_mem;
       le32 blob_flags;
       le32 nr_entries;
       le64 blob_id;
       le64 size;
};

\end{lstlisting}

A blob resource is a container for:

  \begin{itemize*}
  \item a guest memory allocation (referred to as a
  "guest-only blob resource").
  \item a host memory allocation (referred to as a
  "host-only blob resource").
  \item a guest memory and host memory allocation (referred
  to as a "default blob resource").
  \end{itemize*}

The memory properties of the blob resource MUST be described by
\field{blob_mem}, which MUST be non-zero.

For default and guest-only blob resources, \field{nr_entries} guest
memory entries may be assigned to the resource.  For default blob resources
(i.e, when \field{blob_mem} is VIRTIO_GPU_BLOB_MEM_HOST3D_GUEST), these
memory entries are used as a shadow buffer for the host memory. To
facilitate drivers that support swap-in and swap-out, \field{nr_entries} may
be zero and VIRTIO_GPU_CMD_RESOURCE_ATTACH_BACKING may be subsequently used.
VIRTIO_GPU_CMD_RESOURCE_DETACH_BACKING may be used to unassign memory entries.

\field{blob_mem} can only be VIRTIO_GPU_BLOB_MEM_HOST3D and
VIRTIO_GPU_BLOB_MEM_HOST3D_GUEST if VIRTIO_GPU_F_VIRGL is supported.
VIRTIO_GPU_BLOB_MEM_GUEST is valid regardless whether VIRTIO_GPU_F_VIRGL
is supported or not.

For VIRTIO_GPU_BLOB_MEM_HOST3D and VIRTIO_GPU_BLOB_MEM_HOST3D_GUEST, the
virtio-gpu resource MUST be created from the rendering context local object
identified by the \field{blob_id}. The actual allocation is done via
VIRTIO_GPU_CMD_SUBMIT_3D.

The driver MUST inform the device if the blob resource is used for
memory access, sharing between driver instances and/or sharing with
other devices. This is done via the \field{blob_flags} field.

If VIRTIO_GPU_F_VIRGL is set, both VIRTIO_GPU_CMD_TRANSFER_TO_HOST_3D
and VIRTIO_GPU_CMD_TRANSFER_FROM_HOST_3D may be used to update the
resource. There is no restriction on the image/buffer view the driver
has on the blob resource.

\item[VIRTIO_GPU_CMD_SET_SCANOUT_BLOB] sets scanout parameters for a
   blob resource. Request data is
  \field{struct virtio_gpu_set_scanout_blob}. Response type is
  VIRTIO_GPU_RESP_OK_NODATA. Support is optional and negotiated
  using the VIRTIO_GPU_F_RESOURCE_BLOB feature flag.

\begin{lstlisting}
struct virtio_gpu_set_scanout_blob {
       struct virtio_gpu_ctrl_hdr hdr;
       struct virtio_gpu_rect r;
       le32 scanout_id;
       le32 resource_id;
       le32 width;
       le32 height;
       le32 format;
       le32 padding;
       le32 strides[4];
       le32 offsets[4];
};
\end{lstlisting}

The rectangle \field{r} represents the portion of the blob resource being
displayed. The rest is the metadata associated with the blob resource. The
format MUST be one of \field{enum virtio_gpu_formats}.  The format MAY be
compressed with header and data planes.

\end{description}

\subsubsection{Device Operation: controlq (3d)}\label{sec:Device Types / GPU Device / Device Operation / Device Operation: controlq (3d)}

These commands are supported by the device if the VIRTIO_GPU_F_VIRGL
feature flag is set.

\begin{description}

\item[VIRTIO_GPU_CMD_CTX_CREATE] creates a context for submitting an opaque
  command stream.  Request data is \field{struct virtio_gpu_ctx_create}.
  Response type is VIRTIO_GPU_RESP_OK_NODATA.

\begin{lstlisting}
#define VIRTIO_GPU_CONTEXT_INIT_CAPSET_ID_MASK 0x000000ff;
struct virtio_gpu_ctx_create {
       struct virtio_gpu_ctrl_hdr hdr;
       le32 nlen;
       le32 context_init;
       char debug_name[64];
};
\end{lstlisting}

The implementation MUST create a context for the given \field{ctx_id} in
the \field{hdr}.  For debugging purposes, a \field{debug_name} and it's
length \field{nlen} is provided by the driver.  If
VIRTIO_GPU_F_CONTEXT_INIT is supported, then lower 8 bits of
\field{context_init} MAY contain the \field{capset_id} associated with
context.  In that case, then the device MUST create a context that can
handle the specified command stream.

If the lower 8-bits of the \field{context_init} are zero, then the type of
the context is determined by the device.

\item[VIRTIO_GPU_CMD_CTX_DESTROY]
\item[VIRTIO_GPU_CMD_CTX_ATTACH_RESOURCE]
\item[VIRTIO_GPU_CMD_CTX_DETACH_RESOURCE]
  Manage virtio-gpu 3d contexts.

\item[VIRTIO_GPU_CMD_RESOURCE_CREATE_3D]
  Create virtio-gpu 3d resources.

\item[VIRTIO_GPU_CMD_TRANSFER_TO_HOST_3D]
\item[VIRTIO_GPU_CMD_TRANSFER_FROM_HOST_3D]
  Transfer data from and to virtio-gpu 3d resources.

\item[VIRTIO_GPU_CMD_SUBMIT_3D]
  Submit an opaque command stream.  The type of the command stream is
  determined when creating a context.

\item[VIRTIO_GPU_CMD_RESOURCE_MAP_BLOB] maps a host-only
  blob resource into an offset in the host visible memory region. Request
  data is \field{struct virtio_gpu_resource_map_blob}.  The driver MUST
  not map a blob resource that is already mapped.  Response type is
  VIRTIO_GPU_RESP_OK_MAP_INFO. Support is optional and negotiated
  using the VIRTIO_GPU_F_RESOURCE_BLOB feature flag and checking for
  the presence of the host visible memory region.

\begin{lstlisting}
struct virtio_gpu_resource_map_blob {
        struct virtio_gpu_ctrl_hdr hdr;
        le32 resource_id;
        le32 padding;
        le64 offset;
};

#define VIRTIO_GPU_MAP_CACHE_MASK      0x0f
#define VIRTIO_GPU_MAP_CACHE_NONE      0x00
#define VIRTIO_GPU_MAP_CACHE_CACHED    0x01
#define VIRTIO_GPU_MAP_CACHE_UNCACHED  0x02
#define VIRTIO_GPU_MAP_CACHE_WC        0x03
struct virtio_gpu_resp_map_info {
        struct virtio_gpu_ctrl_hdr hdr;
        u32 map_info;
        u32 padding;
};
\end{lstlisting}

\item[VIRTIO_GPU_CMD_RESOURCE_UNMAP_BLOB] unmaps a
  host-only blob resource from the host visible memory region. Request data
  is \field{struct virtio_gpu_resource_unmap_blob}.  Response type is
  VIRTIO_GPU_RESP_OK_NODATA.  Support is optional and negotiated
  using the VIRTIO_GPU_F_RESOURCE_BLOB feature flag and checking for
  the presence of the host visible memory region.

\begin{lstlisting}
struct virtio_gpu_resource_unmap_blob {
        struct virtio_gpu_ctrl_hdr hdr;
        le32 resource_id;
        le32 padding;
};
\end{lstlisting}

\end{description}

\subsubsection{Device Operation: cursorq}\label{sec:Device Types / GPU Device / Device Operation / Device Operation: cursorq}

Both cursorq commands use the same command struct.

\begin{lstlisting}
struct virtio_gpu_cursor_pos {
        le32 scanout_id;
        le32 x;
        le32 y;
        le32 padding;
};

struct virtio_gpu_update_cursor {
        struct virtio_gpu_ctrl_hdr hdr;
        struct virtio_gpu_cursor_pos pos;
        le32 resource_id;
        le32 hot_x;
        le32 hot_y;
        le32 padding;
};
\end{lstlisting}

\begin{description}

\item[VIRTIO_GPU_CMD_UPDATE_CURSOR]
Update cursor.
Request data is \field{struct virtio_gpu_update_cursor}.
Response type is VIRTIO_GPU_RESP_OK_NODATA.

Full cursor update.  Cursor will be loaded from the specified
\field{resource_id} and will be moved to \field{pos}.  The driver must
transfer the cursor into the resource beforehand (using control queue
commands) and make sure the commands to fill the resource are actually
processed (using fencing).

\item[VIRTIO_GPU_CMD_MOVE_CURSOR]
Move cursor.
Request data is \field{struct virtio_gpu_update_cursor}.
Response type is VIRTIO_GPU_RESP_OK_NODATA.

Move cursor to the place specified in \field{pos}.  The other fields
are not used and will be ignored by the device.

\end{description}

\subsection{VGA Compatibility}\label{sec:Device Types / GPU Device / VGA Compatibility}

Applies to Virtio Over PCI only.  The GPU device can come with and
without VGA compatibility.  The PCI class should be DISPLAY_VGA if VGA
compatibility is present and DISPLAY_OTHER otherwise.

VGA compatibility: PCI region 0 has the linear framebuffer, standard
vga registers are present.  Configuring a scanout
(VIRTIO_GPU_CMD_SET_SCANOUT) switches the device from vga
compatibility mode into native virtio mode.  A reset switches it back
into vga compatibility mode.

Note: qemu implementation also provides bochs dispi interface io ports
and mmio bar at pci region 1 and is therefore fully compatible with
the qemu stdvga (see \href{https://git.qemu-project.org/?p=qemu.git;a=blob;f=docs/specs/standard-vga.txt;hb=HEAD}{docs/specs/standard-vga.txt} in the qemu source tree).


\chapter{Reserved Feature Bits}\label{sec:Reserved Feature Bits}

Currently these device-independent feature bits are defined:

\begin{description}
  \item[VIRTIO_F_INDIRECT_DESC (28)] Negotiating this feature indicates
  that the driver can use descriptors with the VIRTQ_DESC_F_INDIRECT
  flag set, as described in \ref{sec:Basic Facilities of a Virtio
Device / Virtqueues / The Virtqueue Descriptor Table / Indirect
Descriptors}~\nameref{sec:Basic Facilities of a Virtio Device /
Virtqueues / The Virtqueue Descriptor Table / Indirect
Descriptors} and \ref{sec:Packed Virtqueues / Indirect Flag: Scatter-Gather Support}~\nameref{sec:Packed Virtqueues / Indirect Flag: Scatter-Gather Support}.
  \item[VIRTIO_F_EVENT_IDX(29)] This feature enables the \field{used_event}
  and the \field{avail_event} fields as described in
\ref{sec:Basic Facilities of a Virtio Device / Virtqueues / Used Buffer Notification Suppression}, \ref{sec:Basic Facilities of a Virtio Device / Virtqueues / The Virtqueue Used Ring} and \ref{sec:Packed Virtqueues / Driver and Device Event Suppression}.


  \item[VIRTIO_F_VERSION_1(32)] This indicates compliance with this
    specification, giving a simple way to detect legacy devices or drivers.

  \item[VIRTIO_F_ACCESS_PLATFORM(33)] This feature indicates that
  the device can be used on a platform where device access to data
  in memory is limited and/or translated. E.g. this is the case if the device can be located
  behind an IOMMU that translates bus addresses from the device into physical
  addresses in memory, if the device can be limited to only access
  certain memory addresses or if special commands such as
  a cache flush can be needed to synchronise data in memory with
  the device. Whether accesses are actually limited or translated
  is described by platform-specific means.
  If this feature bit is set to 0, then the device
  has same access to memory addresses supplied to it as the
  driver has.
  In particular, the device will always use physical addresses
  matching addresses used by the driver (typically meaning
  physical addresses used by the CPU)
  and not translated further, and can access any address supplied to it by
  the driver. When clear, this overrides any platform-specific description of
  whether device access is limited or translated in any way, e.g.
  whether an IOMMU may be present.
  \item[VIRTIO_F_RING_PACKED(34)] This feature indicates
  support for the packed virtqueue layout as described in
  \ref{sec:Basic Facilities of a Virtio Device / Packed Virtqueues}~\nameref{sec:Basic Facilities of a Virtio Device / Packed Virtqueues}.
  \item[VIRTIO_F_IN_ORDER(35)] This feature indicates
  that all buffers are used by the device in the same
  order in which they have been made available.
  \item[VIRTIO_F_ORDER_PLATFORM(36)] This feature indicates
  that memory accesses by the driver and the device are ordered
  in a way described by the platform.

  If this feature bit is negotiated, the ordering in effect for any
  memory accesses by the driver that need to be ordered in a specific way
  with respect to accesses by the device is the one suitable for devices
  described by the platform. This implies that the driver needs to use
  memory barriers suitable for devices described by the platform; e.g.
  for the PCI transport in the case of hardware PCI devices.

  If this feature bit is not negotiated, then the device
  and driver are assumed to be implemented in software, that is
  they can be assumed to run on identical CPUs
  in an SMP configuration.
  Thus a weaker form of memory barriers is sufficient
  to yield better performance.
  \item[VIRTIO_F_SR_IOV(37)] This feature indicates that
  the device supports Single Root I/O Virtualization.
  Currently only PCI devices support this feature.
  \item[VIRTIO_F_NOTIFICATION_DATA(38)] This feature indicates
  that the driver passes extra data (besides identifying the virtqueue)
  in its device notifications.
  See \ref{sec:Basic Facilities of a Virtio Device / Driver notifications}~\nameref{sec:Basic Facilities of a Virtio Device / Driver notifications}.

  \item[VIRTIO_F_NOTIF_CONFIG_DATA(39)] This feature indicates that the driver
  uses the data provided by the device as a virtqueue identifier in available
  buffer notifications.
  As mentioned in section \ref{sec:Basic Facilities of a Virtio Device / Driver notifications}, when the
  driver is required to send an available buffer notification to the device, it
  sends the virtqueue index to be notified. The method of delivering
  notifications is transport specific.
  With the PCI transport, the device can optionally provide a per-virtqueue value
  for the driver to use in driver notifications, instead of the virtqueue index.
  Some devices may benefit from this flexibility by providing, for example,
  an internal virtqueue identifier, or an internal offset related to the
  virtqueue index.

  This feature indicates the availability of such value. The definition of the
  data to be provided in driver notification and the delivery method is
  transport specific.
  For more details about driver notifications over PCI see \ref{sec:Virtio Transport Options / Virtio Over PCI Bus / PCI-specific Initialization And Device Operation / Available Buffer Notifications}.

  \item[VIRTIO_F_RING_RESET(40)] This feature indicates
  that the driver can reset a queue individually.
  See \ref{sec:Basic Facilities of a Virtio Device / Virtqueues / Virtqueue Reset}.

  \item[VIRTIO_F_ADMIN_VQ(41)] This feature indicates that the device exposes one or more
  administration virtqueues.
  At the moment this feature is only supported for devices using
  \ref{sec:Virtio Transport Options / Virtio Over PCI
	  Bus}~\nameref{sec:Virtio Transport Options / Virtio Over PCI Bus}
	  as the transport and is reserved for future use for
	  devices using other transports (see
	  \ref{drivernormative:Basic Facilities of a Virtio Device / Feature Bits}
	and
	\ref{devicenormative:Basic Facilities of a Virtio Device / Feature Bits} for
	handling features reserved for future use.

  \item[VIRTIO_F_QUEUE_STATE(42)] This feature indicates that the device allows the driver
  to access its internal virtqueue state.

  \item[VIRTIO_F_SUSPEND(43)] This feature indicates that the driver can
   suspend the device.
   See \ref{sec:Basic Facilities of a Virtio Device / Device Status Field}.

\end{description}

\drivernormative{\section}{Reserved Feature Bits}{Reserved Feature Bits}

A driver MUST accept VIRTIO_F_VERSION_1 if it is offered.  A driver
MAY fail to operate further if VIRTIO_F_VERSION_1 is not offered.

A driver SHOULD accept VIRTIO_F_ACCESS_PLATFORM if it is offered, and it MUST
then either disable the IOMMU or configure the IOMMU to translate bus addresses
passed to the device into physical addresses in memory.  If
VIRTIO_F_ACCESS_PLATFORM is not offered, then a driver MUST pass only physical
addresses to the device.

A driver SHOULD accept VIRTIO_F_RING_PACKED if it is offered.

A driver SHOULD accept VIRTIO_F_ORDER_PLATFORM if it is offered.
If VIRTIO_F_ORDER_PLATFORM has been negotiated, a driver MUST use
the barriers suitable for hardware devices.

If VIRTIO_F_SR_IOV has been negotiated, a driver MAY enable
virtual functions through the device's PCI SR-IOV capability
structure.  A driver MUST NOT negotiate VIRTIO_F_SR_IOV if
the device does not have a PCI SR-IOV capability structure
or is not a PCI device.  A driver MUST negotiate
VIRTIO_F_SR_IOV and complete the feature negotiation
(including checking the FEATURES_OK \field{device status}
bit) before enabling virtual functions through the device's
PCI SR-IOV capability structure.  After once successfully
negotiating VIRTIO_F_SR_IOV, the driver MAY enable virtual
functions through the device's PCI SR-IOV capability
structure even if the device or the system has been fully
or partially reset, and even without re-negotiating
VIRTIO_F_SR_IOV after the reset.

A driver SHOULD accept VIRTIO_F_NOTIF_CONFIG_DATA if it is offered.

\devicenormative{\section}{Reserved Feature Bits}{Reserved Feature Bits}

A device MUST offer VIRTIO_F_VERSION_1.  A device MAY fail to operate further
if VIRTIO_F_VERSION_1 is not accepted.

A device SHOULD offer VIRTIO_F_ACCESS_PLATFORM if its access to
memory is through bus addresses distinct from and translated
by the platform to physical addresses used by the driver, and/or
if it can only access certain memory addresses with said access
specified and/or granted by the platform.
A device MAY fail to operate further if VIRTIO_F_ACCESS_PLATFORM is not
accepted.

If VIRTIO_F_IN_ORDER has been negotiated, a device MUST use
buffers in the same order in which they have been available.

A device MAY fail to operate further if
VIRTIO_F_ORDER_PLATFORM is offered but not accepted.
A device MAY operate in a slower emulation mode if
VIRTIO_F_ORDER_PLATFORM is offered but not accepted.

It is RECOMMENDED that an add-in card based PCI device
offers both VIRTIO_F_ACCESS_PLATFORM and
VIRTIO_F_ORDER_PLATFORM for maximum portability.

A device SHOULD offer VIRTIO_F_SR_IOV if it is a PCI device
and presents a PCI SR-IOV capability structure, otherwise
it MUST NOT offer VIRTIO_F_SR_IOV.

\section{Legacy Interface: Reserved Feature Bits}\label{sec:Reserved Feature Bits / Legacy Interface: Reserved Feature Bits}

Transitional devices MAY offer the following:
\begin{description}
\item[VIRTIO_F_NOTIFY_ON_EMPTY (24)] If this feature
  has been negotiated by driver, the device MUST issue
  a used buffer notification if the device runs
  out of available descriptors on a virtqueue, even though
  notifications are suppressed using the VIRTQ_AVAIL_F_NO_INTERRUPT
  flag or the \field{used_event} field.
\begin{note}
  An example of a driver using this feature is the legacy
  networking driver: it doesn't need to know every time a packet
  is transmitted, but it does need to free the transmitted
  packets a finite time after they are transmitted. It can avoid
  using a timer if the device notifies it when all the packets
  are transmitted.
\end{note}
\end{description}

Transitional devices MUST offer, and if offered by the device
transitional drivers MUST accept the following:
\begin{description}
\item[VIRTIO_F_ANY_LAYOUT (27)] This feature indicates that the device
  accepts arbitrary descriptor layouts, as described in Section
  \ref{sec:Basic Facilities of a Virtio Device / Virtqueues / Message Framing / Legacy Interface: Message Framing}~\nameref{sec:Basic Facilities of a Virtio Device / Virtqueues / Message Framing / Legacy Interface: Message Framing}.

\item[UNUSED (30)] Bit 30 is used by qemu's implementation to check
  for experimental early versions of virtio which did not perform
  correct feature negotiation, and SHOULD NOT be negotiated.
\end{description}
